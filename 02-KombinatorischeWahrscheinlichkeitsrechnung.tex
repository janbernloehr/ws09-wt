\chapter{Kombinatorische Wahrscheinlichkeitsrechnung}

In der kombinatorischen Wahrscheinlichkeitsrechnung legen wir Laplacesche
W-Räume zugrunde. Die Abzählung der ``günstigen'' und der ``möglichen'' Fälle
erfolgt systematisch mit Hilfe der Kombinatorik.

\begin{defn}
\label{defn:2.1} Gegeben seien $n$ --- nicht notwendiger Weise verschiedene ---
Elemente $a_{1}, \ldots, a_{n}$. Das $n$-Tupel $({\displaystyle a_{i_{1}},
\ldots , a_{i_{n}}})$ mit $i_{j} \in \setd{1, \ldots, n}$, $i_{j} \neq i_{k}$ für
$j \neq k$ $(j,k=1, \ldots , n)$ heißt eine \emph{Permutation} der gegebenen
Elemente.\fishhere
\end{defn}

In der Kombinatorik spielt die mögliche Anzahl an Permutationen eine sehr große
Rolle.

\begin{prop}
\label{prop:2.1}
Die Anzahl der Permutationen von $n$ \textit{verschiedenen} Elementen
ist $n!$.
\end{prop}

Eine Verallgemeinerung für nicht notwendiger weise verschiedene Elemente
liefert der folgende
\begin{prop}
\label{prop:2.2}
Gegeben seien $n$ Elemente; die verschiedenen Elemente darunter
seien mit $a_{1}, \ldots, a_{p}$ bezeichnet $($mit $p \leq n)$. Tritt $a_{i}$
$n_{i}$-fach auf $(i=1, \ldots, p)$, wobei $\sum^{p}_{i=1} n_{i}=n$,
dann können die $n$ Elemente auf
\begin{align*}
\frac{n!}{n_{1}! \ldots n_{p}!}
\end{align*}
verschiedene Arten permutiert werden.\fishhere
\end{prop}

\begin{defn}
\label{defn:2.2}
Sei $A$ eine $n$-elementige Menge, $k \in \mathbb{N}$.
\begin{defnenum}
\item[1.)]
[2.)] Jedes $k$-Tupel $(a_{1}, \ldots, a_{k})$ mit nicht notwendig verschiedenen
[lauter verschiedenen] $a_{i} \in A$ heißt eine \emph{Kombination
$k$-ter Ordnung aus $A$ mit [ohne] Wiederholung und mit
Berücksichtigung der Anordnung}.
\item[3.)]
[4.)] Werden in a) [2.)] Kombinationen, welche dieselben Elemente in
verschiedener
Anordnung enthalten, als äquivalent aufgefasst, so heißen die einzelnen
Äquivalenz\-klassen \emph{Kombinationen $k$-ter Ordnung aus $A$ mit [ohne]
Wiederholung und ohne Berück\-sichtigung der Anordnung}.\fishhere
\end{defnenum}
\end{defn}

\textit{Interpretation} (mit $A= \{1,\ldots ,n\}$): Aufteilung von $k$
Kugeln auf $n$ Zellen, wobei in 1.) [2.)] die Zellen mehrfach [nicht mehrfach]
besetzt werden dürfen und die Kugeln unterscheidbar sind ($a_{i}$ \ldots Nummer der Zelle, in
der die $i$-te Kugel liegt), in 3.) [4.)] die Zellen mehrfach [nicht mehrfach]
besetzt werden dürfen und die Kugeln nicht unterscheidbar sind (ein Repräsentant
der Äquivalenzklasse ist gegeben durch ein $k$-tupel von Zahlen aus
$\{1, \ldots , n\}$, in dem die Zellennummern so oft auftreten, wie die
zugehörige Zelle besetzt ist).\\

\begin{prop}
\label{prop:2.3}
Die Anzahl der Kombinationen $k$-ter Ordnung aus der
$n$-elementigen Menge $A$ -- mit [ohne] Wiederholung und mit [ohne]
Berücksichtigung der Anordnung -- ist gegeben durch

\vspace{0.2cm}
\begin{tabular}{l|l|l}
 & m.\ Wiederholung & o.\ Wiederholung $(1 \leq k \leq n )$ \\ \hline
 m.\ Ber. der Anordnung & $n^{k}$ & $n(n-1) \ldots (n-k+1)$ \\
 \hline
 o.\ Ber. der Anordnung & ${n+k-1 \choose k}$& ${n \choose k}$ \\
\hline
\end{tabular}

\vspace{0.5cm}
Bestimmung von Fakultäten durch Tabellen für $\log n!$  und -- bei großem $n$ --
durch die \emph{Stirlingsche Formel}
\begin{align*}
 n! \cong \left(\frac{n}{e}\right)^{n} \sqrt{2 \pi n } \quad (n \to \infty )
\end{align*}
und ihre Verschärfung
\begin{align*}
\exp \frac{1}{12 n+1}<{\displaystyle\frac{n!}{(\frac{n}{e})^{n}\sqrt{2\pi
n}}} < \exp \frac{1}{12 n}\, , \quad n \in \mathbb{N} \, .\fishhere
\end{align*}
\end{prop}
\begin{proof}
Wir beweisen exemplarisch die Formel für die Anzahl der Kombinationen ohne
Berücksichtigung der Anordnung und mit Wiederholungen. Betrachte dazu $k$
Kugeln, die auf $n$ Zellen verteilt werden sollen.

\begin{figure}[H]
\centering
\begin{pspicture}(0,-0.23)(2.42,0.23)
\psline(0.0,0.21)(0.0,-0.21)
\psline(0.62,0.19)(0.62,-0.21)
\psline(1.22,0.19)(1.22,-0.21)
\psline(1.8,0.19)(1.8,-0.21)
\psline(2.4,0.19)(2.4,-0.21)
\psdots[linecolor=darkblue](0.2,0.11)
\psdots[linecolor=darkblue](0.36,-0.13)
\psdots[linecolor=darkblue](1.36,0.05)
\psdots[linecolor=darkblue](1.56,-0.13)
\psdots[linecolor=darkblue](1.64,0.11)
\end{pspicture} 
\caption{Zur Platzierung von $k$ Kugeln in $n$ Zellen}
\end{figure}

Ohne die äußeren Trennwende gibt es also $k$ Kugeln und $n-1$ Trennwände, d.h.
es gibt insgesamt $(n-1+k)!$ Permutationen. Berücksichtigen wir nun noch die
Ununterscheidbarkeit der Kugeln, so erhalten wir
\begin{align*}
\binom{n+k-1}{k}.\qedhere
\end{align*}
\end{proof}

\begin{bspn}
\textit{Anwendung in der Physik}.
Man beschreibt das makroskopische Verhalten von
$k$ Teilchen in der Weise, dass man den (der Darstellung des Momentanzustandes
dienenden) Phasenraum in $n$ kongruente würfelförmige Zellen zerlegt und die
Wahrscheinlichkeit \mbox{$p(k_{1}, \ldots , k_{n})$} bestimmt, dass sich genau
$k_{i}$
der Teilchen in der $i$-ten Zelle befinden $(i \in \{ 1, \ldots, n\})$. Sei
$\Omega '$ die Menge aller $n$-Tupel $(k_{1}, \ldots, k_{n}) \in \mathbb{N}_{0}^{n}$ von
Besetzungszahlen mit $\sum\limits^{n}_{i=1}k_{i} = k$. W-Maße auf
$\PP(\Omega')$ (charakterisiert durch $p$):
\begin{bspenum}
\item \emph{Maxwell-Boltzmann-Statistik} (Unterscheidbarkeit der
Teilchen, Mehrfachbesetzbarkeit von Zellen)
\begin{align*}
p(k_{1}, \ldots,k_{n} ) = {\displaystyle\frac{k!}{k_{1}! \ldots k_{n}!} \, \,
\frac{1}{n^{k}}}.
\end{align*}
\item
\emph{Bose-Einstein-Statistik} zur Beschreibung des Verhaltens von Photonen
(keine Unterscheidbarkeit der Teilchen, jedoch Mehrfachbesetzbarkeit von Zellen)
\begin{align*}
p(k_{1}, \ldots, k_{n}) = {n+k-1 \choose k} ^{-1}. 
\end{align*}
\item
\emph{Fermi-Dirac-Statistik} zur Beschreibung des Verhaltens von Elektronen,
Protonen und Neutronen (keine Unterscheidbarkeit der Teilchen, keine
Mehrfachbesetzbarkeit von Zellen )
\begin{align*}
p(k_{1}, \ldots, k_{n})=
\begin{cases}
{n \choose k}^{-1}, &  k_{i} \in \{0,1 \} \\
0, &  \mbox{sonst.}\bsphere
\end{cases}
\end{align*}
\end{bspenum}
\end{bspn}
\begin{bspn}
Die \textit{Binomialverteilung} $b(n,p)$ mit Parametern $n \in \mathbb{N}$, $p
\in (0,1)$ ist ein W-Maß auf ${\cal B}$ mit der Zähldichte
\begin{align*}
k\to
\begin{cases}
{n \choose k} \, p^{k} (1-p)^{n-k}  , & k=0,1,\ldots, n \\
0 ,& k = n+1, n+2, \ldots \, ;
\end{cases}
\end{align*}
durch diese wird für $n$ ``unabhängige'' Versuche mit jeweiliger
Erfolgswahrscheinlichkeit $p$ die Wahrscheinlichkeit von $k$ Erfolgen
angegeben.\bsphere
\end{bspn}
