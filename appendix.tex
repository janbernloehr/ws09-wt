\appendix

\chapter{Begriffe und Sätze der Maß- und Integrationstheorie}

Sei $(\Omega, {\mathcal A}, \mu)$ ein Maßraum.

\begin{propn}[Satz von der monotonen Konvergenz (B.\ Levi)]
Für erweitert reellwertige
messbare Funktionen $f_{n}$ mit $f_{n} \geq0$ und $f_{n}
\uparrow f$ existiert $\lim\limits_{n} \int f_{n}\dmu$ und es gilt
\begin{align*}
\lim\limits_{n} \int f_{n}\dmu = \int \lim\limits_{n}
f_{n}\dmu.\fishhere
\end{align*}
\end{propn}

\begin{propn}[Lemma von Fatou]
Für jede Folge $(f_n)$ von erweitert reellwertigen
messbaren Funktionen mit $f_n \ge 0$ $\mu$-f.ü. gilt
\begin{align*}
\int \liminf f_n \dmu \le \liminf \int f_n \dmu.\fishhere 
\end{align*}
\end{propn}

\begin{propn}[Satz von der dominierten Konvergenz (Lebesgue)]
Für erweitert-reellwertige messbare Funktionen $f_n$, $f$ und
$g$ mit $f_n \to f$ $\mufu$, $|f_n| \le g$ $\mufu$ für
alle $n$ und $\int g \dmu < \infty$ existiert $\lim_{n\to\infty} \int f_n\dmu$
und es gilt
\begin{align*}
\lim_{n\to\infty} \int f_n \dmu = \int f\dmu.\fishhere
\end{align*}
\end{propn}

\begin{defnn}
\begin{defnenum}
\item
Das Maß $\mu$ heißt \emph{$\sigma$-endlich}, wenn es ein
Folge von Mengen $A_n\in\AA$ gibt mit $A_n \uparrow
\Omega$ und $\mu(A_n) < \infty$.
\item Ein Maß $\nu$ auf ${\mathcal A}$ heißt \emph{$\mu$-stetig}, falls
\begin{align*}
\forall A \in \AA :  \mu(A)=0 \Rightarrow \nu(A)=0.\fishhere
\end{align*} 
\end{defnenum}
\end{defnn}

\begin{propn}[Satz von Radon-Nikodym]
Seien $(\Omega,\AA)$ Messraum, $\mu$ und $\nu$ $\sigma$-endliche Maße auf
$\AA$ und $\nu$ sei $\mu$-stetig.  Dann existiert eine Funktion
$f:(\Omega,\AA) \to (\R_+, \BB_+)$ mit
\begin{align*}
\forall A \in \AA : \nu(A)=\int_A f\dmu
\end{align*}
$f$ ist eindeutig bis auf Äquivalenz ``=$\mu$-f.ü.''. \\ $f$ ist die
sog.\ \emph{Radon-Nikodym-Ableitung} von $\nu$ nach $\mu$ und wird häufig
kurz durch $\frac{\dnu}{\dmu}$ angegeben.\fishhere
\end{propn}
\begin{proof}
Zum Beweis siehe Elstrodt.\qedhere
\end{proof}

\begin{bsp}
Sei $Q$ ein W-Maß auf $\BB$, $\lambda$ das LB-Maß auf $\BB$ und $\Q$ sei
$\lambda$-stetig. Dann ist die Radon-Nikodyn-Ableitung
\begin{align*}
\frac{\dQ}{\dlambda}
\end{align*}
die gewöhnliche Dichte von $Q$.\bsphere
\end{bsp}

\begin{propn}[Zusatz]
Seien $\mu,\nu,\ph$ $\sigma$-endliche Maße, $\ph$ sei $\nu$-stetig und $\nu$
sei $\mu$-stetig. Dann gilt
\begin{align*}
\frac{\dph}{\dmu} = \frac{\dph}{\dnu}\cdot \frac{\dnu}{\dmu}.\fishhere
\end{align*}
\end{propn}

%\end{appendix}

% 
% \addcontentsline{toc}{section}{Literatur}
%  
\begin{thebibliography}{99}

\bibitem{bauerII} Bauer, H. Wahrscheinlichkeitstheorie; de Gruyter, Berlin
  (1990), 4.\ Aufl.
\bibitem{bauerI} Bauer, H. Maß- und Integrationstheorie; de Gruyter, Berlin
  (1998), 2.~Aufl.
\bibitem{billingsley} Billingsley, P. Probability and Measure; Wiley, New York
  (1995), 3rd ed.
\bibitem{capinski-kopp} Capi\'{n}ski, M., Kopp, E. Measure Integral and
  Probability; Springer (2004), 2nd ed.
\bibitem{dudley} Dudley, R.M. Real Analysis and Probability; Cambridge
  University Press (2002), 2nd ed.
\bibitem{elstrodt} Elstrodt, J. Maß- und Integrationstheorie; Springer (2009),
  6.~Aufl.
\bibitem{feller} Feller, W. An Introduction to Probability and Its
  Applications I,II; Wiley, New York (1970/71), 3rd ed./2nd ed.
\bibitem{fisz} Fisz, M. Wahrscheinlichkeitsrechnung und mathematische
  Statistik; VEB Deutscher Verlag der Wissenschaften, Berlin (1970),
  5.\ Aufl.
\bibitem{gaenssler-stute} Gänssler, P., Stute, W. Wahrscheinlichkeitstheorie;
  Springer, Berlin (1977).
\bibitem{henze} Henze, N. Stochastik für Einsteiger: Eine Einführung in die
  faszinierende Welt des Zufalls; Vieweg+Teubner (2008), 7.~Aufl.
\bibitem{hesse} Hesse, C. Wahrscheinlichkeitstheorie --- Eine Einführung mit
  Beispielen und Anwendungen. Vieweg+Teubner (2009), 2.~Aufl.
\bibitem{hinderer} Hinderer, K. Grundbegriffe der Wahrscheinlichkeitstheorie;
  Springer, Berlin (1985), 3.~korr.\ Nachdruck.\\[-8mm]
\bibitem{jacod-protter} Jacod, J., Protter, P. Probability
  Essentials. Springer, Berlin (2004), 2nd ed.
\bibitem{kallenberg} Kallenberg, O.  Foundations of Modern
  Probability. Springer, New York (2001), 2nd ed.
\bibitem{klenke} Klenke, A. Wahrscheinlichkeitstheorie; Springer, Heidelberg
  (2006)
\bibitem{krengel} Krengel, U. Einführung in die Wahrscheinlichkeitstheorie und
  Statistik; Vieweg, Braunschweig (2003), 7.\ Aufl.
\bibitem{loeve} Loève, M. Probability Theory I,II; Springer, Berlin (1977/78),
  4th ed.
\bibitem{meintrup-schaeffler} Meintrup D., Schäffler S. Stochastik --- Theorie
  und Anwendungen; Springer (2004).
\bibitem{williams-1} Williams, D. Probability with Martingales; Cambridge
  University Press (1991).
\bibitem{wengenroth} Wengenroth, J. Wahrscheinlichkeitstheorie. De Gruyter
  (2008).
\bibitem{williams} Williams, D. Weighing the Odds --- A Course in Probability
  and Statistics; Cambridge University Press (2001).

\end{thebibliography}

%\printindex