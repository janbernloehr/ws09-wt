\chapter{Gegenstand der Vorlesung}

\begin{figure}[H]
\centering
\begin{pspicture}(0,-1.55)(9.38,1.55)

\rput(0.57,1.175){\color{purple}Realität}

\rput(7.15,1.175){\color{purple}Mathematisches Model}

\rput(1.1,-1.045){\color{purple}Aussagen über}
\rput(1.1,-1.365){\color{purple}die Realität}

\rput(7.17,-1.045){\color{purple}Mathematische Eigenschaften}

\psline[linecolor=darkblue]{->}(1.6,1.13)(5.38,1.13)
\psline[linecolor=darkblue]{->}(6.62,0.65)(6.62,-0.79)
\psline[linecolor=darkblue]{->}(4.76,-1.05)(2.42,-1.07)
\psline[linecolor=darkblue]{->}(0.42,-0.73)(0.42,0.25)

\rput[l](0.15,0.875){\color{gdarkgray}\scriptsize Erfahrung}

\rput[l](0.15,0.635){\color{gdarkgray}\scriptsize Daten}

\rput[l](0.15,0.375){\color{gdarkgray}\scriptsize Beobachtung}

\rput(7.25,0.855){\color{gdarkgray}\scriptsize Axiomatische Theorie}

\rput(7.17,-1.405){\color{gdarkgray}\scriptsize Theoreme}

\rput(3.64,1.355){\color{darkblue}\small Modellierung}

\rput(3.63,0.895){\color{darkblue}\small Abstraktion}

\rput(3.68,-1.305){\color{darkblue}\small Interpretation}

\rput(7.53,0.135){\color{darkblue}\small Analyse}

\rput(7.56,-0.345){\color{darkblue}\small Deduktion}

\rput(1.38,-0.285){\color{darkblue}\small Anwendung}
\end{pspicture} 
\caption{Schema Mathematische Modellierung.}
\end{figure}


Gegenstand der Vorlesung ist es nicht, Aussagen darüber zu treffen, was
``Zufall'' oder ``Wahrscheinlichkeit'' konkret bedeutet. Wir haben die Absicht
zu klären, welche Axiome für eine Wahrscheinlichkeit gelten und welche
Eigenschaften sie erfüllen muss - sofern sie existiert.
