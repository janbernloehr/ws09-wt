% =============================================================================
% Titel:		Wahrscheinlichkeitstheorie - Mitschrieb
% Erstellt:	WS 09
% Dozent:	PD Dr. J. Dippon
% Autor:	Jan-Cornelius Molnar
% =============================================================================
%\documentclass[paper=a5,fleqn,DIV=calc,openright,twoside=false,notitlepage,version=last]{scrartcl}
\documentclass[%
	paper=a5,%
	fleqn,%
	DIV=18,%
	%headings=openleft,
	titlepage=false,%
	bibliography=totoc,
	numbers=noendperiod,
	twoside=true]%
	{scrbook}

% =============================================================================
% 					Benötigte Pakete
% =============================================================================
\usepackage{janmcommon}
\usepackage{janmscript}
\usepackage{float}
\restylefloat{figure}
\usepackage{fancyhdr}
%\usepackage[num]{isodate}

\usepackage{marginnote}
\usepackage{makeidx}

\usepackage[mathscr]{euscript}		% Euler Skript
\usepackage{scalefnt}

\makeindex


% =============================================================================
% 					Ana-Theorem-Style
% =============================================================================
% Theorem Umgebungen *MIT* Numerierung
\theoremstyle{graymarginwithblueheader}
\theorembodyfont{\itshape}
\theoremseparator{}
\theoremsymbol{}

\newtheorem{prop}{Satz}[chapter]
\newtheorem{lem}{Lemma}[chapter]
\newtheorem{defn}{Definition}[chapter]
\newtheorem{cor}{Korollar}[chapter]

\theoremstyle{yellowheader}
\theorembodyfont{\normalfont}
\theoremseparator{}
\theoremsymbol{}

\newtheorem{bsp}{Bsp}[chapter]

\theoremstyle{graymarginwithitblackheader}
\theorembodyfont{\normalfont}
\theoremseparator{}
\theoremsymbol{}

\newtheorem{bem}{Bemerkung.}[chapter]

% Theorem Umgebungen *OHNE* Numerierung
\theoremstyle{graymarginwithblueheadern}
\theorembodyfont{\itshape}
\theoremseparator{}
\theoremsymbol{}

\newtheorem{propn}{Satz}
\newtheorem{lemn}{Lemma}
\newtheorem{corn}{Korollar}
\newtheorem{defnn}{Definition}

\theoremstyle{graymarginwithyellowheadern}
\theorembodyfont{\normalfont}
\theoremseparator{}
\theoremsymbol{}

\newtheorem{bspn}{Bsp}

\theoremstyle{graymarginwithitblackheadern}
\theorembodyfont{\normalfont}
\theoremseparator{}
\theoremsymbol{}

\newtheorem{bemn}{Bemerkung.}

% =============================================================================
% 					Überschriften-Style
% =============================================================================
% \renewcommand\thesection{\arabic{section}}
% \renewcommand\thesubsection{\arabic{section}-\Alph{subsection}}
% \renewcommand\thesubsubsection{\small\ensuremath{\blacksquare}\normalsize}
\renewcommand\thesection{\arabic{chapter}-\Alph{section}}
\renewcommand\thesubsection{\small\ensuremath{\blacksquare}\normalsize}
\renewcommand\thebsp{\arabic{bsp}}

\setkomafont{chapter}{\normalfont\bfseries\Huge\color{darkblue}}
\setkomafont{section}{\normalfont\bfseries\Large\color{darkblue}}
\setkomafont{subsection}{\normalfont\bfseries\color{darkblue}}

%\setlength{\oddsidemargin}{-27pt}
%\setlength{\evensidemargin}{-27pt}
% \setlength{\marginparwidth}{1pt}

% \titleformat{\section}%
%   {\normalfont\bfseries\Huge\color{darkblue}}%
%   {\thesection}%
%   {0.6em}%
%   {}%
%   
% \titleformat{\subsection}%
%   {\normalfont\bfseries\Large\color{darkblue}}%
%   {\thesubsection}%
%   {0.6em}%
%   {}%
% 
% \titleformat{\subsubsection}%
%   {\normalfont\bfseries\color{darkblue}}%
%   {\thesubsubsection}%
%   {0.2em}%
%   {}%

% Standardauflistung
\renewcommand{\labelenumi}{{\normalfont(\alph{enumi})}}

% PsTricks Defaults
\psset{linecolor=gdarkgray}
\psset{tickcolor=gdarkgray}
\psset{fillcolor=glightgray}

% Startkapitel 0
\setcounter{chapter}{-1}

% Command definition
\renewcommand{\AA}{\mathcal{A}}

\renewcommand{\atop}[2]{\genfrac{}{}{0pt}{}{#1}{#2}}

\newcommand{\pcal}[1]{
{\text{\scalefont{.82}\ensuremath{\mathscr{#1}}}}
}
\newcommand{\cto}{\overset{\mathrm{c}}{\to}}
\newcommand{\Pto}{\overset{\mathbf{P}}{\to}}
\renewcommand{\Dto}{\overset{\mathcal{D}}{\to}}

\newcommand{\fs}{\ \text{f.s.}}
\newcommand{\Pfs}{\ P\text{-f.s.}}

\renewcommand{\Id}{\mathbf{1}}

\newcommand{\Cov}{\mathbf{Cov}}

\newcommand{\defl}{\coloneq}
\newcommand{\defr}{\eqcolon}

\renewcommand{\le}{\leqslant}
\renewcommand{\ge}{\geqslant}

\renewcommand{\iff}{\Leftrightarrow}
\renewcommand{\bar}[1]{\overline{#1}}

\renewcommand{\CC}{\pcal{C}}

\newcommand{\rvec}[1]{\underline{\vec{#1}}}
\newcommand{\T}{\mathbb{T}}
\renewcommand{\sp}{\mathrm{sp}}
\newcommand{\ad}{\mathrm{ad}}
\newcommand{\codim}{\mathrm{codim}}
\renewcommand{\dph}{\mathrm{d}\ph}

\newcommand{\Poi}{\mathrm{Poi}}
\newcommand{\Exp}{\mathrm{Exp}}

\newcommand{\longto}{\longrightarrow}
\newcommand{\lto}[1]{\overset{L^#1}{\longrightarrow}}


\newcommand{\ii}{\mathrm{i}}
\newcommand{\e}{\mathrm{e}}

\newcommand{\sidenote}[1]{%
\noindent\leavevmode\llap{#1\enspace}%
}

\newenvironment{bemenum}%
	{\begin{enumerate}[label=\textsc{\alph{*}}.,leftmargin=17pt]}{\end{enumerate}}
\newenvironment{defnenum}%
	{\begin{enumerate}[label=\arabic{*}.)]}{\end{enumerate}}
\newenvironment{defnpropenum}%
	{\begin{enumerate}[label={\rm (\alph{*})}]}{\end{enumerate}}
\newenvironment{propenum}%
	{\begin{enumerate}[label=\alph{*})]}{\end{enumerate}}
\newenvironment{equivenum}%
	{\begin{enumerate}[label=(\roman{*})]}{\end{enumerate}}
\newenvironment{bspenum}%
	{\begin{enumerate}[label=\alph{*}.),leftmargin=17pt]}{\end{enumerate}}
\newenvironment{proofenumarm}%
	{\begin{enumerate}[label=\arabic{*}.),leftmargin=17pt]}{\end{enumerate}}
\newenvironment{proofenumarabic}%
	{\begin{enumerate}[label=\arabic{*}.),leftmargin=17pt]}{\end{enumerate}}
\newenvironment{proofenumarabicbr}%
	{\begin{enumerate}[label=(\arabic{*}),leftmargin=17pt]}{\end{enumerate}}
\newenvironment{proofenumroman}%
	{\begin{enumerate}[label=(\roman{*}),leftmargin=17pt]}{\end{enumerate}}
\newenvironment{proofenum}%
	{\begin{enumerate}[label=\alph{*}),leftmargin=17pt]}{\end{enumerate}}

% Select page style
\pagestyle{fancyplain}

% Reset header & footer
\fancyhf{}

% Reset chapter & sectionmark
\renewcommand{\chaptermark}[1]{\markboth{\textsc{#1}}{}}
\renewcommand{\sectionmark}[1]{\markright{\textsc{#1}}{}}

% Clear headrule
\renewcommand{\headrule}{}

% \setlength{\leftmargini}{0pt}

% =============================================================================
% 					Document-Body
% =============================================================================
\begin{document}

%\frontmatter

% Titel
% 
% \begin{titlepage}
% \begin{centering}
% {\huge\bf Wahrscheinlichkeitstheorie - Mitschrieb}
% 
% bei PD Dr. J. Dippon
% 
% Jan-Cornelius Molnar, Version: \today\ \thistime
% 
% \end{centering}
% \end{titlepage}

\begin{titlepage}
\vspace*{2mm}
\noindent\bf\color{gdarkgray}
Jan-Cornelius Molnar

\begin{center}
\vspace*{10mm}
{\noindent\huge\bf\color{darkblue} Wahrscheinlichkeitstheorie}

\vspace*{10mm}
Aufbauend auf einer Vorlesung von PD Dr. J. Dippon

\vspace*{4mm}

Stuttgart, Wintersemester 2009 / 2010
\end{center}

\vspace*{\fill}

\begin{flushright}
\small
Version: \today\ \thistime
\vspace*{5mm}

Für Hinweise auf Druckfehler und Kommentare jeder Art bin ich dankbar.
Viel Spaß!\footnote{\color{gdarkgray}
Jan-Cornelius Molnar
\href{mailto:jan.molnar@studentpartners.de}{jan.molnar@studentpartners.de}}
\end{flushright}
\end{titlepage}

% Inhaltsverzeichnis
\tableofcontents

% Inhalt
%\mainmatter

\fancyhead[RO]{\footnotesize\color{gdarkgray}%
	\marginnote{\Big|\;\textbf{\thesection}}\rightmark}
\fancyhead[LE]{\footnotesize\color{gdarkgray}%
	\marginnote{\;\textbf{\thechapter}\Big|}\leftmark}

% Zweiseitig
\fancyfoot[LE]{\footnotesize\color{gdarkgray}%
 	\thepage}%
 \fancyfoot[RO]{\footnotesize\color{gdarkgray}%
 	\thepage}%
 \fancyfoot[RE,LO]{\tiny\color{gdarkgray}\today\; \thistime}


\chapter{Gegenstand der Vorlesung}

\begin{figure}[H]
\centering
\begin{pspicture}(0,-1.55)(9.38,1.55)

\rput(0.57,1.175){\color{purple}Realität}

\rput(7.15,1.175){\color{purple}Mathematisches Model}

\rput(1.1,-1.045){\color{purple}Aussagen über}
\rput(1.1,-1.365){\color{purple}die Realität}

\rput(7.17,-1.045){\color{purple}Mathematische Eigenschaften}

\psline[linecolor=darkblue]{->}(1.6,1.13)(5.38,1.13)
\psline[linecolor=darkblue]{->}(6.62,0.65)(6.62,-0.79)
\psline[linecolor=darkblue]{->}(4.76,-1.05)(2.42,-1.07)
\psline[linecolor=darkblue]{->}(0.42,-0.73)(0.42,0.25)

\rput[l](0.15,0.875){\color{gdarkgray}\scriptsize Erfahrung}

\rput[l](0.15,0.635){\color{gdarkgray}\scriptsize Daten}

\rput[l](0.15,0.375){\color{gdarkgray}\scriptsize Beobachtung}

\rput(7.25,0.855){\color{gdarkgray}\scriptsize Axiomatische Theorie}

\rput(7.17,-1.405){\color{gdarkgray}\scriptsize Theoreme}

\rput(3.64,1.355){\color{darkblue}\small Modellierung}

\rput(3.63,0.895){\color{darkblue}\small Abstraktion}

\rput(3.68,-1.305){\color{darkblue}\small Interpretation}

\rput(7.53,0.135){\color{darkblue}\small Analyse}

\rput(7.56,-0.345){\color{darkblue}\small Deduktion}

\rput(1.38,-0.285){\color{darkblue}\small Anwendung}
\end{pspicture} 
\caption{Schema Mathematische Modellierung.}
\end{figure}


Gegenstand der Vorlesung ist es nicht, Aussagen darüber zu treffen, was
``Zufall'' oder ``Wahrscheinlichkeit'' konkret bedeutet. Wir haben die Absicht
zu klären, welche Axiome für eine Wahrscheinlichkeit gelten und welche
Eigenschaften sie erfüllen muss - sofern sie existiert.


\cleardoublepage
\chapter{Wahrscheinlichkeitsräume}
\label{chap:1}

% =============================================================
%                                   Algebren, Inhalte und Maße
% =============================================================
\section{Algebren, Inhalte und Maße}
\label{chap:1.a}

Ziel dieses Abschnittes ist es, Abbildungen $\mu$ zu definieren, die einer
Teilmenge eines Raumes $\Omega$ ein ``Maß'' zuordnen,
\begin{align*}
\mu: \PP(\Omega)\to\R.
\end{align*}
Gehen wir von $\Omega=\R$ aus und betrachten zunächst nur Intervalle als
Teilmengen, so ist es nur natürlich diesen auch ihre Länge als ``Maß''
zuzuordnen,
\begin{align*}
\mu((a,b]) = b-a.
\end{align*}
G. Vitali\footnote{Giuseppe Vitali (* 26. August 1875 in Ravenna; † 29. Februar
1932 in Bologna) war ein italienischer Mathematiker.} hat jedoch bereits 1905
gezeigt, dass es nicht möglich ist, ein solches ``Maß'' dann für alle
Teilmengen von $\R$, d.h. auf der Potenzmenge $\PP(\R)$ zu definieren, es ist
also notwendig sich lediglich auf eine Teilmenge von $\PP(\R)$ zurückzuziehen,
d.h. man kann auch nur bestimmen Teilmengen von $\R$ mit einem solchen Maß
``vermessen''.
\clearpage

In der Wahrscheinlichkeitstheorie werden üblicherweise als Definitionsbereich
von Maßen sogenannte $\sigma$-Algebren betrachtet. 

\begin{defn}
\label{defn:1.1}
Sei $\Omega$ eine nichtleere Menge. Ein System $\AA$ von Teilmengen von
$\Omega$ heißt \emph{Algebra}, wenn
\begin{defnenum}
  \item $\varnothing\in\AA$,
  \item $A\in \AA \Rightarrow \Omega\setminus A\in\AA$,
  \item $A,B\in\AA \Rightarrow A\cup B\in\AA$\\
  \qquad oder äquivalent\\
  $A_1,\ldots,A_m\in \AA \Rightarrow \bigcup_{i=1}^m A_i \in\AA$.
\end{defnenum}
das System $\AA$ heißt \emph{$\sigma$-Algebra}, falls zusätzlich gilt,
\begin{enumerate}
  \item[3.*)] $A_n\in\AA$ für $n\in\N$ $\Rightarrow
  \bigcup_{n=1}^\infty A_n\in\AA$.\fishhere
\end{enumerate}
\end{defn}

Diese Definition einer Algebra ist nicht mit der aus der Vorlesung ``Algebra''
zu verwechseln. Dennoch existieren Parallelen. Vereinigung und Schnittbildung
könnte man als Addition und Multiplikation betrachten, das Komplement als
Inverses und die leere Menge als Identität.

\begin{bem}
Ist $\AA$ $\sigma$-Algebra, so ist $\AA$ auch auch eine Algebra.\maphere
\end{bem}

Wie wir noch sehen werden, besitzen Algebren  zahlreiche angenehme
Eigenschaften. Durch die Forderung dass Komplemente und Vereinigungen
enthalten sind, ergibt sich beispielsweise automatisch, dass auch Schnitte und
Differenzen enthalten sind.

\begin{lem}
\label{lem:1.1}
\begin{propenum}
  \item Sei $\AA$ eine Algebra, so gilt
\begin{align*}
&A_1,\ldots,A_m\in\AA \Rightarrow \bigcap_{n=1}^m A_n\in\AA.\\
&A,B\in\AA \Rightarrow A\setminus B \in\AA.
\end{align*}
\item Sei $\AA$ eine $\sigma$-Algebra, so gilt außerdem
\begin{align*}
A_n\in\AA \text{ für }n\in\N \Rightarrow \bigcap_{n=1}^\infty
A_n\in\AA.\fishhere
\end{align*}
\end{propenum}
\end{lem}
Eine Algebra (bzw. $\sigma$-Algebra) enthält somit stets $\varnothing$
und $\Omega$ und ist abgeschlossen gegenüber endlich (bzw. abzählbar) vielen
üblichen Mengenoperationen.
\begin{proof}
\begin{proofenum}
  \item Definition \ref{defn:1.1} besagt, $A_1,\ldots,A_m \in \AA\Rightarrow
  A_1^c,\ldots,A_m^c\in \AA$. Mit den Regeln von De-Morgan folgt,
\begin{align*}
\left(\bigcap_{n=1}^m A_n \right)^c = \bigcup_{n=1}^m A_n^c \in \AA,
\end{align*}
d.h. $\bigcap_{n=1}^m A_n\in\AA$.
\item Seien $A,B\in \AA$. Man sieht leicht, dass $A\setminus B = A\cap B^c$ und
daher ist auch $A\setminus B\in\AA$ als Schnitt von $A$ und $B^c$.\qedhere
\end{proofenum}
\end{proof}

\begin{defn}
\label{defn:1.2}
Sei $\Omega $ eine nichtleere Menge und $\AA$ eine $\sigma$-Algebra in $\Omega$.
Das Paar ($\Omega,\AA$) heißt \emph{Messraum (mesurable space)}; die Elemente
$\in \AA$ heißen $\AA$-\emph{messbare} Mengen.\\
Entsprechendes gilt für $\sigma$-Algebra.\fishhere
\end{defn}

Man startet bei der Definition von Abbildungen auf Messräumen oft nur
auf Teilmengen - wie z.B. den Intervallen in $\R$ - und dehnt anschließend den
Definitionsbereich aus. Für diese Vorgehensweise sind folgende Ergebnisse
zentral.

\begin{lem}
\label{lem:1.2}
\begin{propenum}
  \item Seien $\AA_\alpha$ Algebren in $\Omega$,  $\alpha\in \II$ und $\II$
  beliebige Indexmenge, so ist $\bigcap_{\alpha\in\II} \AA_\alpha$ eine Algebra
  in $\Omega$.
  \item Sei $\CC\subseteq \PP(\Omega)$ ein Mengensystem in $\Omega$,
  so es existiert eine kleinste $\CC$ enthaltende Algebra.\fishhere
\end{propenum}
\end{lem}
\begin{proof}
\begin{proofenum}
\item
Wir führen den Beweis für $\sigma$-Algebren; der für Algebren wird analog
geführt.

  $\AA_\alpha$ sei für jedes $\alpha\in\II$ eine $\sigma$-Algebra in
  $\Omega$. Also ist auch $\varnothing\in \bigcap_{\alpha\in \II} \AA_\alpha$,
  da $\varnothing\in \AA_\alpha$ für alle $\alpha\in\II$. Weiterhin gilt
\begin{align*}
A\in \bigcap_{\alpha\in\II} \AA_\alpha &\Leftrightarrow \forall \alpha\in\II :
A\in \AA_\alpha \Rightarrow \forall \alpha\in\II : A^c\in\AA_\alpha\\
&\Leftrightarrow A^c\in \bigcap_{\alpha\in\II} \AA_\alpha,
\end{align*}
sowie
\begin{align*}
A_1,A_2,\ldots \in \bigcap_{\alpha\in\II} \AA_\alpha &
\Leftrightarrow \forall \alpha\in \II : A_1,A_2,\ldots\in \AA_\alpha \\
&\Rightarrow \forall \alpha\in\II : \bigcup_{n=1}^\infty A_n \in \AA_\alpha 
\Leftrightarrow \bigcup_{n=1}^\infty A_n \in \bigcap_{\alpha\in\II} \AA_\alpha.
\end{align*}
\item Betrachte alle Obermengen von $\CC$, die Algebren sind,
  und bilde den Schnitt,
\begin{align*}
\bigcap_{\atop{\AA_\alpha\text{ Algebra}}{\CC\subseteq\AA_\alpha}} \AA_\alpha
\end{align*}
Aus a) folgt, dass der Schnitt eine Algebra ist. Der Schnitt ist dann
auch \textit{per definitionem} die kleinste $\sigma$-Algebra, die $\CC$
enthält.\qedhere
\end{proofenum}
\end{proof}

\begin{defn}
\label{defn:1.3}
Sei $\CC$ ein Mengensystem in $\Omega$. Die kleinste der $\CC$ enthaltenden
Algebren heißt die \emph{von $\CC$ erzeugte Algebra}; $\CC$ heißt ein
\emph{Erzeugersystem} dieser Algebra.\\
Entsprechendes gilt für $\sigma$-Algebren.\\
Wir bezeichnen die von $\CC$ erzeugte Algebra mit $\FF(\CC)$.\fishhere
\end{defn}

Das Konzept des Erzeugersystems ermöglicht es uns, Eigenschaften nur auf dem
Erzeugersystem nachzuweisen, was oft einfacher ist, und diese Eigenschaften
auf die erzeugte Algebra zu übertragen.

Im $\R^n$ bilden die offenen Teilmengen ein wichtiges Mengensystem, die sie
die Topologie erzeugen und somit Metrik, Norm, Konvergenz, Differenziation,
etc. charakterisieren. Folgende Definition ist daher für alles Weitere zentral.

\begin{defn}
\label{defn:1.4}
Sei $\OO_n$ das System der offenen Mengen des Euklidischen Raumes $\R^n$.
Setze $\BB_n \defl \FF(\OO_n)$. $\BB_n$ wird als $\sigma$-Algebra der
\emph{Borelschen Mengen in $\R^n$} bezeichnet. $\BB\defl \BB_1$.\fishhere
\end{defn}

$\BB_n$ enthält alle abgeschlossenen, alle kompakten und alle höchstens
abzählbaren Mengen in $\R^n$. Es gilt dennoch tatsächlich $\BB_n\subsetneq
\PP(\R^n)$, der Beweis benötigt jedoch die Annahme des Auswahlaxioms.

Die Borelschen Mengen bilden die grundlegenden Mengen, mit denen wir uns
beschäftigen. Wenn wir auf dem $\R^n$ arbeiten, enthält also $\BB_n$ alle ``für
uns interessanten'' Mengen.

%TODO: Bild der halboffenen Mengen in \R und \R^2.

Um das Konzept des Erzeugersystem ausnutzen können, verwenden wir
das System $J_n$ der halboffenen Intervalle (bzw. Rechtecke)
\begin{align*}
(a,b]\defl\setdef{(x_1,\ldots,x_n)\in\R^n}{a_i < x_i\le b_i,\; i=1,\ldots,n}
\end{align*}
in $\R^n$, wobei $a=(a_1,\ldots,a_n)$, $b=(b_1,\ldots,b_n)\in\R^n$ mit $a_i <
b_i$, oder auch das System der offenen Intervalle, der abgeschlossenen
Intervalle, der Intervalle $(-\infty,a]$ oder der Intervalle $(-\infty,a)$.

\begin{prop}
\label{prop:1.1}
Das System $J_n$ ist ein Erzeugersystem von $\BB_n$.\fishhere
\end{prop}
\begin{proof}
\textit{per definitionem} gilt
$(a,b] \defl \setdef{(x_1,\ldots,x_n)\in\R^n}{a_i < x_i\le b_i,\;
i=1,\ldots,n}$. Nun lässt sich jedes offene Intervall als abzählbare
Vereinigung halboffener Intervalle darstellen,
\begin{align*}
(a,b) &= \bigcup_{k\in\N} \left(a,b-\frac{1}{k}\right] \\ &=
\bigcup_{k\in\N} \setdef{(x_1,\ldots,x_n)\in\R^n}{a_i< x_i \le  b_i-\frac{1}{k}}
\in \BB_n.
\end{align*}
Ferner lässt sich jede offene Menge in $\R^n$ als \textit{abzählbare}
Vereinigung von offenen Intervallen mit rationalen Randpunkten
darstellen.\qedhere
\end{proof}

Wir haben nun die notwendigen Vorbereitungen getroffen um ein Maß zu
definieren.

\begin{defn}
\label{defn:1.5}
Sei $\CC$ ein Mengensystem in $\Omega $ mit
$\varnothing \in \CC$. Eine Mengenfunktion 
\begin{align*}
\mu : \CC \to
\overline{\R} \defl \R \cup \{+ \infty, - \infty\}
\end{align*}
heißt (mit den Konventionen
$a+\infty = \infty$ $(a \in \R)$, $\infty + \infty = \infty $ usw.)
\begin{defnenum}
\item ein \emph{Inhalt (content)} auf $\CC$, wenn
\begin{defnpropenum}
\item $\mu$ ist nulltreu, d.h. $\mu(\varnothing) = 0$,
\item $\mu$ ist positiv, d.h. $\forall A \in \CC : \mu
  (A) \geq 0 $,
\item $\mu $ ist additiv, d.h. für paarweise disjunkte $ A_{n} \in
{\CC}$ mit $n=1,2,\ldots,m$ und $\sum^{m}_{n=1} A_{n}\in \CC$
 gilt\footnote{
$A+B$ bzw. $\sum_{n=1}^m A_n$ steht für die disjunkte
Vereinigung von $A$ und $B$ bzw. der $A_n$. Wir fordern bei der Verwendung von
$+$ und $\sum$ implizit, dass $A$ und $B$ bzw. die $A_n$ disjunkt sind.
}
\begin{align*}
\mu\left(\sum^{m}_{n=1} A_{n}\right) = \sum^{m}_{n=1} \mu(A_{n}).
\end{align*}
\end{defnpropenum}
\item
ein \emph{Maß (measure)} auf $\CC$, wenn
\begin{defnpropenum}
\item $\mu$ ist nulltreu, d.h. $\mu (\varnothing)= 0$,
\item $\mu$ ist positiv, d.h. $\forall A \in \CC : \mu
  (A) \geq 0 $,
\item $\mu $ ist $\sigma$-additiv, d.h. für paarweise disjunkte $ A_{n}
\in \CC$ mit $n\in\N$ und $\sum^{\infty}_{n=1} A_{n}\in \CC$
  gilt
\begin{align*}
\mu \left(\sum^{\infty}_{n=1} A_{n}\right) = \sum^{\infty}_{n=1}\mu(A_{n}).
\end{align*}
\end{defnpropenum}
\end{defnenum}
Ist $\AA$ eine $\sigma$-Algebra in $\Omega$, $\mu $
ein Maß auf $\AA$, so heißt $(\Omega, \AA, \mu )$ ein \emph{Maßraum
(measure space)}.\fishhere
\end{defn}
Offensichtlich ist jedes Maß ein Inhalt.

``Natürliche'' Definitionsbereiche für Inhalt und Maß sind Algebren
bzw. $\sigma$-Algebren. In diesem Fall können wir auch auf die Voraussetzung
$\sum^{m}_{n=1}A_{n} \in {\CC}$ bzw. $\sum^{\infty}_{n=1} A_{n} \in {\CC}$)
verzichten. Häufig werden wir nur diese Definitionsbereiche verwenden.

In der Wahrscheinlichkeitstheorie arbeiten wir häufig auf abzählbaren
Messräumen wie z.B. $\N$ dann lassen sich alle Teilmengen messen, indem wir
ihnen als Maß die Anzahl ihrer Elemente zuordnen.

\begin{defn}
\label{defn:1.6}
Sei $(\Omega, \AA)$ ein Messraum und $Z =\setd{z_{1}, z_{2},
\ldots}$ eine höchstens abzählbare Teilmenge von $\Omega $, so heißt
\begin{align*}
\mu: \AA\to \overline{\R}, \text{ mit } \mu (A)\defl \text{Anzahl der Elemente
in } A\cap Z,\quad A \in \AA,
\end{align*}
 ein \emph{abzählendes Maß (counting measure)}.\fishhere
\end{defn}

\begin{bem}
\label{bem:1.2} Sei $\AA$ Algebra in $\Omega$, $\mu $ endlicher Inhalt auf
$\AA$, ferner $A,B \in \AA$ und $ A \subset B$, so gilt
\begin{align*}
\mu(B\setminus A) = \mu (B) - \mu(A).
\end{align*}
Diese Eigenschaft nennt sich \emph{Subtraktivität}.\maphere
\end{bem}
\begin{proof}
Da $B=A+B\setminus A$, gilt $\mu(B) = \mu(A) + \mu(B\setminus A)$.\qedhere
\end{proof}

\clearpage
% =============================================================
%                            Wahrscheinlichkeitsräume und -maße
% =============================================================
\section{Wahrscheinlichkeiträume und -maße}
\label{chap:1.b}

\begin{defn}
\label{defn:1.7}
Sei $\AA$ eine $\sigma$-Algebra in $\Omega $.  Eine
Abbildung $P: \AA \to \R$ heißt ein \emph{Wahrscheinlichkeitsmaß
(W-Maß)} auf $\AA$, wenn
\begin{defnpropenum}
\item $P$ nulltreu, d.h. $P(\varnothing) = 0$,
\item $P$ positiv, d.h.\ $P(A) \geq 0$,
\item $P$ $\sigma$-additiv, d.h.\ für alle paarweise disjunkte $A_{n} \in
  \AA$\; $(n=1,2, \ldots )$ gilt
\begin{align*}
P\left(\sum^{\infty}_{n=1} A_{n}\right) = \sum^{\infty}_{n=1} P(A_{n}),
\end{align*}
\item $P$ normiert, d.h.\ $P(\Omega)= 1$.
\end{defnpropenum}
$(\Omega, \AA,P)$ heißt dann ein \emph{Wahrscheinlichkeitsraum (W-Raum)
(probability space)}.

Ist $(\Omega, \AA,P)$ ein W-Raum, so bezeichnet man die Grundmenge
$\Omega$ auch als \emph{Merkmalraum, Stichprobenraum (sample space)}, die Mengen
$A\in \AA$ als \emph{Ereignisse}, $P(A)$ als \emph{Wahrscheinlichkeit von $A$}
und die Elemente $\omega \in \Omega$ bzw.\ die 1-Punkt-Mengen $\{\omega \}
\subset \Omega $ (nicht notwendig $\in \AA$) als \emph{Elementarereignisse (auch
Realisierungen)}.

Ein W-Raum ist ein normierter Maßraum.\fishhere
\end{defn}

Ein Wahrscheinlichkeitsmaß, ist also ein normiertes Maß auf $(\Omega,\AA)$ mit
Wertebereich $\R$ anstatt $\RA$.

Bei der Definition eines Wahrscheinlichkeitsmaßes, ergeben sich Nulltreue,
Positivität und Normiertheit ganz intuitiv. In der Vergangenheit wurde lange
darüber diskutiert, ob es notwendig ist, die $\sigma$-Additivität für Ereignisse zu fordern. Es
hat sich jedoch herausgestellt, dass dies durchaus sinnvoll und notwendig ist.

Damit haben wir die Axiome der Wahrscheinlichkeitstheorie formuliert und alles
Folgende wird auf diesen Axiomen aufbauen. Nun stellt sich natürlich die
Frage, ob diese Definition einer ``Wahrscheinlichkeit'' mit unserer
Alltagserfahrung übereinstimmt. Im Laufe der Vorlesung werden wir feststellen,
dass dieses ``Modell'' in vielen Fällen eine sehr gute Approximation der
Wirklichkeit darstellt.

\textit{Zur Motivation des Wahrscheinlichkeitsraumes.} Wir suchen nach einem
alternativen Weg, eine Wahrscheinlichkeit einzuführen. Sei dazu $h_n(A)$ die
absolute Häufigkeit des Eintretens eines Ereignisses $A$ (z.B. Würfeln einer
``6'' bei $n$ Würfen) und $H_n(A) = \frac{h_n(A)}{n}$ die relative Häufigkeit
des Auftretens von Ereignis $A$.\\ Offensichtlich gilt für jedes $A$ und jedes
$n$:
\begin{align*}
0\le H_n(A) \le 1,\quad  H_n(\Omega) = 1,\quad  H_n(\varnothing) = 0.
\end{align*}
Für zwei disjunkte Ereignisse $A_1,A_2$ gilt weiterhin
\begin{align*}
H_n(A_1+A_2) = H_n(A_1)+H_n(A_2).
\end{align*}
$H_n$ verfügt also über alle Eigenschafen eines Wahrscheinlichkeistmaßes.

Führt man das (Würfel)-Experiment hinreichend oft durch, kann man das
``\emph{Empirische Gesetz der großen Zahlen}'' vermuten:
\begin{align*}
&\lim\limits_{n\to\infty} H_n(A)\quad\text{existiert}.  
\end{align*}
Im Fall der Existenz kann man dem Ereignis $A$ den obigen Grenzwert als
Wahrscheinlichkeit zuordnen (R.v. Mises 1919\footnote{
Richard Edler von Mises (* 19. April 1883 in Lemberg, Österreich-Ungarn, heute
Lwiw, Ukraine; † 14. Juli 1953 in Boston, Massachusetts) war ein
österreichischer Mathematiker.
}),
\begin{align*}
P(A) = \lim\limits_{n\to\infty} H_n(A).
\end{align*}
Dieser Grenzwert muss aber nicht für alle
möglichen Folgen von Versuchsergebnissen existieren - es lässt sich nicht
ausschließen, dass man in einer Versuchsfolge ausschließlich 6er würfelt.

Die endgültige, axiomatische Formulierung der Wahrschienlichkeitstheorie, wie
wir sie eben eingeführt haben, hat 1933 mit dem Werk ``Grundbegriffe der
Wahrscheinlichkeitsrechnung'' von Kolmogorov\footnote{
Andrei Nikolajewitsch Kolmogorow (* 25. April 1903
in Tambow; † 20. Oktober 1987 in Moskau) war einer der bedeutendsten
Mathematiker des 20. Jahrhunderts. 
Kolmogorow leistete wesentliche Beiträge auf den Gebieten der
Wahrscheinlichkeitstheorie und der Topologie, er gilt als der Gründer der
Algorithmischen Komplexitätstheorie. Seine bekannteste mathematische Leistung
war die Axiomatisierung der Wahrscheinlichkeitstheorie.}
seinen Abschluss gefunden. Der axiomatische Ansatz hat sich bisher als der erfolgreichste erwiesen. Wir zahlen aber einen Preis, denn dieser Ansatz erklärt nicht, was Wahrscheinlichkeit eigentlich
bedeutet.

\subsection{Binomialverteilung}

Oft sind nur endlich viele Elementarereignisse möglich und alle diese treten
mit der gleichen Wahrscheinlichkeit ein. Wir sprechen in diesem Fall von einem
Laplace-Experiment.
\begin{defnn}
Ein W-Raum $(\Omega, \AA,P)$ mit
$\Omega $ endlich, $\AA = {\cal P} (\Omega )$ und
\begin{align*}
P(\{\omega \}) = \frac{1}{|\Omega |},\qquad \omega \in \Omega,\quad
|\Omega | = \text{ Anzahl der Elemente in }\Omega,
\end{align*}
heißt \emph{Laplacescher W-Raum}:
\begin{align*}
\text{Für $A \in \AA$ gilt $P(A)= \frac{|A|}{|\Omega|}=
\frac{\mbox{Anzahl der ``günstigen'' Fälle}}
{\mbox{Anzahl der ``möglichen'' Fälle}}$}.\fishhere
\end{align*}
\end{defnn}

\begin{bsp}
\label{bsp:1.1:1}
Seien $n\in\N$, $p\in[0,1]$ fest. Betrachte $n$ gleichartige Experimente ohne
gegenseitige Beeinflussung mit den möglichen Ausgängen
\begin{align*}
\text{0 (``Misserfolg'', ``Kopf'') und 1
(``Erfolg'', ``Zahl'')}
\end{align*}
mit jeweiliger Erfolgswahrscheinlichkeit $p$.
Dies bezeichnet man auch als $n$-Tupel von
Bernoulli-Versuchen.\footnote{Jakob I. Bernoulli (* 6. Januar 1655 in Basel; †
16. August 1705 ebenda) war ein Schweizer Mathematiker und Physiker.}
\bsphere
\end{bsp}

\textit{Wie groß ist die Wahrscheinlichkeit in $n$ Bernoulli-Versuchen genau
$k$, wobei $k\in\setd{0,1,\ldots,n}$. Erfolge zu erhalten?}

Dieses Experiment entspricht der Platzierung von $k$ Kugeln auf $n$ Plätzen ohne
Mehrfachbelegung eines Platzes.
\begin{figure}[H]
\begin{tabular}{l|l}
\textbf{Experiment} & \textbf{Wahrscheinlichkeit}\\\hline
$\underbrace{1\; \ldots\; 1}_{k} \underbrace{0\;\ldots\; 0}_{n-k}$ & 
$p\cdots p \cdot (1-p)\cdots (1-p) = p^k(1-p)^{n-k}$\\
$0\; 1\; 1 \ldots 1\; 0 \ldots 0$ & 
$(1-p)p\cdots p(1-p)\cdots (1-p) = p^k(1-p)^{n-k}$.
\end{tabular}
\end{figure}
Die Reihenfolge der Erfolge und Miserfolge ist unerheblich für die
Wahrscheinlichkeit. Das Experiment besitzt genau
\begin{align*}
\binom{n}{k} = \dfrac{n!}{k!(n-k)!}
\end{align*}
mögliche Ausgänge. Aufsummation liefert die Gesamtwahrscheinlichkeit,
\begin{align*}
W = \binom{n}{k} p^k(1-p)^{n-k} \defr b(n,p;k),
\end{align*}
wobei $\binom{n}{k} = 0$, falls
$k\notin\setd{0,1,\ldots,n}$.
\begin{defnn}
Durch $P(\setd{k}) \defl b(n,p;k)$ für $k\in\N_0$ ist ein W-Maß auf $\BB$
definiert, wobei für $A\subseteq \N_0$,
\begin{align*}
P(A) = \sum\limits_{k\in\N_0\cap A} b(n,p;k) \defl \sum\limits_{k\in\N_0\cap A}
\binom{n}{k} p^k(1-p)^{n-k}.
\end{align*}
$P$ ist ein W-Maß und heißt \emph{Binomialverteilung} $b(n,p)$ mit Parameteren
$n\in\N$ und $p\in[0,1]$.\fishhere
\end{defnn}

\begin{bsp}
\label{bsp:1.2}
500g Teig und 36 Rosinen seien rein zufällig verteilt. Daraus
werden 10 Brötchen mit je 50g Teil geformt. Greife ein Brötchen heraus.

\textit{Wie groß ist die Wahrscheinlichkeit, dass dieses
Brötchen 2 oder mehr Rosinen enthält?}

Rosine Nr. 1-36 ist unabhängig von den anderen mit Wahrscheinlichkeit
$p=\frac{1}{10}$ im betrachteten Brötchen. Es handelt sich also um 36
Bernoulli Versuche mit jeweiliger Erfolgswahrscheinlichkeit $p=\frac{1}{10}$. 
\begin{align*}
P(A) &= \sum\limits_{k=2}^{36} b(n,p;k) = \sum\limits_{k=2}^{36}
\binom{36}{k}\frac{1}{10^k}\left(1-\frac{1}{10}\right)^{36-k} \\
&=1-b\left(36,\frac{1}{10};0\right) - b\left(36,\frac{1}{10};1\right) \approx
0.89.\bsphere
\end{align*}
\end{bsp}

Wir wollen das Beispiel der Binomialverteilung nun dahingehend verallgemeinern,
dass wir die Menge $\Omega$ auf die ganze reelle Achse erweitern.

\begin{defnn}
Sei $\Omega=\R$, $\AA = \BB$ und 
$(p_k)_{k\in\N_0}$ eine reelle Zahlenfolge mit
\begin{align*}
0\le p_k \le 1\quad \text{ für } k\in\N_0,\qquad \sum\limits_{k\in\N_0} p_k = 1.
\end{align*}
$(p_k)$ bezeichnet man als \emph{Zähldichte}. Durch $P(\setd{k}) = p_k, \forall
k\in\N_0$ also
\begin{align*}
P(A) = \sum\limits_{k\in A\cap \N_0} p_k,\qquad A\in\BB.
\end{align*}
wird ein W-Maß auf $\BB$ definiert.
$P$ ist auf $\N_0$ \emph{konzentriert}, d.h. $P(\N_0)=1$ und besitzt die
Zähldichte $(p_k)_{k\in\N_0}$.\fishhere
\end{defnn}
\begin{proof}
Nulltreue, Positivität und Normiertheit sind klar. Wir weisen nun die
$\sigma$-Additivität nach: Seien $A_n$ für $n\in\N$ paarweise disjunkt. Ohne
Einschränkung können wir davon ausgehen, dass $A_n\subseteq\N$, es gilt also,
\begin{align*}
P\left(\sum\limits_{n=1}^\infty A_n\right) = \sum\limits_{k\in \sum_n A_n} p_k.
\end{align*}
Da die Reihe absolut konvergent ist, dürfen wir den großen Umordnungssatz
aus der Analysis anwenden, und die Reihenglieder umgruppieren,
\begin{align*}
\ldots = \sum\limits_{n=1}^\infty \underbrace{\sum\limits_{k\in A_n} p_k}_{\defl
P(A_n)} = \sum_{n=1}^\infty P(A_n).\qedhere
\end{align*}
\end{proof}

Für Beispiel \ref{bsp:1.1:1} gilt $(b(n,p;k))_{k\in\N_0} = (p_k)_{k\in\N_0}$.

\subsection{Poissonverteilung}

Seien nun $p_n\in (0,1)$ mit $n\cdot p_n\to \lambda\in(0,1)$ für $n\to\infty$,
so ist
\begin{align*}
b(n,p_n;k) &= \binom{n}{k} p_n^k(1-p_n)^{n-k} \\ &= \frac{n(n-1)\cdots
(n-k+1)}{k!} p_n^k (1-p_n)^{n-k} \\&
= \frac{n(n-1)\cdots(n-k+1)}{k!}p_n^k \cdot 
(1-p_n)^{-k}\left[(1-p_n)^{1/p_n}\right]^{np_n}
\end{align*}
Für den Grenzübergang $n\to\infty$ gilt,
\begin{align*}
&\frac{n(n-1)\cdots(n-k+1)}{k!}p_n^k \to \frac{\lambda^k}{k!},\\
&(1-p_n)^{-k} \to 1,\\
&(1-p_n)^{1/p_n} \to \e^{-1}.
\end{align*}
Wir erhalten also insgesamt,
\begin{align*}
b(n,p_n;k) \overset{n\to\infty}{\to}
\frac{\lambda^k}{k!}\e^{-\lambda}\defr\pi(\lambda;k).
\end{align*}
$\pi(\lambda;k)$ definiert eine Zähldichte, denn
\begin{align*}
\sum\limits_{k\in\N_0} \pi(\lambda;k)
= \e^{-\lambda}\underbrace{\sum\limits_{k=0}^\infty
\frac{\lambda^k}{k!}}_{\defl\e^\lambda} = 1.
\end{align*}
\begin{defn}
Das zu $\pi(\lambda;k)$ gehörige W-Maß $\pi(\lambda)$ auf $\BB$ ist gegeben
durch,
\begin{align*}
\pi(\lambda)(\setd{k}) \defl \pi(\lambda;k)
\end{align*}
und heißt \emph{Poisson-Verteilung}\footnote{Siméon-Denis Poisson (* 21. Juni
1781 in Pithiviers (Département Loiret); † 25. April 1840 in Paris) war ein
französischer Physiker und Mathematiker.} mit Parameter $\lambda$.\fishhere
\end{defn}

Die Poission-Verteilung hat sehr viele angenehme Eigenschaften, insbesondere
ist sie numerisch gut handhabbar. Daher approximiert man oft eine
Binomialverteilung durch eine Poissonverteilung. Die Approximation ist gut für
$p$ ``klein'' und $n$ ``groß''.

\begin{bsp}
\label{bsp:1.3}
\textit{Eine Anwendung zu Beispiel \ref{bsp:1.2}.}
Gegeben seien $m$ gleichgroße mehrfach besetzbare Zellen. 
\begin{figure}[H]
\centering
\begin{pspicture}(0,-0.62)(4.98,0.6)
\psline(0.02,0.46)(3.62,0.46)
\psline(0.0,0.58)(0.0,0.34)
\psline(0.42,0.58)(0.42,0.34)
\psline(0.82,0.58)(0.82,0.34)
\psline(1.22,0.58)(1.22,0.34)
\psline(1.62,0.58)(1.62,0.34)
\psline(2.02,0.58)(2.02,0.34)
\psline[linecolor=darkblue](2.42,0.58)(2.42,0.34)
\psline[linecolor=darkblue](2.82,0.58)(2.82,0.34)
\psline[linecolor=darkblue](2.42,0.46)(2.82,0.46)
\psline(3.22,0.58)(3.22,0.34)
\psline(3.62,0.58)(3.62,0.34)
\psline[linecolor=darkblue]{<-}(2.62,0.34)(2.98,-0.26)
\rput(3.44,-0.395){\color{gdarkgray}vorgegebene Zelle Z}
\end{pspicture} 
\caption{Zur Anwendung von \ref{bsp:1.2}.}
\end{figure}

$n$ Teilchen seien rein zufällig auf die Zellen verteilt. Die
Wahrscheinlichkeit, dass Teilchen Nr. 1 in Zelle $Z$ landet, ist
$\frac{1}{m}$. Also ist die Wahrscheinlichkeit, dass $k$ Teilchen in $Z$ landen
$b(n,\frac{1}{m};k)$. Die ``Belegungsintensität'' ist $\frac{n}{m}$.

Für $n\to\infty$ und $m\to\infty$ gelte $\frac{n}{m}\to\lambda\in(0,\infty)$.
In der Grenze ist die Wahrscheinlichkeit, dass genau $k\in\N_0$ Teilchen in der
Zelle $Z$ landen gegeben durch $\pi(\lambda;k)$. Damit können wir das Beispiel
\ref{bsp:1.2} wieder aufgreifen.

Betrachte eine große Anzahl von Rosinen in einer großen Teigmenge. Teile den
Teig in $m$ gleichgroße Brötchen auf, mit $\lambda\defl\frac{n}{m}$. Nehmen wir
nun ein Brötchen heraus, so ist die Wahrscheinlichkeit, dass es genau $k$
Rosinen enthält
\begin{align*}
\approx \e^{-\lambda}\frac{\lambda^k}{k!}.
\end{align*}
Im Zahlenbeispiel $n=36$, $m=10$, $\lambda=3.6$ gilt
\begin{align*}
\sum\limits_{k=2}^{36} \e^{-\lambda} \frac{\lambda^k}{k!} \approx
1-\e^{-\lambda}-\lambda \e^{-\lambda} \approx 0.874,\quad \text{statt
}0.89.\bsphere
\end{align*}
\end{bsp}

\subsection{Weitere Eigenschaften von Wahrscheinlichkeitsmaßen}

Die reellen Zahlen $\R$ sind geordnet und für ``monoton
fallende'' bzw. ``monoton steigende'' Folgen wissen wir, dass ein eigentlicher
oder uneigentlicher Grenzwert existiert. Ein Mengensystem lässt sich durch
``$\subseteq$'' oder ``$\supseteq$'' halbordnen, somit lässt sich der
Monotoniebegriff in gewisser Weise übertragen.

\begin{defnn}
Die Konvergenz \emph{von unten} bzw. \emph{von oben} ist für die Mengen $A$ und
$A_{n}$ $(n=1,2,\ldots )$ folgendermaßen definiert:
\begin{align*}
&A_{n}\uparrow A:A_{1} \subset A_{2} \subset A_{3} \subset \ldots , \;
\bigcup^{\infty}_{n=1} A_{n} = A,\\
&A_{n} \downarrow A: A_{1} \supset A_{2} \supset A_{3} \supset \ldots ,\;
\bigcap^{\infty}_{n=1} A_{n} = A.\fishhere
\end{align*}
\end{defnn}

Da wir nun über eine Art Konvergenzbegriff für Mengen verfügen, lässt sich nach
der Stetigkeit von Abbildungen von Mengensystemen in die reellen Zahlen fragen.
Natürlich handelt es sich hierbei nicht um Konvergenz und Stetigkeit im Sinne
der Analysis, denn wir haben auf $\AA$ keine Norm zur Verfügung.

\begin{prop}
\label{prop:1.2}
Sei $\AA$ eine Algebra in $\Omega $, $\mu$ ein \textit{endlicher} (d.h. $\mu
(\Omega) < \infty $) Inhalt auf $\AA$.  Dann sind die folgenden Aussagen
äquivalent:
\begin{propenum}
\item
\label{prop:1.2:1}
$\mu$ ist $\sigma $-additiv
\item
\label{prop:1.2:2}
$\mu$ ist stetig von unten,\\
d.h. $[A_{n} \in \AA$, $A_{n} \uparrow A \in \AA \Rightarrow
\mu (A_{n}) \to \mu (A) ]$
\item
\label{prop:1.2:3}
$\mu $ ist stetig von oben, \\
d.h. $[A_{n} \in \AA$, $ A_{n} \downarrow A \in \AA \Rightarrow
\mu (A_{n}) \to \mu (A) ] $
\item
\label{prop:1.2:4}
$\mu$ ist $\varnothing$-stetig, \\
d.h. $[A_{n} \in \AA$, $ A_{n} \downarrow \varnothing \Rightarrow
\mu (A_{n} )\to 0]\, .$
\end{propenum}
Auch ohne die Voraussetzung der Endlichkeit von $\mu $ gilt a)
$\Longleftrightarrow $ b).\fishhere
\end{prop}

\begin{proof}
\ref{prop:1.2:1}$\Rightarrow$\ref{prop:1.2:2}: Sei $A_n = A_1+A_2\setminus A_1
+ \ldots + A_n\setminus A_{n-1}$ und $A=A_1+\sum\limits_{k=2}^\infty
A_{k}\setminus A_{k-1}$. Wir verwenden die Additivität von $\mu$ und erhalten
somit,
\begin{align*} 
\mu\left(A_n\right) &= \mu(A_1) + \mu(A_2\setminus A_1) + \ldots +
\mu(A_n\setminus A_{n-1})\\
&= \mu(A_1) + \sum\limits_{k=2}^n \mu(A_k\setminus A_{k-1})
 \overset{n\to\infty}{\to} \mu(A_1) + \sum\limits_{k=2}^\infty \mu(A_k\setminus
 A_{k-1}) = \mu(A).
\end{align*}
Die übrigen Implikationen folgen analog.\qedhere
\end{proof}

Da Wahrscheinlichkeitsräume \textit{per definitionem} endlich sind, ist Satz
\ref{prop:1.2} dort stets anwendbar. Insbesondere ist jedes W-Maß stetig.

\begin{lem}
\label{lem:1.3}
Sei $\AA$ eine Algebra in $\Omega $, $\mu$ ein Inhalt auf
$\AA$.
\begin{propenum}
\item\label{lem:1.3:1} $\mu $ ist monoton, \\ d.h. $[A,B \in \AA$, $A \subset B
  \Rightarrow \mu (A) \leq \mu (B) ]\, .$
\item\label{lem:1.3:2} $\mu$ ist subadditiv, \\ d.h. $[A_{1}, \ldots , A_{m}
\in \AA \Rightarrow \mu ( \bigcup^{m}_{n=1}A_{n}) \leq
  \sum\limits^{m}_{n=1} \mu (A_{n})]\, .$
\item\label{lem:1.3:3}
Ist $\mu$ ein Maß, dann ist $\mu$ sogar $\sigma$-subadditiv, \\
d.h. $[A_{n} \in \AA$ für $n\in\N$, $
\bigcup^{\infty}_{n=1}A_{n}\in \AA \Rightarrow
\mu ( \bigcup^{\infty}_{n=1} A_{n}) \leq
\sum\limits^{\infty}_{n=1} \mu (A_{n})]\, .$\fishhere
\end{propenum}
\end{lem}
\begin{proof}
\ref{lem:1.3:1}: Seien $A,B\in\AA$ und $A\subseteq B$. Nun gilt
$B=A+B\setminus A$, also
\begin{align*}
&\mu(B) = \mu(A)+\underbrace{\mu(B\setminus A)}_{\ge 0},\\
\Rightarrow &\mu(A)\le \mu(B).
\end{align*}
\ref{lem:1.3:2}: Wir untersuchen den Spezialfall $n=2$. Seien $A_1,A_2\in\AA$,
so ist
\begin{align*}
\mu(A_1\cup A_2) &= \mu(A_1+A_2\setminus A_1) = \mu(A_1) +
\underbrace{\mu(A_2\setminus A_1)}_{\le\mu(A_2)}
\\ &\le \mu(A_1)+\mu(A_2).
\end{align*}
Der allgemeine Fall folgt durch Induktion.

\noindent\ref{lem:1.3:3}: Für $N\in\N$ gilt,
\begin{align*}
\mu(A_1\cup \ldots \cup A_N) \le \sum\limits_{n=1}^N \mu(A_n) \le
\sum\limits_{n=1}^\infty \mu(A_n).
\end{align*}
Beim Grenzübergang für $N\to\infty$ erhalten wir so,
\begin{align*}
&\mu(A_1\cup \ldots \cup A_N) \to \mu\left(\bigcup_{n=1}^\infty A_n\right),
\end{align*}
und die Behauptung folgt.\qedhere
\end{proof}

W-Maße sind also $\sigma$-additiv, $\sigma$-subadditiv, monoton und stetig.
Diese Eigenschaften sind zwar unscheinbar, es lassen sich damit aber sehr
starke Aussagen beweisen, wie wir gleich sehen werden.

Erinnern wir uns aber zunächst an den \textit{limes superior} einer reellen
Zahlenfolge $(a_n)$,
\begin{align*}
\limsup\limits_{n\to\infty} a_n \defl
\inf\limits_{n\in\N}\sup\limits_{k\ge n} a_k.
\end{align*}
Wir können nun diese Definitionen auf Mengen übertragen.
\begin{defnn}
Sei $(\Omega,\AA,P)$ ein W-Raum, $A_n\in\AA$ für $n\in\N$
\begin{align*}
\limsup\limits_{n\to\infty} A_n= \bigcap_{n=1}^\infty \bigcup_{k=n}^\infty A_k
&\defl \setdef{\omega\in\Omega}{\text{es ex. unendlich viele $n$ mit
$\omega\in A_n$}}\\
&= \text{``Ereignis, dass unendlich viele $A_n$ eintreten''}.\fishhere
\end{align*}
\end{defnn}

\begin{prop}[1.\ Lemma von Borel und Cantelli]
\label{prop:1.3}
Sei $(\Omega, \AA, P)$ ein W-Raum und $A_{n}\in\AA$ für $n \in \N$,
dann gilt
\begin{align*}
\sum\limits^{\infty}_{n=1} P(A_{n}) < \infty \quad \Longrightarrow \quad & P
(\limsup A_{n}) = 0,
\end{align*}
d.h.$\Pfs$ gilt: $A_{n} $ tritt nur endlich oft auf.\fishhere
\end{prop}

\begin{bsp}
\textit{Lernen durch Erfahrung}, d.h. je öfter man eine Tätigkeit durchführt, je
kleiner ist die Wahrscheinlichkeit, dass ein Fehler auftritt.\\ Betrachte eine
Folge von Aufgaben. Ein Misserfolg in der $n$-ten Aufgabe tritt mit der
Wahrscheinlichkeit
\begin{align*}
P(A_n) = \frac{1}{n^2}
\end{align*}
auf. Es gilt dann
\begin{align*}
&\sum\limits_{n=1}^\infty P(A_n) = \sum\limits_{n=1}^\infty \frac{1}{n^2} <
\infty.
\end{align*}
$\overset{\ref{prop:1.3}}{\Rightarrow}$ Mit Wahrscheinlichkeit 1 wird von
einem zufälligen Index $n$ an \textit{jede} Aufgabe immer richtig
gelöst.\bsphere
\end{bsp}
\begin{proof}[Beweis des 1. Lemma von Borel und Cantelli.]
Sei $B\defl\limsup_n A_n$, d.h.
\begin{align*}
B = \bigcap_{n=1}^\infty \bigcup_{k\ge n} A_k.
\end{align*}
Offensichtlich gilt $\bigcup_{k\ge n} A_k \downarrow B$. Aufgrund der Monotonie
von $P$ gilt somit,
\begin{align*}
P(B) \le P\left(\bigcup_{k\ge n} A_k \right) \le \sum\limits_{k=n}^\infty
P(A_k) \to 0,
\end{align*}
da die Reihe nach Voraussetzung konvergent ist. Also ist $P(B)=0$.\qedhere
\end{proof}

\subsection{Fortsetzung von Wahrscheinlichkeitsmaßen}

Bisher haben wir ausschließlich mit Maßen gearbeitet, die auf $\N_0$
konzentiert sind, d.h. $P(\N_0) = 1$. Unser Ziel ist es jedoch ein Maß auf
$\BB_n$ zu definieren, das jedem Intervall (Rechteck) seinen
elementargeometrischen Inhalt zuordnet. 

Betrachten wir wieder das Erzeugersystem $J_n$ der halboffenen Intervalle,
so bildet dieses einen Halbring.
\begin{defnn}
$h\subseteq \PP(\Omega)$ heißt \emph{Halbring} über $\Omega$, falls
\begin{defnenum}
  \item $\varnothing\in h$,
  \item $A,B\in h \Rightarrow A\cap B\in h$,
  \item $A,B\in h$, $A\subseteq B \Rightarrow \exists k\in\N: C_1,\ldots,C_k\in
  h : B\setminus A = \sum_{i=1}^k C_i$.\fishhere
\end{defnenum} 
\end{defnn}
Auf dem Halbring $J_n$ können wir das gesuchte Maß definieren durch,
\begin{align*}
\mu: J_n \to\R,\quad 
(a_i,b_i] \mapsto \mu((a_i,b_i]) = \prod\limits_{i=1}^n (b_i-a_i).
\end{align*}
Mit Hilfe des folgenden Satzes können wir $\mu$ \textit{eindeutig} auf $\BB_n$
fortsetzen.
\begin{prop}[Fortsetzungs- und Eindeutigkeitssatz]
\label{prop:1.4}
Sei $\mu$ ein Maß auf einem Halbring $h$ über $\Omega$, so besitzt
$\mu$ eine Fortsezung $\mu^*$ zu einem Maß auf $\FF(h)$.
\\
Gilt außerdem $\Omega = \bigcup_{i=1}^\infty A_i$ mit $\mu(A_i)<\infty$, so ist
$\mu^*$ eindeutig.\fishhere
\end{prop}
\begin{proof}
Siehe Elstrodt, Kap. III, $\mathsection\mathsection$ 4.5.\qedhere
\end{proof}

Insbesondere lassen sich auf Halbringen bzw. Algebren über W-Räumen definierte
Maße immer eindeutig fortsetzen, da hier bereits $\mu(\Omega)<\infty$.

\begin{bem}
\label{bem:1.3}
$\mu^*$ in Satz \ref{prop:1.4} ist für $A\in\FF(\AA)$ gegeben durch
\begin{align*}
\mu^*(A) = \inf \setdef{ \sum\limits^{\infty }_{n=1} \mu(B_{n})}{B_{n} \in
\AA \; \; (n=1,2,\ldots ) \mbox{ mit } A \subset \bigcup\limits_{n}
B_{n}}.\maphere
\end{align*}
\end{bem}

\begin{bem}
\label{bem:1.4}
Es ist im Allgemeinen nicht möglich, $\mu$ von $\AA$ auf ${\PP}(\Omega)$
fortzusetzen. H. Lebesque\footnote{
Henri Léon Lebesgue (* 28. Juni 1875 in Beauvais; † 26.
Juli 1941 in Paris) war ein französischer Mathematiker.
}
hat sogar bewiesen, dass es unter Annahme des
Auswahlaxioms nicht möglich ist, ein Maß auf $\PP([0,1])$ fortzusetzen. Ist
$\Omega$ jedoch abzählbar, so kann $P$ stets auf die Potenzmenge fortgesetzt
werden.\maphere
\end{bem}

\begin{prop}[Korollar]
\label{prop:1.5}
Es gibt genau ein Maß $\lambda$ auf $\BB_n$, das jedem
beschränkten (nach rechts halboffenen) Intervall in $\R^n$ seinen
elementargeometrischen Inhalt zuordnet.\fishhere
\end{prop}

\begin{defn}
\label{defn:1.8}
Das Maß $\lambda$ in Korollar \ref{prop:1.5} heißt \emph{Lebesgue--Borel--Maß
(L-B-Maß)} in $\R^n$.\fishhere
\end{defn}

\begin{bsp}
\label{bsp:1.5}
\textit{Rein zufälliges Herausgreifen eines Punktes aus dem Intervall $[0,1]$}.

Sei $\Omega=[0,1]$, $\AA\defl[0,1]\cap\BB\defl\setdef{B\cap[0,1]}{B\in\BB}$.
Das W-Maß $P$ auf $\AA$ soll jedem Intervall in $[0,1]$ seine elementare Länge
zuordnen. Also ist
\begin{align*}
P=\lambda\big|_{[0,1]\cap\BB}
\end{align*}
eine Restriktion von $\lambda$ auf $[0,1]\cap\BB$.\bsphere
\end{bsp}

Das $\lambda$-Maß hat die Eigenschaft, dass nicht nur die leere Menge
sondern auch jede höchstens abzählbare Menge das Maß Null hat. Eine Menge
$N\subseteq \Omega$ mit $\mu(N)=0$ nennt man \emph{Nullmenge}.

Oft kann man Eigenschaften nicht auf ganz $\Omega$ nachweisen aber auf
$\Omega$ bis auf eine Nullmenge. Solche Eigenschaften treten vor allem in der
Maßtheorie auf und sind daher auch in der Wahrscheinlichkeitstheorie von großer
Bedeutung.

\begin{defnn}
Sei Maßraum $(\Omega, \AA, \mu )$, W-Raum $(\Omega, {\AA},P)$. Eine Aussage
gilt
\begin{defnenum}
  \item \emph{$\mu$-fast überall} ($\mu$-f.ü.) in $\Omega$ bzw. \emph{$P$-fast
sicher} ($P$-f.s.) in $\Omega$, falls sie überall bis auf einer Menge $\in
{\AA}$ vom $\mu$-Maß bzw.\ $P$-Maß Null [oder eine Teilmenge hiervon] gilt.
\item \emph{$L$-fast überall} ($L$-f.ü.) in $\R^{n}$ \ldots, wenn sie
überall bis auf einer Menge $\in \BB_{n}$ vom L-B-Maß Null [oder eine Teilmenge
hiervon] gilt.\fishhere   
\end{defnenum}
\end{defnn}

\clearpage
% =============================================================
%                                   Verteilungsfunktionen
% =============================================================
\section{Verteilungsfunktionen}
\label{chap:1.c}


W-Maße sind Abbildungen der Art
\begin{align*}
P : \sigma\subseteq \PP(\Omega) \to \R.
\end{align*}
Ihr Definitionsbereich ist somit ein Mengensystem. Wir konnten einige
elementare Eigenschaften wie Stetigkeit und Monotonie aus der reellen Analysis
auf diese Abbildungen übertragen, dennoch lassen sie sich nicht immer so leicht
handhaben wie Funktionen.

Betrachten wir nun ein W-Maß auf $\BB$, so ist durch folgende Vorschrift eine
Funktion gegeben
\begin{align*}
F: \R\to\R,\quad F(x) \defl P((-\infty,x]),\quad x\in\R.
\end{align*}
$F$ wird als die zu $P$ gehörende \emph{Verteilungsfunktion (Verteilungsfunktion)} bezeichnet.
Verteilungsfunktionen auf $\R$ lassen sich in der Regel viel leichter handhaben
als Maße auf $\BB$.

In diesem Abschnitt werden wir zeigen, dass eine Bijektion zwischen Maßen und
Verteilungsfunktionen auf einem W-Raum existiert, wir also jedem Maß eine
Verteilungsfunktion zuordnen können und umgekehrt. Durch die
Verteilungsfunktionen werden die Maße ``greifbar''.

\clearpage

\begin{bsp}
\label{bsp:1.6}
\begin{bspenum}
  \item Würfeln mit Wahrscheinlichkeiten $\underbrace{p_1}_{\ge
  0},\ldots,\underbrace{p_6}_{\ge 0}$ für die Augenzahlen $1,\ldots,6$ wobei
  $p_1+\ldots+p_6 =1$.
  
Als zugehörigen Wahrscheinlichkeitsraum wählen wir $\Omega=\R$ mit $\AA=\BB$.
Das W-Maß auf $\BB$ erfüllt $P(\setd{1})=p_1,\ldots,P(\setd{6}) = p_6$.

Zur Verteilungsfunktion betrachte die Skizze. 
\item Betrachte erneut Beispiel \ref{bsp:1.5}:
Sei $(\Omega,\AA,P) =
([0,1],[0,1]\cap\BB, \lambda\big|_{[0,1]\cap\BB})$ abgewandelt zu
$(\Omega,\AA,P)=(\R,\BB,P)$. Das Lebesgue-Maß ist nur dann ein
Wahrscheinlichkeitsmaß, wenn man den zugrundeliegenden Raum auf ein Intervall
der Länge Eins einschränkt.

Wir setzten daher $P(B)\defl\lambda([0,1]\cap B)$. Die zugehörige
Verteilungsfunktion ist dann gegeben durch,
\begin{align*}
F(x) = \begin{cases}
       0,& x\le 0,\\
       x, & x\in(0,1),\\
       1, & x\ge 1.\bsphere
       \end{cases}
\end{align*}

\begin{figure}[htpb]
\centering
\begin{pspicture}(-0.4,-1.25)(5.36,1.4)
\psline{->}(0.34,-0.75)(5.02,-0.75)
\psline{->}(1.54,-0.95)(1.54,1.35)

\psline(1.94,-0.63)(1.94,-0.87)
\psline(2.34,-0.63)(2.34,-0.87)
\psline(2.74,-0.63)(2.74,-0.87)
\psline(3.14,-0.63)(3.14,-0.87)
\psline(3.54,-0.63)(3.54,-0.87)
\psline(3.94,-0.63)(3.94,-0.87)

\psline(1.42,1.05)(1.66,1.05)
\psline[linecolor=darkblue](0.36,-0.75)(1.92,-0.75)
\psline[linecolor=darkblue]{*-}(1.94,-0.45)(2.34,-0.45)
\psline[linecolor=darkblue]{*-}(2.34,-0.15)(2.74,-0.15)
\psline[linecolor=darkblue]{*-}(2.74,0.15)(3.14,0.15)
\psline[linecolor=darkblue]{*-}(3.14,0.45)(3.54,0.45)
\psline[linecolor=darkblue]{*-}(3.54,0.73)(3.94,0.73)
\psline[linecolor=darkblue]{*-}(3.92,1.05)(5.34,1.05)

\psline(1.62,-0.45)(1.46,-0.45)
\psline(1.62,-0.15)(1.46,-0.15)
\psline(1.62,0.73)(1.46,0.73)

\rput(1.19,1.05){\color{gdarkgray}1}

\rput[r](1.4,-0.45){\color{gdarkgray}\small$p_1$}
\rput[r](1.4,-0.1){\color{gdarkgray}\small$p_1+p_2$}
\rput[r](1.4,0.67){\color{gdarkgray}\small$p_1+...+p_5$}

\rput(1.94,-1.045){\color{gdarkgray}1}
\rput(2.34,-1.045){\color{gdarkgray}2}
\rput(2.74,-1.045){\color{gdarkgray}3}
\rput(3.14,-1.045){\color{gdarkgray}4}
\rput(3.54,-1.045){\color{gdarkgray}5}
\rput(3.94,-1.045){\color{gdarkgray}6}

\end{pspicture}
\quad
\begin{pspicture}(-0.1,-1.43)(4.22,1.43)
\psline{->}(0.68,-1.17)(0.68,1.25)
\psline{->}(0.06,-0.97)(4.1,-0.95)
\psline[linecolor=darkblue](0.26,-0.97)(0.68,-0.97)
\psline[linecolor=darkblue](0.68,-0.97)(2.08,0.63)
\psline[linecolor=darkblue](2.08,0.63)(3.88,0.63)
\psline(2.08,-0.87)(2.08,-1.05)
\psline(0.6,0.63)(0.76,0.63)

\rput(0.5,0.63){\color{gdarkgray}1}
\rput(2.08,-1.285){\color{gdarkgray}1}
\rput(4.09,-1.125){\color{gdarkgray}$x$}
\rput(0.31,1.235){\color{gdarkgray}$F(x)$}
\end{pspicture} 
\caption{Verteilungsfunktionen zum Würfelexperiment und zu Beispiel
\ref{bsp:1.5}}
\end{figure}
\end{bspenum}
\end{bsp}

\begin{defn}
\label{defn:1.9}
Eine Funktion $F: \R \to \R$,
die monoton wachsend (im Sinne von nicht fallend) und rechtsseitig stetig
ist mit
\begin{align*}
\lim\limits_{x\to - \infty} F(x) =0, \qquad \lim\limits_{x \to + \infty} F(x)
= 1,
\end{align*}
heißt (eindimensionale) \emph{Verteilungsfunktion (Verteilungsfunktion)}.\fishhere
\end{defn}

Der folgende Satz stellt nun eine eindeutige Beziehung zwischen einem Maß auf
$\BB$ und einer eindimensionalen Verteilungsfunktion her.

\begin{prop}
\label{prop:1.6}
Zu jedem W-Maß $P$ auf $\BB$ gibt es genau eine Verteilungsfunktion
$F: \R \to \R$, so dass gilt
\begin{align*}
F(b) -F(a) = P ((a,b]), \; \; a \leq b, \; \; a,b \in \R\tag{*}
\end{align*}
gilt, und umgekehrt. Es existiert also eine Bijektion $P
\longleftrightarrow F$.\fishhere
\end{prop}

\begin{proof}
Wir wählen für die Verteilungsfunktion den Ansatz,
\begin{align*}
F(x)\defl P((-\infty,x]),\qquad x\in\R.
\end{align*}
 \begin{enumerate}[label=\arabic{*}.),leftmargin=17pt]
  \item Die (*)-Bedingung ist erfüllt, denn es gilt für $a,b\in\R$ mit $a<b$,
\begin{align*}
P((a,b]) = P((-\infty,b]\setminus(-\infty,a]) = P((-\infty,b])-P((-\infty,a]).
\end{align*}
  \item $F$ ist monoton steigend, denn $P$ ist monoton.
  \item $F$ ist rechtsstetig, denn für $x_n\downarrow x$ folgt
  $(-\infty,x_n]\downarrow (-\infty,x]$. Somit gilt nach Satz \ref{prop:1.2},
\begin{align*}
F(x_n)=P((-\infty,x_n]) \downarrow P((-\infty,x])=F(x).
\end{align*}
\item$F(x)\to 0$ für $x\to -\infty$, denn
\begin{align*}
x_n\downarrow -\infty \Rightarrow (-\infty,x_n]\downarrow \varnothing.
\end{align*}
Mit Satz \ref{prop:1.2} folgt daraus,
\begin{align*}
F(x_n)=P((-\infty,x_n]) \downarrow P(\varnothing) = 0.
\end{align*}
\item $F(x)\to 1$ für $x\to\infty$, denn
\begin{align*}
x_n\to \infty \Rightarrow (-\infty,x_n)\uparrow \R.
\end{align*}
Ebenfalls mit Satz \ref{prop:1.2} folgt
\begin{align*}
F(x_n)=P((-\infty,x_n])\to P(\R) = 1.
\end{align*}
\item \textit{Eindeutigkeit}. Seien $F,F^*$ zwei Verteilungsfunktion gemäß der
(*)-Bedingung in Satz \ref{prop:1.6}. Dann unterscheiden sich $F$ und $F^*$
lediglich um eine Konstante, d.h.
\begin{align*}
F^* = F + c.
\end{align*}
Nun gilt aber $F^*(x)$, $F(x)\to 1$ für $x\to+\infty$, also $c=0$, d.h.
$F=F^*$.\qedhere
\end{enumerate}
\end{proof}

Wir betrachten nun zwei sehr wichtige Verteilungsfunktionen als Beispiele.
\begin{bsp}
\label{bsp:1.7}
Die Funktion
$\Phi :\R \to [0,1] $ mit
\begin{align*}
\Phi (x) \defl \frac{1}{\sqrt{2 \pi }} \int\limits^{x}_{-\infty }
\underbrace{\e^{-\frac{t^{2}}{2}}}_{\ph(t)} dt, \; \; x \in \R,
\end{align*}
ist eine Verteilungsfunktion, denn $\Phi$ ist streng monoton, da $\ph > 0$, die
Rechtsstetigkeit folgt aus der Stetigkeit des Integrals. $F(x)\to 0$ für
$x\to-\infty$ ist offensichtlich, einzig $F(x)\to 1$ für $x\to\infty$
muss man explizit nachrechnen (Funktionentheorie).

Der Integrand $\ph(t)$ heißt übrigens Gaußsche Glockenkurve und das zu $\Phi$
gehörige W-Maß auf $\BB$, \emph{standardisierte Normverteilung $N(0,1)$}.

\begin{figure}[!htpb]
\centering
\begin{pspicture}(-2.8,-1)(2.8,3)

 \psaxes[labels=none,ticks=none,linecolor=gdarkgray,tickcolor=gdarkgray]{->}%
 (0,0)(-2.7,-0.5)(2.7,2.5)[\color{gdarkgray}$t$,-90][\color{gdarkgray}$\ph(t)$,0]

\psplot[linewidth=1.2pt,%
	     linecolor=darkblue,%
	     algebraic=true]%
	     {-2.5}{2.5}{2*(2.71828)^(-(3/2.5*x)^2/2)}

\rput(1.4,1.4){\color{gdarkgray}$\ph$}
\end{pspicture}
\begin{pspicture}(-2.8,-1)(2.8,3)
 \psaxes[labels=none,ticks=none,linecolor=gdarkgray,tickcolor=gdarkgray]{->}%
 (0,0)(-2.7,-0.5)(2.7,2.5)[\color{gdarkgray}$t$,-90][\color{gdarkgray}$\Phi(t)$,0]

 \psplot[linewidth=1.2pt,%
	     linecolor=darkblue,%
	     algebraic=true]%
	     {-2.5}{2.5}{%
2*(0.5 + 0.3989422804014327*x -% 
 0.06649038006690544*x^3 + %
 0.009973557010035819*x^5 - %
 0.0011873282154804543*x^7 + %
 0.00011543468761615529*x^9 -%
 0.000009444656259503615*x^11 + %
 0.0000006659693516316651*x^13 -% 
 0.00000004122667414862689*x^15)%
 }
\rput(2.1,2.2){\color{gdarkgray}$\Phi$}
\end{pspicture}
\caption{Gaußsche Glockenkurve und standardisierte Normalverteilung}
\end{figure}

Eine Verallgemeinerung davon ist die \emph{Normalverteilung (Gauß-Verteilung)
$N(a,\sigma^2)$}, mit $a\in\R$ und $\sigma > 0$. $N(a,\sigma^2)$ ist ein W-Maß
auf $\BB$ mit Verteilungsfunktion
\begin{align*}
F(x) \defl \frac{1}{\sqrt{2\pi}}
\int\limits^{x}_{-\infty}\e^{-\frac{(t-a)^{2}}{2 \sigma^{2}}}\dt,\qquad x \in
\R.
\end{align*}
Diese Funktion ergibt sich mittels Substitution aus der Verteilungsfunktion von
$N(0,1)$, alle Eigenschaften übertragen sich daher entsprechen.\bsphere
\end{bsp}
\begin{bsp}
Wähle $\lambda > 0$ fest, so ist eine Verteilungsfunktion gegeben durch,
\begin{align*}
F: \R\to\R,\quad F(x) =
\begin{cases}
0, & x\le 0\\
1-\e^{-\lambda x}, & x > 0.
\end{cases}
\end{align*}
\begin{figure}[!htbp]
\centering
\begin{pspicture}(-1.8,-0.8)(4.2,2.2)
 \psaxes[labels=none,ticks=none,linecolor=gdarkgray,tickcolor=gdarkgray]{->}%
 (0,0)(-1.5,-0.5)(4,2)[\color{gdarkgray}$x$,-90][\color{gdarkgray}$F(x)$,0]

\psline[linewidth=1.2pt,linecolor=darkblue](-1,0)(0,0)

\psline[linewidth=1.2pt,linestyle=dotted](0,1)(3.5,1)

\psplot[linewidth=1.2pt,%
	     linecolor=darkblue,%
	     algebraic=true]%
	     {0}{3.5}{1-(2.71828)^(-x)}
	     
\psyTick(1){\color{gdarkgray}1}

\end{pspicture}
\caption{Verteilungsfunktion der Exponentialverteilung.}
\end{figure}

Das zugehörige W-Maß heißt \emph{Exponentialverteilung} mit Parameter
$\lambda$ (kurz $\exp(\lambda)$).\bsphere
\end{bsp}

\begin{bem}
\label{bem:1.5}
Definition \ref{defn:1.9} und Satz \ref{prop:1.6} lassen sich auf $\BB_{n}$
und $\R^{n}$ als Definitionsbereiche von $P$ bzw. $F$ 
anstatt $\BB$ bzw. $\R$ verallgemeinern. Dabei ist die Verteilungsfunktion $F$
zu einem W-Maß $P$ auf $\BB_{n}$ gegeben durch
\begin{align*}
F(x_{1}, \ldots ,x_{n})= P(\{(u_{1}, \ldots , u_{n})\in \R^{n}\mid
u_{1}\leq x_{1},\ldots, u_{n} \leq x_{n}\} )
\end{align*}
 für $ (x_{1}, \ldots, x_{n})
\in \R^{n}$.\maphere
\end{bem}

\begin{figure}[!htpb]
\centering
\begin{pspicture}(0,-1.27)(8.5,2.15)
\psframe[linestyle=none,fillcolor=glightgray,fillstyle=solid](3.1,1.13)(0,-1.26)

\psline{->}(2.12,-1.26)(2.1,1.93)
\psline{->}(0.0,0.51)(4.04,0.53)

\psline[linecolor=darkblue](0,1.13)(3.1,1.13)(3.1,-1.26)

\rput(2.57,1.955){\color{gdarkgray}$\R^2$}

\rput(4.1,1.3){\color{gdarkgray}$x=(x_1,x_2)$}

\rput[l](3.2,-1.025){\color{gdarkgray}$\setdef{(u1,u2)\in\R^2}{u_1\le x_1,
u_2\le x_2}$}

\rput(1.88,1.3){\color{gdarkgray}$x_2$}

\rput(3.36,0.32){\color{gdarkgray}$x_1$}
\end{pspicture} 
\caption{Zur Verallgemeinerung der Verteilungsfunktion.}
\end{figure}

\clearpage
% =============================================================
%                                   Bedingte Wahrscheinlichkeiten
% =============================================================
\section{Bedingte Wahrscheinlichkeiten}
\label{chap:1.d}

Wenn wir beim Durchführen eines Zufallsexperiments wissen, dass bereits
Ereignis $A$ eingetreten ist, so ändert sich die Wahrscheinlichkeit dafür, dass
auch noch $B$ eintritt. Für diese Wahrscheinlichkeit schreiben wir
\begin{align*}
P(B\mid A).
\end{align*}
$P(B\mid A)$ ist die bedingte Wahrscheinlichkeit, dass unter der Voraussetzung,
dass $A$ eingetreten ist, nun $B$ eintritt.

Bedingte Wahrscheinlichkeiten spielen in vielen aktuellen Fragestellungen der
Wahrscheinlichkeitstheorie eine zentrale Rolle. Beispielsweise in der
Finanzmathematik, denn es ist offensichtlich, dass der nächste Aktienkurs
von allen vorherigen abhängt.


\begin{bsp}
Unter Studenten einer Vorlesung wird eine Umfrage gemacht.
\begin{figure}[H]
\begin{tabular}{l|c|c}
 & m & w\\\hline
 Sport & 12 & 18\\
 Kein Sport & 16 & 20
\end{tabular}
\caption{Ergebnis der Umfrage.}
\end{figure}
Wir nummerieren die Personen so, dass Nr. 1-12 die Sport-treibenden Frauen, 
13-30 die Sport-treibenden Männer, 31-46 die nicht Sport-treibenden Frauen und
47-66 die nicht Sport-treibenden Männer sind.

Nun modelieren wir einen W-Raum für die rein zufällige Auswahl einer Person,
\begin{align*}
\Omega = \setd{1,\ldots,66},\quad \AA = \PP(\Omega),\quad P(\setd{\omega}) =
\frac{1}{66}.
\end{align*}
Ereignis $A\defl\setd{1,\ldots,12,31,\ldots,46}$, die ausgewählte Persion ist
weiblich,\\
Ereignis $B\defl\setd{1,\ldots,30}$, die ausgewählte Person treibt Sport.

Es sei bekannt, dass die ausgewählte Person Sport treibt. 
\textit{Wie groß ist die Wahrscheinlichkeit, dass die ausgewählte Person eine
Frau ist?} Abzählen der Elemente in $\Omega$ ergibt,
\begin{align*}
\frac{12}{30} = \frac{2}{5}. 
\end{align*}
Andererseits erhalten wir durch
\begin{align*}
\frac{P(A\cap B)}{P(B)} = \frac{\frac{12}{66}}{\frac{30}{66}} = \frac{2}{5}
\end{align*}
dasselbe Ergebnis. Wir verwenden dies nun zur Definition der bedingten
Wahrscheinlichkeit.\bsphere
\end{bsp}

\begin{defn}
\label{defn:1.10}
Es sei $(\Omega, \AA, P)$ ein W-Raum, $A \in {\cal
A}$, $B \in \AA$ mit $P(B) >0$.
Dann heißt
\begin{align*}
P(A\mid B)\defl \frac{P(A\cap B)}{P(B)}
\end{align*}
die \emph{bedingte Wahrscheinlichkeit (conditional probability)} von $A$ unter
der Bedingung~$B$.\fishhere
\end{defn}

Es folgen einige elemenatre Eigenschaften der bedingten Wahrscheinlichkeit.
\begin{prop}
\label{prop:1.7}
Sei $(\Omega, \AA, P)$ ein W-Raum.
\begin{propenum}
\item
Bei festem $B \in \AA$ mit $P(B) > 0 $ ist $P(\cdot\mid B)$ ein W-Maß auf
$\AA$, das auf $B$ konzentriert ist [d.h. $P(B\mid B) =1$].
\item
Für $A,B \in \AA$ mit $P(A) > 0$, $P(B) > 0$ gilt
\begin{align*}
P(A\mid B)= P(B\mid A)\cdot \frac{P(A)}{P(B)}.
\end{align*}
\item Für $A_{n} \in \AA$ $(n=1, \ldots ,m)$ mit
  $P(\bigcap^{m-1}_{n=1}A_{n})>0$ gilt
\begin{align*}
P\left(\bigcap^{m}_{n=1}A_{n}\right) = &P(A_{1}) \cdot P(A_{2}\mid A_{1})\cdot
P(A_{3}\mid A_{1}\cap A_{2}) \cdot \\ & \ldots \cdot
P\left(A_{m}\mid \bigcap\limits^{m-1}_{n=1}A_{n}\right).\fishhere
\end{align*}
\end{propenum}
\end{prop}

\begin{proof}
\begin{proofenum}
  \item Der Beweis ist eine leichte Übungsaufgabe.
  \item Betrachte die rechte Seite,
\begin{align*}
P(B\mid A)\cdot \frac{P(A)}{P(B)} = \frac{P(A\cap B)}{P(A)}\frac{P(A)}{P(B)}
=\frac{P(A\cap B)}{P(B)} \defr P(A|B).
\end{align*}
\item Vollständige Induktion. Für $n=2$
\begin{align*}
P(A_1\cap A_2) = P(A_1)P(A_2|A_1).
\end{align*}
Der Induktionsschulss ist eine leichte Übung.\qedhere
\end{proofenum}
\end{proof}
 
\begin{prop}
\label{prop:1.8}
Sei $(\Omega, \AA, P)$ ein W-Raum, $\setdef{B_{n}}{n \in I}$ eine höchstens
abzählbare Familie paarweise disjunkter Ereignisse $B_{n} \in \AA$ mit
$P(B_{n}) > 0 \; (n \in I)$ und $\sum\limits_{n \in I} B_{n} = \Omega $. Dann
gilt für $A\in\AA$
\begin{propenum}
\item die \emph{Formel von der totalen Wahrscheinlichkeit}
\begin{align*}
P(A) = \sum\limits_{n \in I} P(B_{n})P(A\mid B_{n}).
\end{align*}
\item falls $P(A)>0$ die \emph{Formel von Bayes}
\begin{align*}
P(B_{k}\mid A)= {\displaystyle\frac{P(B_{k})\cdot P(A\mid B_{k})}
  {\sum\limits_{n\in I} P(B_{n})P(A\mid B_{n})}},\qquad k \in I.\fishhere
\end{align*}
\end{propenum}
\end{prop}
Mit der Formel von der totalen Wahrscheinlichkeit kann man somit mit bedingten
Wahrscheinlichkeiten auf die Wahrscheinlichkeit für ein bestimmtes
Ereignis rückschließen. Die Formel von Bayes hingegen erlaubt es die Bedingung
der Bedingten Wahrscheinlichkeit umzukehren.

\begin{proof}
\begin{proofenum}
  \item $A=A\cap \Omega = \sum\limits_{n\in I} (A\cap B_n)$. Weiterhin gilt
  aufgrund der $\sigma$-Additivität,
\begin{align*}
P(A) = \sum\limits_{n\in I} P(A\cap B_n) = \sum\limits_{n\in I}
P(B_n)P(A\mid B_n).
\end{align*}
\item Satz \ref{prop:1.7} besagt, $P(B_k) = P(A|B_k)\frac{P(B_k)}{P(A)}$. Die
Behauptung folgt nun unter Verwendung des Teils a).\qedhere
\end{proofenum}
\end{proof}

\begin{bsp}
\label{bsp:1.7}
\textit{Zur Veranschaulichung des Satzes \ref{prop:1.8}}. Betrachte einen Weg
von $0$ über $b_1$ oder $b_2$ oder $b_3$ oder \ldots nach $a$ oder
$\overline{a}$. [$a$ könnte z.B. für das Auftreten von Krebs,
$\overline{a}$ für das Nichtauftreten von Krebs, $b_1$ für Rauchen, $b_2$
für das Ausgesetzsein von giftigen Dämpfen, $b_3$\ \ldots stehen].

\begin{figure}[!htbp]
\centering
\begin{pspicture}(0,-1.77)(5.68,1.79)

\psbezier[linecolor=darkblue](0.2,0.61)(0.94,1.39)(1.48,1.53)(2.92,1.41)
\psbezier[linecolor=purple](0.22,0.61)(0.94,0.11)(1.78,-0.09)(2.88,0.19)
\psbezier[linecolor=yellow](0.2,0.61)(0.16,-0.91)(1.44,-1.09)(2.86,-1.03)
\psbezier[linecolor=darkblue](2.92,1.41)(3.58,1.75)(4.94,1.63)(5.32,1.09)
\psbezier[linecolor=darkblue](2.92,1.41)(2.88,0.23)(3.84,-0.19)(5.24,-0.09)
\psbezier[linecolor=purple](2.92,0.19)(3.32,-0.21)(4.24,-0.35)(5.22,-0.09)
\psbezier[linecolor=purple](2.9,0.19)(3.8,-0.07)(5.34,0.31)(5.34,1.11)
\psbezier[linecolor=yellow](2.86,-1.03)(4.0,-1.75)(5.0,-1.31)(5.24,-0.09)
\psbezier[linecolor=yellow](2.86,-1.03)(4.58,-1.13)(5.34,0.29)(5.34,1.09)

\psdots(0.22,0.61)
\psdots(2.92,1.43)
\psdots(2.9,0.19)
\psdots(2.86,-1.03)
\psdots(5.34,1.11)
\psdots(5.24,-0.09)

\rput(0.09,0.775){\color{gdarkgray}$0$}

\rput(5.56,1.215){\color{gdarkgray}$a$}
\rput(5.48,-0.165){\color{gdarkgray}$\overline{a}$}

\rput(3.17,1.275){\color{gdarkgray}$b_1$}
\rput(2.85,-0.065){\color{gdarkgray}$b_2$}
\rput(2.8,-1.265){\color{gdarkgray}$b_3$}

\rput(1.07,1.595){\color{darkblue}$P(B_1)$}
\rput(1.73,0.295){\color{purple}$P(B_2)$}
\rput(1.09,-1.125){\color{yellow}$P(B_3)$}
\end{pspicture} 
\caption{Wegediagramm.}
\end{figure}

An jeder Verzweigung $b_1$, $b_2$, $b_3$, \ldots wird ein Zufallsexperiment
durchgeführt. Das Ereignis $B_k$ sei definiert durch, dass der Weg über $b_k$
führt.

Die Formel von der totale Wahrscheinlichkeit ergibt,
\begin{align*}
P(A) = \sum_n P(A|B_n) P(B_n).
\end{align*}
\textit{Deutung}. Betrachte $B_1,B_2,\ldots$ als ``Ursachen'' und $A$ als
``Wirkung''.
\begin{align*}
P(B_k)\text{ bezeichnet man als sog. ``\emph{a priori} Wahrsch.'' von }B_k.
\end{align*}
Nun stehe $A$ durch Erfahrung zur Verfügung.
\begin{align*}
P(B_k|A)\text{ bezeichnet man als sog. ``\emph{a posteriori} Wahrsch.'' von
}B_k.
\end{align*}
$P(B_k|A)$ ist die Wahrscheinlichkeit, dass der Weg über $b_k$ verlaufen ist,
wobei bereits bekannt sei, dass die Ankunft in $a$ (= Ereignis $A$)
stattgefunden hat.

Die Formel von Bayes erlaubt also den Rückschluss von einer Wirkung auf ihre
Ursache - zumindest in einem Wahrscheinlichkeitstheoretischen Sinne.

\textit{Anwendung}. Ein Arzt beobachtet das Symptom $a$, welches ausschließlich
die Ursachen $b_1$, $b_2$, \ldots haben kann. Gesucht ist $P(B_k|A)$, die
Wahrscheinlichkeit für die jeweilige Ursache unter Berücksichtigung des
Eintretens von $a$.\bsphere
\end{bsp}
\begin{bsp}
\textit{Betrieb mit einer Alarmanalge}. Bekannt seien folgende Daten:
\begin{itemize}[label=-]
  \item Bei Einbruch erfolgt Alarm mit Wahrscheinlichkeit 0.99,
  \item bei Nichteinbruch mit Wahrscheinlichkeit 0.005.
  \item Die Einbruchswahrscheinlichkeit ist 0.001. 
\end{itemize}
 
Gesucht ist nun die Wahrscheinlichkeit, dass bei einem Alarm auch tatsächlich
ein Einbruch stattfindet. Übertragen wir dies auf unser Modell, so erhalten
wir folgendes Schema:
\begin{figure}[H]
\begin{tabular}{l|l}
\textbf{Ereignis} & \textbf{Beschreibung}\\\hline
$E$ & Einbruch\\
$\e^c$ & kein Einbruch\\
$A$ & Alarm\\
$A^c$ & kein Alarm
\end{tabular}
\end{figure}
Die Wahrscheinlichkeiten der einzelnen Ereignisse sind,
\begin{align*}
P(A|E) = 0.99,\quad P(A|\e^c) = 0.005,\quad P(E) = 0.001.
\end{align*}

Gesucht ist $P(E|A)$. Mit dem Satz von Bayes erhalten wir,
\begin{align*}
P(E|A) &= \frac{P(A|E)P(E)}{P(A|E)P(E)+P(A|\e^c)P(\e^c)}
= \frac{0.99\cdot 0.001}{0.99\cdot 0.001+0.005\cdot 0.999}
\\ 
&= \frac{22}{133} \approx 0.165.
\end{align*}
D.h. die Wahrscheinlichkeit, dass bei Auslösung eines Alarms,
auch tatsächlich ein Einbruch stattfindet, beträgt lediglich 16.5\%.

Die Wahrscheinlichkeit für einen Alarm ist nach der Formel von der totalen
Wahrscheinlichkeit
\begin{align*}
P(A) = P(A|E)P(E)+P(A|\e^c)P(\e^c) = 0.006.\bsphere
\end{align*}
\end{bsp}

\cleardoublepage
\chapter{Kombinatorische Wahrscheinlichkeitsrechnung}

In der kombinatorischen Wahrscheinlichkeitsrechnung legen wir Laplacesche
W-Räume zugrunde. Die Abzählung der ``günstigen'' und der ``möglichen'' Fälle
erfolgt systematisch mit Hilfe der Kombinatorik.

\begin{defn}
\label{defn:2.1} Gegeben seien $n$ --- nicht notwendiger Weise verschiedene ---
Elemente $a_{1}, \ldots, a_{n}$. Das $n$-Tupel $({\displaystyle a_{i_{1}},
\ldots , a_{i_{n}}})$ mit $i_{j} \in \setd{1, \ldots, n}$, $i_{j} \neq i_{k}$ für
$j \neq k$ $(j,k=1, \ldots , n)$ heißt eine \emph{Permutation} der gegebenen
Elemente.\fishhere
\end{defn}

In der Kombinatorik spielt die mögliche Anzahl an Permutationen eine sehr große
Rolle.

\begin{prop}
\label{prop:2.1}
Die Anzahl der Permutationen von $n$ \textit{verschiedenen} Elementen
ist $n!$.
\end{prop}

Eine Verallgemeinerung für nicht notwendiger weise verschiedene Elemente
liefert der folgende
\begin{prop}
\label{prop:2.2}
Gegeben seien $n$ Elemente; die verschiedenen Elemente darunter
seien mit $a_{1}, \ldots, a_{p}$ bezeichnet $($mit $p \leq n)$. Tritt $a_{i}$
$n_{i}$-fach auf $(i=1, \ldots, p)$, wobei $\sum^{p}_{i=1} n_{i}=n$,
dann können die $n$ Elemente auf
\begin{align*}
\frac{n!}{n_{1}! \ldots n_{p}!}
\end{align*}
verschiedene Arten permutiert werden.\fishhere
\end{prop}

\begin{defn}
\label{defn:2.2}
Sei $A$ eine $n$-elementige Menge, $k \in \mathbb{N}$.
\begin{defnenum}
\item[1.)]
[2.)] Jedes $k$-Tupel $(a_{1}, \ldots, a_{k})$ mit nicht notwendig verschiedenen
[lauter verschiedenen] $a_{i} \in A$ heißt eine \emph{Kombination
$k$-ter Ordnung aus $A$ mit [ohne] Wiederholung und mit
Berücksichtigung der Anordnung}.
\item[3.)]
[4.)] Werden in a) [2.)] Kombinationen, welche dieselben Elemente in
verschiedener
Anordnung enthalten, als äquivalent aufgefasst, so heißen die einzelnen
Äquivalenz\-klassen \emph{Kombinationen $k$-ter Ordnung aus $A$ mit [ohne]
Wiederholung und ohne Berück\-sichtigung der Anordnung}.\fishhere
\end{defnenum}
\end{defn}

\textit{Interpretation} (mit $A= \{1,\ldots ,n\}$): Aufteilung von $k$
Kugeln auf $n$ Zellen, wobei in 1.) [2.)] die Zellen mehrfach [nicht mehrfach]
besetzt werden dürfen und die Kugeln unterscheidbar sind ($a_{i}$ \ldots Nummer der Zelle, in
der die $i$-te Kugel liegt), in 3.) [4.)] die Zellen mehrfach [nicht mehrfach]
besetzt werden dürfen und die Kugeln nicht unterscheidbar sind (ein Repräsentant
der Äquivalenzklasse ist gegeben durch ein $k$-tupel von Zahlen aus
$\{1, \ldots , n\}$, in dem die Zellennummern so oft auftreten, wie die
zugehörige Zelle besetzt ist).\\

\begin{prop}
\label{prop:2.3}
Die Anzahl der Kombinationen $k$-ter Ordnung aus der
$n$-elementigen Menge $A$ -- mit [ohne] Wiederholung und mit [ohne]
Berücksichtigung der Anordnung -- ist gegeben durch

\vspace{0.2cm}
\begin{tabular}{l|l|l}
 & m.\ Wiederholung & o.\ Wiederholung $(1 \leq k \leq n )$ \\ \hline
 m.\ Ber. der Anordnung & $n^{k}$ & $n(n-1) \ldots (n-k+1)$ \\
 \hline
 o.\ Ber. der Anordnung & ${n+k-1 \choose k}$& ${n \choose k}$ \\
\hline
\end{tabular}

\vspace{0.5cm}
Bestimmung von Fakultäten durch Tabellen für $\log n!$  und -- bei großem $n$ --
durch die \emph{Stirlingsche Formel}
\begin{align*}
 n! \cong \left(\frac{n}{e}\right)^{n} \sqrt{2 \pi n } \quad (n \to \infty )
\end{align*}
und ihre Verschärfung
\begin{align*}
\exp \frac{1}{12 n+1}<{\displaystyle\frac{n!}{(\frac{n}{e})^{n}\sqrt{2\pi
n}}} < \exp \frac{1}{12 n}\, , \quad n \in \mathbb{N} \, .\fishhere
\end{align*}
\end{prop}
\begin{proof}
Wir beweisen exemplarisch die Formel für die Anzahl der Kombinationen ohne
Berücksichtigung der Anordnung und mit Wiederholungen. Betrachte dazu $k$
Kugeln, die auf $n$ Zellen verteilt werden sollen.

\begin{figure}[H]
\centering
\begin{pspicture}(0,-0.23)(2.42,0.23)
\psline(0.0,0.21)(0.0,-0.21)
\psline(0.62,0.19)(0.62,-0.21)
\psline(1.22,0.19)(1.22,-0.21)
\psline(1.8,0.19)(1.8,-0.21)
\psline(2.4,0.19)(2.4,-0.21)
\psdots[linecolor=darkblue](0.2,0.11)
\psdots[linecolor=darkblue](0.36,-0.13)
\psdots[linecolor=darkblue](1.36,0.05)
\psdots[linecolor=darkblue](1.56,-0.13)
\psdots[linecolor=darkblue](1.64,0.11)
\end{pspicture} 
\caption{Zur Platzierung von $k$ Kugeln in $n$ Zellen}
\end{figure}

Ohne die äußeren Trennwende gibt es also $k$ Kugeln und $n-1$ Trennwände, d.h.
es gibt insgesamt $(n-1+k)!$ Permutationen. Berücksichtigen wir nun noch die
Ununterscheidbarkeit der Kugeln, so erhalten wir
\begin{align*}
\binom{n+k-1}{k}.\qedhere
\end{align*}
\end{proof}

\begin{bspn}
\textit{Anwendung in der Physik}.
Man beschreibt das makroskopische Verhalten von
$k$ Teilchen in der Weise, dass man den (der Darstellung des Momentanzustandes
dienenden) Phasenraum in $n$ kongruente würfelförmige Zellen zerlegt und die
Wahrscheinlichkeit \mbox{$p(k_{1}, \ldots , k_{n})$} bestimmt, dass sich genau
$k_{i}$
der Teilchen in der $i$-ten Zelle befinden $(i \in \{ 1, \ldots, n\})$. Sei
$\Omega '$ die Menge aller $n$-Tupel $(k_{1}, \ldots, k_{n}) \in \mathbb{N}_{0}^{n}$ von
Besetzungszahlen mit $\sum\limits^{n}_{i=1}k_{i} = k$. W-Maße auf
$\PP(\Omega')$ (charakterisiert durch $p$):
\begin{bspenum}
\item \emph{Maxwell-Boltzmann-Statistik} (Unterscheidbarkeit der
Teilchen, Mehrfachbesetzbarkeit von Zellen)
\begin{align*}
p(k_{1}, \ldots,k_{n} ) = {\displaystyle\frac{k!}{k_{1}! \ldots k_{n}!} \, \,
\frac{1}{n^{k}}}.
\end{align*}
\item
\emph{Bose-Einstein-Statistik} zur Beschreibung des Verhaltens von Photonen
(keine Unterscheidbarkeit der Teilchen, jedoch Mehrfachbesetzbarkeit von Zellen)
\begin{align*}
p(k_{1}, \ldots, k_{n}) = {n+k-1 \choose k} ^{-1}. 
\end{align*}
\item
\emph{Fermi-Dirac-Statistik} zur Beschreibung des Verhaltens von Elektronen,
Protonen und Neutronen (keine Unterscheidbarkeit der Teilchen, keine
Mehrfachbesetzbarkeit von Zellen )
\begin{align*}
p(k_{1}, \ldots, k_{n})=
\begin{cases}
{n \choose k}^{-1}, &  k_{i} \in \{0,1 \} \\
0, &  \mbox{sonst.}\bsphere
\end{cases}
\end{align*}
\end{bspenum}
\end{bspn}
\begin{bspn}
Die \textit{Binomialverteilung} $b(n,p)$ mit Parametern $n \in \mathbb{N}$, $p
\in (0,1)$ ist ein W-Maß auf ${\cal B}$ mit der Zähldichte
\begin{align*}
k\to
\begin{cases}
{n \choose k} \, p^{k} (1-p)^{n-k}  , & k=0,1,\ldots, n \\
0 ,& k = n+1, n+2, \ldots \, ;
\end{cases}
\end{align*}
durch diese wird für $n$ ``unabhängige'' Versuche mit jeweiliger
Erfolgswahrscheinlichkeit $p$ die Wahrscheinlichkeit von $k$ Erfolgen
angegeben.\bsphere
\end{bspn}


\cleardoublepage
\chapter{Zufallsvariablen und Verteilungen}

\section{Meßbare Abbildungen und Zufallsvariablen}

Wir betrachten folgende Zufallsexperimente:
\begin{defnenum}
  \item Zufällige Anzahl der Erfolge bei $n$ Bernoulli-Versuchen.
  \item Zufällige Anzahl der an einem Schalter in einem bestimmten
  Zeitintervall $[0,T]$ eintreffenden Kunden.
  \item  Zufällige Wartezeit (von $0$ an gemessen) bis zum Eintreffen des
  nächsten Kunden.
  \item Zufälliger Abstand eines aus dem Intervall $[-1/2,1/2]$
  herausgegriffenen Punktes von $0$.
  \item Zufällige Lage eines Partikels im $\R^3$ zu einem bestimmten Zeitpunkt.
\end{defnenum}

Die zufälligen Größen können durch sogenannte ``Zufallsvariablen''
modelliert werden.

\begin{bsp}
\label{bsp:3.1}
Zu 4.): Betrachte den W-Raum $(\Omega,\AA,P)$ mit $\Omega = [-1/2,1/2]$,
$\AA=\BB\cap\Omega$ und $P=\lambda\big|_{\AA}$ (Restriktion
des LB-Maßes auf Mengen aus $\AA$). Ein Punkt $\omega\in\Omega$ hat Abstand
$Y(\omega)\defl\abs{\omega}$ von Null. $Y$ nennen wir Zufallsvariable.

Die Wahrscheinlichkeit, dass $Y\le x$ mit $x\in[0,1/2]$ fest, ist gegeben durch
\begin{align*}
P[Y\le x] \defl P\left(\setdef{\omega\in \Omega}{Y(\omega)\le x} \right)
= [-x,x]\in \AA.
\end{align*}
Allgemeiner ist die Wahrscheinlichkeit, dass $Y\in B$ für $B\in\BB$, gegeben
durch
\begin{align*}
P[Y\in B] \defl P\left(\setdef{\omega\in\Omega}{Y(\omega)\in B}\right).
\end{align*}
Damit $P[Y\in B]$ überhaupt definiert ist, muss $[Y\in B]$ eine messbare Menge
sein, d.h. im Definitionsbereich von $P$ liegen. Im Folgenden
erarbeiten wir die notwendigen Vorraussetzungen, dass dem so ist.\bsphere
\end{bsp}

\begin{defn}
\label{defn:3.1}
Seien $\Omega$, $\Omega '$ zwei nichtleere Mengen, $A' \subset
\Omega '$ und $X: \Omega \to \Omega '$ eine Abbildung. Die Menge
\begin{align*}
\setdef{\omega \in \Omega}{X(\omega ) \in A'} \defr X^{-1} (A')\defr [X \in
A']
\end{align*}
(in $\Omega $) heißt das \emph{Urbild} von $A'$ bezüglich der Abbildung $X$; 
die somit
definierte Abbildung $X^{-1}:{\cal P} (\Omega ') \to {\cal P} (\Omega )$ heißt
\emph{Urbildfunktion} (zu unterscheiden von einer inversen Funktion!).\fishhere
\end{defn}

In vielen Fällen untersuchen wir nicht nur einzelne Mengen sondern ganze
Mengensysteme und verwenden daher folgende Bezeichnung als Abkürzung.

\begin{bemn}[Bezeichnung.]
Sei $X: \Omega \to \Omega '$ und ${\cal C}'$ Mengensystem in
$\Omega '$. So ist
\begin{align*}
X^{-1}({\cal C}') \defl \setdef{ X^{-1} (A ')}{ A' \in {\cal C}'},
\end{align*}
wobei $X^{-1}({\CC}')$ ist ein Mengensystem in $\Omega$ ist.\maphere 
\end{bemn}

Um zu klären, unter welchen Voraussetzungen $[X\in B]$ messbar ist, wenn $B$
messbar ist, müssen wir untersuchen, wie sich die Urbildfunktion $X^{-1}$ auf
$\sigma$-Algebren im Bildraum auswirkt.

\begin{prop}
\label{prop:3.1}
Sei $X:\Omega \to \Omega' $.
\begin{propenum}
\item\label{prop:3.1:1}
$X^{-1}$ und Mengenoperationen $\cup$, $\sum$, $\cap$, $^{c}$, $\backslash$
sind vertauschbar.\\
z.B. ist $X^{-1}\left(\bigcup\limits_{\alpha \in I} A'_{\alpha}\right) =
\bigcup\limits_{\alpha \in I} X^{-1}(A'_{\alpha})\qquad$ für $A'_{\alpha}\subset
\Omega ', \alpha \in $ Indexbereich $I$.
\item\label{prop:3.1:2}
$X^{-1}(\emptyset) = \emptyset; \quad X^{-1}(\Omega ') = \Omega$.\\ 
$A' \subset B'\subset \Omega ' \Longrightarrow X^{-1} (A') \subset X^{-1}(B')$.
\item\label{prop:3.1:3}
Ist $\AA'$ $\sigma$-Algebra in $\Omega '$, so ist $X^{-1}(\AA') $
$\sigma$-Algebra in $\Omega$.
\item\label{prop:3.1:4}
${\CC}'\subset {\cal P}(\Omega') \Longrightarrow X^{-1} ({\cal F}_{\Omega' }
({\CC}')) = {\cal F}_{\Omega}(X^{-1}({\CC}'))$.\fishhere
\end{propenum}
\end{prop}
\begin{proof}
\ref{prop:3.1:1}-\ref{prop:3.1:4}: Der Beweis sei als Übungsaufgabe
überlassen.\qedhere
\end{proof}

Sei $X:(\Omega,\AA)\to (\Omega',\AA')$, so ist $X^{-1}(\AA')$ stets eine
$\sigma$-Algebra. Damit die Urbilder messbarer Mengen auch tatsächlich messbar
sind, müssen wir also fordern, dass $X^{-1}(\AA')\subseteq \AA$.

\begin{defn}
\label{3.2}
Seien $(\Omega ,\AA)$, $(\Omega ', \AA')$ Messräume.
Die Abbildung $X:\Omega \to \Omega '$ heißt $\AA$-${\cal A'}$-messbar
[kurz: \emph{messbar}; measurable], wenn gilt:
\begin{align*}
\forall {A' \in \AA'} : X^{-1}(A') \in \AA,
\quad \text{d.h. } X^{-1}(\AA') \subseteq \AA,
\end{align*}
d.h. Urbilder von messbaren Mengen in $\Omega '$ sind messbare Mengen in
$\Omega $.\\
 In diesem Falle verwenden wir die Schreibweise $X$: $(\Omega, \AA) \to (\Omega
 ', \AA')$.\fishhere
\end{defn}

Wie wir noch sehen werden, ist die Messbarkeit einer Abbildung eine
\textit{wesentlich} schwächere Voraussetzung als beispielsweise Stetigkeit oder
gar Differenzierbarkeit. Es erfordert sogar einiges an Aufwand, eine nicht
messbare Abbildung zu konstruieren.

\begin{bem}
\label{bem:3.1}
In Satz \ref{prop:3.1} \ref{prop:3.1:3} ist $X^{-1}(\AA')$ die kleinste der
$\sigma$-Algebren $\AA$ in $\Omega$ mit $\AA$-$\AA'$-Messbarkeit
von $X$.\maphere
\end{bem}

Wir fassen nun alle nötigen Vorraussetzungen in der folgenden Definition
zusammen.

\begin{defn}
\label{defn:3.3}
Sei $(\Omega, \AA, P)$ ein W-Raum, $(\Omega ', {\cal
A}')$ ein Messraum. Die Abbildung
\begin{align*}
X:(\Omega , \AA) \to (\Omega ', \AA')
\end{align*}
heißt \emph{Zufallsvariable $(ZV)$} auf $(\Omega, \AA,P)$
[mit Zustandsraum $\Omega '$] (ausführlich: $(\Omega ', \AA')$-Zufallsvariable auf
$(\Omega, \AA, P)$; random variable).

\begin{defnenum}
  \item $X: (\Omega,\AA) \to (\R, {\cal B})$ heißt \emph{reelle Zufallsvariable} 
(oft auch nur ZV).
\item $X: (\Omega, \AA) \to (\overline{\R}, \overline{\cal B})$ heißt 
\emph{ erweitert-reelle Zufallsvariable}.
\item $X: (\Omega, \AA) \to (\R^{n}, {\cal B}_{n})$ heißt 
\emph{$n$-dimensionaler Zufallsvektor}.
\item $X(\omega)$ für ein $\omega \in \Omega $ heißt eine \emph{Realisierung}
der Zufallsvariable $X$.   
\end{defnenum}
Bezeichnung. $\overline{\BB}\defl \setdef{B, B \cup \{+ \infty \}, B \cup \{
-\infty \},  B\cup \{- \infty , + \infty \}}{B \in {\BB}}$.\fishhere
\end{defn}

Zufallsvariablen ermöglichen es uns für ein Experiment lediglich einmal einen
W-Raum $(\Omega,\AA,P)$ zu modellieren und dann für jeden Aspekt, der uns
interessiert, eine Zufallsvariable zu konstruieren. $X(\omega)$ kann man als
Messung interpretieren, $X(\omega)$ ist der von $\omega$ abhängige Messwert.

Außerdem lassen sich Zufallsvariablen verknüpfen, wir können sie addieren,
multiplizieren, hintereinanderausführen, \ldots

Messbarkeit für alle Elemente der Bild-$\sigma$-Algebra nachzuweisen, kann sich
als äußert delikat herausstellen. Der folgende Satz besagt jedoch, dass
es genügt sich auf ein Erzeugersystem der Bild-$\sigma$-Algebra
zurückzuziehen.

\begin{prop}
\label{prop:3.2}
Seien $(\Omega, \AA)$, $(\Omega ', {\cal A '})$ Messräume, $ X:
\Omega \to \Omega '$ und ${\EE}'$ Erzeugersystem  von $\AA'$.
Dann ist $X$ genau dann messbar, wenn
\begin{align*}
\forall M\in {\EE}' : X^{-1}(M)\in \AA,\qquad
\text{d.h. } X^{-1}({\EE}') \subseteq \AA.\fishhere
\end{align*}
\end{prop}
\begin{proof}
$\Rightarrow$: Gilt \textit{per definitionem}.\\
$\Leftarrow$: Sei $X^{-1}(\EE')\subseteq \AA $, zu zeigen ist nun, dass
$X^{-1}(\AA')\subseteq \AA$. Es gilt
\begin{align*}
X^{-1}(\AA') = X^{-1}(\FF_{\Omega'}(\EE')) = \FF_\Omega(X^{-1}(\EE')) \subseteq
\AA,
\end{align*}
denn $X^{-1}(\EE')\subseteq \AA$ und $\FF_\Omega(X^{-1}(\EE'))$ ist die
kleinste $\sigma$-Algebra, die $X^{-1}(\EE')$ enthält und $\AA$ ist
$\sigma$-Algebra.\qedhere
\end{proof}
 
Ein einfaches aber äußerst nützliches Korollar ergibt sich, wenn wir uns auf
reellwertige messbare Abbildungen beschränken.
\begin{prop}[Korollar]
\label{prop:3.3}
Sei $(\Omega, \AA)$ ein Messraum und $X: \Omega \to \overline{\R}$
Abbildung. Dann sind folgende Aussagen äquivalent:
\begin{equivenum}
  \item $X$ $\AA$-$\overline{\BB}$-messbar.
  \item $\forall \alpha \in \R : [X \leq \alpha] \in \AA$.
  \item $\forall \alpha \in \R : [X < \alpha] \in \AA$.
  \item $\forall \alpha \in \R : [X \ge \alpha] \in \AA$.\fishhere
\end{equivenum}
\end{prop}
\begin{proof}
Wird geführt mit Satz \ref{prop:3.2} und der Tatsache, dass
$\setdef{(-\infty,\alpha]}{\alpha\in\R}$ ein Erzeugersystem von
$\overline{\BB}$ ist.\qedhere
\end{proof}

Insbesondere sind somit die Mengen $[X\le \alpha]$, $[X<\alpha]$, \ldots\ für
jede reelle Zufallsvariable $X$ messbar.

\begin{cor}
\label{cor:3.1}
Für zwei Abbildungen $X,Y : (\Omega, \AA)\to
(\overline{\R}, \overline{\BB})$ gilt
\begin{align*}
[X< Y], \quad 
[X\leq Y], \quad 
[X = Y], \quad 
[X \neq Y] \in \AA,
\end{align*}
wobei $[X<Y]\defl \setdef{\omega \in \Omega}{X(\omega) < Y(\omega )}$.\fishhere
\end{cor}
\begin{proof}
Nach dem Prinzip von Archimedes gilt
$[X<Y]=\bigcup_{\alpha\in\R} [X<\alpha < Y]$. Außerdem ist
\begin{align*}
\bigcup_{\alpha\in\R} [X<\alpha < Y] = \bigcup_{\alpha\in\Q} [X<\alpha <
Y] = \bigcup_{\alpha\in\Q} [X<\alpha] \cap [Y\le \alpha]^c.
\end{align*}
Für jedes $\alpha\in\Q$ ist $[X<\alpha] \cap [Y\le \alpha]^c$ messbar und die abzählbare
Vereinigung messbarer Mengen ist messbar.

Die übrigen Fälle folgen sofort, denn
\begin{align*}
&[X\le Y] = [X>Y]^c \in \AA,\\
&[X=Y] = [X\le Y]\cap [X\ge Y]\in \AA\\
&[X\neq Y] = [X=Y]^c \in \AA.\qedhere
\end{align*}
\end{proof}

\begin{prop}
\label{prop:3.4}
Sei $\varnothing \neq A \subset \R^{m}$ und $A \cap {\BB}_{m} \defl
\setdef{A \cap B}{B\in {\BB}_{m}}$. Jede stetige Abbildung $g: A \to
\R^{n}$ ist $A \cap {\BB}_{m}$-${\BB}_{n}$-messbar.\fishhere
\end{prop}
\begin{proof}
$\OO_n$, das System der offenen Mengen auf $\R^n$, ist ein
Erzeugersystem von $\BB_n$. Nach \ref{prop:3.2} genügt es  zu
zeigen, dass
\begin{align*}
g^{-1}(\OO_n) \subseteq A\cap\BB_m.
\end{align*} 
$g$ ist stetig, daher sind die Urbilder offener Mengen offene Mengen
$\cap\; A$.\qedhere
\end{proof}

Für Abbildungen $\R^m\opento\R^n$ ist somit Messbarkeit eine \textit{wesentlich}
schwächere Voraussetzung als Stetigkeit.
\begin{bsp}
Die \emph{Dirichletfunktion} $f: \R\to\R$ gegeben durch,
\begin{align*}
f(x) = 
\begin{cases}
1, & x\in\Q,\\
0, & x\in\R\setminus\Q,
\end{cases}
\end{align*}
ist messbar aber sicher \textit{nicht} stetig.\bsphere
\end{bsp}

\begin{prop}
\label{prop:3.5}
Seien $X: (\Omega_{1}, \AA_{1}) \to (\Omega _{2}, \AA_{2})$,
$Y:(\Omega _{2}, \AA_{2}) \to (\Omega _{3}, \AA_{3})$.
Die zusammengesetzte Abbildung $Y \circ X: \Omega _{1} \to \Omega _{3}$ ist dann
$\AA_{1}$-$\AA_{3}$-messbar.\fishhere
\end{prop}
\begin{proof}
Offensichtlich gilt $(Y\circ X)^{-1}(\AA_3) = X^{-1}\circ Y^{-1}(\AA_3) 
\subseteq \AA_1$.\qedhere
\end{proof}

\begin{cor}
\label{cor:3.2}
Für $X: (\Omega , \AA)\to (\R^{m}, {\BB}_{m})$ und
eine stetige Abbildung $g: \R^{m} \to \R^{n}$ ist $g \circ X: \Omega \to \R^{n}$
 $\AA$-${\BB}_{n}$-messbar.\fishhere
\end{cor}

Insbesondere ist die Komposition von Zufallsvariablen bzw. von
Zufallsvariablen mit stetigen Funktionen wieder eine Zufallsvariable.  Für eine
reelle Zufallsvariable $X$ sind somit auch $\abs{X}$, $X^2$, $\sqrt{\abs{X}}$,
\ldots\ Zufallsvariablen.

\begin{prop}
\label{prop:3.6}
Sei $(\Omega , \AA)$ Messraum.
\begin{propenum}
\item
  Seien $X_{n} :(\Omega , \AA) \to (\R , {\BB})$ für $n = 1, \ldots, m$. Dann
  ist
\begin{align*}
 Y: \Omega \to \R^{m},\; \omega\mapsto Y(\omega)\defl (X_{1} (\omega )),
  \ldots , X_{m}(\omega )),\qquad \omega\in\Omega,
\end{align*}
$\AA$-${\cal B}_{m}$-messbar. Für $g: (\R^{m},
{\BB}_{m}) \to (\R, {\BB})$ ist
\begin{align*}
g \circ Y: \Omega \to \R
\end{align*}
$\AA$-${\BB}$-messbar.
\item
Seien $X_{1,2}: (\Omega, \AA) \to (\R, {\BB})$.
Dann sind auch die Abbildungen
\begin{defnpropenum}
  \item $\alpha X_{1} + \beta X_{2} \quad (\alpha , \beta \in \R)'$,
  \item $X_{1}X_{2}$,
  \item $\dfrac{X_{1}}{X_{2}}$ (falls existent)   
\end{defnpropenum}
$\AA$-$\BB$-messbar.\fishhere
\end{propenum}
\end{prop}
\begin{proof}
\begin{proofenum}
\item Ein Erzeugersystem $\CC$ von $\BB_m$ ist gegeben durch
\begin{align*}
(-\infty,\alpha_1]\times \ldots \times (-\infty,\alpha_m],
\end{align*}
wobei $(\alpha_1,\ldots,\alpha_m)\in\R^m$.
\begin{figure}[!htbp]
\centering
\begin{pspicture}(0,-2.13)(6,2.15)
\psframe[linestyle=none,fillcolor=glightgray,fillstyle=solid](3.1,1.13)(0,-2.13)

\psline{->}(2.12,-2.13)(2.1,1.93)
\psline{->}(0.0,0.51)(4.04,0.53)

\psline[linecolor=darkblue](0,1.13)(3.1,1.13)(3.1,-2.13)

\rput(2.57,1.955){\color{gdarkgray}$\R^2$}

\rput(4.1,1.3){\color{gdarkgray}$(\alpha_1,\alpha_2)$}

\rput(1.88,1.3){\color{gdarkgray}$\alpha_2$}

\rput(3.36,0.32){\color{gdarkgray}$\alpha_1$}
\end{pspicture}
\caption{Erzeuger der $\BB_n$.}
\end{figure}
Es genügt zu zeigen, dass $Y^{-1}(\CC)\subseteq \AA$.
Nun gilt
\begin{align*}
&Y^{-1}\left((-\infty,\alpha_1]\times \ldots \times (-\infty,\alpha_m]
\right)\\
&\qquad = \setdef{\omega\in \Omega}{x_1(\omega) \le \alpha_1, \ldots,
X_m(\omega) \le \alpha_m} \\
&\qquad = [X_1\le \alpha_1]\cap \ldots \cap [X_m\le
\alpha_m]\in \AA.\qedhere
\end{align*}
\item Die Messbarkeit folgt mit Korollar \ref{cor:3.2} und der Stetigkeit von
Summen- und Produktabbildung $(x,y)\mapsto x+y$, $(x,y)\mapsto x\cdot
y$.\qedhere
\end{proofenum}
\end{proof}

Eine Abbildung $X:\Omega\to \R^n$ ist also genau dann messbar, wenn jede
Komponente $X_n:\Omega\to\R$ messbar ist. Insbesondere sind Produkte und
Summen von Zufallsvariablen stets messbar.

\begin{prop}
\label{prop:3.7}
Sei  $(\Omega, \AA)$ Messraum, $X_{n} : (\Omega, \AA) \to
(\overline{\R}, \overline{\BB})$ für $n= 1,2, \ldots$.
Dann sind
\begin{defnpropenum}
  \item $\inf\limits_{n} X_{n}$,
  \item $\sup\limits_{n} X_{n}$,
  \item $\limsup\limits_{n} X_{n}$,
  \item $\liminf \limits_{n} X_{n}$,
  \item $\lim\limits_{n} X_{n}$ (falls existent)
\end{defnpropenum}
$\AA$-$\overline{\BB}$-messbar.\fishhere
\end{prop}
\begin{proof}
\begin{proofenumarabicbr}
  \item $\forall\alpha\in\R : [\inf X_n < \alpha] = \bigcup_{n\in\N} [X_n <
  \alpha] \in \AA$, d.h. $\inf X_n$ ist $\AA$-$\overline{\BB}$-messbar.
  \item $\sup=-\inf(-X_n)$ ist $\AA$-$\overline{\BB}$-messbar.
  \item $\limsup X_n = \inf_n \sup_{k\ge n} X_k$ 
 ist $\AA$-$\overline{\BB}$-messbar.
 \item $\liminf X_n = \sup_n \inf_{k\ge n} X_k$ ist
 $\AA$-$\overline{\BB}$-messbar.
 \item $\lim X_n = \limsup X_n$, falls $\lim_n X_n$ existiert.\qedhere
\end{proofenumarabicbr}
\end{proof}

Da Messbarkeit eine so schwache Voraussetzung ist, ist sie auch ein sehr
stabiler Begriff, da sie auch unter Grenzwertbildung und Komposition erhalten
bleibt. Probleme treten erst bei überabzählbar vielen Operationen auf.

\clearpage
\section{Bildmaße und Verteilungen}

In diesem Abschnitt werden wir eine Beziehung zwischen
Zufallsvariablen und Maßen herstellen. Jeder Zufallsvariablen
\begin{align*}
X:(\Omega,\AA,P)\to (\Omega',\AA')
\end{align*}
lässt sich eindeutig ein Maß
\begin{align*}
P_X : \AA' \to \R,\quad A'\mapsto P_X(A) \defl P(X^{-1}(A)) 
\end{align*}
das sogenannte \emph{Bildmaß} zuordnen. Die Verteilungsfunktion des
Bildmaßes
\begin{align*}
F(t) = P_X((-\infty,t])
\end{align*}
nennen wir auch \emph{Verteilungsfunktion von $X$}. $F$ lässt sich oft leichter
handhaben als $X$, und aus den Eigenschaften von $F$ lassen sich Rückschlüsse
auf $X$ machen.

Analog lässt sich zu jedem Maß
\begin{align*}
\mu:\AA'\to\R
\end{align*}
über $\Omega'$ eine Zufallsvariable
\begin{align*}
Y: (\Omega,\AA,Q)\to (\Omega',\AA')
\end{align*}
finden, so dass $P_Y=\mu$. 

\clearpage

Zunächst ist natürlich zu klären, ob das Bildmaß überhaupt ein Maß ist.

\begin{prop}
\label{prop:3.8}
Seien $(\Omega, \AA)$ und $ (\Omega ', \AA')$ Messräume und
 $X:(\Omega, \AA) \to (\Omega ', \AA')$ eine Abbildung. Sei $\mu$ ein
Maß auf $\AA$. Durch
\begin{align*}
\mu_{X}(A') & \defl  \mu(X^{-1}(A')) \\
              & =   \mu \left(\setdef{\omega \in \Omega}{X(\omega ) \in
              A'}\right) \defr \mu [ X \in A']; \quad A' \in \AA'
\end{align*}
wird ein Maß (das sogenannte \emph{Bildmaß}) $\mu _{X}$ auf $\AA'$
definiert.\\
Ist $\mu $ ein W-Maß auf $\AA$, dann ist $\mu _{X}$ ebenfalls ein W-Maß auf
$\AA'$.\fishhere
\end{prop}
\begin{proof}
\begin{proofenumroman}
  \item $\mu_X \ge 0$ ist klar.
  \item $\mu_X(\varnothing) = \mu(X^{-1}(\varnothing)) = \mu(\varnothing) = 0$.
  \item Seien $A_1',A_2',\ldots\in\AA'$ paarweise disjunkt, so gilt
\begin{align*}
\mu_X\left(\sum\limits_{i\ge 1} A_i'\right) &= 
\mu\left(X^{-1}\left(\sum\limits_{i\ge 1} A_i'\right)\right)
= \mu\left(\sum\limits_{i\ge 1}\left(X^{-1}(A_i')\right)\right)\\
&= \sum\limits_{i\ge 1}\mu\left(X^{-1}(A_i')\right).
\end{align*}
\item Sei $\mu$ ein Wahrscheinlichkeitsmaß so ist
\begin{align*}
\mu_X(\Omega') = \mu(X^{-1}(\Omega')) = \mu(\Omega) = 1.\qedhere
\end{align*}
\end{proofenumroman}
\end{proof}

Es genügt also, dass $X$ messbar ist, damit $P_X$ tatsächlich ein Maß ist.
Insbesondere ist für jede Zufallsvariable $P_X$ ein Maß.

\begin{defn}
\label{defn:3.4}
Sei $X$ eine $(\Omega ', \AA')$-Zufallsvariable auf dem W-Raum
$(\Omega, \AA, P)$.
Das W-Maß $P_{X}$ im Bild-W-Raum $(\Omega ', \AA', P_{X})$ heißt
\emph{Verteilung} der Zufallsvariable $X$.\\
Sprechweise:
\begin{itemize}[label=-]
  \item Die Zufallsvariable $X$ liegt in $A'\in \AA'$.
  \item $X$ nimmt Werte in $A' \in \AA'$ an mit Wahrscheinlichkeit $P_{X}(A') =
  P[X \in A']$.
  \item Wenn $P[X \in A'] = 1$, sagt man $X$ liegt $P$-fast sicher
  ($P$-f.s.) in $A'$.\fishhere
\end{itemize}
\end{defn}

Betrachten wir die Verteilung $P_X$ einer reellen Zufallsvariablen $X$, so
besitzt diese eine Verteilungsfunktion
\begin{align*}
F(t) = P_X((-\infty,t]) = P(X^{-1}(-\infty,t]).
\end{align*}
Da es sich bei der Verteilungsfunktion um eine reelle Funktion
handelt, kann man oft viel leichter mit ihr Rechnen, als mit der
Zufallsvariablen selbst.

Die Eigenschaften der Verteilungsfunktion charakterisieren die Zufallsvariable,
man klassifiziert Zufallsvariablen daher nach Verteilung (binomial-, poisson-~,
exponentialverteilt, \ldots).

\begin{defnn}
Besitzen zwei Zufallsvariaben dieselbe Verteilung (also dieselbe
Verteilungsfunktion) heißen sie \emph{gleichverteilt}.\fishhere
\end{defnn}

\begin{bsp}
Wir betrachten $n$ Bernoulli-Versuche mit jeweiliger Erfolgswahrscheinlichkeit
$p$. $\Omega$ besteht aus der Menge der Elementarereignisse $\omega$ $(=$
$n$-Tupel aus Nullen und Einsen$)$, $\AA=\PP(\Omega)$.

Das Wahrscheinlichkeitsmaß $P$ auf $\AA$ ist gegeben durch
\begin{align*}
P(\setd{\omega}) = p^k(1-p)^{n-k},
\end{align*}
falls $\omega$ aus $k$ Einsen und $n-k$ Nullen besteht.

Wir definieren uns eine Zufallsvariable $X$ durch:
\begin{align*}
X: \Omega \to \R,\quad X(\omega) = \text{Anzahl der Einsen in $\omega$}.
\end{align*}
$X$ ist $\AA-\BB$-messbar, denn $X^{-1}(\BB)\subseteq \PP(\Omega) = \AA$.

$X : (\Omega,\AA,P)\to(\R,\BB)$ gibt also die zufällige Anzahl der Erfolge in
$n$ Bernoulli-Versuchen an.

Das Bildmaß zu $X$ ist gegeben durch
\begin{align*}
P_X(\setd{k}) = P[X=k] = \binom{n}{k}p^k(1-p)^{n-k} \defr b(n,p;k).
\end{align*}
Um das Bildmaß auf allgemeine Mengen in $\BB$ fortzusetzen, setzen wir zu
$B\in\BB$
\begin{align*}
P_X(B) = \sum\limits_{k\in\N_0\cap B} b(n,p;k).
\end{align*}
$P_X$ ist die sogenannte \emph{Binominalverteilung} $b(n,p)$. $X$ ist
$b(n,p)$-verteilt.\bsphere
\end{bsp}

\begin{bem}
\label{bem:3.2}
Seien $(\Omega, \AA, P)$ ein W-Raum und
$(\Omega ', \AA')$, $(\Omega '', \AA'')$  Messräume.
Seien
$X: (\Omega , \AA) \to (\Omega ', \AA')$ und $Y: (\Omega ', \AA')\to
(\Omega '', \AA'')$, dann gilt
\begin{align*}
P_{Y \circ X} = (P_{X})_{Y}.\maphere
\end{align*}
\end{bem}
\begin{proof}
$Y\circ X$ ist wieder messbar. Für $A''\in\AA''$ gilt somit,
\begin{align*}
P_{Y\circ X}(A'') &= P((Y\circ X)^{-1}(A'')) =
P(X^{-1}(Y^{-1}(A''))) \\ &= P_X(Y^{-1}(A'') = (P_X)_Y(A'').\qedhere
\end{align*}
\end{proof}

\subsection{Produktmessräume}

Betrachten wir die Vektorräume $\R^n$ und $\R^m$, so können wir diese durch das
karthesische Produkt verknüpfen zu $V=\R^n\times \R^m$. $V$ ist dann wieder
ein Vektorraum und jedes Element $v\in V$ lässt sich darstellen als
$v=(x,y)$, wobei $x\in\R^n$ und $y\in\R^m$.

Eine solche Verknüpfung lässt sich auch für Messräume definieren.

\begin{defn}
\label{defn:3.5}
Seien $(\Omega _{i}, \AA_{i})$ Messräume für $i= 1, \ldots ,n$.
Die \emph{Produkt-$\sigma$-Algebra}
$\bigotimes^{n}_{i=1} \AA_{i} $ wird definiert als die von dem
Mengensystem
\begin{align*}
\setdef{
\prod\limits^{n}_{i=1}A_{i}
}{
A_{i} \in \AA_{i},\quad  i= 1, \ldots ,n}
\end{align*}
erzeugte $\sigma$-Algebra.\\
$\bigotimes^{n}_{i=1}(\Omega _{i}, \AA_{i}) \defl
\left(\prod\limits^{n}_{i=1} \Omega _{i},\,
\bigotimes^{n}_{i=1} \AA_{i}\right)$
heißt \emph{Produkt-Messraum}.\fishhere
\end{defn}

Man bildet also das karthesische Produkt der $\Omega_i$ und die
Produkt-$\sigma$-Algebra der $\AA_i$ und erhält so wieder einen Messraum
$\bigotimes_{i=1}^n (\Omega_i,\AA_i)$.

\begin{bsp}
$(\R^n\times\R^m$, ${\BB}_{n} \bigotimes {\BB}_{m}) =
(\R^{n+m},{\BB}_{n+m})$.\bsphere
\end{bsp}

\begin{bem}
\label{bem:3.3}
Sei $X_{i}: (\Omega , \AA) \to (\Omega _{i},
\AA_{i})$ messbar für $i=1, \ldots , n$. Die Abbildung
\begin{align*}
X: \Omega \to \prod^{n}_{i=1} \Omega _{i}
\end{align*}
mit
\begin{align*}
X(\omega )\defl (X_{1} (\omega ), \ldots , X_{n}(\omega )), \qquad \omega \in
\Omega
\end{align*}
ist dann $\AA$-$\bigotimes^{n}_{i=1}\AA_{i}$-messbar.\maphere
\end{bem}
\begin{proof}
Betrachte ein Erzeugendensystem von $\bigotimes_{i=1}^n A_i$, z.B. die Quader
\begin{align*}
\CC = \setdef{\prod\limits_{i=1}^n A_i}{A_i\in\AA_i}.
\end{align*}
Der Beweis wird dann wie im eindimensionalen Fall geführt.\qedhere
\end{proof}

Insbesondere ist eine Abbildung
\begin{align*}
X: \Omega\to\R^n,\qquad \omega\mapsto (x_1,\ldots,x_n)
\end{align*}
genau dann messbar, wenn es jede Komponete $X_i:\Omega\to \R$ ist.

\begin{bem}
\label{bem:3.4}
Seien $(\Omega _{i}, \AA_{i})$ Messräume für $i=1,\ldots ,n$. Ein W-Maß auf
$\bigotimes^{n}_{i=1} \AA_{i}$ ist eindeutig festgelegt durch seine Werte
auf dem Mengensystem
\begin{align*}
\setdef{\prod\limits^{n}_{i=1} A_{i}}{A_{i}
\in \AA_{i},\;i=1, \ldots ,n}.\maphere
\end{align*}
\end{bem}
\begin{proof}
Fortsetzungs- und Eindeutigkeitssatz \ref{prop:1.4} mit der $\sigma$-Algebra
gegeben durch die endliche Summe von Quadern der Form
\begin{align*}
\prod\limits_{i=1}^n A_i,\qquad A_i\in\AA_i.\qedhere
\end{align*}
\end{proof}

\begin{defn}
\label{defn:3.6}
Seien $X_{i}: (\Omega,\AA,P)\to(\Omega _{i}, \AA_{i})$ Zufallsvariablen für
$i=1, \ldots ,n$ und $X\defl (X_{1},
\ldots, X_{n})$ Zufallsvariable auf $(\Omega, \AA, P)$.

\begin{defnenum}
  \item Die Verteilung $P_{X}$ der Zufallsvariablen $X$  - erklärt durch
\begin{align*}
P_{X} (A) \defl P\left[(X_{1}, \ldots , X_{n}) \in A \right]
\end{align*}
für $A \in \bigotimes^{n}_{i=1} \AA_{i}$ oder auch nur für
$A= \prod\limits^{n}_{i=1} A_{i} \; (A_{i} \in \AA_i, \; i=
1, \ldots ,n)$ - heißt die \emph{gemeinsame Verteilung} der Zufallsvariablen $X_{i}$.
\item
Die Verteilungen ${P_{X_{i}}}$ - erklärt durch
\begin{align*}
P_{X_{i}}(A_{i}) &\defl P[X_{i} \in A_{i}] \\ &= P[X \in \Omega_{1}
\times \ldots \times \Omega _{i-1} \times A_{i} \times \Omega _{i+1} \times
\ldots \times \Omega _{n}]
\end{align*}
für $A_{i} \in \AA_{i}$ - heißen \emph{Randverteilungen} von
$P_{X} \quad (i=1,\ldots , n)$.\fishhere  
\end{defnenum}
\end{defn}

\begin{bemn}[Wichtiger Spezialfall.]
$(\Omega _{i}, \AA_{i}) =
(\R, {\BB})$ für $i=1, \ldots, n$. Dann sind die $X_i$ reelle
Zufallsvariablen und $X$ ein Zufallsvektor.\maphere
\end{bemn}

\begin{bem}[Bemerkungen.]
\label{bem:3.5}
Seien die Bezeichnungen wie in Definition \ref{defn:3.6}.
\begin{bemenum}
\item
Die Verteilung $P_{X}$ ist ohne Zusatzvoraussetzung durch ihre Randverteilungen
\textit{nicht} eindeutig festgelegt.
\item
Die Projektionsabbildung
\begin{align*}
\pi _{i} : \prod^{n}_{k=1} \Omega _{k} \to \Omega_{i},\qquad (\omega _{1},
\ldots, \omega _{n}) \to \omega _{i}
\end{align*}
ist messbar
und somit gilt
\begin{align*}
P_{X_{i}} = (P_{X})_{\pi _{i}},\qquad i=1, \ldots , n.\maphere
\end{align*}
\end{bemenum}
\end{bem}

Aufgrund der Messbarkeit der Projektionen sind die Randverteilungen
einer gemeinsamen Verteilung eindeutig bestimmt. Die Umkehrung, dass die
Randverteilungen auch die gemeinsame Verteilung eindeutig bestimmen ist im
Allgemeinen falsch. Wir werden uns in Kapitel \ref{chap:5} ausführlicher damit
beschäftigen.

\begin{bem}[Bemerkungen.]
\label{bem:3.6}
\begin{bemenum}
\item
Sei $Q$ ein W-Maß auf $(\R^n, {\BB}_{n})$.
Dann existieren reelle Zufallsvariablen $X_{1}, \ldots , X_{n}$
auf einem geeigneten W-Raum $(\Omega ,\AA,P)$ so,
dass ihre gemeinsame Verteilung mit $Q$ übereinstimmt: \\
Auf dem W-Raum $(\Omega, \AA,P) \defl (\R^{n}, {\BB}_{n}, Q)$, wird
$X_{i} : \Omega \to \R$ definiert als die Projektionsabbildung
\begin{align*}
(x_{1}, \ldots, x_{n}) \to x_{i} \quad (i=1, \ldots,n);
\end{align*}
hierbei ist also $X\defl (X_{1}, \ldots ,X_{n})$ die auf
$\R^n$ definierte identische Abbildung.
\item
Wir können die Aussage aus a) unmittelbar auf einen beliebigen Produkt-Meßraum
$\left(\prod\limits^{n}_{i=1}\Omega _{i}, \;
\bigotimes^{n}_{i=1} \AA_{i}\right)$
statt $(\R^{n}, {\BB}_{n})$ verallgemeinern.
\item Sonderfall zu b): Ist $(\Omega, \AA,Q)$ ein W-Raum, $X:\Omega \to
\Omega$ die identische Abbildung, so gilt $P_{X} =Q$. Jedes W-Maß lässt sich somit
als eine Verteilung auffassen (und - definitionsgemäß - umgekehrt).\maphere
\end{bemenum}
\end{bem}

Dieser Zusammenhang zwischen W-Maß und Verteilung hat eine große Bedeutung, wie
wir mit der Einführung des Maßintegrals im folgenden Kapitel sehen werden.


\cleardoublepage
\chapter{Erwartungswerte und Dichten}

\section{Erwartungswert, Varianz und Lebesgue Integral}

Gegeben seien ein W-Raum $(\Omega, \AA, P)$ und eine reelle Zufallsvariable $X$ auf
$(\Omega, \AA, P)$ mit Verteilung $P_{X}$ (auf ${\BB}$) und Verteilungsfunktion
$F:\mathbb{R} \to \mathbb{R}$.\\

Der Erwartungswert $\E X$ der Zufallsvariable $X$ gibt einen ``mittleren Wert'' von $X$
bezüglich $P$ an.

\begin{bsp}
Wir betrachten einen fairen Würfel.
$X$ gebe die zufällige Augenzahl an. Den Erwartungswert von $X$ erhalten wir,
indem wir die möglichen Werte von $X$ mit ihrer Wahrscheinlichkeit
multiplizieren und aufsummieren,
\begin{align*}
\E X = 1\cdot \frac{1}{6} + 2\cdot \frac{1}{6} + \ldots + 6\cdot \frac{1}{6} =
3.5.
\end{align*}
Für \emph{diskrete} Zufallsvariablen, das sind Zufallsvariablen deren
Verteilung auf $\N_0$ konzentriert ist, erhalten wir somit,
\begin{align*}
\E X = \sum\limits_{k=1}^\infty k p_k,\qquad
\text{mit } p_k = P[X=k].\bsphere
\end{align*}
\end{bsp}

Diese Definition lässt sich allerdings nicht auf den Fall einer
Zufallsvariablen mit einer auf einem  Kontinuum konzentrierten Verteilung
übertragen. Um diesen Fall einzuschließen, gehen wir von der diskreten Summe
zum kontinuierlichen Integral über, wir wählen also Integralansatz, um den
Erwartungswert einzuführen.

\subsection{Erwartungswert mittels Riemann-Stieltjes-Integral}

\textit{Bezeichnungen.}
\begin{align*}
&X^{+}(\omega)\defl
\begin{cases}
X(\omega), &\text{falls } X(\omega) > 0,\\
0,          &\text{sonst }
\end{cases}\\
&X^{-}(\omega)\defl
\begin{cases}
-X(\omega), \; &\text{ falls } X (\omega) < 0,\\
0,              &\text{sonst. }
\end{cases}
\end{align*}

Wir wählen zunächst den naiven Ansatz, den Erwartungswert von $X$ als Integral
über den Wertebereich von $X$ zu definieren.

\textit{1.\ Schritt}: Sei zunächst $X$ positiv ($X\geq 0$), d.h. $P_{X}(\R_{+})=
1$ und $F(x) =0$ für $x < 0$. Wir ersetzen nun die Summe aus dem Beispiel durch
ein Integral,
\begin{align*}
\E X\defl 
\int_{\R_{+}} x\,P_{X}(\dx)
\defl
\int_{\R_{+}}x\,\dF(x) \defl
\lim\limits_{a \to \infty}
\int\limits_{[0,a]} x\dF(x),
\end{align*}
wobei es sich hier um ein \textit{Riemann-Stieltjes-Integral} von $x$ bezüglich
$F$ handelt. Da der Integrand $x$ auf $\R_+$ positiv, gilt $0 \leq \E X \leq \infty$.

Für $F(x) = x$ entspricht das Riemann-Stieltjes-Integral dem gewöhnlichen
Riemann-Integral,
\begin{align*}
\int_{\R_{+}}x\,\dF(x) = \int_0^\infty x \dx.
\end{align*}
Allgemeiner entspricht das Riemann-Stieltjes-Integral 
für stetig differenzierbares $F$ dem
Riemann-Integral,
\begin{align*}
\int_{\R_{+}}x\,\dF(x) = \int_0^\infty x\ F'(x)\dx.
\end{align*}

\textit{2.\ Schritt}: Sei $X$ beliebig und $\E X^{+}$, $\E X^{-}$ nicht beide
$\infty$, so ist der Erwartungswert von $X$ definiert als
\begin{align*}
\E X\defl \int_{\R}x\, P_{X}(\dx) \defl \E X^{+} - \E X^{-}.\fishhere
\end{align*}

\begin{defn}[Erster Versuch]
\label{defn:4.1}
des \emph{Erwartungswertes $\E X$ von $X$} als ein
Integral auf dem Wertebereich von $X$:
\begin{align*}
\E X \defl \int_{\R} x\, P_{X}(\dx).\fishhere
\end{align*}
\end{defn}

Für den Erwartungswert ist diese Definition ausreichend und historisch ist man
auch genau so vorgegangen. Wir integrieren hier jedoch lediglich die stetige
Funktion $x\mapsto x$. In der Wahrscheinlichkeitstheorie arbeitet man aber
meist mit unstetigen Funktionen und für diese ist das
Riemann-Stieltjes-Integral \textit{nicht} ausreichend.

\subsection{Erwartungswert mittels Maß-Integral}

Wir benötigen einen allgemeineren Integrationsbegriff, der es uns erlaubt,
lediglich messbare (also insbesondere auch unstetige) Funktionen zu
integrieren.

Dazu machen wir einen neuen Integralansatz, wobei wir diesmal über den
Definitionsbereich von $X$ also $\Omega$ integrieren.

Sei also $X: (\Omega,\AA)\to (\R,\BB)$ messbar.

\textit{0.\ Schritt}: Sei $X$ positiv und nehme nur endlich viele
Werte an. Dann existiert eine Darstellung
\begin{align*}
X= \sum\limits^{N}_{i=1} \alpha _{i}\,\Id _{A_{i}},\qquad
\alpha_{i} \in \R_+,\quad A_{i} \in \AA
\end{align*}
mit $A_i$ paarweise disjunkt und $\sum^{N}_{i=1} A_{i} = \Omega $.

$\Id_A$ bezeichnet die sogenannte \emph{Indikatorfunktion} einer Menge $A$,
\begin{align*}
\Id_A(x) \defl
\begin{cases}
1, & x\in A,\\
0, & x\notin A.
\end{cases}
\end{align*}
Somit können wir den Erwartungswert definieren als
\begin{align*}
\E X\defl \int_{\Omega} X\dP\defl \sum\limits^{N}_{i=1} \alpha_{i}\, P(A_{i}).
\end{align*}

Diese Definition schließt z.B. den Fall des Würfels ein, jedoch sind wir
beispielsweise noch nicht in der Lage, den Erwartungswert eines zufälligen 
Temperaturwertes anzugeben.

\textit{1.\ Schritt}: Sei nun $X$ lediglich positiv. In der Maßtheorie wird
gezeigt, dass dann eine Folge von Zufallsvariablen $X_{n} \geq 0$ existiert,
wobei $X_n$ jeweils nur endlich viele Werte annimmt und die Folge $(X_n)$
monoton gegen $X$ konvergiert, $X_{n}(\omega) \uparrow X(\omega)$ für $n
\to \infty$ und $\omega \in \Omega$.

Somit können wir den Erwartungswert von $X$ definieren als
\begin{align*}
\E X \defl \int_{\Omega}X\dP \defl \lim\limits_{n \to
\infty }\E X_{n},
\end{align*}
wobei der (uneigentliche) Grenzwert existiert, da die $\E X_n$ monoton wachsen. 

\textit{2.\ Schritt}: Seien $\E X^{+}$, $\E X^{-}$ nicht beide $\infty$.
\begin{align*}
\E X\defl\int_{\Omega } X\dP\defl \E X^{+} - \E X^{-}.
\end{align*}

Somit erhalten wir die endgültige Definition des Erwartungswerts. 

\begin{defn}
\label{defn:4.2}
des Erwartungswertes $\E X$ von $X$ als ein
Integral auf dem Definitionsbereich von $X$:
\begin{align*}
\E X \defl \int_{\Omega}X(\omega)P(\domega) \defr \int_{\Omega}
X\dP.\fishhere
\end{align*}
\end{defn}

\begin{bem}[Bemerkungen zu Definition \ref{defn:4.2}]
\label{bem:4.1}
\begin{bemenum}
\item\label{bem:4.1:1}
Die einzelnen Schritte - insbesondere bei der 2.\ Definition über das
Maß-Interal - sind sinnvoll. Für Interessierte werden die Details in der
Maßtheorie, einem Teilgebiet der Analysis, behandelt.
\item\label{bem:4.1:2}
Die beiden Definitionen sind äquivalent.
\item\label{bem:4.1:3}
Existiert das  Integral
\begin{align*}
\int_{\R} x\ P_{X}(\dx)
\end{align*}
in 1.\ Definition, so ändert es seinen Wert nicht, wenn man es gemäß 2.\
Definition erklärt. Insbesondere gilt dann,
\begin{align*}
\int_\Omega X\dP = \int_\R x\, P_X(\dx) = \int_\R x \dF(x).
\end{align*}
\item\label{bem:4.1:4}
Der Begriff $\int_{\Omega}X\ \dP$ in Definition \ref{defn:4.2} lässt sich
unmittelbar verallgemeinern auf den Fall eines Maßraumes $(\Omega , \AA, \mu )$
(anstatt eines W-Raumes) und liefert
\begin{align*}
\int_{\Omega }X(\omega)\ \mu (\domega )\defr \int_{\Omega }
X \dmu.
\end{align*}

\textit{Wichtige Spezialfälle}:
\begin{defnenum}
\item
Sei $(\Omega, \AA) = (\R^{n}, {\BB}_{n})$ oder etwas allgemeiner
$\Omega = B$ mit $B \in {\BB}_{n}$,  $\AA = B\cap \BB_n$ und $\mu =
\lambda\Big|_{\AA}$ Restriktion des L-B-Maß(es) auf $\AA$.
\begin{align*}
\int_{\R^{n}}X\dlambda\quad \text{bzw.} \quad \int_{B} X\dmu
\end{align*}
wird als \emph{Lebesgue--Integral} bezeichnet.
Im Falle $n=1$ schreiben wir auch
\begin{align*}
\int_{\R}X(x)\dx \quad \text{bzw.} \quad \int_{B} X(x)\dx.
\end{align*}
\item
Sei $(\Omega,\AA)= (\R, {\BB})$ und $H: \R \to \R$ eine maßdefinierende Funktion
mit zugehörigem Maß $\mu$, d.h.
\begin{align*}
H(b)-H(a) = \mu ((a,b]),\qquad -\infty< a <b < \infty,
\end{align*}
so schreiben wir
\begin{align*}
\int_{\R} X(x)\dH(x) \defl \int_{\R} X(x)\ H(\dx)\defl
\int_{\R} X\ \dmu
\end{align*}
Die Verallgemeinerung auf $\Omega = B \in {\BB}$ folgt analog.
\item Sei $(\Omega,\AA)=(\N_0,\PP(\N_0))$ und $\mu$ das Zählmaß, d.h.
\begin{align*}
\mu(A) = \text{Anzahl der Elemente in }A,
\end{align*}
so gilt für eine Funktion $f:\N_0\to\R$,
\begin{align*}
\int_{\N_0} f\dmu = \sum\limits_{k=0}^\infty f(k).
\end{align*}
Reihen sind also lediglich ein Spezialfall des Maß-Integrals.
\end{defnenum}
\item\label{bem:4.1:5}
Sei $X$ eine erweitert-reelle Zufallsvariable auf einem W-Raum $(\Omega,
\AA,P)$ mit $P[|X| = \infty] = 0$ und
\begin{align*}
Y(\omega ) \defl
\begin{cases}
X(\omega),& \text{falls } \abs{X(\omega)} < \infty, \; \; \omega \in \Omega,\\
0 & \text{sonst},
\end{cases}
\end{align*}
so definiert man $\E X\defl \E Y$, falls $\E X$ existiert. 
\maphere
\end{bemenum}
\end{bem}

Für das Maß- bzw. spezieller das Lebesgue-Integral existieren zahlreiche sehr
allgemeine Sätze und elegante Beweisstrategien, die uns für das
Riemann-Stieltjes-Integral nicht zur Verfügung stehen, weshalb wir im Folgenden
fast ausschließlich von dieser Definition ausgehen werden.

Wir haben bisher nicht geklärt, wie man ein Maß- bzw. ein  Lebesgue-Integral
konkret berechnet, wenn $X$ nicht nur endlich viele Werte annimmt. Ist die
Verteilungsfunktion stetig differenzierbar, so gilt
\begin{align*}
\int_\Omega X\dP = \int_\R x\, F'(x)\dx,
\end{align*}
wobei wir letzteres Integral durchaus als
Lebesgue-Integral mit all seinen angenehmen Eigenschaften auffassen können,
wir können damit aber auch rechnen wie mit dem Riemann-Integral (Substitution,
partielle Integration, \ldots). Für allgemeineres $F$ müssen wir uns darauf
beschränken, abstrakt mit dem Integral arbeiten zu können. 

\begin{bem}
\label{bem:4.2}
Sei $(\Omega, \AA, P)$ ein W-Raum. Für $A \in \AA$ gilt
$P (A) = \E\,\Id _{A}$.\maphere
\end{bem}

\begin{defn}
\label{defn:4.3}
Existiert für die reelle Zufallsvariable $X$ auf dem W-Raum
$(\Omega, \AA,P)$ ein endlicher Erwartungswert $\E X$, so heißt $X$
\emph{integrierbar} (bezüglich $P$). Analog für $\int_{\Omega } X\dmu $ in
Bemerkung \ref{bem:4.1} \ref{bem:4.1:4}.\fishhere
\end{defn}

Ist $X$ eine positive Zufallsvariable, so können wir den Erwartungswert
direkt mit Hilfe der Verteilungsfunktion berechnen.
\begin{lem}
\label{lem:4.1}
Für eine reelle Zufallsvariable $X \geq 0$ mit Verteilungsfunktion $F$ gilt
\begin{align*}
\E X = \int^{\infty}_{0} (1-F(x))\dx.\fishhere
\end{align*}
\end{lem}
\begin{proof}
Wir verwenden Definition \ref{defn:4.1} mithilfe des R-S-Integrals,
\begin{align*}
\E X &= 
\int_{\R_+} x\dF(x) = \lim\limits_{a\to\infty}
\int_{[0,a]} x\dF(x)\\
&\overset{\text{part.int.}}{=}
\lim\limits_{a\to\infty} \left(
x\,F(x)\Big|_{x=0}^a - 
\int_0^a 1F(x)\dx
\right)
\overset{(*)}{=}
\lim\limits_{a\to\infty}
\left(a - \int_0^a F(x)\dx \right)\\
&=
\lim\limits_{a\to\infty}
\left(\int_0^a (1-F(x))\dx \right)
= 
\int_0^\infty (1-F(x))\dx.
\end{align*}
\begin{proofenumarabic}
\item Zu (*). Sei $\E X < \infty$
\begin{align*}
\infty &> \E X = \int\limits_{\R_+} x\dF(x)
\ge \int\limits_{[a,\infty]}x\dF(x)
\ge \int\limits_{[a,\infty]}a\dF(x)\\
&\ge a\int\limits_{(a,\infty)}\dF(x)
= a(1-F(a)) \ge 0.
\end{align*}
\item Außerdem:
\begin{align*}
\int\limits_{[a,\infty]} x\dF(x) \overset{a\to\infty}{\to} 0,
\end{align*}
da $\int_{[0,a]} x\dF(x) \to \int_{\R_+} x\dF(x)$, also
\begin{align*}
a(1-F(a))\overset{a\to\infty}{\to}0.
\end{align*}
\end{proofenumarabic}
Der Fall $\E X = \infty$ wird gesonderd behandelt.\qedhere
\end{proof}

\subsection{Eigenschaften des Erwartungswerts}

Der Erwartungswert einer reellen Zufallsvariablen $X$,
\begin{align*}
\E X = \int_\Omega X\dP = \int_\R x \dP_X,
\end{align*}
ist ein spezielles Maß-Integral. Daher übertragen sich alle Eigenschaften
dieses Integrals auf den Erwartungswert.

\begin{prop}
\label{prop:4.1}
Sei $X$ eine reelle Zufallsvariable auf einem W-Raum $(\Omega, \AA, P)$.
\begin{propenum}
\item
$X$ integrierbar $\Leftrightarrow X^{+}$ und $X^{-}$ integrierbar
$\Leftrightarrow |X| $ integrierbar.
\item Existiert eine reelle integrierbare Zufallsvariable $Y \geq 0 $ mit
\begin{align*}
|X| \leq Y  \text{ P-f.s., so ist } X \text{ integrierbar}.
\end{align*}
\item
Ist $X$ integrierbar und  existiert eine reelle Zufallsvariable $Y$ mit
$Y = X$ P-f.s., dann existiert $\E Y = \E X $.\fishhere
\end{propenum}
\end{prop}
\begin{proof}
Der Beweis wird in der Maßtheorie geführt.\qedhere
\end{proof}

Im Gegensatz zum klassischen uneigentlichen Riemann-Integral ist es beim
Maß-Integral nicht möglich, dass sich positive und negative Anteile gegenseitig
eliminieren und so eine Funktion deren Positiv- oder Negativanteil alleine
nicht integrierbar sind, als ganzes integrierbar wird. Es gibt also durchaus
Funktionen die (uneigentlich) Riemann- aber nicht Lebesgue-integrierbar sind.
Für ``viel mehr Funktionen'' gilt jedoch das Gegenteil.

Beim Riemann-Integral ändert eine Abänderung einer Funktion an endlich vielen
Punkten den Integralwert nicht, beim Lebesgue-Integral hingegen ändert eine
Abänderung einer Zufallsvariabeln auf einer Menge vom Maß Null ihren
Erwartungswert nicht.

\begin{bsp}
Die Dirichletfunktion $f:\R\to\R$ mit
\begin{align*}
f(x) = 
\begin{cases}
1, & x\in \Q,\\
0, & x\in \R\setminus\Q.
\end{cases}
\end{align*}
ist \textit{nicht} Riemann- aber Lebesgue-integirerbar. Das Lebesgue-Integral
lässt sich auch sehr leicht berechnen, denn $f\ge 0$, also gilt nach Definition
\begin{align*}
\int_\R f(x)\dx = 1\cdot \lambda(\Q) + 0\cdot \lambda(\R\setminus\Q) = 1\cdot
\underbrace{\lambda(\Q)}_{=0},
\end{align*}
denn $\Q$ ist abzählbare Teilmenge von $\R$.\bsphere
\end{bsp}

Ferner ist der Erwartungswert linear, monoton und erfüllt die
Dreiecksungleichung.

\begin{prop}
\label{prop:4.2}
Seien $X, Y $ reelle integrierbare Zufallsvariablen auf einem W-Raum
$(\Omega, \AA, P)$.
\begin{propenum}
\item
Es existiert $\E(X+Y)= \E X +\E Y$.
\item
Es existiert $\E(\alpha X)= \alpha \E X$ für $\alpha \in \R$.
\item
$X \geq Y \Rightarrow \E X \geq \E Y $.
\item
$|\E X|\leq \E|X| $.
\item
$X \geq 0, \; \; \E X =0 \Rightarrow X = 0 $ P-f.s.\fishhere
\end{propenum}
\end{prop}

\begin{proof}[Beweisidee.]
Wir beweisen lediglich die erste Behauptung, der Rest folgt analog. Dazu
bedienen wir uns dem \textit{Standardtrick} für Beweise in der Maßtheorie.

\textit{1. Schritt}. Reduktion auf $X\ge 0$ und $Y\ge 0$, dann Reduktion auf
einfache Funktionen
\begin{align*}
X= \sum\limits_{i=1}^n \alpha_i \Id_{A_i},\qquad Y=
\sum\limits_{j=1}^m \beta_j \Id_{B_j}
\end{align*}
mit $\sum\limits_{i=1}^n A_i = \Omega = \sum\limits_{j=1}^m B_j$. Sei
$C_{ij}=A_i\cap B_j\in \AA$, so gilt
\begin{align*}
&X = \sum\limits_{i,j} \gamma_{ij} \Id_{C_{ij}},\qquad
Y = \sum\limits_{i,j} \delta_{ij} \Id_{C_{ij}},\\
\Rightarrow &
X+Y = \sum\limits_{i,j} (\gamma_{ij}+\delta_{ij})\Id_{C_{ij}}.
\end{align*}

\textit{2. Schritt}. Nun seien $X\ge 0$ und $Y\ge 0$ beliebig und $X_n$, $Y_n$
Folgen von einfachen Funktionen mit $X_n\uparrow X$, $Y_n\uparrow Y$ und
$X_n+Y_n=Z_n\uparrow Z=X+Y$.

Für jedes $n\in\N$ sind $X_n$ und $Y_n$ einfach und damit $\E$ linear. Es gilt
also
\begin{align*}
\E (X_n+Y_n) = \E X_n + \E Y_n \le \E X + \E Y
\end{align*}
sowie
\begin{align*}
\E X_n + \E Y_n  = \E (X_n+Y_n) = \E Z_n \le \E Z = \E (X+ Y).
\end{align*}
Nach dem Grenzübergang für $n\to\infty$ erhalten wir somit
\begin{align*}
\E X + \E Y = \E (X+Y).
\end{align*}

\textit{3. Schritt}. Für $X$ und $Y$ integrierbar betrachten wir
\begin{align*}
X=X_+-X_-,\qquad Y = Y_+-Y_-.
\end{align*}
Nach dem eben gezeigten, können wir die Linearität von $\E$ für Positiv und
Negativteil verwenden.
\begin{align*}
\E X + \E Y &= (\E X_+ \E Y_+) - (\E X_- + \E Y_-)
= \E (X_++Y_+) - \E (X_-+Y_-)\\
&= \E (X+Y).\qedhere
\end{align*}
\end{proof}

\begin{bsp}
Wir betrachten $n$ Bernoulli-Versuche mit jeweiliger Erfolgswahrscheinlichkeit
$p$. Die Zufallsvariable $X_i$ nehme Werte $0$ bzw. $1$ an, falls im $i$-ten
Versuch ein Misserfolg bzw. Erfolg aufgetreten ist ($i=1,\ldots,n$).

$X=X_1+\ldots+X_n$ gibt die Anzahl der Erfolge an.
\begin{align*}
\E X = \E X_1 + \ldots \E X_n \overset{!}{=} n \E X_1 = np,
\end{align*}
da alle $X_i$ dieselbe Verteilungsfunktion besitzen.

Der Erwartungswert einer $b(n,p)$-verteilten Zufallsvariablen ist also $np$.
Ein alternativer (evtl. ungeschickterer) Rechenweg ist
\begin{align*}
\E X = \sum\limits_{k=0}^n kP[X=k] = \sum\limits_{k=0}^n
k\binom{n}{k}p^k(1-p)^{n-k} = \ldots = np.\bsphere
\end{align*}
\end{bsp}

Für das Riemann-Integral ist für die Vertauschbarkeit von
Integration und Grenzwertbildung
\begin{align*}
\lim\limits_{n\to\infty}\int_{[a,b]} f_n(x)\dx
=
\int_{[a,b]} \lim\limits_{n\to\infty}f_n(x)\dx
\end{align*}
im Allgemeinen die gleichmäßige Konvergenz der $f_n$ erforderlich.
Ein entscheidender Vorteil des Lebesgue-Integrals ist die Existenz von
Konvergenzsätzen, die wesentlich schwächere Voraussetzungen für die
Vertauschbarkeit haben.

\begin{prop}[Satz von der monotonen Konvergenz (B. Levi)]
\label{prop:4.3}
Für reelle
Zufallsvariablen
$X_{n}$ auf $(\Omega, \AA, P)$ mit $X_{n} \geq0 \; \; (n \in
\N)$,
$X_{n} \uparrow X \; \; (n \to \infty) $ existiert
\begin{align*}
\E X = \lim\limits_{n} \E X_{n}.
\end{align*}
Hierbei $\E X \defl \infty$, falls $P[X= \infty ] > 0$. Entsprechend für Reihen von
nichtnegativen Zufallsvariablen. -- Analog für $\int_{\Omega } X_{n}\dmu $ gemäß
Bemerkung \ref{bem:4.1}d.\fishhere
\end{prop}
\begin{proof}
Die Messbarkeit von $X$ folgt aus Satz \ref{prop:3.7}.
% \begin{align*}
% [X\le \alpha] = \bigcap_{n\in\N} [X_n\le \alpha],\qquad \alpha\in\R.
% \end{align*}
Die $X_n$ sind monoton wachsend und positiv, also ist $(\int_\Omega X_n\dP)_n$
monotone Folge, somit existiert ihr Grenzwert. Setze
\begin{align*}
c \defl \lim\limits_{n\to\infty}\int_\Omega X_n\dP \le \int_\Omega X\dP.
\end{align*}
Es genügt nun für jede einfache Funktion $Y$ mit $Y\le X$ zu zeigen, dass
\begin{align*}
c \ge \int_\Omega Y\dP.
\end{align*}
Sei dazu
\begin{align*}
Y = \sum\limits_{i=1}^m \alpha_i\, \Id_{[Y= \alpha_i]},\qquad \alpha_i \ge 0.
\end{align*}
Sei $\delta <1 $ beliebig aber fest und
\begin{align*}
A_n \defl [X_n\ge \delta Y].
\end{align*}
Wegen $X_n\uparrow X$ folgt $A_n\uparrow \Omega$ also auch 
$A_n\cap [Y=\alpha_j]\uparrow [Y=\alpha_j]$ für $j=1,\ldots,m$.
\begin{align*}
\delta \int_\Omega Y\dP &= \delta \sum\limits_{j=1}^m \alpha_j
P[Y=\alpha_j] = \delta \sum\limits_{j=1}^m \alpha_j
\lim\limits_{n\to\infty}P(A_n\cap[Y=\alpha_j])\\
&= \lim\limits_{n\to\infty}
\sum\limits_{j=1}^m \delta \alpha_j P(A_n\cap [Y=\alpha_j])
= \lim\limits_{n\to\infty} \int_\Omega \underbrace{\delta Y\Id_{A_n}}_{\le
X_n}\dP\\ &\le \lim\limits_{n\to\infty} \int_\Omega X_n\dP = c.
\end{align*}
Da die Ungleichung für jedes $\delta < 1$ gilt, folgt
\begin{align*}
\int_\Omega Y \dP \le c.\qedhere
\end{align*}
\end{proof}

\section{Dichtefunktion und Zähldichte}

\begin{defn}
\label{defn:4.4}
Sei $X$ ein $n$-dimensionaler Zufallsvektor mit Verteilung
$P_{X}$ auf ${\BB}_{n}$ und Verteilungsfunktion $F: \R^{n} \to \R$.
\begin{defnenum}
\item
Nimmt $X$ (eventuell nach Vernachlässigung einer $P$-Nullmenge in $\Omega$)
höchstens abzählbar viele Werte an, so ist $P_{X}$ eine sogenannte
\emph{diskrete W-Verteilung}.

Ist $X$ eine reelle Zufallsvariable und $P_{X}$ auf $\N_{0}$ konzentriert, so
heißt die Folge $(p_{k})_k$ mit
\begin{align*}
p_{k} \defl P_{X}(\setd{k}) = P[X=k],\qquad k \in \N_{0},
\end{align*}
(wobei $\sum_{k=0}^\infty p_{k}=1$) \emph{Zähldichte} von $X$ bzw. $P_{X}$.
\item
Gibt es eine Funktion
\begin{align*}
f: (\R^{n}, {\BB}_{n}) \to (\R_{+}, {\BB}_{+})
\end{align*}
für die das Lebesgue-Integral
\begin{align*}
\int_{\prod^{n}_{i=1}(-\infty,x_{i}]}f\dlambda 
\end{align*}
existiert und mit
\begin{align*}
P_{X}\left(\prod\limits^{n}_{i=1}(- \infty , x_{i}]\right) = F(x_{1},
\ldots , x_{n}),\qquad (x_{1}, \ldots , x_{n}) \in \R^{n}
\end{align*}
übereinstimmt, so heißt
$f$ \emph{Dichte(funktion)} von $X$ bzw. $P_{X}$ bzw. $F$.

$P_{X}$ und
$F$ heißen \emph{totalstetig}, falls sie eine Dichtefunktion
besitzen.\fishhere
\end{defnenum}
\end{defn}

\begin{bem}[Bemerkungen.]
\label{bem:4.3}
\begin{bemenum}
\item
Eine totalstetige Verteilungsfunktion ist stetig.
\item
Besitzt die $n$-dimensionale Verteilungsfunktion
\begin{align*}
F: \R^{n} \to \R
\end{align*}
eine Dichtefunktion
\begin{align*}
f: \R^{n} \to \R_{+},
\end{align*}
so existiert L-f.ü.
\begin{align*}
\frac{\partial^{n}F}{\partial x_{1} \ldots \partial x_{n}},
\end{align*}
und L-f.ü., d.h. insbesondere an den Stetigkeitsstellen von~$f$, gilt
\begin{align*}
\frac{\partial^{n}F}{\partial x_{1} \ldots \partial x_{n}} = f.
\end{align*}
\item
Ein uneigentliches Riemann-Integral einer nichtnegativen ${\BB}_{n}$-${\cal
B}$-messbaren Funktion lässt sich als Lebesgue-Integral deuten.\maphere
\end{bemenum}
\end{bem}

Sind Dichtefunktion bzw. Zähldichte bekannt, so können wir den abstrakten
Ausdruck $P_X$ durch ein Integral über eine konkrete Funktion bzw. durch eine
Reihe ersetzen,
\begin{align*}
P_X(A) = \int_A f\dlambda \quad \text{bzw.} \quad \sum\limits_{k\in\N_0\cap A}
p_k.
\end{align*}

Zufallsvariablen mit Dichtefunktion bzw. Zähldichte, sind für
uns somit sehr ``zugänglich''. Es lassen sich jedoch nicht für alle
Zufallsvariablen Dichten finden, denn dazu müsste jede Verteilungsfunktion
$\fu{L}$ differenzierbar und Differentiation mit der Integration über $\R^n$
vertauschbar sein.

\begin{prop}
\label{prop:4.4}
Sei $X$ eine reelle Zufallsvariable mit Verteilung $P_{X}$.
\begin{propenum}
\item\label{prop:4.4:1}
Ist $P_{X}$ auf $\N_{0}$ konzentriert mit Zähldichte $(p_{k})$, dann gilt
\begin{align*}
\E X = \sum\limits_{k=1}^\infty k\,p_{k}.
\end{align*}
\item\label{prop:4.4:2}
Besitzt $X$ eine Dichtefunktion $f: \R \to \R_{+}$, so existiert das
Lebesgue-Integral
\begin{align*}
\int_{\R} xf(x)\dx \defr \int_{\R} xf(x)\; \lambda(\dx)
\end{align*}
genau dann, wenn $\E X$ existiert, und es gilt hierbei
\begin{align*}
\E X= \int_{\R} xf(x)\dx.\fishhere
\end{align*}
\end{propenum}
\end{prop}
\begin{proof}
\textit{Beweisidee zu Satz \ref{prop:4.4}\ref{prop:4.4:1}}
Für den Erwartungswert gilt nach \ref{defn:4.2},
\begin{align*}
\E X = \int_\Omega X \dP.
\end{align*}
$X$ nimmt höchstens abzählbar viele verschiedene Werte in $\N_0$ an, wir können
das Integral also schreiben als,
\begin{align*}
\int_\Omega X \dP = \lim\limits_{n\to\infty} \sum\limits_{k=0}^n k\cdot P[X=k]
= \sum\limits_{k=0}^\infty k\, p_k. 
\end{align*}
\textit{Spezialfall zu \ref{prop:4.4}\ref{prop:4.4:2}}. Sei $X\ge 0$ und $F$
stetig differenzierbar auf $(0,\infty)$. Wir verwenden die Definition des
Erwartungswerts als Riemann-Stieltjes-Integral
\begin{align*}
\E X = \lim\limits_{x\to\infty} \int\limits_{[0,x]}
t\dF(t)
\overset{!}{=} \lim\limits_{x\to\infty} \int\limits_{[0,x]}
t F'(t)\dt = \int\limits_{[0,\infty]}xf(x)\dx.
\end{align*}
(!) gilt, da $F$ stetig differenzierbar. Da $f$ nur auf der positiven rellen
Achse Werte $\neq 0$ annimmt, gilt somit
\begin{align*}
\E X = \int_{\R}xf(x)\dx.
\end{align*}
Mit Hilfe des Transformationssatzes werden wir später einen vollständigen
Beweis geben können.\qedhere
\end{proof}

Für eine Zufallsvariable $X$ auf $(\Omega,\AA,P)$ mit Dichtefunktion bzw.
Zähldichte haben wir mit Satz \ref{prop:4.4} eine konkrete Möglichkeit den
Erwartungswert zu berechnen. Wir betrachten dazu nun einige Beispiele.
\begin{bsp}
Sei $X$ $b(n,p)$-verteilt, d.h. $X$ hat die Zähldichte
\begin{align*}
p_k = P[X=k]= {n \choose k}p^{k}(1-p)^{n-k},\quad (k=0,1, \ldots,n)
\end{align*}
mit $n \in \N$ und $p\in [0,1]$. Für den Erwartungswert gilt $\E X=np$.\bsphere
\end{bsp}
\begin{bsp}
Sei $X$ $\pi(\lambda )$-verteilt, d.h.
\begin{align*}
p_k=P[X=k]=\e^{- \lambda}{\displaystyle
\frac{\lambda^{k}}{k!}},\qquad  k \in \N_{0},
\end{align*}
mit $\lambda >0$, so gilt $\E X = \lambda $.
\begin{proof}
Zu berechnen ist der Wert folgender Reihe,
\begin{align*}
\E X = \sum_{k=0}^\infty k\e^{-\lambda}\frac{\lambda^k}{k!}.
\end{align*}
Wir verwenden dazu folgende Hilfsunktion,
\begin{align*}
g(s) \defl \sum\limits_{k=0}^{\infty} \e^{-\lambda}\frac{\lambda^k}{k!}s^k
= \e^{-\lambda}\e^{\lambda s},\qquad s\in [0,1].
\end{align*}
$g$ bezeichnet man auch als \emph{erzeugende Funktion} der Poisson-Verteilung.
Wir werden in Kapitel \ref{chap:6} genauer darauf eingehen. Die Reihe
konvergiert auf $[0,1]$ gleichmäßig, wir können also gliedweise
differenzieren und erhalten somit,
\begin{align*}
g'(s) = \sum\limits_{k=1}k \e^{-\lambda} \frac{\lambda^k}{k!}s^{k-1}
= \lambda \e^{-\lambda} \e^{\lambda s}.
\end{align*}
Setzen wir $s=1$, so erhalten wir
\begin{align*}
\E X = \sum\limits_{k=1}k \e^{-\lambda} \frac{\lambda^k}{k!} = g'(1)
= \lambda.\qedhere\bsphere
\end{align*}
\end{proof}
\end{bsp}
\begin{bsp}
Sei $X$ $N(a, \sigma ^{2})$-verteilt, d.h. $X$ besitzt die Dichtefunktion $f:
\R \to \R_{+}$ mit
\begin{align*}
f(x) \defl \frac{1}{\sqrt{2 \pi }\sigma }\,\e^{-\frac{(x-a)^{2}}{2\sigma
^{2}}},\qquad a \in \R,\quad \sigma > 0.
\end{align*}
Es gilt $\E X=a$.
\begin{proof}
Wir wenden Satz \ref{prop:4.4}\ref{prop:4.4:2} an und erhalten,
\begin{align*}
\E X =\int_\R xf(x)\dx
= \int_\R (x-a)f(x)\dx + \int_\R a f(x) \dx.
\end{align*}
$f(x)$ ist symmetrisch bezüglich $x=a$, d.h. der linke Integrand ist
asymmetrisch bezüglich $x=a$ und damit verschwindet das Integral.
\begin{align*}
\Rightarrow \E X = a \int_\R f(x)\dx = a,
\end{align*}
da $f$ Dichte.\qedhere\bsphere
\end{proof}
\end{bsp}
\begin{bsp}
Sei $X$ $\exp(\lambda)$ verteilt mit $\lambda > 0$, so besitzt $X$ die
Dichtefunktion $f$ mit,
\begin{align*}
f(x) =
\begin{cases}
0,& x < 0,\\
\lambda \e^{-\lambda x},& x \ge 0.
\end{cases}
\end{align*}
Es gilt $\E X = \frac{1}{\lambda}$.
\begin{proof}
Eine Anwendung von Satz \ref{prop:4.4}\ref{prop:4.4:2} ergibt,
\begin{align*}
\E X &= \int_\R x f(x)\dx = \int_0^\infty x \lambda \e^{-\lambda x}\dx
\overset{\text{part.int.}}{=}
-x\e^{-\lambda x}\Big|_{0}^\infty + \int_0^\infty \e^{-\lambda x}\dx\\
&= -\frac{1}{\lambda}\e^{-\lambda x}\Big|_0^\infty
= \frac{1}{\lambda}.\qedhere\bsphere
\end{align*}
\end{proof}
%\item
%
%\bsphere
%\end{enumerate}
\end{bsp}
\begin{bsp}
$X$ sei auf $[a,b]$ gleichverteilt, d.h. mit Dichtefunktion $f:\R\to\R_+$,
\begin{align*}
f(x) = 
\begin{cases}
0, & x\notin[a,b],\\
\frac{1}{b-a}, & x\in [a,b].
\end{cases}
\end{align*}
Dann ist $\E X = \frac{a+b}{2}$.
\begin{proof}
$f$ ist eine Rechteckfunktion, ihr Erwartungswert ist die
Intervallmitte.\qedhere\bsphere
\end{proof}
\end{bsp}

\subsection{Der Transformationssatz}

\begin{prop}[Transformationssatz]
\label{prop:4.5}
Sei $X:(\Omega,\AA,P)\to \R$ eine reelle Zufallsvariable mit Verteilung
$P_{X}$ und Verteilungsfunktion $F$, sowie
\begin{align*}
g: (\R, \BB) \to (\R, {\BB}).
\end{align*}
$\E(g \circ X)$ existiert genau dann,
wenn $\int_{\R} g\dP_{X}$ existiert, und es gilt hierbei nach dem
Transformationssatz für Integrale
\begin{align*}
\E(g \circ X) = \int_{\R}g\dP_{X} = : \int_{\R}g\dF\defr
\int_{\R}g(x)\dF(x).
\end{align*}
Spezialfälle sind:
\begin{align*}
\E(g\circ X) =
\begin{cases}
\sum_{k\ge 0}g(k)p_{k}, & \mbox{ unter der Voraussetzung von Satz
\ref{prop:4.4}a},\\ \int_{\R}g(x)f(x)\dlambda(x),
& \mbox{ unter der Voraussetzung von Satz \ref{prop:4.4}b}.
\end{cases}
\end{align*}
Insbesondere gilt (mit $g = \Id_{A}$, wobei $A \in {\BB}$)
\begin{align*}
P[X\in A] &=P_{X}(A) \\ &=
\begin{cases}
\sum_{k \in A\cap \N_{0}} p_{k} & \mbox{ u. V.
von Satz \ref{prop:4.4}a}\\
\int_{\R} \Id_{A} (x)f(x)\dlambda(x) \defr \int_{A}f(x)\,dx &
\mbox{ u. V. von Satz \ref{prop:4.4}b}.\fishhere
\end{cases}
\end{align*}
\end{prop}
\begin{proof}
Wir führen den Beweis nach der \textsc{Standardprozedur}.

\textit{0. Schritt:}
Beweis der Behauptung für $g=\Id_A$ und $A\in\BB$.
\begin{align*}
\int_\R \Id_A(x) \dP_X = P_X(A) = P(X^{-1}(A))
= \int_\Omega X^{-1}(A)\dP = \int_\Omega \Id_A(X) \dP. 
\end{align*}

\textit{1. Schritt:}
$g \ge 0$, wobei $g$ Linearkombination einfacher Funktionen.

\textit{2. Schritt:} $g\ge 0$. Verwende den Satz von der monotonen Konvergenz.

\textit{3. Schritt:} $g=g^+-g^-$.\qedhere
\end{proof}

Wir können somit einen vollständigen Beweis von \ref{prop:4.4}\,\ref{prop:4.4:2}
geben.
\begin{proof}
Sei $X$ reelle Zufallsvariable und $g=\id$, d.h. $g(x)=x$, so gilt
\begin{align*}
\E X = \E(g\circ X) = \int_\R g\dP_X = \int_\R \id \dP_X.
\end{align*}
Die Verteilungsfunktion $F$ erzeugt das Maß $P_X$, somit gilt
\begin{align*}
\int_\R \id \dP_X = \int_\R x\dF(x).  
\end{align*}
Besitzt $X$ die Dichte $f$, d.h. $F'=f\fu{L}$, so gilt
\begin{align*}
\int_\R x\dF(x) = \int_\R x f(x) \dlambda(x).\qedhere 
\end{align*}
\end{proof}

Wir betrachten einige Beispiele zu Satz \ref{prop:4.5}
\begin{bsp}
Ein Schütze trifft einen Punkt des Intervalls $[-1,1]$ gemäß einer
Gleichverteilung und erhält den Gewinn $\frac{1}{\sqrt{\abs{x}}}$, wenn er den
Punkt $x$ trifft, wobei wir $\frac{1}{0}\defl\infty$ gewichten.

\textit{Gesucht ist der mittlere Gewinn des Schützen}. Dazu konstruieren
wir eine Zufallsvariable $X$, die den zufällig getroffenen Punkt angibt. $X$
ist nach Voraussetzung gleichverteilt, d.h. die zugehörige Dichte gegeben durch,
\begin{align*}
f(x) =
\begin{cases}
\frac{1}{2}, & x\in [-1,1],\\
0, & \text{sonst}.
\end{cases}
\end{align*}

\begin{figure}[!htpb]
\centering
\begin{pspicture}(-2.2,-0.7)(2.2,1.7)

 \psaxes[labels=none,ticks=none,linecolor=gdarkgray,tickcolor=gdarkgray]{->}%
 (0,0)(-2,-0.5)(2,1.5)[\color{gdarkgray}$x$,-90][\color{gdarkgray}$f(x)$,0]

\psline[linecolor=darkblue](-1.5,0)(-1,0)(-1,1)(1,1)(1,0)(1.5,0)

\rput(1.4,1.4){\color{gdarkgray}$f$}
\end{pspicture}
\caption{Dichtefunktion zu $X$.}
\end{figure}

Den mittleren Gewinn erhalten wir durch den Erwartungswert,
\begin{align*}
\E(\text{Gewinn}) &= \E \frac{1}{\sqrt{X}}
= \int_\R \frac{1}{\sqrt{\abs{x}}}f(x)\dx
= \frac{1}{2}\int_{[-1,1]} \frac{1}{\sqrt{\abs{x}}}\dx\\
&= \int_0^1 \frac{1}{\sqrt{x}}\dx
= 2\sqrt{x}\bigg|_{0}^1 = 2.\bsphere
\end{align*}
\end{bsp}
\begin{bsp}
Ein Punkt wird rein zufällig aus der Einheitskreisscheibe $K$ von $0$
ausgewählt. Wie groß ist der zufällige Abstand dieses Punktes von $0$?

Um diese Fragestellung zu modellieren, konstruieren wir einen zweidimensionalen
Zufallsvektor $X=(X_1,X_2)$, der auf $K$ gleichverteilt ist. Die Dichtefunktion
$f:\R^2 \to \R_+$ ist dann gegeben durch,
\begin{align*}
f(x_1,x_2) = 
\begin{cases}
\frac{1}{\pi}, & (x_1,x_2)\in K,\\
0, & \text{sonst}.
\end{cases}
\end{align*}
Gesucht ist nun $\E\sqrt{X_1^2 + X_2^2}$. Man arbeitet überlicherweise nicht
auf $\Omega$.
\begin{enumerate}
  \item[] \textit{1. Weg (üblich)}. Anwendung des Transformationssatzes,
\begin{align*}
\E\sqrt{X_1^2 + X_2^2}
&\overset{\ref{prop:4.5}}{=}
\int_{\R^2}
\sqrt{x_1^2+x_2^2}
f(x_1,x_2)\dlambda(x_1,x_2)\\
&\overset{\text{pol.koord.}}{=} \frac{1}{\pi}
\int\limits_{\ph=0}^{2\pi}
\int\limits_{r=0}^1
r\cdot r 
\dlambda(r,\ph)
= \frac{2}{3}.
\end{align*}
\item[] \textit{2. Weg}.
Betrachte die Zufallsvariable $Y=\sqrt{X_1^2+X_2^2}$ mit der
zugehörigen Verteilungsfunktion
\begin{align*}
&F(y) \defl P[Y\le y] = 
\begin{cases}
0, & \text{falls } y \le 0,\\
\frac{1}{\pi}\pi y^2 = y^2, & \text{falls } y\in (0,1),\\
1, & \text{falls } y \ge 1.
\end{cases}
\end{align*}
\begin{figure}[!htpb]
\centering
\begin{pspicture}(-2.2,-1)(2.2,3)

 \psaxes[labels=none,ticks=none,linecolor=gdarkgray,tickcolor=gdarkgray]{->}%
 (0,0)(-2,-0.5)(2,2.5)[\color{gdarkgray}$x$,-90][,0]

\psline[linewidth=1.2pt,linecolor=darkblue](-1.7,0)(0,0)
\psline[linewidth=1.2pt,linecolor=darkblue](1,1)(1.7,1)
\psplot[linewidth=1.2pt,%
	     linecolor=darkblue,%
	     algebraic=true]%
	     {0}{1}{%
x^2 %
}

\psline[linecolor=purple](-1.7,0)(0,0)(1,2)
\psline[linecolor=purple](1,0)(1.7,0)

\rput(1.4,1.4){\color{darkblue}$F$}
\rput(0.8,2.1){\color{purple}$f$}
\end{pspicture}
\caption{Dichtefunktion zu $X$.}
\end{figure}

$F$ ist bis auf den Punkt $y=1$ stetig differenzierbar, es gilt also
\begin{align*}
\E Y =
\int_0^1 y F'(y)\dy = 2\int_0^1 y^2\dy = \frac{2}{3}.
\end{align*}
Welcher Weg (schneller) zum Ziel führt, hängt vom Problem ab.\bsphere 
\end{enumerate}
\end{bsp}

\begin{propn}[Zusatz]
Sei $X$ ein $n$-dimensionaler Zufallsvektor auf $(\Omega, {\cal
A},P)$ und
\begin{align*}
g: (\R^n,\BB_n)\to(\R,\BB),
\end{align*}
so existiert $\E(g\circ X)$ genau dann, wenn $\int_{\R^n} g\dP_X$ existiert. In
diesem Fall gilt
\begin{align*}
\E (g\circ X) = \int_\Omega g\circ X \dP = \int_{\R^n} g \dP_X. 
\end{align*}
Besitzt $X$ außerdem eine Dichtefunktion $f: \R^{n} \to \R_{+}$, so gilt
\begin{align*}
\E (g\circ X) = \int_{\R^n} g(x)f(x)\dlambda(x).
\end{align*}
Insbesondere gilt für $A\in\BB_{n}$,
\begin{align*}
P[X\in A]=P_{X}(A)= \int_{\R^{n}}\Id _{A}(x)f(x)\dx
\defr \int_{A}f(x)\dx.
\end{align*}
[Falls das entsprechende Riemann-Integral existiert, so stimmen Riemann- und
Lebesgue-Integral überein.]\fishhere
\end{propn}
\begin{proof}
Der Beweis erfolgt wie im eindimensionalen Fall. Die letzte
Behauptung folgt sofort für $g=\Id_A$.\qedhere
\end{proof}

\begin{bem}
\label{bem:4.5}
Das Integral $\int_{A}f(x)\dx $ im Zusatz zu Satz \ref{prop:4.4} ist entspricht
dem $\lambda$-Maß der Ordinatenmenge
\begin{align*}
\setdef{(x,y) \in \R^{n+1}}{x \in A,\; 0\leq y \leq f(x)}
\end{align*}
von $f|_{A}$ in $\R^{n+1}$.\maphere
\end{bem}

\section{Momente von Zufallsvariablen}

\begin{defn}
\label{defn:4.5}
Sei $X$ eine reelle Zufallsvariable und $k \in \N$. Im Falle der Existenz
heißt $\E X^{k}$ das \emph{$k$-te  Moment} von $X$ und
$\E(X-\E X)^ {k}$ das \emph{$k$-te  zentrale Moment} von $X$.\\
Ist $X$ integrierbar, dann heißt
\begin{align*}
\V(X)\defl\E(X-\E X)^ {2}
\end{align*}
die \emph{Varianz} von $X$ und
\begin{align*}
\sigma (X) \defl _{_{_{ +}}}\!\! \sqrt{\V(X)}
\end{align*}
die \emph{Streuung} von $X$.\fishhere
\end{defn}

Das erste Moment einer Zufallsvariablen haben wir bereits als Erwartungswert
kennengelernt. Die Varianz $\V(X)$, gibt die ``Schwankung'' der Zufallsvariablen
$X$ als ``mittleren Wert'' ihrer quadratischen Abweichung von $\E X$ an.

Für die Existenz der Varianz ist es nicht notwendig, die quadratische
Integrierbarkeit von $X$ explizit zu fordern, da durch die Integrierbarkeit von
$X$ gesichert ist, dass $(X-\E X)^2$ eine nicht-negative reelle Zufallsvariable
und damit $\E(X-\E X)^2$ definiert ist (wenn auch möglicherweise $\infty$).

\begin{lem}
\label{lem:4.2}
Sei $X$ eine reelle Zufallsvariable und $0\leq \alpha < \beta < \infty $. Dann
gilt
\begin{align*}
\E|X|^{\beta}< \infty \Rightarrow \E|X|^{\alpha}< \infty.\fishhere
\end{align*}
\end{lem}

\begin{proof}
$\E\abs{X}^\alpha \le \E \left(\max\setd{1,\abs{X}^\beta}\right) \le 1 + \E
\abs{X}^\beta < \infty$.\qedhere
\end{proof}

Da $\E X^k$ genau dann existiert, wenn $\E \abs{X}^k$ existiert, impliziert die
Existenz des $k$-ten Moments somit die Existenz aller niedrigeren Momente $\E
X^l$ mit $l\le k$.

\begin{prop}
\label{prop:4.6}
Sei $X$ eine reelle integrierbare Zufallsvariable. Dann gilt
\begin{propenum}
\item
$\V(X) =\E X^ {2}-(\E X)^ {2}$.
\item
$\V(aX+b) = a^ {2}\V(X),\quad a, b \in \R$.\fishhere
\end{propenum}
\end{prop}

Im Gegensatz zum Erwartungswert ist die Varianz ist also insbesondere
\textit{keine} lineare Operation.
\begin{proof}
Den Fall $\E X^2 = \infty$ behandeln wir durch Stutzung von $X$ in Höhe $c$ und
gehen dann zum Grenzwert für $c\to\infty$ über.

Betrachten wir also den Fall $\E X^2 < \infty$.
\begin{proofenum}
  \item $\V(X) = \E(X-\E(X))^2 = \E(X^2-2(\E X) X + (\E X)^2 )$.
Nach Voraussetzung ist $\E X<\infty$ und somit,
\begin{align*}
\V(X) = \E X^2 - 2(\E X)(\E X) + (\E X)^2 = \E X^2 - (\E X)^2.
\end{align*}
\item Unter Verwendung der Linearität von $\E$ erhalten wir sofort,
\begin{align*}
\V(a X + b) &= \E(a X +b - \E(a X + b))^2 = \E(aX +b -a \E X -b)^2 \\ &=
a^2\E(X-\E X)^2
= a^2\V(X).\qedhere
\end{align*}
\end{proofenum}
\end{proof}

\begin{bsp}
$X$ sei $b(n,p)$-verteilt mit $n\in \N$, $p \in [0,1]$. 
$\V(x)=np(1-p)$. Wir werden dafür später einen einfachen Beweis gegeben
können.\bsphere
\end{bsp}
\begin{bsp}
$X$ sei $\pi(\lambda)$-verteilt mit $\lambda >0$.
Die Berechnung der Varianz erfolgt mittels der Erzeugendenfunktion,
\begin{align*}
&g(s) = \sum\limits_{k=0}^n \e^{-\lambda} \frac{\lambda^k}{k!} s^k
= \e^{-\lambda}\e^{\lambda s},\\
&g''(s) = \sum\limits_{k=2}^n k(k-1)\e^{-\lambda} \frac{\lambda^k}{k!} s^{k-2}
= \lambda^2\e^{-\lambda}\e^{\lambda s},\quad s\in[0,1].
\end{align*}
Auswerten ergibt
\begin{align*}
g''(1) = \underbrace{\sum\limits_{k=2}^n k(k-1)\e^{-\lambda}
\frac{\lambda^k}{k!}}_{\ref{prop:4.5} = \E(X(X-1))}
= \lambda^2.
\end{align*}
Somit erhalten wir,
\begin{align*}
&\E X^2 = \E(X(X-1)) + \E X = \lambda^2 + \lambda,\\
&\V X = \E X^2 - (\E X)^2 = \lambda^2 + \lambda - \lambda^2 = \lambda.\bsphere
\end{align*}
\end{bsp}
\begin{bsp}
$X$ sei $N(a,\sigma^ {2})$-verteilt mit $a \in \R$, $\sigma >
0$, dann ist $Y=\frac{1}{\sigma}(X-a)$ $N(0,1)$ verteilt.
Wir können uns also auf den Fall $a=0$, $\sigma=1$ zurückziehen, wobei
man leicht zeigen kann, dass $\E\abs{X}^n < \infty, \forall n\in\N$.

Weiterhin gilt
\begin{align*}
\E X^{2k+1} = 0,\qquad k\in\N_0,
\end{align*}
da der Integrand eine ungerade Funktion ist und daher das Integral
verschwindet. Insbesondere ist $\E X =0$.

Für geraden Exponenten erhalten wir
\begin{align*}
\E X^{2k} &\overset{\ref{prop:4.5}}{=}
\frac{1}{\sqrt{2\pi}}
\int_\R x^{2k} \e^{-\frac{k^2}{2}}\dx\\
&\overset{\text{part. int.}}{=}
\underbrace{\frac{2}{\sqrt{2\pi}} \frac{1}{2k+1}
x^{2k+1}\e^{-\frac{x^2}{2}}\bigg|_0^\infty}_{=0} - \frac{2}{\sqrt{2\pi}}
\int\limits_0^\infty \frac{1}{2k+1} x^{2k+1}(-x)\e^{-\frac{x^2}{2}}\dx\\
&=
\frac{2}{\sqrt{2\pi}}\frac{1}{2k+1}
\int\limits_0^\infty  x^{2k+2}\e^{-\frac{x^2}{2}}\dx
= \frac{1}{2k+1}\E X^{2k+2}.
\end{align*}
Somit gilt für jedes $k\in\N_0$, $\E X^{2k+2} = (2k+1)\E X^{2k}$. Insbesondere
erhalten wir
\begin{align*}
&\E X^0 = \E 1 = 1\quad \Rightarrow \E X^2 = 1
\quad  \Rightarrow \V(X) = 1.
\end{align*}
Für allgemeines $N(a,\sigma^2)$ verteiltes $X$ erhalten wir somit,
\begin{align*}
&\E \left(\frac{1}{\sigma}(X-a)\right) = 0\Rightarrow \E X = a,\\
&\V \left(\frac{1}{\sigma}(X-a)\right) = 1\Rightarrow \V X =
\sigma^2.\bsphere
\end{align*}
\end{bsp}

Die folgenden Ungleichungen haben zahlreiche Anwendungen und gehören zu den
wichtigsten Hilfsmitteln der Stochastik.

\begin{prop}
\label{prop:4.7}
Sei $X$ eine reelle Zufallsvariable auf einem W-Raum $(\Omega, {\cal
A},P)$.
Dann gilt für jedes $\varepsilon >0$, $r>0$:
\begin{align*}
P[| X | \geq \varepsilon ] \leq \varepsilon ^ {-r} \E | X | ^ {r},\qquad
\text{\emph{Markoffsche Ungleichung}}.
\end{align*}

Für $r=2$ erhält man die sog. \emph{Tschebyschevsche Ungleichung}, die bei
integrierbarem $X$ auch in der Variante
\begin{align*}
P[| X-\E X | \geq \varepsilon ] \leq \varepsilon ^ {-2} \V(X)
\end{align*}
angegeben wird.\fishhere
\end{prop}
\begin{proof}
Zum Beweis verwenden wir eine Zufallsvariable $Y$, die $X$ ``stutzt'',
\begin{align*}
Y \defl
\begin{cases}
1, & \text{falls } \abs{X} \ge \ep,\\
0, & \text{sonst}.
\end{cases}
\end{align*}
Offensichtlich ist $Y\le \frac{\abs{X}^r}{\ep^r}$, d.h. $\E Y \le \ep^{-r}\E
\abs{X}^r$, wobei $\E Y = P[\abs{X}\ge \ep]$.\qedhere
\end{proof}

Als Anwendung beweisen wir das Bernoullische schwache Gesetz der großen Zahlen. 

\begin{bsp}
Sei $Y_n$ eine Zufallsvariable, die die Anzahl der Erfolge in $n$
Bernoulli-Versuchen mit jeweiliger Erfolgswahrscheinlichkeit $p\in[0,1]$ (fest)
angibt. $Y_n$ ist also $b(n,p)$-verteilt und es gilt
\begin{align*}
\E Y_n = np,\qquad \V Y_n = np(1-p).
\end{align*}
Die relative Häufigkeit der Anzahl der Erfolge ist gegeben durch
$\dfrac{Y_n}{n}$.

Sei nun $\ep > 0$ beliebig aber fest, so gilt
\begin{align*}
P\left[\abs{\frac{Y_n}{n}-p}\ge \ep\right] \le \frac{\V
\left(\frac{Y_n}{n}\right)}{\ep^2} = \frac{1}{\ep^2n^2}\V(Y_n) =
\frac{p(1-p)}{\ep^2n}\to0,\quad n\to\infty.
\end{align*}
Zusammenfassend erhalten wir das \emph{Bernoullische schwache Gesetz der großen
Zahlen},
\begin{align*}
\forall \ep > 0 : P\left[ \abs{\frac{Y_n}{n} -p} \ge \ep \right] \to 0,\qquad
n\to\infty.
\end{align*}

Eines der Ziele dieser Vorlesung ist die Verallgemeinerung dieses Gesetzes auf
das starke Kolmogorovsche Gesetz der großen Zahlen in Kapitel
\ref{chap:9}.\bsphere
\end{bsp}

% \begin{prop}[Transformationssatz für Dichten]
% \label{prop:4.8}
% [Verallgemeinerung s. Hinderer S. 148]
% Sei $Q$ ein W-Maß  auf ${\BB}_{2}$ und $T:  \R^{2}\to
% \R^{2}$ eine injektive stetig-differenzierbare Abbildung.
% Es sei $R\defl Q_{T}$ das
% Bild-W-Maß von $Q$ bzgl.\ $T$ [d.h. $R(B) =Q(T^ {-1}(B))$, $B\in {\BB}_{2}$]
% und $\Delta $ der Betrag der Funktionaldeterminante von $T$. Hat $R$ die Dichte
% $g$, so hat $Q$ die Dichte $(g \circ T)\Delta$.\fishhere
% \end{prop}

\cleardoublepage
\chapter{Unabhängigkeit}
\label{chap:5}

Es ist eine zentrale Fragestellung der Wahrscheinlichkeitstheorie, inwiefern
sich zufällige Experimente gegenseitig beeinflussen. In \ref{chap:1.d}
haben wir bereits die Bedingte Wahrscheinlichkeit definiert und festgestellt,
dass sich Zufallsexperimente tatsächlich beeinflussen können und damit die
Wahrscheinlichkeit für das Eintreten eines Ereignises vom Eintreten eines
vorangegangen Ereignisses abhängen kann.

In diesem Kapitel wollen wir die Eigenschaften von unabhängigen Ereignissen und
insbesondere unabhängigen Zufallsvariablen studieren. Solche Zufallsvariablen
haben viele angenehme Eigenschaften und lassen sich besonders ``leicht''
handhaben ---
Die wirklich interessanten Experimente sind jedoch gerade
\textit{nicht} unabhängig.

\clearpage

\section{Unabhängige Ereignisse und Zufallsvariablen}
\label{chap:5.a}

Wir betrachten einige Beispiele zum Einstieg.

\begin{bsp}
\textit{Zwei Ausspielungen eines Würfels ``ohne gegenseitige Beeinflussung''}.
  
Betrachte dazu einen laplaceschen W-Raum $(\Omega,\AA,P)$ mit
$\Omega=\setd{1,\ldots,6}^2$ und $\AA=\PP(\Omega)$.

Sei $A$ das Ereignis, dass im ersten Wurf eine 1 erzielt wird, $B$ das
Ereignis, dass im zweiten Wurf eine gerade Zahl erscheint. Einfaches Abzählen
ergibt,
\begin{align*}
&P(A) = \frac{6}{36} = \frac{1}{6},\\
&P(B) = \frac{18}{36} = \frac{1}{2}.
\end{align*}
Die Wahrscheinlichkeit, dass beide Ereignisse gleichzeitig eintreten ist,
\begin{align*}
P(A\cap B) = \frac{3}{36} = \frac{1}{12} = \frac{1}{6}\cdot\frac{1}{2} =
P(A)\cdot P(B).
\end{align*}
Eine solche Situation ($P(A\cap B)=P(A)\cdot P(B)$) ist typisch bei Ereignissen,
bei denen es physikalisch keine gegenseite Beeinflussung gibt.\bsphere
\end{bsp}


Wir werden den Fall, dass die Wahrscheinlichkeit des gemeinsamen Eintretens mit
dem Produkt der einzelnen Wahrscheinlichkeiten übereinstimmt als Grundlage für
die Definition der stochastischen Unabhängigkeit verwenden. 
Es ist wichtig, zwischen ``stochastisch unabhängig'' und ``physikalisch
unabhängig'' zu unterscheiden, denn im Allgemeinen lassen sich aus der
stochastischen Unabhänigkeit keine Rückschlüsse auf die physikalische machen.

Sei $(\Omega,\AA,P)$ ein W-Raum, $A$ und $B\in\AA$ mit $P(B)>0$, dann gilt
\begin{align*}
P(A\cap B) = P(A\mid B)P(B).
\end{align*}
Falls $A$ von $B$ ``unabhängig'' ist, muss gelten $P(A\mid B)=P(A)$ und damit
\begin{align*}
P(A\cap B) = P(A)P(B).
\end{align*} 

\begin{defn}
\label{defn:5.1}
Sei $(\Omega,\AA, P)$ ein W-Raum. Eine Familie $\setdef{A_{i}}
{i \in I}$ von Ereignissen $A_{i} \in \AA$ heißt \emph{unabhängig}
(ausführlich: stochastisch unabhängig bzgl. $P$), falls für jede nichtleere
endliche Menge $K\subset I$ gilt
\begin{align*}
P\left(\bigcap_{k\in K }A_{k}\right)
  = \prod\limits_{k\in K}P(A_{k}).\fishhere
\end{align*}
\end{defn}

In dieser Definition werden keine Anforderungen an die Indexmenge $I$ gemacht,
unendliche (z.B. überabzählbare) Indexmengen sind also durchaus zugelassen.

Im Folgenden sei mit ``unabhängig'' stets stochastisch unabhängig gemeint. Auf
logische oder physikalische Unabhängigkeit lassen sich aus der stochastischen
Unabhängigkeit im Allgemeinen keine Rückschlüsse machen.

\begin{defn}
\label{defn:5.2}
Sei $(\Omega,  \AA,P)$ ein W-Raum. Eine Familie $\setdef{X_{i}}
{i\in I}$ von $(\Omega_{i}, \AA_{i})$ - Zufallsvariablen auf $(\Omega, \AA, P)$
heißt \emph{unabhängig}, wenn gilt: Für jede nichtleere endliche Indexmenge
$\setd{i_{1},\ldots, i_{n}}\subset I$ und jede Wahl von Mengen
$A_{i_{\nu}}\in \AA_{i_{\nu}}$ $(\nu = 1,\ldots,n)$ ist
\begin{align*}
P\left[ X_{i_{1}} \in A_{i_{1}},\ldots,X_{i_{n}} \in A_{i_{n}}\right] =
\prod\limits_{\nu =1}^ {n} P\big[X_{i_{\nu}} \in A_{i_{\nu}}\big].
\end{align*}
Sprechweise: Unabhängigkeit der Zufallsvariablen statt Unabhängigkeit der Familie der
Zufallsvariablen.\fishhere
\end{defn}

Für die Unabhängigkeit von Zufallsvariablen ist es somit nicht notwendig
explizit zu fordern, dass alle Zufallsvariablen den selben Wertebereich
teilen. Die Voraussetzung, dass alle auf dem selben W-Raum
$(\Omega,\AA,P)$ definiert sind, lässt sich jedoch nicht abschwächen.

\begin{lem}
\label{lem:5.1}
Eine Familie von Ereignissen ist genau dann unabhängig,
wenn jede endliche Teilfamilie unabhängig ist. Entsprechendes gilt für
Zufallsvariablen.\fishhere
\end{lem}
\begin{proof}
Die Aussage folgt direkt aus den Definitionen \ref{defn:5.1} und
\ref{defn:5.2}.\qedhere
\end{proof}

\begin{bem}[Bemerkung zu Definition \ref{defn:5.1} bzw. Definition
\ref{defn:5.2}.]
\label{bem:5.1}
Die paarweise Unabhängigkeit impliziert im Allgemeinen
\textit{nicht} die Unabhängigkeit.\maphere
\end{bem}
\begin{proof}[Beweis durch Gegenbeispiel.]
Betrachte zwei Ausspielungen eines echten Würfels ohne gegenseitige
Beeinflussung. Dieses Experiment können wir durch einen Laplaceschen W-Raum
$(\Omega,\AA,P)$ mit $\abs{\Omega}=36$ modellieren.

Sei $A_i$ das Ereignis, dass im $i$-ten Wurf eine ungerade Zahl auftritt
$P(A_i)=\frac{1}{2}$ und $B$ das Ereignis, dass die Summe der Augenzahlen
ungerade ist $P(B)=\frac{1}{2}$.
\begin{align*}
P(A_1\cap A_2) = \frac{9}{36} = \frac{1}{4},\quad
P(A_1\cap B) = \frac{9}{36} = \frac{1}{4},\quad
P(A_2\cap B) = \frac{9}{36} = \frac{1}{4},
\end{align*}
somit sind die Ereignisse paarweise unabhängig aber
\begin{align*}
P(A_1\cap A_2 \cap B) = P(\varnothing) = 0 \neq
\frac{1}{2}\frac{1}{2}\frac{1}{2} = \frac{1}{8}.\qedhere
\end{align*}
\end{proof}

\begin{bem}
\label{bem:5.2}
Sei $(\Omega, \AA,P)$ ein W-Raum und  $A_{i}\in {\cal
A}$, $i\in I$ Ereignisse, mit (reellen) Zufallsvariablen und
Indikatorfunktionen $\Id_{A_{i}}$ auf $(\Omega, \AA, P)$.
\begin{align*}
\setdef{ A_{i} }{ i\in I } \mbox{ unabhängig } \Leftrightarrow \setdef{\Id
_{A_{i}}}{ i\in I } \mbox{ unabhängig }.\maphere
\end{align*}
\end{bem}

\subsection{Unabhängige Zufallsvariablen}

\begin{prop}
\label{prop:5.1}
Gegeben seien $(\Omega_{i}, \AA_{i})$ - Zufallsvariablen $X_{i} $ auf
einem W-Raum $(\Omega, \AA, P)$ und $g_{i}:(\Omega _{i}, {\cal
A}_{i}) \to (\Omega'_{i},\AA'_{i})$, $ i\in I$ Abbildungen. Dann gilt:
\begin{align*}
\setdef{X_{i}}{i\in I} \mbox{ unabhängig } \Rightarrow \setdef{g_{i}\circ
X_{i}}{i\in I} \mbox{ unabhängig }\fishhere
\end{align*}
\end{prop}
\begin{proof}
Aufgrund von Lemma \ref{lem:5.1} genügt es den Beweis auf einer endlichen
Indexmenge $I=\setd{1,\ldots,n}$ zu führen.
\begin{align*}
&P\left[g_1\circ X_1 \in A_1',\ldots,g_n\circ X_n\in A_n'\right]
=
P\left[ \forall {i\in\setd{1,\ldots,n}} : X_i\in g_i^{-1}(A_i') \right]\\
&\quad
= \prod\limits_{i=1,\ldots,n} P\left[X_i\in g_i^{-1}(A_i')\right] = 
\prod\limits_{i=1,\ldots,n} P\left[g_i\circ X_i \in A_i'\right],
\end{align*}
da die $X_i$ nach Voraussetzung unabhängig sind.\qedhere
\end{proof}

Sind also die Zufallsvariablen $\setd{X_i}$ unabhängig, so sind es
insbesondere auch $\setd{X_i^2}$, $\setd{\sqrt{X_i}}$, \ldots

\begin{bsp}
Betrachte erneut zwei Ausspielungen eines echten Würfels ohne gegenseitige
Beeinflussung mit W-Raum $(\Omega,\AA,P)$.

Sei $X_{1},X_2$ die Augenzahl im 1. bzw. 2. Wurf. Da $\setd{X_1,X_2}$
unabhängig, ist auch $\setd{X_1^2,X_2^3}$ unabhängig.\bsphere
\end{bsp}

\begin{prop}
\label{prop:5.2}
Sei eine unabhängige Familie $\setdef{X_{i}}{i\in I}$ reeller
Zufallsvariablen auf einem W-Raum $(\Omega, \AA, P)$ und eine Zerlegung
$I=\sum_{j\in J} I_{j} $ mit $|I_{j}| < \infty$ gegeben.
Das $|I_{j}|$-Tupel $\setdef{X_{k}}{k\in I_{j}}$ sei mit $Y_{j}$ bezeichnet für
$j\in J$.
\begin{propenum}
\item
Die Familie $\setdef{Y_{j}}{j\in J}$ ist unabhängig.
\item
Sind Abbildungen $g_{j} :(\R^{\abs{I_{j}}}, {\BB}_{\abs{
I_{j}}} )\to (\R, {\BB})$, $ j\in J$, gegeben,
so ist die Familie $\setdef{g_{j} \circ Y_{j}}{j\in J}$ unabhängig.\fishhere
\end{propenum}
\end{prop}
\begin{proof}
\begin{proofenum}
  \item \textit{Beweisidee}. Betrachte den Spezialfall $Y_1=X_1$,
  $Y_2=(X_2,X_3)$. Nach Annahme gilt, dass die $X_i$ unabhängige reelle
  Zufallsvariablen sind. Zu zeigen ist nun für $B_1\in\BB, B_2\in\BB_2$
\begin{align*}
P\left[Y_1 \in B_1,\ Y_2\in B_2\right] = P[Y_1\in B_1]\cdot P[Y_2\in B_2].
\end{align*}
Dabei genügt es nach Satz \ref{prop:1.4}, die Aussage auf Intervallen
$I_1\in\BB_1$ bzw. Rechtecken $I_2\times I_3\in\BB_2$ nachzuweisen. Also
\begin{align*}
& P\left[X_1\in I_1, (X_2,X_3)\in I_2\times I_3\right]
= P\left[X_1\in I_1, X_2\in I_2, X_3\in I_3 \right]\\
&\quad= P\left[X_1\in I_1\right]P\left[X_2\in I_2, X_3\in I_3 \right]\\
&\quad= P\left[X_1\in I_1\right]P\left[(X_2,X_3)\in I_2\times I_3\right].
\end{align*}
\item Anwendung von Satz \ref{prop:5.1}.\qedhere
\end{proofenum}
\end{proof}
 
\begin{bem}
\label{bem:5.3}
Satz \ref{prop:5.2} lässt sich auf $(\Omega_{i}, \AA_{i})$ - Zufallsvariablen
und Abbildungen $ g_{j}: \bigotimes_{i\in I_{j}}(\Omega_{i},
\AA_{i})\to (\Omega'_{j}, \AA'_{j})$ verallgemeinern.\maphere
\end{bem}

\begin{prop}
\label{prop:5.3}
Seien $X_{i}$ $(i=1,\ldots,n)$ reelle Zufallsvariablen auf einem W-Raum
$(\Omega, \AA,P)$ mit Verteilungsfunktionen $F_{i}$. Der Zufallsvektor
$X\defl (X_{1},\ldots ,X_{n})$ habe die $n$-dimensionale Verteilungsfunktion\ F.\\[-7mm]
\begin{propenum}
\item
$\setd{X_{1},\ldots ,X_{n}}$ ist genau dann unabhängig, wenn gilt
\begin{align*}
\forall {(x_{1},\ldots, x_{n})\in \R^ {n}} : F(x_{1},\ldots, x_{n}) =
\prod\limits^ {n}_{i=1} F_{i}(x_{i})\tag{*}
\end{align*}
\item
Existieren  zu $F_{i}$ $(i=1,\ldots,n)$ bzw. $F$, Dichten $f_{i}$ bzw.\ $f$,
dann gilt
\begin{align*}
\text{(*)}
\Leftrightarrow 
f(x_{1},\ldots,x_{n})=\prod\limits_{i=1}^ {n} f_{i}(x_{i})
\text{ für $\lambda$-fast alle }(x_{1},\ldots, x_{n}) \in
\R^n.\fishhere
\end{align*}
\end{propenum}
\end{prop}
\begin{proof}
\begin{proofenum}
  \item Seien $\setd{X_1,\ldots,X_n}$ unabhängig, so ist 
  Definition \ref{defn:5.2} äquivalent zu
\begin{align*}
&\Leftrightarrow \forall B_1,\ldots,B_n\in\BB\quad:  &&P\left[X_1\in
B_1,\ldots,X_n\in B_n\right]\\ &&&\quad = P[X_1\in B_1]\cdots P[X_n\in B_n],\\
&\Leftrightarrow
\forall x_1,\ldots,x_n\in \R\quad: && P\left[X_1\le x_1,\ldots,X_n\le x_n\right]
\\ &&&\quad = P[X_1\le x_1]\cdots P[X_n\le x_n]\\
&\Leftrightarrow
\forall x_1,\ldots,x_n\in \R\quad: && F(x_1,\ldots,x_n) = F_1(x_1)\ldots
F_n(x_n).\qedhere
\end{align*}
\end{proofenum}
\end{proof}

Unabhänige Zufallsvariablen haben somit die angenehme Eigenschaft, dass die
\emph{gemeinsame Dichte} (die Dichte der gemeinsamen Verteilung) dem Produkt
der \emph{Randverteilungsdichten} (der Dichten der Randverteilungen) entspricht.
Insbesondere ist für unabhänige Zufallsvariablen die gemeinsame Verteilung
\textit{eindeutig} durch die Randverteilungen bestimmt.

\begin{bsp}
Seien $X,Y$ unabhängig, wobei $X$ $\exp(\lambda)$ und $Y$ $\exp(\tau)$
verteilt, so besitzen sie die Dichten
\begin{align*}
f_X(x) = \lambda \e^{-\lambda x},\qquad
f_Y(y) = \tau \e^{-\tau x}.
\end{align*}
Die gemeinsame Dichte ist nach Satz \ref{prop:5.3} gegeben durch
\begin{align*}
f(x,y) = \lambda\tau \e^{-\lambda x}\e^{-\tau y}.
\end{align*}
Betrachten wir andererseits die gemeinsame Verteilung, so gilt
\begin{align*}
F(s,t) = P[X\le s, Y\le t] \overset{!}{=} P[X\le s]P[Y\le t]
\end{align*}
da $X$ und $Y$ unabhängig. Weiterhin ist
\begin{align*}
P[X\le s]P[Y\le t] &= F_X(s)F_Y(t) = 
\left(\int_0^s \lambda \e^{-\lambda x}\dx\right)
\left(\int_0^t \tau \e^{-\tau y}\dy\right)\\
&= 
\int_0^s \int_0^t \underbrace{ \lambda\tau \e^{-\lambda x}\e^{-\tau
y}}_{f(x,y)}\dx\dy.\bsphere
\end{align*}
\end{bsp}

\begin{prop}
\label{prop:5.4}
Seien $X_{1},\ldots, X_{n}$ unabhängige reelle Zufallsvariablen auf einem
W-Raum $(\Omega, \AA,P)$ mit endlichen Erwartungswerten. Dann existiert
\begin{align*}
\E(X_{1}\cdot \ldots\cdot X_{n})= \E X_{1} \cdot \ldots \cdot \E X_{n}.
\end{align*}
--- Entsprechendes gilt für komplexe Zufallsvariablen, wobei wir $\C$ mit dem
$\R^2$ identifizieren und komplexe Zufallsvariablen als $X_{k}:(\Omega,
\AA)\to (\R^ {2}, {\cal B}_{2})$ mit
\begin{align*}
\E X_{k}\defl \E \mbox{ Re } X_{k}+i\E\mbox{ Im } X_{k} \quad
(k=1,\ldots,n).\fishhere
\end{align*}
\end{prop}
\begin{proof}
Wir beschränken uns zunächst auf reelle Zufallsvariablen. Mit Satz
$\ref{prop:5.2}$ ist eine Reduktion auf den Fall $n=2$ möglich. Zu zeigen ist
also, dass $\E (X_1 X_2) = (\E X_1)(\E X_2)$.

Ohne Einschränkung sind $X_1,X_2\ge 0$, ansonsten betrachten wir eine Zerlegung
in $X_1=X_{1,+}-X_{1,-}$. Mit Hilfe des des Satzes von der monotonen
Konvergenz können wir uns auf einfache Funktionen der Form
\begin{align*}
x\mapsto  \sum\limits_{j=1}^n \lambda_j \Id_{A_j}(x)
\end{align*}
zurückziehen. Wir beweisen die Aussage nun für Indikatorfunktionen $X_1 =
\Id_{A_1}$, $X_2=\Id_{A_2}$ mit $A_1,A_2\in\AA$.
\begin{align*}
\E (X_1 X_2) &= \E(\Id_{A_1}\Id_{A_2}) = \E (\Id_{A_1\cap A_2}) = P(A_1\cap
A_2) = P(\Id_{A_1}=1, \Id_{A_2}=1)\\ & = P(\Id_{A_1}=1)P(\Id_{A_2}=1)
= P(A_1)P(A_2)  = (\E \Id_{A_1})(\E \Id_{A_2}) \\ &= (\E X_1)(\E X_2).\qedhere  
\end{align*}
\end{proof}

Der Erwartungswert ist somit linear und vertauscht mit der Multiplikation.

\begin{prop}[Satz von Bienaymé]
\label{prop:5.5}
Seien $X_{1}, \ldots, X_{n}$ paarweise unabhängige reelle Zufallsvariablen mit
endlichen Erwartungswerten. Dann gilt
\begin{align*}
\V\left( \sum^ {n}_{j=1}X_{j}\right )=\sum^ {n}_{j=1} \V(X_{j})\, .\fishhere
\end{align*}
\end{prop}
\begin{proof}
Addieren wir zum Erwartungswert eine Konstante, ändert sich die Varianz nicht,
wir können also ohne Einschränkung annehmen $\E X_j=0$ ($j=1,\ldots,n$).
\begin{align*}
\V\left(\sum\limits_{j=1}^n X_j\right)
&= \E\left(\sum_{j=1}^n X_j\right)^2
= \sum\limits_{i=1}^n \E X_i^2 + 2\sum\limits_{i<j} \E(X_iX_j)\\
&=\sum\limits_{i=1}^n \E X_i^2 + 2\underbrace{\sum\limits_{i<j}
\E(X_i)\E(X_j)}_{=0} = \sum\limits_{i=1}^n \E X_i^2
= \sum\limits_{i=1}^n \V X_i.\qedhere
\end{align*}
\end{proof}

Für unabhängige Zufallsvariablen vertauscht die Varianz mit \textit{endlichen}
Summen, wobei für reelle Konstanten $a,b\in\R$ und eine Zufallsvariable $X$ gilt
\begin{align*}
\V(aX +b) = a^2\V(X).
\end{align*}
Dies ist kein Widerspruch, denn setzen wir $Y\equiv b$, so ist $Y$
Zufallsvariable und $\E Y = b$. Somit ist $\V Y = \E Y^2 - (\E Y)^2 = b^2 - b^2
= 0$, also
\begin{align*}
\V(aX +b) = \V(aX+Y) = \V(aX)+\V(Y) = a^2\V(X).
\end{align*}

\begin{bem}
\label{bem:5.4}
In Satz \ref{prop:5.5} genügt statt der Unabhängigkeit von
$\setd{X_{1},\ldots,X_{n}}$ die sog. \emph{paarweise Unkorreliertheit} zu
fordern, d.h.
\begin{align*}
\forall{j\neq k} : \E((X_{j}-\E X_{j}) (X_{k}-\E X _{k}))=0.
\end{align*}
Dies ist tatsächlich eine schwächere Voraussetzung, denn sind $X$ und $Y$
unabhängig, so gilt
\begin{align*}
\E((X-\E X)(Y-\E Y)) &= \E(X\cdot Y - (\E X)Y- (\E Y)X + (\E X)(\E Y))\\
&= \E(X\cdot Y) - (\E X)(\E Y) - (\E Y)(\E X) + (\E X)(\E Y))\\
&= \E(X)(\E Y) - (\E X)(\E Y) = 0.\maphere
\end{align*}
\end{bem}

\subsection{Faltungen}

Sind $X, Y$ zwei unabhängige reelle Zufallsvariablen auf einem
W-Raum $(\Omega, \AA, P)$ mit Verteilungen $P_{X}, P_{Y}$, so ist nach
Satz \ref{prop:5.3} und Satz \ref{prop:2.1} die Verteilung $P_{(X,Y)}$ des
$2$-dimensionalen Zufallsvektors durch $P_X$ und $P_Y$ eindeutig festgelegt.

Setzen wir $g : (x,y)\mapsto x+y$, so ist $g$ messbar und damit auch
die Verteilung $P_{X+Y}=P_{g\circ(X,Y)}$ von $X+Y$ eindeutig durch $P_X$ und
$P_Y$ festgelegt.

\textit{Wie ermittelt man nun diese Verteilung?} 

\begin{defn}
\label{defn:5.3}
Sind $X, Y$ zwei unabhängige reelle Zufallsvariablen auf einem
W-Raum $(\Omega, \AA, P)$ mit Verteilungen $P_{X}, P_{Y}$, so wird die
Verteilung $P_{X+Y} \defr P_{X} \ast P_{Y}$ der Zufallsvariable $X+Y$ als
\emph{Faltung} von $P_{X}$ und $P_{Y}$ bezeichnet.\fishhere
\end{defn}

\begin{bem}
\label{bem:5.5}
Die Faltungsoperation ist kommutativ und assoziativ.\maphere
\end{bem}

\begin{prop}
\label{prop:5.6}
Seien $X,Y$ zwei unabhängige reelle Zufallsvariablen.\\[-7mm]
\begin{propenum}
\item
Besitzen $X$ und $Y$ Dichten $f$ bzw. $g$, so besitzt $X+Y$ eine [als Faltung
von $f$ und $g$ bezeichnete] Dichte $h$ mit
\begin{align*}
h(t) =\int_{\R}f(t-y)g(y)\dy =
\int_{\R}g(t-x)f(x) \dx,\quad t \in \R.
\end{align*}
\item
Besitzen $X$ und $Y$ Zähldichten $(p_{k})_{k\in \mathbb{N}_{0}}$ bzw.
$(q_{k})_{k\in \mathbb{N}_{0}}$, so besitzt $X+Y$ eine [als Faltung von
$(p_{k})$ und $(q_{k})$ bezeichnete] Zähldichte $(r_{k})_{k\in \mathbb{N}_{0}}$
mit
\begin{align*}
r_{k}=\sum_{j=0}^ {k} p_{k-j}q_{j}=\sum\limits^ {k}_{i=0}q_{k-i}p_{i}\, , \;
k\in \mathbb{N}_{0}\, .\fishhere
\end{align*}
\end{propenum}
\end{prop}

%TODO: Faltungsbild
Den Beweis verschieben wir auf das nächste Kapitel, denn mit Hilfe der
charakteristischen Funktionen werden wir über die für einen einfachen Beweis
nötigen Mittel verfügen.


\begin{bsp}
Sei $X_1$ eine $b(n_1,p)$-verteilte und $X_2$ eine $b(n_2,p)$-verteilte
Zufallsvariable. Sind $X_1$ und $X_2$ unabhängig, dann ist $X_1+X_2$ eine
$b(n_1+n_2,p)$-verteilte Zufallsvariable.

\begin{proof}[Einfacher Beweis.]
Betrachte $n_1+n_2$ Bernoulli-Versuche mit jeweiliger Erfolgswahrscheinlichkeit
$p$. Sei nun $Y_1$ die Zufallsvariable, die die
Anzahl der Erfolge der ersten $n_1$ Versuche angibt und $Y_2$ die
Zufallsvariable, die die Anzahl der Erfolge in den letzten $n_2$ Versuchen
angibt. Somit sind $Y_1$ und $Y_2$ unabhängig und $Y_1+Y_2$ gibt die Anzahl der
Erfolge in $n_1+n_2$ Bernoulli-Versuchen an.

Scharfes Hinsehen liefert $P_{X_1}=P_{Y_1}$ und $P_{X_2}=P_{Y_2}$, dann gilt
auch
\begin{align*}
P_{X_1+X_2}=P_{Y_1+Y_2}=b(n_1+n_2,p).\qedhere
\end{align*}
Somit folgt für unabhängige Zufallsvariablen $X_1,\ldots,X_n$, die
$b(1,p)$-verteilt sind, dass
$X_1+\ldots+X_n$ eine $b(n,p)$-verteilte Zufallsvariable ist.\bsphere
\end{proof}
\end{bsp}

\begin{bsp}
Wir können jetzt einen einfachen Beweis dafür geben, dass eine
$b(n,p)$-verteilte Zufallsvariable die Varianz $np(1-p)$ besitzt. Ohne
Einschränkung lässt sich diese Zufallsvariable als Summe von $n$ unabhängigen
$b(1,p)$-verteilten Zufallsvariablen $X_1,\ldots,X_n$ darstellen,
\begin{align*}
\V(X_1+\ldots+X_n) \overset{\ref{prop:5.5}}{=} n\V(X_1) = np(1-p),
\end{align*}
da $\V(X_1) = \E X_1^2 - (\E X_1)^2 = 0^2\cdot (1-p) + 1^2\cdot p - p^2 =
p(1-p)$.\bsphere
\end{bsp}

Die Unabhängigkeit der Zufallsvariablen ist für die Aussage des Satzes
\ref{prop:5.6} notwendig, denn er basiert darauf, dass für unabhängige
Zufallsvariablen die gemeinsame Dichte das Produkt der Randverteilungsdichten ist. Für
allgemeine (nicht unabhängige) Zufallsvariablen wird diese Aussage und damit der
Satz falsch.


\begin{bem}[Bemerkung (Satz von Andersen und Jessen).]
\label{bem:5.6}
Gegeben seien Messräume $(\Omega_{i}, \AA_{i})$ und
W-Maße $Q_{i}$ auf $\AA_{i}$, $i\in I$. Dann existiert ein W-Raum $(\Omega,
{\cal A }, P)$ und $(\Omega_{i}, \AA_{i})$-Zufallsvariablen $X_{i}$ auf $(\Omega,
\AA, P)$, $i \in I$, mit Unabhängigkeit von $\setdef{X_{i}}{i \in I}$ und
$P_{X_{i}}=Q_{i}$.\maphere
\end{bem}

\section{Null-Eins-Gesetz}

In Kapitel \ref{chap:1.b} haben wir das 1. Lemma von Borel und Cantelli
bewiesen, welches für einen W-Raum $(\Omega,\AA,P)$ und $A_n\in\AA$ die Aussage
macht
\begin{align*}
\sum\limits_{n=1}^n P(A_n)<\infty\Rightarrow P(\limsup A_n) = 0.
\end{align*}

Das 2. Lemma von Borel und Cantelli ist für unabhängige Ereignisse 
das Gegenstück zu dieser Aussage.

\begin{prop}[2.\ Lemma von Borel und Cantelli]
\label{prop:5.7}
Sei $(\Omega, \AA,P)$ ein W-Raum.
Die Familie \\$\setdef{A_{n}}{n \in \N}$ von Ereignissen $(\in \AA)$
sei unabhängig. Dann gilt:
\begin{align*}
\sum\limits_{n=1}^ {\infty} P(A_{n}) = \infty
\Rightarrow
P(\limsup A_{n})= 1.
\end{align*}
In diesem Fall tritt $A_n$ P-f.s. unendlich oft auf.\fishhere
\end{prop}
\begin{proof}
Sei $B\defl\limsup_n A_n \defl \bigcap_{n=1}^\infty \bigcup_{k\ge n} A_k$. Zu zeigen
ist nun, dass $P(B) = 1$, d.h. $P(B^c) = 0$. Mit den De-Morganschen Regeln folgt,
\begin{align*}
P(B^c) = P\left(\bigcup_{n=1}^\infty\bigcap_{k\ge n} A_k^c \right).
\end{align*}
Aufgrund der $\sigma$-Additivität genügt es nun zu zeigen, dass die Mengen
$\bigcap_{k\ge n} A_k^c$ das $P$-Maß Null haben.

Sei also $n\in\N$ fest. 
\begin{align*}
\bigcap_{k=n}^N A_k^c \downarrow \bigcap_{k\ge n} A_k^c,\qquad N\to\infty.
\end{align*}
Somit genügt es wegen der Stetigkeit der W-Maße von oben zu zeigen, dass
\begin{align*}
P\left(\bigcap_{k=n}^N A_k^c\right) \downarrow 0.
\end{align*}
Aufgrund der Unabhängigkeit der $A_1,A_2,\ldots$ gilt
\begin{align*}
P\left(\bigcap_{k=n}^N A_k^c\right) 
= \prod\limits_{k=n}^N P\left(A_k^c\right)
= \prod\limits_{k=n}^N (1-P\left(A_k\right)).
\end{align*}
\begin{figure}[!htpb]
\centering
\begin{pspicture}(-1.7,-1.5)(2.7,3.2)

 \psaxes[labels=none,ticks=none,linecolor=gdarkgray,tickcolor=gdarkgray]{->}%
 (0,0)(-1.5,-1.3)(2.5,3)[\color{gdarkgray}$x$,-90][,0]

\psplot[linewidth=1.2pt,%
	     linecolor=darkblue,%
	     algebraic=true]%
	     {-1.5}{2}{%
1-x %
}
\psplot[linewidth=1.2pt,%
	     linecolor=purple,%
	     algebraic=true]%
	     {-1}{2.2}{%
(2.71828)^(-x) %
}

\rput(2,0.6){\color{purple}$\e^{-x}$}
\rput(1,-0.8){\color{darkblue}$1-x$}
\end{pspicture}
\caption{$\e^{-x}$ mit Tangente $1-x$ in $(0,1)$.}
\end{figure}

Eine geometrische Überlegung zeigt $1-x\le \e^{-x}$ (siehe Skizze), also gilt
auch $1-P\left(A_k\right) \le \e^{-P(A_k)}$ und somit
\begin{align*}
P\left(\bigcap_{k=n}^N A_k^c\right)  \le \e^{-\sum\limits_{k=n}^N P(A_k)}  \to
0,\qquad N\to\infty.\qedhere
\end{align*}
\end{proof}

\begin{corn}
Sind die Ereignisse $A_1,A_2,\ldots$ unabhängig, gilt
\begin{align*}
P(\limsup A_n) = 0\text{ oder }1,
\end{align*}
abhängig davon ob
\begin{align*}
\sum\limits_{n\ge 1} P(A_n) < \infty\text{ oder }
\sum\limits_{n\ge 1} P(A_n) = \infty.\fishhere 
\end{align*}
\end{corn}


Die Voraussetzung der Unabhängigkeit der $(A_n)$ im 2. Lemma von Borel-Cantelli
ist notwendig, damit die Aussage gilt, wie folgendes Beispiel zeigt.
\begin{bsp}
Sei $A\in\AA$ mit $P(A) =\frac{1}{2}$ und $\forall n\in\N : A_n = A$. Dann gilt
$\limsup A_n = A$ und $\sum_{n\ge 1} P(A_n) = \infty$. Dennoch ist
\begin{align*}
P\left(\limsup A_N\right) = P(A) = \frac{1}{2}.\bsphere
\end{align*}
\end{bsp}
\begin{bsp}
Betrachte eine unabhängige Folge von Versuchen $A_n$ mit
Misserfolgswahrscheinlichkeit $\frac{1}{n}$. Da
\begin{align*}
\sum_{n\ge 1}P\left(A_n\right) = \infty,
\end{align*}
folgt, dass die Wahrscheinlichkeit für unendlich viele Misserfolge gleich $1$
ist (obwohl $P(A_n)\to 0$).\bsphere
\end{bsp}

\begin{defn}
\label{defn:5.4}
Sei $(\Omega, \AA)$ ein Messraum und $\AA_{n} \subset \AA$  $\sigma$-Algebren
für $n\in \N$. Mit
\begin{align*}
\TT_{n} \defl \FF\left({\bigcup^
{\infty}_{k=n}} \AA_{k}\right)
\end{align*}
bezeichnen wir die von $\AA_{n}, {\AA}_{n+1}, \ldots $ erzeugte
$\sigma$-Algebra.
\begin{align*}
\TT_{\infty}\defl {\bigcap^ {\infty}_{n=1}} {\TT}_{n}
\end{align*}
heißt die $\sigma$-Algebra der \emph{terminalen Ereignisse (tail events)}
der Folge $(\AA_{n})$.\fishhere
\end{defn}

\begin{bsp}
Seien $X_{n}: (\Omega, \AA)\to (\R, {\BB})$ und
$\AA_{n} \defl {\FF}(X_{n}) \defl X^ {-1}_{n}({\BB})$ für $n\in\N$.
$\TT_n$ ist somit die von den Ereignissen, die lediglich von $X_n$,
$X_{n+1}$, \ldots\ abhängen, erzeugte $\sigma$-Algebra.
\begin{bspenum}
\item
$\left[(X_{n}) \text{ konvergiert}\right]\in {\TT}_{\infty}$, denn es gilt
\begin{align*}
\left[(X_{n}) \text{ konvergiert}\right] &=
\left[\lim_{n\to\infty} X_{n} = X\in \R\right]
=
\left[\lim_{n\to\infty} \abs{X_{n} - X} = 0 \right]\\
&=
\left[\forall \ep > 0\exists N\in\N \forall n\ge N : \abs{X_{n} - X} <\ep
\right]
\end{align*}
$\Q$ liegt dicht in $\R$, wir können uns also für die Wahl von $\ep$ auf $\Q$
zurückziehen,
\begin{align*}
\left[\forall \ep\in\Q > 0\exists N\in\N \forall n\ge N : \abs{X_{n} - X}
<\ep\right]
\end{align*}
da alle $\ep >1$ uninteressant sind, können wir auch schreiben
\begin{align*}
&\left[\forall k\in\N \exists N\in\N \forall n\ge N : \abs{X_{n} - X}
<\frac{1}{k}\right]\\
&= 
\bigcap_{k\in\N}
\left[\exists N\in\N \forall n\ge N : \abs{X_{n} - X}
<\frac{1}{k}\right]
\end{align*}
und da es auf endlich viele $N$ nicht ankommt,
\begin{align*}
=&\bigcap_{k\in\N}\bigcup_{N\ge k}
\left[\forall n\ge N : \abs{X_{n} - X} <\frac{1}{k}\right]\\
=&\bigcap_{k\in\N}\underbrace{\bigcup_{N\ge k}
\left[\sup_{n\ge N} \abs{X_{n} - X} <\frac{1}{k}\right]}_{\in
\TT_k}.
\end{align*}
Also handelt es sich um ein terminales Ereignis.
\item
$\left[\sum_n X_{n}\text{ konvergiert}\right]\in {\TT}_{\infty}$, denn
\begin{align*}
\left[\sum_n X_{n}\text{ konvergiert}\right] = 
\left[\sum_{n\ge k} X_{n}\text{ konvergiert}\right],\quad \forall k\in\N
\end{align*}
da es auf endlich viele nicht ankommt also $\left[\sum_{n} X_{n}\text{
konvergiert}\right]\in T_k$ für $k\in\N$.\bsphere
\end{bspenum}
\end{bsp}
\begin{bsp}
Sei $(\Omega, {\cal A })$ ein Messraum, $A_{n} \in \AA$ Ereignisse $(n \in
\N)$ und weiterhin\\
$\AA_{n}\defl \setd{\emptyset, A_{n}, A_{n}^ {c},\Omega}$ $(n\in
\N)$.
\begin{align*}
&\limsup\limits_{n\to \infty} A_{n} \defl \bigcap^{\infty}_{n=1}
\bigcup^ {\infty}_{k=n}  A_{k}= \setdef{\omega \in
\Omega}{\omega \in A_{n}\text{ für unendliche viele }n} \in {\TT}_{\infty},\\
&\liminf\limits_{n\to \infty} A_{n} \defl \bigcup\limits^{\infty}_{n=1}
\bigcap^ {\infty}_{k=n} A_{k} = \setdef{ \omega \in
\Omega}{\omega \in A_{n}\text{ von einem Index an}} \in {\TT}_{\infty}.\bsphere
\end{align*}
\end{bsp}

\begin{prop}[Null-Eins-Gesetz von Kolmogorov]
\label{prop:5.8}
Sei $(\Omega,\AA, P)$ ein W-Raum und $(\AA_{n})$ eine unabhängige Folge
von $\sigma$-Algebren $\AA_{n} \subset \AA$.
Dann gilt für jedes terminale Ereignis $A$ von $(\AA_{n})$ entweder
$P(A)=0$ oder $P(A) = 1$.\fishhere
\end{prop}

Für den Beweis benötigen wir das folgende Resultat.
\begin{lemn}
Seien $\EE_1$ und $\EE_2$ \emph{schnittstabile} Mengensysteme von Ereignissen,
d.h.
\begin{align*}
A,B\in \EE_{1},\EE_{2}\Rightarrow A\cap B\in \EE_{1},\EE_{2}.
\end{align*}
Sind $\EE_1$ und $\EE_2$ unabhängig, so sind es auch $\FF(\EE_1)$ und
$\FF(\EE_2)$.\fishhere
\end{lemn}
\begin{proof}
Den Beweis findet man z.B. in [Wengenroth].\qedhere
\end{proof}

\begin{proof}[Beweis von Satz \ref{prop:5.8}.]
Sei also $A\in\TT_\infty$. Es genügt zu zeigen, dass $A$ von sich selbst
unabhängig ist, d.h.
\begin{align*}
P(A\cap A) = P(A)P(A),
\end{align*}
denn dann ist $P(A)=0$ oder $1$. Betrachte dazu die Mengen
\begin{align*}
&\TT_{n+1}=\FF\left(\bigcup_{k\ge n+1} \AA_k\right),\\
&\DD_n\defl\FF\left(\bigcup_{k\le n} \AA_k\right),
\end{align*}
und wende das vorige Lemma an, so sind $\TT_{n+1}$ und $\DD_n$ unabhängig.

Weiterhin sind aus dem selben Grund $\TT_{\infty}$ und $\DD_n$ unabhängig für
alle $n\in\N$. Daher ist auch
\begin{align*}
\GG =\FF\left(\bigcup_{n\in\N} \DD_n\right)
\end{align*}
unabhängig von $\TT_{\infty}$, wobei gerade $\GG = \TT_1 \supset \TT_{\infty}$.
Somit ist $\TT_{\infty}$ unabhängig von jeder Teilmenge von $\TT_1$
insbesondere von sich selbst. Somit folgt die Behauptung.\qedhere
\end{proof}

Wir betrachten nun einige Anwendungen des 0-1-Gesetzes.
\begin{bsp}
Sei $\setdef{A_n}{n\in\N}$ unabhängig, $\AA_n \defl \setd{\varnothing, A_n,
A_n^c,\Omega}$, somit sind
\begin{align*}
\AA_1, \AA_2, \ldots
\end{align*} 
unabhängig. Mit dem 0-1-Gesetz gilt daher,
\begin{align*}
P\left(\underbrace{\limsup A_n}_{\in\TT_\infty}\right) = 0 \text{ oder } 1.
\end{align*}

Diese Aussage erhalten wir auch mit dem 1. und 2. Lemma von Borel und Cantelli.
Insofern kann man das 0-1-Gesetz als Verallgemeinerung dieser Lemmata
betrachten. Die Aussagen von Borel und Cantelli liefern jedoch darüber hinaus
noch ein Kriterium, wann die Voraussetzungen des 0-1-Gesetzes erfüllt
sind.\bsphere
\end{bsp}
\begin{bsp}
Sei $\setdef{X_n}{n\in\N}$ eine Folge reeller unabhängiger Zufallsvariablen,
dann gilt
\begin{align*}
&P\left[ (X_n) \text{ konvergiert} \right] = 0\text{ oder 1},\\
&P\left[ \sum_n X_n \text{ konvergiert} \right] = 0\text{ oder 1}.\bsphere
\end{align*}
\end{bsp}

\cleardoublepage
\chapter{Erzeugende und charakteristische Funktionen, Faltungen}
\label{chap:6}

Die erzeugenden und charakteristischen Funktionen sind ein methodisches
Hilfsmittel zur Untersuchung von W-Maßen.

Bisher haben wir gesehen, dass Verteilungen und Verteilungsfunktionen 
manchmal schwierig ``zu fassen'' sind.
Erzeugende und charakteristische Funktionen ermöglichen den Übergang von
Verteilungen definiert auf einer $\sigma$-Algebra zu Funktionen definiert auf
$\R$. Dadurch erhalten wir die Möglichkeit, Konzepte aus der Analysis wie
Integration, Differentiation und Grenzwertbildung anzuwenden.

\section{Erzeugende Funktionen}

\begin{defn}
\label{defn:6.1}
Es sei $\mu :{\BB} \to \R$ ein auf
$\mathbb{N}_{0}$ konzentriertes W-Maß mit Zähldichte $(b_{k})_{k\in
  \mathbb{N}_{0}}$ bzw.\ $X$ eine reelle Zufallsvariable auf einem W-Raum $(\Omega, {\cal
  A}, P)$, deren Verteilung $P_{X}$ auf $\mathbb{N}_{0}$ konzentriert ist, mit
Zähldichte $(b_{k})_{k\in \mathbb{N}_{0}}$\,. Dann heißt die auf $[0,1]$ (oder
auch $\{s\in \mathbb{C}\mid$ $| s | \leq 1 \}$) definierte Funktion $g$ mit
\begin{align*}
g(s) \defl\sum_{k=0}^ {\infty} b_{k} \, s^ {k} =
\begin{cases}
\sum \mu  (\{k\})s^ {k},\\
\text{bzw.}\\
\sum P[X=k]s^ {k}=\sum P_{X}(\{k\})s^ {k} =\E s ^ {X}.
\end{cases}
\end{align*}
die \emph{erzeugende Funktion} von $\mu $ bzw.~$X$.\fishhere
\end{defn}

\begin{bem}
\label{bem:6.1}
In Definition \ref{defn:6.1} gilt $| g (s)| \leq \sum b_{k}|
s | ^ {k}\leq \sum b_{k}=1$ für $| s | \leq 1$, da es sich bei $(b_k)$
um eine Zähldichte handelt.\maphere
\end{bem}

Bei der erzeugenden Funktion handelt es sich also um die formale Potenzreihe
\begin{align*}
g(s) = \sum\limits_{k=0}^\infty b_k s^k,
\end{align*}
die nach obiger Bemerkung auf dem Intervall $[0,1]$ bzw. der
abgeschlossenen Kreisscheibe $\setdef{s\in\C}{\abs{s}\le 1}$ absolut und
gleichmäßig bezüglich $s$ konvergiert.

Von besonderem Interesse sind die erzeugenden Funktionen, die neben der
Darstellung als Potenzreihe auch über eine explizite Darstellung verfügen, denn
dann lassen sich $g(s)$, $g'(s)$, \ldots\ meißt leicht berechnen und man
erspart sich kombinatorische Tricks, um die Reihen auf eine Form mit bekanntem
Wert zu bringen.


\begin{bem}[Bemerkungen.]
\begin{bemenum}
\item
Die Binomialverteilung $b(n,p)$ mit $n\in \N$ und $p\in [0,1]$ hat - mit
$q\defl 1-p$ - die erzeugende Funktion $g$,
\begin{align*}
g(s) &\defl \sum\limits_{k=0}^\infty \binom{n}{k} p^k(1-p)^{n-k} s^k
= \sum\limits_{k=0}^n \binom{n}{k} (p\cdot s)^k(1-p)^{n-k}\\ &
= (p\cdot s + (1-p))^n = (ps+q)^n.
\end{align*}
Die Reihe lässt sich im Gegensatz zu dem Ausdruck $(ps+q)^n$ nur mit einigem
an Aufwand direkt berechnen.

\item
Die Poissonverteilung $\pi(\lambda )$ mit $\lambda \in (0,\infty)$ hat die
erzeugende Funktion $g$ mit
\begin{align*}
g(s) = \sum\limits_{k=0}^\infty \e^{-\lambda} \frac{\lambda^k}{k!}s^k
= \e^{-\lambda} \sum\limits_{k=0}^\infty \e^{-\lambda} \frac{(\lambda s)^k}{k!}
= \e^{-\lambda}\e^{s\lambda} = \e^{-\lambda(1-s)}. 
\end{align*}
\item
Als \emph{negative Binomialverteilung} oder \emph{Pascal-Verteilung}
$Nb(r,p)$ mit Parametern
$r\in \mathbb{N}$, $p\in (0,1)$ wird ein W-Maß auf ${\BB}$ (oder auch
$\overline{{\BB}}$)  bezeichnet, das auf $\mathbb{N}_{0}$ konzentriert ist
und --- mit $q\defl 1-p$ --- die Zähldichte
\begin{align*}
k \to Nb (r,p;k) \defl {r+k-1\choose k} p^ {r}q^ {k},\,\, k\in \mathbb{N}_{0}
\end{align*}
besitzt. Speziell wird $Nb(1,p)$ --- mit Zähldichte
\begin{align*}
k\to p(1-p)^ {k}, \;\, k\in \mathbb{N}_{0}
\end{align*}
--- als \emph{geometrische Verteilung} mit Parameter $p\in (0,1)$
bezeichnet.

Ist $(X_{n})_{n\in \mathbb{N}}$ eine unabhängige Folge von $b(1,p)$-verteilten Zufallsvariablen,
so ist die erweitert-reelle Zufallsvariable
\begin{align*}
X\defl \inf\setdef{n\in\N}{\sum\limits^{n}_{k=1} X_{k}=r} -r,
\end{align*}
mit der Konvention $ \inf \emptyset \defl\infty$, $Nb(r,p)$-verteilt. $X$ gibt die
Anzahl der Misserfolge bis zum r-ten Erfolg bei der zu $(X_{n})$ gehörigen
Folge von Bernoulli-Versuchen an.
\begin{align*}
\E X ={\dfrac{rq}{p}}, \V(X) ={\dfrac{rq}{p^ {2}}}.
\end{align*}
$Nb(r,p)$ hat die erzeugende Funktion $g$ mit $g(s) =p^ {r}(1-qs)^
{-r}$.\maphere
\end{bemenum}
\end{bem}

\begin{prop}[Eindeutigkeitssatz für erzeugende Funktionen]

\label{prop:6.1}
Besitzen zwei auf ${\BB}$ definierte, aber auf $\mathbb{N}_{0}$ konzentrierte
W-Maße bzw.\ Verteilungen dieselbe erzeugende Funktion, so stimmen sie
überein.\fishhere
\end{prop}
\begin{proof}
Dies folgt aus dem Identitätssatz für Potenzreihen, der besagt, dass zwei
auf einer nichtleeren offenen Menge identische Potenzreihen dieselben
Koeffizienten haben.\qedhere
\end{proof}

Es besteht also eine Bijektion zwischen diskreten Verteilungen und erzeugenden
Funktionen. Insbesondere haben zwei diskrete Zufallsvariablen mit derselben
erzeugenden Funktion auch dieselbe Verteilung.

\begin{prop}
\label{prop:6.2}
Sei $X$ eine reelle Zufallsvariable, deren Verteilung auf
$\mathbb{N}_{0}$ konzentriert ist, mit erzeugender Funktion $g$.\\[-7mm]
\begin{propenum}
\item
$g$ ist auf der offenen Kreisscheibe $\setdef{s\in \C}{| s | < 1}$ unendlich oft
differenzierbar, sogar analytisch und für $j\in \mathbb{N}$ gilt
\begin{align*}
g^{(j)} (1-) \defl\lim\limits_{0\le s \uparrow 1} g^ {(j)} (s) =\E[X(X-1)\ldots
(X-j+1)] \quad (\leq \infty),
\end{align*}
insbesondere $g'(1-) =\E X$.
\item
Falls $\E X<\infty$, so gilt $\V(X) = g'' (1-)+g'(1-) -(g'(1-))^
{2}$.\fishhere
\end{propenum}
\end{prop}
\begin{proof}
\begin{proofenum}
\item Die Differenzierbarkeit bzw. die Analytizität auf der offenen
Kreisscheibe wird in der Funktionentheorie bewiesen. Auf dieser Kreisscheibe
können wir gliedweise differenzieren und erhalten somit,
\begin{align*}
g^{(j)}(s) &= \sum\limits_{k=j}^\infty b_k k(k-1)\cdots (k-j+n)s^{k-j},\quad
0\le s < 1.
\end{align*}
Auf dem Rand gehen wir
gegebenfalls zum Grenzwert über und erhalten mit 
dem Satz von der monotonen Konvergenz für $s\uparrow 1$,
\begin{align*}
g^{(j)}(s) \to \sum\limits_{k=j}^\infty b_k k(k-1)\cdots (k-j+n).
\end{align*} 
Diese Reihe können wir auch als Erwartungswert der Zufallsvariablen
\begin{align*}
X(X-1)\cdots(X-j+1)
\end{align*}
betrachten (siehe Satz \ref{prop:4.5}), d.h.
\begin{align*}
\lim\limits_{s\uparrow 1} g^{(j)}(s) = \E(X(X-1\cdots (X-j+1))).
\end{align*}
Insbesondere erhalten wir $g'(1-) = \E X$.
\item Da $X$ integrierbar, können wir die Varianz darstellen als
\begin{align*}
\V X &= \E X^2 - (\E X)^2 = \E(X(X-1)) + \E X - (\E X)^2
\\ &= g''(1-) + g'(1-) - (g'(1-))^2.\qedhere
\end{align*}
\end{proofenum}
\end{proof}

Kennen wir eine explizite Darstellung der erzeugenden Funktion einer diskreten
Zufallsvariablen, lassen sich mit Satz \ref{prop:6.2} ihre Momente oftmals
leicht berechnen.

\begin{prop}
\label{prop:6.3}
Seien $X_{1},\ldots,X_{n}$ unabhängige reelle Zufallsvariablen, deren
Verteilungen jeweils auf $\N_{0}$ konzentriert sind, mit erzeugenden
Funktionen $g_{1},\ldots,g_{n}$. Für die erzeugende Funktion $g$ der Summe
$X_{1}+\ldots+X_{n}$ gilt dann $g=\prod^ {n}_{j=1}g_{j}$.\fishhere
\end{prop}

\begin{proof}
Sei $\abs{s}<1$, dann gilt \textit{per definitionem}
\begin{align*}
g(s) = \sum\limits_{k=0}^\infty P[X_1+\ldots+X_n=k]s^k.
\end{align*}
Nun interpretieren wir die erzeugende Funktion als Erwartungswert wie in
\ref{defn:6.1}, d.h.
\begin{align*}
g(s) = \E\ s^{X_1+\ldots+X_n}
= \E\prod\limits_{i=1}^n s^{X_i}.
\end{align*}
Da $X_1,\ldots,X_n$ unabhängig, sind nach Satz \ref{prop:5.4} auch
$s^{X_1},\ldots,s^{X_n}$ unabhängig, d.h.
\begin{align*}
g(s) = \prod\limits_{i=1}^n\E s^{X_i} = \prod\limits_{i=1}^n g_i(s).\qedhere
\end{align*}
\end{proof}

\begin{bsp}
Seien $X_1,X_2,\ldots$ unabhängige $b(1,p)$ verteilte Zufallsvariablen, mit
\begin{align*}
P[X_i=1] = p,\quad P[X_i=0] = \underbrace{1-p}_{q},
\end{align*}
so können wir sie als Folge von Bernoulli-Versuchen interpretieren mit
jeweiliger Erfolgswahrscheinlichkeit $p$,
\begin{align*}
[X_i=1],\qquad \text{Erfolg im i-ten Versuch}.
\end{align*}
Die erweiterte reellwertige Zufallsvariable $Z_1$ gebe die Anzahl der
Misserfolge bis zum 1. Erfolg an. $Z_1$ ist erweitert reellwertig, da die
Möglichkeit besteht, dass $Z_1$ unendlich wird.

Nach dem 1. Erfolg wiederholt sich die stochastische Situation. $Z_2$ gebe die
Anzahl der Misserfolge nach dem 1. bis zum 2. Erfolg an. 

Betrachte nun eine Realisierung
\begin{align*}
\underbrace{0,0,0,0}_{Z_1},1,\underbrace{0,0}_{Z_2},1\underbrace{,}_{Z_3}1,0,\ldots
\end{align*}
Die Zufallsvariablen $Z_1,Z_2,\ldots$ sind unabhängig und identisch verteilt.
\begin{proof}[Nachweis der Unabhänigkeit und der identischen Verteilung.]
%,
%auch die Bernoulli-Versuche unabhängig und identisch verteilt sind.
Wir zeigen zuerst die identische Verteilung,
\begin{align*}
P[Z_1=k]&=q^kp,\qquad k\in\N_0\\
P[Z_1=\infty] &= 1-\sum\limits_{k=0}^\infty q^kp
= 1-p\frac{1}{1-q} = 0,\quad \text{für }p > 0,\\
P[Z_2=k] &= \sum\limits_{j=1}^\infty P[Z_1=j, Z_2=k]
= \sum\limits_{j=1}^\infty q^jpq^kp\\
&= p^2q^k \frac{1}{1-q}=q^kp,\\
P[Z_2=\infty] &= \ldots = 0.
\end{align*}
Induktiv erhalten wir
\begin{align*}
\forall j,k\in\N_0 : P[Z_1=j,Z_2=k]=q^jpq^kp = P[Z_1=j]P[Z_2=k]
\end{align*}
insbesondere sind $Z_1$ und $Z_2$ unabhängig.\qedhere

$Z_1$ hat die erzeugende Funktion,
\begin{align*}
h(s) &= \sum\limits_{k=0}^\infty q^kp s^k = p\frac{1}{1-qs} = p(1-qs)^{-1},\quad
\abs{s}<1,\\
h'(s) &= pq(1-qs)^{-2},\quad h'(1-) = \frac{pq}{(1-q)^2} = \frac{q}{p},\\
h''(s) &= 2pq^2(1-qs)^{-3},\quad  h''(1-) = \frac{2pq^2}{(1-q)^3} =
\frac{2q^2}{p^2}.
\end{align*}
Nach Satz \ref{prop:6.2} gilt somit
\begin{align*}
\E Z_1 = \frac{q}{p},\quad \V Z_1 = \frac{2q^2}{p^2} + \frac{q}{p} -
\left(\frac{q}{p}\right)^2 = \frac{q(q+p)}{p^2} = \frac{q}{p^2}.
\end{align*}
Sei nun $r\in\N$ fest, so hat die Zufallsvariable
\begin{align*}
X\defl \sum\limits_{i=1}^r Z_i
\end{align*}
nach Satz \ref{prop:6.3} die erzeugende Funktion $g=h^r$, also 
\begin{align*}
g(s) = h(s)^r = p^r(1-qs)^{-r}.
\end{align*}
Aus der Analysis kennen wir folgenden Spezialfall der binomischen Reihe,
\begin{align*}
(1+x)^\alpha = \sum\limits_{k=0}^\infty \binom{\alpha}{k}x^k,\qquad \abs{x}<
1,\quad \binom{\alpha}{k} = \frac{\alpha(\alpha-1)\cdots (\alpha-k-1)}{k!}.
\end{align*}
Diese Definition ist auch für $\alpha\in\R$ sinnvoll.
Wir können somit $g$ darstsellen als
\begin{align*}
&g(s) = \sum\limits_{k=0}^\infty p^r\binom{-r}{k}(-1)^k q^ks^k
= \sum\limits_{k=0}^\infty p^r\binom{r+k-1}{k} q^ks^k,\\
\overset{\ref{prop:6.1}}{\Rightarrow} & P[X=k] = \binom{r+k-1}{k}p^rq^k,\qquad
k\in\N_0.
\end{align*}
Die Zufallsvariable $X$ gibt die Anzahl der Misserfolge bis zum $r$-ten Erfolg
an. Also ist eine äquivalente Definition von $X$ gegeben durch
\begin{align*}
X\defl\inf \setdef{n\in\N}{\sum\limits_{k=1}^n X_k = r} - r
\end{align*}
wobei $X_k$ angibt, ob im $k$-ten Versuch ein Erfolg aufgetreten ist.

Eine Zufallsvariable, welche die Anzahl der Misserfolge bis zum $r$-ten Erfolg
bei Bernoulli-Versuchen mit Erfolgswahrscheinlichkeit $p\in (0,1)$ angibt, hat
somit die Zähldichte
\begin{align*}
\binom{r+k-1}{k}p^rq^k,\qquad k\in\N_0, r\in\N, p\in (0,1),
\end{align*}
wobei $N_b(r,p;k) \defl \binom{-r}{k}(-1)^kp^rq^k = \binom{r+k-1}{k}p^rq^k$.
$Nb(r,p)$ heißt \emph{negative Binomial-Verteilung} oder
\emph{Pascal-Verteilung} mit Parametern $r\in\N$, $p\in (0,1)$.

Für $r=1$ heißt $Nb(1,p)$ mit Zähldichte $(pq^k)_{k\in\N}$ \emph{geometrische
Verteilung} mit Parameter $p\in(0,1)$. Wir können nun die negative
Binomial-Verteilung darstellen als
\begin{align*}
Nb(r,p) = \underbrace{Nb(1,p)*\ldots*Nb(1,p)}_{r-\text{mal}}.\bsphere
\end{align*}
\end{proof}
\end{bsp}

\subsection{Konvergenzsätze der Maßtheorie}

Wir haben bereits den Satz von der monotonen Konvergenz kennengelernt, der eine
Aussage über Vertauschbarkeit von Grenzwertbildung und Integration für 
positive, monotone Funktionenfolgen macht. Im Allgemeinen sind die
Funktionenfolgen, mit denen wir in der Wahrscheinlichkeitstheorie arbeiten
weder monoton noch positiv, wir benötigen also allgemeinere Aussagen.

\begin{lemn}[Lemma von Fatou]
Sei $(\Omega,\AA,P)$ ein Maßraum. Für jede Folge von nichtnegativen erweitert
reellen messbaren Funktionen $f_n$ gilt,
\begin{align*}
\int_\Omega \liminf\limits_{n\to\infty} f_n\dmu \le \liminf\limits_{n\to\infty}
\int_\Omega f_n\dmu \le
\limsup\limits_{n\to\infty}
\int_\Omega f_n\dmu.
\end{align*}
Gilt außerdem $\abs{f_n}\le g$ mit $\int_\Omega g \dmu < \infty$, so gilt auch
\begin{align*}
\limsup\limits_{n\to\infty}
\int_\Omega f_n\dmu
\le
\int_\Omega \limsup\limits_{n\to\infty}f_n\dmu.\fishhere
\end{align*} 
\end{lemn}
\begin{proof}
Wir beweisen lediglich die erste Ungleichung, die andere ist trivial. Setze
\begin{align*}
g_n \defl \inf\limits_{m\ge n} f_m \uparrow \liminf_n f_n \defr g.
\end{align*}
Auf diese Funktionenfolge können wir den Satz von der monotonen Konvergenz
anwenden und erhalten somit,
\begin{align*}
\int_{\Omega} \liminf f_n \dmu &= \lim\limits_{n\to\infty}
\int_\Omega
\inf\limits_{m\ge n} f_n \dmu
\le
\lim_{n\to\infty}
\inf\limits_{m\ge n} 
\int_\Omega
 f_n \dmu\\
 &=
 \liminf_{n} 
\int_\Omega
 f_n \dmu.\qedhere
\end{align*}
\end{proof}

Man kann das Lemma von Fatou als Verallgemeinerung des Satzes von der monotonen
Konvergenz von monotonen nichtnegativen auf lediglich nichtnegative
Funktionenfolgen betrachten. Dadurch wird jedoch auch die Aussage auf ``$\le$''
anstatt ``$=$'' abgeschwächt.

\begin{propn}[Satz von der dominierten Konvergenz]
Sei $(\Omega,\AA,\mu)$ ein Maßraum und $f_n$, $f$, $g$ erweitert reellwertige
messbare Funktionen mit $f_n\to f$ $\mufu$, $\abs{f_n}\le g$ $\mufu$ sowie
$\int_\Omega g\dmu < \infty$. Dann existiert 
\begin{align*}
\lim\limits_{n\to\infty} \int_\Omega f_n \dmu
\end{align*}
und es gilt
\begin{align*}
\lim\limits_{n\to\infty} \int_\Omega f_n \dmu
=  \int_\Omega \lim\limits_{n\to\infty} f_n \dmu
= \int_\Omega f\dmu.\fishhere
\end{align*}
\end{propn}
Man findet den Satz in der Literatur auch unter dem Namen Satz von der
majorisierten Konvergenz bzw. Satz von Lebesgue.
\begin{proof}
Setze $N\defl [f_n\nrightarrow f]\ \cup\ \bigcup_{n\in\N} [\abs{f_n}>g]\ \cup\ 
[g=\infty]\in \AA$, so ist nach Voraussetzung $\mu(N) =0$. Setze nun
\begin{align*}
\tilde{f}_n \defl f_n\cdot\Id_{N^c},\quad \tilde{f} \defl f\cdot\Id_{N^c},\quad
\tilde{g}\defl g\cdot\Id_{N^c}.
\end{align*}
Dann gilt $\tilde{f}_n+\tilde{g}\ge 0$, $\tilde{g}-\tilde{f}_n\ge 0$ auf ganz
$\Omega$ und für alle $n\in\N$. Mit dem Lemma von Fatou folgt,
\begin{align*}
\int_\Omega \tilde{f}\dmu
&=
\int_\Omega \liminf_n (\tilde{f}_n + \tilde{g})\dmu
- \int_\Omega \tilde{g}\dmu\\ 
&\le \liminf_n \int_\Omega (\tilde{f}_n+\tilde{g})\dmu - \int_\Omega \tilde{g}
\dmu = \liminf_n \int_\Omega \tilde{f}_n \dmu \\ &  \le \limsup_n \int_\Omega
\tilde{f}_n\dmu
= -\liminf_n \int_\Omega
-\tilde{f}_n\dmu
 \\ &= \int_\Omega \tilde{g} \dmu - \liminf_n \int_\Omega
(\tilde{g}-\tilde{f}_n)\dmu\\ &\le
\int_\Omega \tilde{g}\dmu - \int_\Omega \liminf_n (\tilde{g}-\tilde{f}_n)\dmu
= \int_\Omega \tilde{f}\dmu.
\end{align*}
Insbesondere gilt $\liminf_n \int_\Omega \tilde{f}_n \dmu  = \limsup_n
\int_\Omega \tilde{f}_n\dmu$ und somit ist alles gezeigt.\qedhere
\end{proof}


\section{Charakteristische Funktionen}


\begin{defn}
\label{defn:6.2}
Sei $\mu$ ein W-Maß auf ${\BB}$ bzw.\ $X$ eine
reelle Zufallsvariable mit Verteilungsfunktion $F$. Dann heißt die auf
$\R$ definierte (i.a.\ komplexwertige) Funktion $\ph$
mit
\begin{align*}
\ph (u) \defl \int_{\R} \e^{iux} \mu(\dx)
\end{align*}
bzw.
\begin{align*}
\ph(u) \defl \E \e^ {iuX} = \int_{\R }\e^ {iux}\,P_{X}(\dx) =
\int _{\R}\e^ {iux} \dF(x)
\end{align*}
die \emph{charakteristische} Funktion von $\mu$ bzw.\ $X$.\fishhere
\end{defn}

Diese Definition ist tatsächlich sinvoll, denn wir können $\e^{iux}$ für
$u,x\in\R$ stets durch $1$ majorisieren. Es gilt also
\begin{align*}
\abs{\ph(u)}\le \E\abs{\e^{iu X}} \le \E 1 =1
\end{align*}
und daher existiert $\ph(u)$ für jedes $u\in\R$.

\begin{bem}[Bemerkungen.]
\label{bem:6.3}
In Definition \ref{defn:6.2} gilt\\[-7mm]
\begin{bemenum}
\item $\ph (0) = 1$,
\item $\abs{\ph (u)} \leq 1$, $ u\in \R$,
\item $\ph $ gleichmäßig stetig in $\R$,
\item $\ph (-u) = \overline{\varphi (u)}$, $ u \in \R$.\maphere
\end{bemenum}
\end{bem}
\begin{proof}
Nachweis von Bemerkung \ref{bem:6.3} c)
\begin{align*}
\abs{\ph(u+h)-\ph(u)}
&= \abs{\int_\R \e^{i(u+h)x}-\e^{iux}\mu(\dx)}
= \abs{\int_\R (\e^{ihx}-1)\e^{iux}\mu(\dx)}\\
&\le \int_\R\abs{\e^{ihx}-1}\mu(\dx)
\end{align*}
Nun verschwindet der Integrand für $h\to 0$, d.h. für jede Nullfolge $h_n$ gilt
\begin{align*}
\abs{\e^{ih_nx}-1}\to 0,\qquad n\to\infty.
\end{align*}
Außerdem ist $\abs{\e^{ih_nx}-1}\le \abs{\e^{ih_nx}}+1 = 2$, und $\int_\R 2\dmu =
2\mu(\R) = 2$, wir können also den Satz von der dominierten Konvergenz
anwenden.\qedhere
\end{proof}

Die charakteristische Funktion ist im Wesentlichen die Inverse
Fouriertransformierte von $P_X$, der Verteilung von $X$. Besitzt $X$ eine
Dichte $f$, so gilt nach dem Transformationssatz
\begin{align*}
\ph_X(u) = \int_\R \e^{iux}f(x)\dx.
\end{align*}

\begin{bem}
\label{bem:6.4}
Die Normalverteilung $N(a,\sigma ^ {2})$ mit $a \in \R$ und $\sigma ^ {2} \in
(0,\infty )$ besitzt die charakteristische Funktion $\ph : \R\to\C $ mit
\begin{align*}
\ph(u) = \e^{iau} \e^ {-\frac{\sigma^ {2}u^ {2}}{2}}.
\end{align*}
Insbesondere besitzt die Standardnormalverteilung $N(0,1)$ die charakteristische
Funktion $\ph $ mit $\ph(u) = \e^ {-\frac{u^ {2}}{2}}$.\maphere
\end{bem}
\begin{proof}[Beweisskizze.]
\textit{1. Teil}. Wir betrachten zunächst $N(0,1)$. Die
Standardnormalverteilung besitzt die Dichte
\begin{align*}
f(x) = \frac{1}{\sqrt{2\pi}}\e^{-\frac{x^2}{2}},
\end{align*}
somit gilt für die charakteristische Funktion,
\begin{align*}
\ph(u) &= \int_\R \e^{iux} \frac{1}{\sqrt{2\pi}}\e^{-\frac{1}{2}x^2}\dx
= \frac{1}{\sqrt{2\pi}} 
\int_\R \e^{i(x-iu)^2-\frac{1}{2}u^2}\dx\\
&=
\frac{1}{\sqrt{2\pi}}
\e^{-\frac{1}{2}u^2}
\int_\R \e^{\frac{1}{2}(x-iu)^2}\dx.
\end{align*}
Um das Integral zu berechnen, betrachten wir $z\mapsto
\e^{\frac{1}{2}z^2}$; diese Funktion einer komplexen Variablen $z$ ist
komplex differenzierbar (analytisch, holomorph). Wir verwenden nun den
Residuensatz, ein Resultat aus der Funktionentheorie, der besagt, dass das
Integral einer holomorphen Funktion über jede geschlossene Kurve $c$
verschwindet,
\begin{align*}
\int_c \e^{\frac{1}{2}z^2} \dz = 0. 
\end{align*}

Um dies auszunutzen, wählen wir eine spezielle geschlossene Kurve
$c$ (siehe Skizze) bestehend aus den Geradenstücken
$c_1,c_2,c_3$ und $c_4$.
\begin{figure}[!htpb]
\centering
\begin{pspicture}(0,-1.12)(3.56,1.1)
\psline{->}(0.04,0.06)(3.44,0.06)
\psline{->}(1.72,-0.94)(1.74,1.08)

\psline[linecolor=darkblue](3,-0.74)(0.44,-0.74)(0.44,0.06)(3,0.06)(3,-0.74)(0.44,-0.74)

\psline[linecolor=darkblue]{->}(0.44,0.06)(0.44,-0.34)
\psline[linecolor=darkblue]{->}(0.44,-0.74)(2.24,-0.74)
\psline[linecolor=darkblue]{->}(3,-0.74)(3,-0.2)
\psline[linecolor=darkblue]{->}(3,0.06)(0.98,0.06)

%\psline(1.22,0.06)(0.44,0.06)
%\psline(0.44,-0.34)(0.44,-0.74)
%\psline(2.24,-0.74)(3.04,-0.74)
%\psline(3.04,0.06)(3.04,-0.3)

\rput(1.19,0.325){\color{gdarkgray}$c_1$}
\rput(2.34,-0.955){\color{gdarkgray}$c_3$}
\rput(3.35,-0.315){\color{gdarkgray}$c_4$}
\rput(0.17,-0.335){\color{gdarkgray}$c_2$}

\psline(0.44,0.22)(0.44,0.04)
\psline(3,0.22)(3,0.04)

\rput(0.44,0.405){\color{gdarkgray}$-R$}
\rput(3,0.405){\color{gdarkgray}$R$}
\end{pspicture} 
\caption{Zur Konstruktion der geschlossenen Kurve $c$.}
\end{figure}

 Für diese Geradenstücke können wir die Integrale
leicht berechnen bzw. abschätzen,
\begin{align*}
&\int_{c_1} \e^{-\frac{1}{2}z^2}\dz \overset{R\to\infty}{\to} -\sqrt{2\pi},\\
&\int_{c_2,c_4} \e^{-\frac{1}{2}z^2}\dz = \int_{c_2} \e^{-\frac{1}{2}z^2}\dz +
\underbrace{\int_{c_4} \e^{-\frac{1}{2}z^2}\dz}_{-\int_{c_2}
\e^{-\frac{1}{2}z^2}\dz} = 0,\\
&\int_{c_3} \e^{-\frac{1}{2}z^2}\dz \overset{R\to\infty}{\to} \int_\R
\e^{-\frac{1}{2}(x-iu)^2}\dx.
\end{align*}

Für ganz $c$ gilt nach Voraussetzung,
\begin{align*}
&0 = \int_c \e^{-\frac{1}{2}z^2} \dz =
\sum_{i=1}^4
\int_{c_i}
\e^{-\frac{1}{2}z^2}\dz
= 
\int_{c_1} \e^{-\frac{1}{2}z^2}\dz +
\int_{c_3} \e^{-\frac{1}{2}z^2}\dz
\\
\Rightarrow 
&\int_\R
\e^{-\frac{1}{2}(x-iu)^2}\dx = \sqrt{2\pi}.
\end{align*}
Somit besitzt $N(0,1)$ die charakteristische Funktion
\begin{align*}
\ph(u) = \frac{1}{\sqrt{2\pi}}\e^{-\frac{1}{2}u^2}\int_\R
\e^{-\frac{1}{2}(x-iu)^2}\dx
= \e^{-\frac{1}{2}u^2}.
\end{align*}
\textit{2. Teil}. Für $N(a,\sigma^2)$ allgemein ist die charakteristische
Funktion gegeben durch,
\begin{align*}
\ph(u) = \int_\R
\e^{iux}\frac{1}{\sqrt{2\pi}\sigma}\e^{\frac{(x-a)^2}{2\sigma^2}}\dx
\end{align*}
Die Substitution $y=\frac{x-a}{\sigma}$ liefert die Behauptung.\qedhere
\end{proof}

\begin{bem}
\label{bem:6.5}
Als \emph{Exponentialverteilung} $\exp (\lambda )$ mit
Parameter $\lambda \in (0,\infty)$ wird ein W-Maß auf ${\BB}$ oder
$\overline{\BB}$
bezeichnet, das eine Dichtefunktion $f$ mit
\begin{align*}
f(x)= \begin{cases}
\lambda \e^ {-\lambda x}, &  x>0     \\
0,                       & x\leq 0
\end{cases}
\end{align*}
besitzt.

Die zufällige Lebensdauer eines radioaktiven Atoms wird durch eine
exponentialverteilte Zufallsvariable angegeben. Ist $X$ eine erweitert-reelle
Zufallsvariable, deren Verteilung auf $\overline{\R}_{+}$ konzentriert ist,
mit $P[0<X<\infty]>0$, so gilt
\begin{align*}
&\forall s, t\in (0,\infty) :
 P[X>t+s \mid X>s] =P[X>t].
\end{align*}
Diese Eigenschaft nennt man \emph{Gedächtnislosigkeit}. Eine auf
$\overline{\R}_{+}$ konzentrierte reell erweiterte Zufallsvariable ist genau
dann gedächtnislos, wenn sie exponentialverteilt ist.
\begin{proof}
``$\Leftarrow$'': Sei $X$ also $\exp(\lambda)$ verteilt. Per definitionem der
bedingten Wahrscheinlichkeit gilt,
\begin{align*}
P[X>t+s\mid X>s] = \frac{P[X>t+s,\;X>s]}{P[X>s]}
= \frac{1-F(t+s)}{1-F(s)}. 
\end{align*}
Die Verteilungsfunktion erhält man durch elementare Integration,
\begin{align*}
F(x) = 
\begin{cases}
\int_0^x \lambda \e^{-\lambda t}\dt = 1-\e^{-\lambda x}, & x > 0,\\
0, & x\le 0.
\end{cases}
\end{align*}
Somit ergibt sich
\begin{align*}
\frac{1-F(t+s)}{1-F(s)} = \frac{\e^{-\lambda (t+s)}}{\e^{-\lambda s}} =
\e^{-\lambda t} = 1- F(t) = P[X>t].
\end{align*}
``$\Rightarrow$'': Sei $X$ gedächtnislos, d.h. $P[X>t+s\mid X>s]=P[X>s]$. Unter
Verwendung der Definition der bedingten Wahrscheinlichkeit erhalten wir
\begin{align*}
\underbrace{P[X>t+s]}_{H(t+s)} =
\underbrace{P[X>t]}_{H(t)}\underbrace{P[X>s]}_{H(s)}.
\end{align*}
Wir interpretieren diese Gleichung als Funktionalgleichung und suchen nach
einer Lösung $H:\R_+\to\R_+$, die diese Gleichung erfüllt. Da
\begin{align*}
H(t)=P[X>t]=1-F(t)
\end{align*}
erhalten wir aus den Eigenschaften der Verteilungsfunktion
für $H$ außerdem die Randbedingungen $H(0)=1$ und
$\lim\limits_{t\to\infty}H(t) = 0$. Offensichtlich ist $H(t)=\e^{-\lambda t}$
für $\lambda\in(0,\infty)$ eine Lösung.

Man kann zeigen, dass dies in der Klasse auf $\overline{\R}_+$ konzentrierten
Verteilungen die einzige Lösung ist (siehe Hinderer $\mathsection$ 28).\qedhere
\end{proof}

Für die Beschreibung von technischen Geräten oder Bauteilen ist es oft
notwending ein ``Gedächtnis'' miteinzubauen. Dafür eignet sich die sog.
Weibull-Verteilung\footnote{Ernst Hjalmar Waloddi Weibull (* 18. Juni 1887; †
12. Oktober 1979 in Annecy) war ein schwedischer Ingenieur und Mathematiker.},
welche eine Verallgemeinerung der Exponential-Verteillung darstellt.

Für eine Zufallsvariable $X$ mit $P_{X}= \exp (\lambda )$ gilt
\begin{align*}
\E X=\frac{1}{\lambda},\qquad \V(X) =\frac{1}{\lambda ^{2}}.
\end{align*}
\begin{proof}
Der Nachweis folgt durch direktes Rechnen,
\begin{align*}
&\E X = \int\limits_{-\infty}^\infty xf(x)\dx
 = \int\limits_0^\infty
\lambda x\e^{-\lambda x}\dx 
= -x\e^{-\lambda x}\bigg|_{x=0}^\infty + \int_0^\infty \e^{-\lambda x}\dx
= \frac{1}{\lambda}\\
&\V X = \E X^2- (\E X)^2
= \int\limits_0^\infty \lambda x^2\e^{-\lambda x}
- \frac{1}{\lambda^2}
= \frac{1}{\lambda^2}.\qedhere
\end{align*}
\end{proof}
exp $(\lambda )$ hat die charakteristische Funktion $\ph$  mit $\ph(u)
=\frac{\lambda }{\lambda - iu}$.
\begin{proof}
$\ph(u)=\int_{0}^\infty \e^{iux}\lambda \e^{-\lambda x}\dx
=
\lambda \int_{0}^\infty \e^{-(\lambda-iu)x}\dx
 =
\frac{\lambda}{\lambda-iu}$.\qedhere\maphere
\end{proof}
\end{bem}

\begin{bem}
\label{bem:6.6}
Als \emph{Gamma-Verteilung} $\Gamma _{\lambda , \nu }$ mit
Parametern $\lambda , \nu \in (0,\infty )$ wird ein W-Maß auf ${\BB}$
oder $\overline{\BB}$ bezeichnet, das eine Dichtefunktion $f$ mit
\begin{align*}
f(x) = \begin{cases}
\frac{\lambda ^ {\nu }}{\Gamma (\nu )} x^ {\nu -1}e ^ {-\lambda x},& x>0, \\
0, & x\leq 0,
\end{cases}
\end{align*}
besitzt. Hierbei ist $ \Gamma _{\lambda ,1} = \exp (\lambda)$ Die
Gamma-Verteilung stellt also ebenfalls eine Verallgemeinerung der
Exponential-Verteilung dar. $\Gamma _{\frac{1}{2},\frac{n}{2}}$ wird als
\emph{Chi-Quadrat-Verteilung} $\chi^ {2}_{n}$ mit $n$ Freiheitsgraden
bezeichnet ($n\in\N$).

Für eine Zufallsvariable $X$ mit $P_{X} =\Gamma _{\lambda , \nu }$ gilt
\begin{align*}
\E X=\frac{\nu}{\lambda},\qquad \V(X) = \frac{\nu }{\lambda ^ {2}}.
\end{align*}
$\Gamma _{\lambda , \nu }$ hat die charakteristische Funktion $\varphi $ mit
$\varphi (u) =  \left( \frac{\lambda}{\lambda -iu}\right) ^ {\nu }$.
\begin{proof}
$\E X$ und $\V X$ folgen durch direktes Integrieren oder unter Verwendung der
charakteristischen Funktion.

Die charakteristische Funktion ist gegeben durch folgendes Integral
\begin{align*}
\ph(u) = \int\limits_0^\infty
\frac{\lambda^\nu}{\Gamma(\nu)}\e^{iux}x^{\nu-1}\e^{-\lambda x}\dx.
\end{align*}
Dieses wollen wir nicht direkt berechnen, sondern in eine Differentialgleichung
umformen und so elegant lösen. Unter Verwendung des Satzes über dominierte
Konvergenz lässt sich nachweisen, dass  Differentiation
und Integration hier vertauscht werden können. Somit erhalten wir
\begin{align*}
\ph'(u) &= \int\limits_0^\infty \partial_u
\frac{\lambda^\nu}{\Gamma(\nu)}\e^{iux}x^{\nu-1}\e^{-\lambda x}\dx
= \frac{\lambda^\nu}{\Gamma(\nu)}
 \int\limits_0^\infty
ix \e^{iux}x^{\nu-1}\e^{-\lambda x}\dx\\
&=
\frac{i\lambda^\nu}{\Gamma(\nu)}
 \int\limits_0^\infty
\e^{(iu-\lambda)x}x^{\nu}\dx\\
&=
 \underbrace{\frac{i\lambda^\nu}{\Gamma(\nu)}
x^\nu\frac{\e^{(iu-\lambda)x}}{iu-\lambda}\bigg|_0^\infty}_{=0}
- \frac{i\lambda^\nu}{\Gamma(\nu)}
 \int\limits_0^\infty
\nu x^{\nu-1}\frac{\e^{(iu-\lambda)x}}{iu-\lambda}\dx\\
&=
\frac{i\nu}{\lambda-iu}
 \int\limits_0^\infty
 \frac{\lambda^\nu}{\Gamma(\nu)}
x^{\nu-1}\e^{(iu-\lambda)x}\dx
=
 \frac{i\nu}{\lambda-iu}\ph(u).
\end{align*}
Dies liefert nach Zerlegung der DGL in Real- und Imaginäranteil und
anschließendem Lösen der Anfangswertprobleme $\Re\ph(0) = 1$ und $\Im\ph(0) =
0$ die eindeutige Lösung
\begin{align*}
\ph(u) = \left(\frac{\lambda}{\lambda-iu}\right)^\nu.\qedhere\maphere
\end{align*}
\end{proof}
\end{bem}

\subsection{Eindeutigkeitssatz und Umkehrformeln}

Per se ist nicht klar, ob zwei unterschiedliche Verteilungsfunktionen auch
unterschiedliche charakteristische Funktionen besitzten bzw. inwiefern
eine charakteristische Funktion die Verteilung ``charakterisiert''. Der folgende
Satz beantwortet die Frage.

\begin{prop}[Eindeutigkeitssatz für charakteristische Funktionen]
\label{prop:6.4}
Besitzen zwei auf ${\BB}$ definierte W-Maße bzw.\ Verteilungen dieselbe
charakteristische Funktion, so stimmen sie überein.

Ist $\mu $ ein W-Maß auf
${\BB}$ bzw.\ $P_{X}$ die Verteilung einer reellen Zufallsvariablen $X$ mit
zugehöriger Verteilungsfunktion $F$ und zugehöriger charakteristischer
Funktion $\ph $ und sind $a,b$ mit $a<b$ Stetigkeitsstellen von $F$, so
gilt die Umkehrformel
\begin{align*}
\underbrace{F(b) -F(a)}_{{ = \mu ((a,b])  \text{ bzw. } P_{X}((a,b])}}
= \lim\limits_{U\to \infty} \frac{1}{2\pi}  \int^ {U}_{-U} \frac{\e^ {-iua}-\e^
{-iub}}{iu} \ph (u) \du.\fishhere
\end{align*}
\end{prop}

Wir erhalten mit Hilfe dieses Satzes zunächst nur die Eindeutigkeit auf den
halboffenen Intervallen. Der Fortsetzungssatz besagt jedoch, dass damit auch
die Fortsetzung auf die Menge der Borelschen Mengen eindeutig ist.

\begin{proof}
Es genügt, die Umkehrformel zu beweisen. Wir verwenden dazu die Dirichlet-Formel
\begin{align*}
\lim\limits_{\atop{A\to-\infty}{B\to\infty}}\frac{1}{\pi}\int\limits_{A}^B
\frac{\sin(v)}{v}\dv = 1,
\end{align*}
die man leicht unter Verwendung der Funktionentheorie beweisen kann.
Wir verwenden außerdem die Identität,
\begin{align*}
\e^{-iua}-\e^{-iub} = -\e^{-iuy}\bigg|_{y=a}^b
= iu \int\limits_{y=a}^b \e^{-iuy}\dy,
\end{align*}
somit folgt nach Anwendung des Satzes von Fubini
\begin{align*}
&\lim\limits_{U\to\infty} \frac{1}{2\pi} \int\limits_{u=-U}^U
\frac{1}{iu}\left(\e^{-iua}-\e^{-iub}\right)\ph(u)\du\\
&=\lim\limits_{U\to\infty} \frac{1}{2\pi} \int\limits_{u=-U}^U
\int\limits_{x=-\infty}^\infty
\frac{1}{iu}\left(\e^{-iua}-\e^{-iub}\right)\e^{iux}F(\dx) \du\\
&=\lim\limits_{U\to\infty} \frac{1}{2\pi} 
\int\limits_{x=-\infty}^\infty
\int\limits_{u=-U}^U
\int\limits_{y=a}^b \e^{iu(x-y)}\dy F(\dx) \du\\
&=\lim\limits_{U\to\infty} \frac{1}{2\pi} 
\int\limits_{x=-\infty}^\infty
\int\limits_{y=a}^b
\int\limits_{u=-U}^U
\e^{iu(x-y)}  \du\dy F(\dx)\\
&=\lim\limits_{U\to\infty} \frac{1}{2\pi} 
\int\limits_{x=-\infty}^\infty
\int\limits_{y=a}^b
\frac{\e^{iU(x-y)}-\e^{-iU(x-y)}}{i(x-y)}
\dy F(\dx)\\
&=\lim\limits_{U\to\infty} \frac{1}{\pi} 
\int\limits_{x=-\infty}^\infty
\int\limits_{y=a}^b
\frac{\sin(U(x-y))}{(x-y)}
\dy F(\dx)
\end{align*}
Nun substituieren wir $\nu=U(x-y)$, d.h. ``$\dx = \frac{1}{U}\dnu$'',
\begin{align*}
&=
\int\limits_{x=-\infty}^\infty
\lim\limits_{U\to\infty}
\underbrace{\frac{1}{\pi}  
\int\limits_{y=U(x-a)}^{U(x-b)}
\frac{\sin(v)}{v}
\dv}_{\defl G(U,x)} F(\dx)\\
&=
\int\limits_{-\infty}^\infty
\lim\limits_{U\to\infty} G(U,x)F(\dx)
\end{align*}
Für den Integranden erhalten wir unter Verwendung der Formel von Dirichlet,
\begin{align*}
\lim\limits_{U\to\infty} G(U,x)=\begin{cases} 1, & \text{falls } a<x<b,\\
0, & \text{falls } x<a,\\
0, & \text{falls } x> b.
\end{cases}
\end{align*}
Falls wir den Satz von der dominierten Konvergenz anwenden können, erhalten wir
für das Integral
\begin{align*}
\int\limits_{-\infty}^\infty \lim\limits_{U\to\infty} G(U,x)F(\dx)
= F(b)-F(a),
\end{align*}
denn das Verhalten des Integranden an den Unstetigkeitsstellen ist für den
Integralwert nicht von Bedeutung.

Wir müssen also noch nachweisen, dass wir den Satz über die dominierte
Konvergenz auch anwenden können. Setze $n=U$ und $f_n=G(U,\cdot)$,
$G(U,\cdot)$ ist beschränkt, d.h.
\begin{align*}
\abs{G(U,\cdot)} \le g < \infty,\qquad g\in \R. 
\end{align*}
Nun ist die konstante Funktion $g$ integrierbar bezüglich der durch die
Verteilungsfunktion $F$ induzierten Verteilung $\mu$.
$(\mu(a,b]=F(b)-F(a))$.\qedhere
\end{proof}

Mit dem Eindeutigkeitssatz erhält man aus $\ph$ die Verteilungsfunktion, ist
$\ph$ darüber hinaus integrierbar, gilt wesentlich mehr.

\begin{prop}
\label{prop:6.5}
Gegeben sei eine Verteilungsfunktion\ $F$ mit charakteristischer Funktion
$\ph$; es sei $\int^{\infty}_{-\infty} \abs{\ph(u)}\du < \infty
$. Dann besitzt $F$ eine Dichte $f$, die stetig und beschränkt ist, und es gilt
die Umkehrformel
\begin{align*}
f(x) = \frac{1}{2\pi }  \int_{\R} \e^ {-iux} \varphi (u) \du,\quad
x\in \R.\fishhere
\end{align*}
\end{prop}

\begin{proof}
Sei also $\ph$ integrierbar, d.h. $\int_\R \abs{\ph}(x) \dx = c <\infty$.
Wir verwenden die Umkehrformel
\begin{align*}
F(b)-F(a) = \lim\limits_{U\to\infty} \frac{1}{2\pi}\int\limits_{-U}^U
\frac{\e^{-iua}-\e^{-iub}}{iu}\ph(u)\du
\end{align*}
und bilden so den Differenzenquotienten für $F$, falls $F$ in $x$ stetig,
\begin{align*}
\frac{F(x+h)-F(x)}{h}
&= \lim\limits_{U\to\infty} \frac{1}{2\pi}\int\limits_{-U}^U
-\left(\frac{\e^{-iu(x+h)}-\e^{-iux}}{iuh}\right)\ph(u)\du.
\end{align*}
Diesen Ausdruck können wir durch $\frac{1}{2\pi}\int_\R \abs{\ph}(u)\du$
majorisieren, denn
\begin{align*}
\abs{\frac{\e^{-iu(x+h)}-\e^{-iux}}{iuh}}
= 
\frac{1}{\abs{uh}}
\abs{\int_{x}^{x+h} -iu \e^{-ius}\ds}
\le \frac{1}{\abs{h}} \abs{\int_{x}^{x+h}} \ds
= 1.
\end{align*}
% 
%  indem wir auf den Integranden den Mittelwertsatz bezüglich
% $x$ anwenden,
% \begin{align*}
% \frac{\e^{-iu(x+h)}-\e^{-iux}}{iuh}=
% -\frac{iu \e^{iu\xi}}{iu} = - \e^{iu\xi},\qquad \xi\in[x,x+h].
% \end{align*}
Somit können wir den Satz von der dominierten Konvergenz anwenden und die
Grenzwertbildung für $h\to 0$ mit der Grenzwertbildung für $U\to\infty$ und der
Integration  vertauschen.
\begin{align*}
\lim\limits_{h\to 0}
\frac{F(x+h)-F(x)}{h}
&=
\lim\limits_{h\to0}\lim\limits_{U\to\infty}
\frac{1}{2\pi}\int\limits_{-U}^U
-\left(\frac{\e^{-iu(x+h)}-\e^{-iux}}{iuh}\right)\ph(u)\du\\ 
&= \frac{1}{2\pi}\int\limits_{\R}
\e^{-iux}\ph(u)\du \defr f(x).
\end{align*}
$f$ ist offensichtlich Dichte zu $F$ und beschränkt, denn
\begin{align*}
\abs{f(x)} \le 
\frac{1}{2\pi}\int\limits_{\R}
\abs{\e^{-iux}}\abs{\ph(u)}\du
= 
\frac{1}{2\pi}\int\limits_{\R}
\abs{\ph(u)}\du < \infty,
\end{align*}
und außerdem stetig, denn
\begin{align*}
\abs{f(y)-f(x)}
&\le \int_\R \abs{\e^{iuy}-\e^{iux}}\abs{\ph(u)}\du\\
&\le \sup_{x,y\in U}  \abs{\e^{iuy}-\e^{iux}}\int_\R
\abs{\ph(u)}\du \\ &=
c\sup_{x,y\in U}  \abs{\e^{iuy}-\e^{iux}}.
\end{align*}
Der rechte Ausdruck wird klein für $\abs{x-y}$ klein, denn $\e^{iu\cdot}$ ist
stetig.\qedhere
\end{proof}

Ist $X$ Zufallsvariable mit integrierbarer charakteristischer Funktion $\ph$,
so besitzt $X$ eine Dichte und diese entspricht im Wesentlichen der
Fouriertransformierten von $\ph$. Wie wir bereits festgestellt haben, erhalten
wir $\ph$ als inverse Fouriertransformierte dieser Dichte zurück.

\begin{prop}
\label{prop:6.6}
Sei $X$ eine reelle Zufallsvariable mit Verteilung $P_{X}$ und
charakteristischer Funktion $\ph $. Falls für $j\in \mathbb{N}$ gilt:
\begin{align*}
&\E\abs{X}^{j} < \infty,
\end{align*}
so besitzt $\ph$
eine stetige $j$-te Ableitung
$\ph^{(j)}$ mit
\begin{align*}
\ph^{(j)}(u) = i^{j}\int_{\R}
x^ {j} \e^ {iux} P_{X}(\dx),\quad u\in \R.
\end{align*}
Insbesondere $\ph^{(j)} (0) = i^ {j} \E X^ {j}$.\fishhere
\end{prop}

\begin{proof}
Wir beweisen die Behauptung für $j=1$, der Rest folgt induktiv.
\begin{align*}
\ph(u) &= \int\limits_{-\infty}^\infty \e^{iux}P_X(\dx)\\
\ph'(u) &= \lim\limits_{h\to0} \frac{\ph(u+h)-\ph(u)}{h}
= \lim\limits_{h\to 0}\int\limits_\R \frac{\e^{i(u+h)x}-\e^{iux}}{h} P_X(\dx)\\
&= \lim\limits_{h\to 0}\int\limits_\R
\e^{iux}\underbrace{\left(\frac{\e^{ihx}-1}{h}\right)}_{\to ix\e^{iux}} P_X(\dx)
\end{align*}
Wir müssen also nachweisen, dass wir den Satz von der dominierten Konvergenz
anwenden können,
\begin{align*}
\abs{\e^{iux}\underbrace{\left(\frac{\e^{ihx}-1}{h}\right)}_{\to ix\e^{iux}}} = 1
\abs{\frac{\e^{ihx}-1}{h}} < \const\abs{x},\qquad x,h\in\R
\end{align*}
und $\int_\R \abs{x}P_X(\dx) = \E \abs{X} < \infty$.

Somit erhalten wir für das Integral,
\begin{align*}
\ph'(u) = i\int_\R x \e^{iux}P_X(\dx)
\end{align*}
$\ph'$ ist gleichmäßig stetig, denn
\begin{align*}
\abs{\ph'(u+h)-\ph'(u)} \le \int_\R \abs{x}\abs{\e^{i(u+h)x}-\e^{iux}}P_X(\dx)
\end{align*}
Nun ist $\abs{\e^{i(u+h)x}-\e^{iux}}$ nach der Dreiecksungleichung durch $2$
beschränkt und konvergiert für $h\to 0$ gegen Null, somit haben wir eine
integrierbare Majorante und mit dem Satz von der dominierten Konvergenz folgt,
\begin{align*}
\abs{\ph'(u+h)-\ph'(u)} \to 0,\qquad h\to 0.\qedhere
\end{align*} 
\end{proof}

Erinnern wir uns an die erzeugende Funktion $g$ einer auf $\N_0$ konzentrierten
Verteilung so besagt Satz \ref{prop:6.2},
\begin{align*}
g^{(j)}(1-) = \E (X(X-1)\cdots(X-j+1)).
\end{align*}
Für charakteristische Funktionen reeller Zufallsvariablen gilt nun,
\begin{align*}
\ph^{(j)}(0) = i^j\E X^j.
\end{align*}

Erzeugende und charakteristische Funktionen ermöglichen es uns
Integrationsaufgaben,
\begin{align*}
\int_\R x^j P_X,\qquad \sum_{k\ge 0} k^j b_k 
\end{align*}
auf Differentiationsaufgaben $\ph^{j}(u)\big|_{u=0}$, $g^{(j)}(1-)$ 
zurückzuführen, die sich meist wesentlich leichter lösen lassen.


\begin{prop}
\label{prop:6.7}
Seien $X_{1},\ldots ,X_{n}$ unabhängige reelle Zufallsvariablen mit
charakteristischen Funktionen $\varphi_{1},\ldots ,\varphi _{n}$. Für die
charakteristische Funktion der Summe
\begin{align*}
X_{1}+\ldots +X_{n}
\end{align*}
gilt dann
\begin{align*}
\ph = \prod\limits^ {n}_{j=1} \ph _{j}.\fishhere
\end{align*}
\end{prop}

\begin{proof}
$\ph(u)=\E \e^{iu(X_1+\ldots+X_n)} = \E(\e^{iu X_1}\cdots \e^{iu X_n}) 
\overset{\ref{prop:5.4}}{=}(\E \e^{iu X_1})\cdots (\E \e^{iu X_n})$.\qedhere
\end{proof}

Für die erzeugende und charakteristische Funktion einer Summe von unabhängigen
Zufallsvariablen gelten also analoge Aussagen. Beide beruhen auf der
Multiplikativität des Erwartungswerts für unabhänige Zufallsvariablen.

\section{Faltungen}

In Abschnitt \ref{chap:5.a} haben wir bereits für zwei reelle unabhängige
Zufallsvariablen $X$ und $Y$ die Faltung $P_X*P_Y\defl P_{X+Y}$ als Verteilung von
$X+Y$ definiert. Wir wollen die Definition nun noch etwas Verallgemeinern und
die Argumentationslücken aus \ref{chap:5.a} schließen. 

\begin{defn}
\label{defn:6.3}
Zu zwei W-Maßen $P,Q$ auf ${\BB}$ wird das W-Maß $P\ast
Q$ auf ${\BB}$, die sog. \emph{Faltung} von $P$ und $Q$, folgendermaßen
definiert:
\begin{defnenum}
\item
Sei $T: \R \times \R \to \R$ mit $T(x,y)=x+y$;
\begin{align*}
P\ast Q\defl (P\otimes Q)_{T}
\end{align*}
($P\otimes Q$ W-Maß auf ${\BB}_{2}$ mit $(P\otimes Q)(B_{1}\times B_{2})=
P(B_{1})Q(B_{2}), \;\, B_{1,2}\in {\BB}$, sog.\ Produkt-W-Maß von $P$ und
$Q$)

oder --- äquivalent ---
\item
Man wähle ein unabhängiges Paar reeller Zufallsvariablen $X,Y$ mit Verteilungen
$P_{X}=P$, $P_{Y}=Q$ und setze
\begin{align*}
P\ast Q\defl P_{X+Y}.\fishhere
\end{align*}
\end{defnenum}
\end{defn}

\begin{proof}[Beweis der Äquivalenz in \ref{defn:6.3}.]
Sei $Z=(X,Y)$ gemäß b), dann gilt
\begin{align*}
P_{X+Y} = P_{T\circ Z} = (P_Z)_T
\overset{!}{=} (P_X\otimes P_Y)_T.
\end{align*}
Um (!) nachzuweisen, genügt es zu zeigen, dass
\begin{align*}
\forall B_1,B_2\in\BB : \underbrace{(P_Z)(B_1\times B_2)}_{P[Z\in B_1\times
B_2]} = \underbrace{(P_X\otimes P_Y)(B_1\times B_2)}_{P_X(B_1)P_Y(B_2)}.\tag{*}
\end{align*}
Da $X,Y$ unabhängig sind, gilt
\begin{align*}
P[Z\in B_1\times B_2] &= P[(X,Y)\in B_1\times B_2] = P[X\in B_1, Y\in B_2]\\
&= P[X\in B_1]P[Y\in B_2]
\end{align*}
und somit ist die Identität (*) gezeigt. Mit Hilfe des Fortsetzungssatzes liegt
die Identität auf allen Borelschen Mengen aus $\BB_2$ vor.\qedhere 
\end{proof}

\begin{prop}
\label{prop:6.8}
\begin{propenum}
\item Die Faltungsoperation ist kommutativ und assoziativ.
\item
Sei $(X,Y)$ ein unabhängiges Paar reeller Zufallsvariablen. Für $X,Y$ und
$X+Y$ seien die Verteilungsfunktionen mit $F,G,H$, die Dichten --- falls existent --- mit
$f,g,h$ und die Zähldichten --- falls $P_{X}$, $P_{Y}$ auf $\mathbb{N}_{0}$
konzentriert sind --- mit $(p_{k})$, $(q_{k})$, $(r_{k})$ bezeichnet. Dann
gilt
\begin{align*}
H(t) = \int_{\R}F(t-y)\,G(\dy) = \int_{\R}G(t-x)\,F(dx), \;\, t\in
\R.
\end{align*}
Falls $f$ existiert, so existiert $h$ mit
\begin{align*}
h(t) =\int_{\R} f(t-y)\, G(\dy) \quad  \text{für (L-f.a.) }
t \in \R;
\end{align*}
falls zusätzlich $g$ existiert, so gilt
\begin{align*}
h(t) =\int_{\R}f(t-y) g(y)\dy= \int_{\R}g(t-x) f(x)\dx
\text{ \;für ($L$-f.a.) }t\in \R.
\end{align*}
Falls $(p_{k})_{k\in \mathbb{N}_{0}}, \;\, (q_{k})_{k\in \mathbb{N}_{0}}$
existieren, so existiert $(r_{k})_{k\in \mathbb{N}_{0}}$ mit
\begin{align*}
r_{k}= {\displaystyle\sum\limits^ {k}_{i=0}p_{k-i}q_{i}= \sum^
{k}_{i=0}q_{k-i}p_{i} \;\, (k\in \mathbb{N}_{0})}.\fishhere
\end{align*}
\end{propenum}
\end{prop}
\begin{proof}
\begin{proofenum}
\item Assoiziativität und Kommutativität der Faltungsoperation folgen direkt aus der
Assoziativität und Kommutativität der Addition reeller
Zahlen.
\item Siehe auch Satz \ref{prop:5.6}. Sei
\begin{align*}
H(t) = P[X+Y\le t] = P[(X,Y)\in B]
\end{align*}
mit $B\defl\setdef{(x,y)}{x+y\le t}$, also
% \begin{pspicture}(-1,-1)(4,3)
% 
% \psaxes[labels=none,ticks=none,linecolor=gdarkgray,tickcolor=gdarkgray]{->}%
%  (0,0)(-0.5,-0.5)(3.5,2.5)[\color{gdarkgray}$x$,-90][\color{gdarkgray}$y$,0]
% 
% \psline[linecolor=darkblue](-1,3)(3,0)
% 
% \rput(-1,1){\color{gdarkgray}$B$}
% \end{pspicture}
\begin{align*}
H(t) &= P_{(X,Y)}(B) = (P_X\otimes P_Y)(B) = \int_\R
P_X((-\infty,t-y])P_Y(\dy)\\
&= \int_\R F(t-y)G(\dy) \overset{\text{Komm. von $X+Y$}}{=}  \int_\R
G(t-x)F(\dx).
\end{align*}
Falls zu $F$ eine Dichte $f$ existiert, dann gilt für $t\in\R$,
\begin{align*}
H(t) &= \int_\R \left[\int\limits_{-\infty}^{t-y}
f(\tau)\dtau\right]G(\dy) \overset{\text{Fubini}}{=}
\int\limits_{-\infty}^{t}\left[\int_\R
f(\tau-y)G(\dy)\right]\dtau.
\end{align*}
Falls zusätzlich $G$ eine Dichte $g$ besitzt, dann ist $G(\dy)=g(y)\dy$ und
somit besitzt $H$ eine Dichte der Form
\begin{align*}
h(t) = \int\limits_\R f(t-y)g(y)\dy
=
\int\limits_\R g(t-x)f(x)\dx,
\end{align*}
wobei $h(t)$ nur $\fu{\lambda}$ definiert ist. 
Falls $X$ und $Y$ Zähldichten $(p_k)$ und $(q_k)$ besitzen, dann ist
\begin{align*}
r_k &\defl P[X+Y=k] = P[\exists i\in \setd{0,1,\ldots,k}: X=k-i, Y=i]\\
&= \sum\limits_{i=0}^k P[X=k-i, Y=i]
=\sum\limits_{i=0}^k P[X=k-i]P[Y=i]\\
&=\sum\limits_{i=0}^k p_{k-i}q_i
=\sum\limits_{i=0}^k q_{k-i}p_i.\qedhere
\end{align*}
\end{proofenum}
\end{proof}

Sind $X$ und $Y$ reelle unabhängige Zufallsvariablen mit Dichten $f$ und $g$,
so ist mit dem eben bewiesenen die Dichte von $X+Y$ gegeben durch
\begin{align*}
\int_\R f(x-y)g(y)\dy = \int_\R g(y-x)f(x)\dx.  
\end{align*}
Dieser Ausdruck wird als \emph{Faltung von $f$ mit $g$} bezeichnet, geschrieben
\begin{align*}
f*g\defr f_{X+Y}.
\end{align*}

\begin{bem}[Bemerkungen.]
\label{bem:6.7}
\begin{bemenum}
\item
Für $n_{1,2} \in \N$, $\, p\in [0,1]$ gilt
\begin{align*}
b(n_{1},p)\ast b(n_{2},p) =b(n_{1}+n_{2},p).
\end{align*}
Die Summe von $n$ unabhängigen $b(1,p)$ verteilten Zufallsvariablen ist also
$b(n,p)$-verteilt.
\item
Für $\lambda _{1,2} >0$ gilt
$\pi(\lambda _{1}) \ast \pi(\lambda _{2})= \pi
(\lambda _{1} +\lambda _{2})$.
\begin{proof}
$\pi(\lambda_1),\pi(\lambda_2)$ haben nach Bemerkung \ref{prop:6.2} die
erzeugenden Funktionen
\begin{align*}
g_1(s) = \e^{\lambda_1(s-1)},\qquad g_2(s) = \e^{\lambda_2(s-1)}.
\end{align*}
Im Satz \ref{prop:6.3} haben wir gezeigt, dass die Verteilungsfunktion einer
Zufallsvariable $X+Y$, wobei $X$ und $Y$ unabhängig und $X$ $\pi(\lambda_1)$
und $Y$ $\pi(\lambda_2)$ verteilt ist, die Form hat
\begin{align*}
\pi(\lambda_1)*\pi(\lambda_2).
\end{align*}
Diese hat die erzeugende Funktion
\begin{align*}
g_1(s)g_2(s) = \e^{(\lambda_1+\lambda_2)(s-1)},
\end{align*}
welche ebenfalls  die erzeugende Funktion von
$\pi(\lambda_1+\lambda_2)$ ist. Mit Satz \ref{prop:6.1} folgt nun, dass
$\pi(\lambda_1+\lambda_2)=\pi(\lambda_1)*\pi(\lambda_2)$.\qedhere
\end{proof}
\item
Für $r_{1,2} \in \mathbb{N}$, $\, p \in (0,1)$ gilt
\begin{align*}
Nb(r_{1},p) \ast Nb(r_{2},p)=Nb (r_{1} +r_{2},p).
\end{align*}
Die Summe von $r$ unabhängigen $Nb (1,p)$-verteilten Zufallsvariablen ist also
$Nb(r,p)$-verteilt.
\item
Für $a_{1,2}\in \R$, $\, \sigma^ {2}_{1,2}\in (0,\infty)$ gilt
\begin{align*}
N(a_{1},\sigma^ {2}_{1})\ast N(a_{2},\sigma ^ {2}_{2}) =
N(a_{1}+a_{2}, \sigma ^ {2}_{1} +\sigma ^ {2}_{2}).
\end{align*}
\item
Für $\lambda \in (0,\infty )$, $\, \nu _{1,2} \in (0,\infty)$ gilt
\begin{align*}
\Gamma _{\lambda,\nu _{1}}\ast \Gamma _{\lambda , \nu _{2}} =
\Gamma _{\lambda , \nu _{1} +\nu _{2}}.
\end{align*}
Die Summe von $n$ unabhängigen $\exp (\lambda)$-verteilten Zufallsvariablen ist
$\Gamma_{\lambda , n}$-verteilt. \\
Die Summe der Quadrate von $n$ unabhängigen jeweils $N(0,1)$-verteilten Zufallsvariablen
ist $\chi^ {2}_{n}$-verteilt.\maphere
\end{bemenum}
\end{bem}

Es lässt sich auch für nicht unabhängige reelle Zufallsvariablen $X$ und $Y$
nach der Verteilung von $X+Y$ fragen. Um diese zu berechnen kann man stets
folgenden Ansatz wählen,
\begin{align*}
F_{X+Y}(t) &= P[X+Y\le t] = P[g\circ (X,Y)\le t] = P[(X,Y)\in
g^{-1}(-\infty,t]]\\
&= P_{(X,Y)}[g^{-1}(-\infty,t]] = \int_{g^{-1}(-\infty,t]} 1 \dP_{(X,Y)}.
\end{align*}
Besitzen $X$ und $Y$ eine gemeinsame Dichte, so kann man das
eigentliche Integrationsgebiet $g^{-1}(-\infty,t]\cap \setd{f_{(X,Y)}\neq 0}$
(meißt geometrisch) bestimmen und so das Integral berechnen.

\cleardoublepage
\chapter{Spezielle Verteilungen}
\label{chap:7}

Wir wollen in diesem Kapitel die wesentlichen Eigenschaften der bisher
vorgestellten Verteilungen zusammenfassen.

\section{Diskrete Verteilungen}

Verteilungen, die auf höchstens abzählbare Mengen konzentriert sind, heißen
\emph{diskret}. Nimmt eine Zufallsvariable nur endlich oder
abzählbar viele Werte an, so ist ihre Verteilung $P_X$ diskret.

\begin{defn}
\label{defn:7.1}
Als \emph{Binomialverteilung} $b(n,p)$ mit Parametern $n\in
\N$, $\, p\in (0,1)$ oder auch $[0,1]$ wird ein W-Maß auf ${\BB}$
bezeichnet, das auf $\{0,1,\ldots,n\}$ konzentriert ist und die Zähldichte
\begin{align*}
k\mapsto b(n,p;k)\defl {n\choose k} p^ {k}(1-p)^ {n-k}, \;\, k\in \N_{0}
\end{align*}
besitzt.\fishhere
\end{defn}

\begin{prop}
\label{prop:7.1}
Sei $n\in \N$, $\, p\in [0,1]$; $\, q\defl 1-p$.\\[-7mm]
\begin{propenum}
\item
Ist $(X_{1}, \ldots , X_{n})$ ein $n$-Tupel unabhängiger reeller Zufallsvariablen mit
\begin{align*}
P[X_{i} =1] = p,\qquad P[X_{i}=0] = q,\qquad i=1,\ldots,n
\end{align*}
d.h. mit jeweiliger $b(1,p)$-Verteilung, so ist $\sum^{n}_{i=1} X_{i}$ \; $b(n,p)$-verteilt.
\item Für eine Zufallsvariable $X$ mit $P_{X}= b(n,p)$ gilt
\begin{align*}
\E X=np,\qquad\V(X) =npq.
\end{align*} 
\item
$b(n,p)$ hat die erzeugende Funktion $g$ mit $g(s) = (q+ps)^ {n}$.
\item
Für $n_{1,2} \in \N$ gilt
\begin{align*}
b(n_{1},p) \ast b(n_{2},p) = b (n_{1}+n_{2},p),
\end{align*}
d.h.\ für zwei unabhängige Zufallsvariablen $X_{1}$, $X_{2}$ mit
$P_{X_{1}}=b(n_{1},p)$ und $P_{X_{2}}= b(n_{2},p)$ gilt
$ P_{X_{1}+X_{2}}=b(n_{1}+n_{2},p)$.\fishhere
\end{propenum}
\end{prop}

\begin{bem}
\label{bem:7.1}
Das $n$-Tupel $(X_{1},\ldots, X_{n})$ von Zufallsvariablen aus Satz
\ref{prop:7.1} beschreibt ein $n$-Tupel von Bernoulli-Versuchen, d.h.\
Zufallsexperimenten ohne gegenseitige Beeinflussung mit Ausgängen 1
(``Erfolg'') oder 0 (``Misserfolg'') und jeweiliger Erfolgswahrscheinlichkeit
$p$; $\sum^ {n}_{k=1} X_{i}$ gibt die Anzahl der Erfolge in $n$
Bernoulli-Versuchen an.\maphere
\end{bem}

\begin{defn}
\label{defn:7.2}
Als \emph{Poisson-Verteilung} $\pi (\lambda )$ mit
Parameter $\lambda > 0$ wird ein W-Maß auf ${\BB}$ bezeichnet, das auf
$\N_{0}$ konzentriert ist und die Zähldichte
\begin{align*}
k\mapsto \pi (\lambda ; k) \defl \e^ {-\lambda }\frac{\lambda ^ {k}}{k!} \, , \;\,
k\in \N_{0}
\end{align*}
besitzt.\fishhere
\end{defn}

\begin{prop}
\label{prop:7.2}
Sei $\lambda >0$.\\[-7mm]
\begin{propenum}
\item
Ist $(b(n,p_{n}))_n$ eine Folge von Binomialverteilungen mit $n\,p_{n}\to
\lambda$ für $n\to~\infty$, dann konvergiert
\begin{align*}
b(n,p_{n}; k)\to \pi (\lambda; k) \qquad (n\to \infty) \mbox{ für alle } k\in
\N_{0}.
\end{align*}
\item
Für eine Zufallsvariable $X$ mit $P_{X}= \pi (\lambda ) $ gilt
\begin{align*}
\E X =\V(X) =\lambda.
\end{align*}
\item
$\pi (\lambda )$ hat die erzeugende Funktion $g$ mit $g(s) = \e^ {-\lambda
+\lambda s}$.
\item
Für $\lambda _{1,2} > 0$ gilt
\begin{align*}
\pi (\lambda _{1})\ast \pi (\lambda _{2}) = \pi (\lambda _{1} +\lambda _{2})\,
.\fishhere
\end{align*}
\end{propenum}
\end{prop}

\begin{bem}
\label{bem:7.2}
Bei einer rein zufälligen Aufteilung von $n$
unterscheidbaren Teilchen auf $m$ gleichgroße mehrfach besetzbare Zellen in
einem Euklidischen Raum wird die Anzahl der Teilchen in einer vorgegebenen
Zelle durch eine $b(n,\frac{1}{m})$-verteilte Zufallsvariable angegeben.

Der Grenzübergang $n\to \infty$, $m\to \infty$ mit {\it Belegungsintensität}
$\frac{n}{m} \to \lambda > 0 $ führt gemäß Satz \ref{prop:7.2}a zu einer
$\pi(\lambda)$-verteilten Zufallsvariable.\maphere
\end{bem}

\begin{defn}
\label{defn:7.3}
Als \emph{negative Binomialverteilung} oder \emph{
Pascal-Verteilung} $Nb (r,p)$ mit Parametern $r\in \N$, $p\in (0,1)$
wird ein W-Maß auf ${\BB}$ (oder auch $\overline{\BB}$) bezeichnet, das
auf $\N_{0}$ konzentriert ist und --- mit $q\defl 1-p$ --- die Zähldichte
\begin{align*}
k\mapsto Nb(r,p;k) \defl {r+k-1\choose k}p^ {r} q^ {k},\quad k\in \N
\end{align*}
besitzt. Speziell wird $Nb (1,p)$ --- mit Zähldichte
$k\mapsto p(1-p)^ {k},\; k\in \N_{0}$ ---
als \emph{geometrische Verteilung} mit Parameter $p\in (0,1)$
bezeichnet.\fishhere
\end{defn} 

\begin{prop}
\label{prop:7.3}
Sei $r \in \N$, $p\in (0,1)$, $q\defl 1-p$.
\begin{propenum}
\item
Sei $(X_{n})_{n\in \N}$ eine unabhängige Folge von $b(1,p)$-verteilten
Zufallsvariablen. Die erweitert-reelle Zufallsvariable
\begin{align*}
X\defl \inf\setdef{n\in \N}{\sum\limits^{n}_{k=1} X_{k}=r}-r
\end{align*}
mit $\inf \emptyset \defl\infty $ ist $Nb(r,p)$-verteilt.
\item
Für eine Zufallsvariable $X$ mit $P_{X} =Nb (r,p)$ gilt
\begin{align*}
\E X =\frac{rq}{p},\qquad \V(X) = \frac{rq}{p^ {2}}.
\end{align*}
\item
$Nb(r,p)$ hat die erzeugende Funktion $g$ mit $g(s) = p^ {r} (1-qs)^ {-r}$.
\item Für $r_{1,2} \in \N$  gilt
\begin{align*}
Nb(r_{1},p)\ast Nb (r_{2},p) =Nb (r_{1}+r_{2},p).
\end{align*}
Die Summe von $r$  unabhängigen $Nb(1,p)$-verteilten Zufallsvariablen ist somit
$Nb(r,p)$-verteilt.\fishhere
\end{propenum}
\end{prop}

\begin{bem}
\label{bem:7.3}
Für die Folge $(X_{n})$ in Satz \ref{prop:7.3}a wird durch die
erweitert-reelle Zufallsvariable
\begin{align*}
T\defl \inf \setdef{n\in \N}{\sum\limits^ {n}_{k=1} X_{k}=r}
\end{align*}
mit $\inf \emptyset \defl\infty$ die {\it Wartezeit} bis zum $r$-ten
Auftreten der Eins in $(X_{n})$ und durch die erweitert-reelle Zufallsvariable $X=T-r$  die
Anzahl der Misserfolge bis zum $r$-ten Erfolg bei der zu $(X_{n})$ gehörigen
Folge von Bernoulli-Versuchen angegeben.\maphere
\end{bem}

\begin{defn}
\label{defn:7.4}
Als \emph{hypergeometrische Verteilung} mit Parametern 
$n,r, s\in \N$, wobei $n\leq r+s$, wird ein W-Maß auf ${\BB}$
bezeichnet, das auf $\{0,1,\ldots,n\}$ --- sogar auf $\{\max \,(0,n-s),\ldots
,\min\, (n,r)\}$
--- konzentriert ist und die Zähldichte
\begin{align*}
k\mapsto
\begin{cases}
{\dfrac{{r\choose k} {s\choose n-k}}{{r+s\choose n}} }, & \quad
k=0,1,\ldots,n\\[2mm]
0, & \quad k=n+1, n+2,\ldots
\end{cases}
\end{align*}

besitzt.\fishhere
\end{defn}

\begin{bem}
\label{bem:7.4}
Werden aus einer Urne mit $r$ roten und $s$ schwarzen
Kugeln rein zufällig $n$ Kugeln ohne Zurücklegen gezogen ($n,r,s$ wie in
Definition \ref{defn:7.4}), so wird die Anzahl der gezogenen roten Kugeln durch
eine Zufallsvariable angegeben, die eine hypergeometrische Verteilung mit
Parametern $n,r,s $ besitzt. Anwendung der hypergeometrischen Verteilung in der
Qualitätskontrolle.\maphere
\end{bem}

\clearpage

\section{Totalstetige Verteilungen}

Totalstetige Verteilungen sind die Verteilungen, die eine Dichtefunktion
\begin{align*}
f: \R \to \R_+
\end{align*}
besitzen. Sie sind dann natürlich auch stetig, denn für ihre
Verteilungsfunktion gilt
\begin{align*}
F(t) = \int_{-\infty}^t f(x)\dx,
\end{align*}
aber nicht jede stetige Verteilung besitzt eine Dichtefunktion. Mit Hilfe der
Dichtefunktion können wir Randverteilungen, Erwartungswert, Varianz, Momente
\ldots\ sehr leicht berechnen. Totalstetige Verteilungen sind also
sehr angenehm.

\begin{defn}
\label{defn:7.5}
Als \emph{Gleichverteilung auf $ (a,b)$} mit $-\infty < a < b < \infty $
wird ein W-Maß auf ${\BB}$ bezeichnet, das eine Dichte
\begin{align*}
f=\frac{1}{b-a}\Id_{(a,b)}
\end{align*}
besitzt.\fishhere
\end{defn}

\begin{prop}
\label{prop:7.4}
Für eine Zufallsvariable $X$ mit Gleichverteilung auf $(a,b)$ gilt
\begin{align*}
\E X=\frac{a+b}{2},\qquad \V(X) = \frac{(b-a)^{2}}{12}.\fishhere
\end{align*}
\end{prop} 

\begin{defn}
\label{defn:7.6}
Als (eindimensionale) \emph{Normalverteilung} oder
\emph{Gauß-Verteilung} $N(a,\sigma ^ {2})$  mit Parametern $a\in \R$,
$\sigma > 0$ wird ein W-Maß auf ${\BB}$ bezeichnet, das eine Dichte $f$ mit
\begin{align*}
f(x) = \frac{1}{\sqrt{2 \pi }\sigma}\e^ {-\frac{(x-a)^ {2}}{2\sigma ^ {2}}}\, ,
\quad x\in \R
\end{align*}
besitzt. Speziell heißt $N(0,1)$ \emph{standardisierte
Normalverteilung}.\fishhere
\end{defn}

\begin{bem}[Bemerkungen.]
\label{bem:7.5}
\begin{bemenum}
\item
Sei $f$ wie in Definition \ref{defn:7.6}. Der Graph von $f$ heißt Gaußsche
Glockenkurve. Im Fall $a=0$ ist $f(0) ={\displaystyle\frac{1}{\sqrt{2 \pi
}\sigma}}$ und hat $f$ zwei Wendepunkte ${\displaystyle(\pm \sigma ;
\frac{1}{\sqrt{2 \pi e}\sigma })}\, $.
\begin{figure}
\centering
\begin{pspicture}(-2.8,-1)(2.8,3)

 \psaxes[labels=none,ticks=none,linecolor=gdarkgray,tickcolor=gdarkgray]{->}%
 (0,0)(-2.7,-0.5)(2.7,2.5)[\color{gdarkgray}$t$,-90][\color{gdarkgray}$\ph(t)$,0]

\psplot[linewidth=1.2pt,%
	     linecolor=darkblue,%
	     algebraic=true]%
	     {-2.5}{2.5}{2*(2.71828)^(-(3/2.5*x)^2/2)}
	     
\psxTick(0.91){\color{gdarkgray}\sigma}
\psxTick(-0.91){\color{gdarkgray}-\sigma}

\psyTick(2){\color{gdarkgray}\frac{1}{\sqrt{2\pi}\sigma}}

\psline[linestyle=dashed](0.913,0)(0.913,1.1)
\psline[linestyle=dashed](-0.913,0)(-0.913,1.1)

\rput(1.4,1.4){\color{gdarkgray}$\ph$}
\end{pspicture}
\caption{Gaußsche Glockenkurve mit Wendepunkten}
\end{figure}
\item
Dichte- und die Verteilungsfunktion $\phi $ von $N(0,1)$ sind tabelliert.
\item
Ist die Zufallsvariable $X$ $N(a,\sigma ^ {2})$-verteilt, so ist
${\displaystyle\
\frac{X-a}{\sigma }}$ $N(0,1)$-verteilt. Es gilt hierbei
\begin{center}
$\begin{array}{lllll}
P[a-\sigma \leq X\leq a+\sigma ]& = & 2\phi (1) -1& \approx & 0,683\\[2mm]
P[a-2\sigma \leq X\leq a+2\sigma ]& = & 2\phi (2) -1& \approx & 0,954\\[2mm]
P[a-3\sigma \leq  X\leq a+3\sigma ]& =  & 2\phi (3) -1& \approx & 0,997\, .
\end{array}$
\end{center}
\item
Anwendung der Normalverteilung z.B.\ in der Theorie der
Beobachtungsfehler.\maphere
\end{bemenum}
\end{bem}

\begin{prop}
\label{prop:7.5}
 Sei $a \in \R$, $\sigma > 0$.
\begin{propenum}
\item Für eine Zufallsvariable $X$ mit Verteilung $N(a,\sigma ^ {2})$ gilt:
\begin{align*}
\E(X-a)^ {2k-1} = 0,\qquad \E(X-a)^ {2k} =\sigma ^ {2k} \prod
\limits^ {k}_{j=1} (2j-1), \quad k\in \N\, ;
\end{align*}
insbesondere $\E X =a$, $\V(X) = \sigma ^ {2}$. Die Zufallsvariable $Y=cX+b$ mit
$ 0 \neq c\in \R$, $\, b\in \R$ hat die Verteilung $N(ca+b,\,\; c^ {2} \sigma
^{2})$.
\item
Die charakteristische Funktion von $N(a,\sigma ^ {2})$ ist $\varphi $ mit
\begin{align*}
\ph (u) = \e^ {iau} \e^ {-\frac{\sigma ^ {2} u^ {2}}{2}}\, .
\end{align*}
\item
Für $a_{1,2} \in \R$, $\sigma _{1,2} >0$ gilt
\begin{align*}
N(a_{1}, \sigma ^ {2}_{1}) \ast N(a_{2},\sigma ^ {2}_{2}) = N (a_{1} +a_{2},
\sigma ^ {2}_{1} +\sigma ^ {2}_{2})\, .\fishhere
\end{align*}
\end{propenum}
\end{prop}

\begin{defn}
\label{defn:7.7}
Als \emph{Exponentialverteilung} $\exp (\lambda)$ mit
Parameter $\lambda >0$ wird ein W-Maß auf ${\BB}$ oder $\overline{\BB}$
bezeichnet, das eine Dichtefunktion $f$ mit
\begin{align*}
f(x)=
\begin{cases}
\lambda \e^ {-\lambda x}, & x>0,\\
0, & x\leq 0,
\end{cases}
\end{align*}
besitzt.\fishhere
\end{defn}

\begin{prop}
\label{prop:7.6}
Sei $\lambda >0$.
\begin{propenum}
\item
Sei $X$ eine erweitert-reelle Zufallsvariable, deren Verteilung auf
$\overline{\R}_{+}$ konzentriert sei, mit $P[0<X<\infty] >0$.
$X$ erfüllt
\begin{align*}
\forall s,t\in (0,\infty ) :
P[X>t+s\mid X>s] = P[X>t].
\end{align*}
genau dann, wenn $P_{X}$  eine Exponentialverteilung ist. Diese Eigenschaft
heißt ``\emph{Gedächtnislosigkeit}''.
\item Für eine Zufallsvariable $X$ mit $P_{X} = \exp (\lambda ) $ gilt
\begin{align*}
\E X = \frac{1}{\lambda}, \qquad \V(X) = 
\frac{1}{\lambda ^ {2}}.
\end{align*}
\item
$\exp (\lambda )$ hat die charakteristische Funktion $\varphi $  mit
\begin{align*}
\ph (u) = {\displaystyle \frac{\lambda}{\lambda -iu}}.\fishhere
\end{align*}
\end{propenum}
\end{prop}

\begin{bem}
\label{bem:7.6}
Die zufällige Lebensdauer eines radioaktiven Atoms wird
durch eine exponentialverteilte Zufallsvariable angegeben.\maphere
\end{bem}

\begin{defn}
\label{defn:7.8}
Als \emph{Gamma-Verteilung} $\Gamma _{\lambda, \nu}$ mit
Parametern $\lambda, \nu >0$ wird ein W-Maß auf ${\BB}$ oder $\overline{\cal
B}$ bezeichnet, das eine Dichtefunktion $f$ mit
\begin{align*}
f(x) = \left\{ \begin{array}{ll}
{\displaystyle\frac{\lambda ^ {\nu}}{\Gamma (\nu)}}x^ {\nu -1} \e^ {-\lambda
x},& x>0\\[2mm]
0 ,& x\leq 0
\end{array}\right.
\end{align*}
besitzt. Hierbei gilt $\Gamma _{\lambda,1} = \exp (\lambda)$. $\Gamma
_{\frac{1}{2}, \frac{n}{2}}$ wird als Chi-Quadrat-Verteilung $\chi ^ {2}_{n}$ mit $n (\in
\N)$ Freiheitsgraden bezeichnet.\fishhere
\end{defn}

\begin{prop}
\label{prop:7.7}
Seien $\lambda, \nu  >0,\;\, n\in \N$.
\begin{propenum}
\item
Für eine Zufallsvariable $X$ mit $P_{X}=\Gamma _{\lambda, \nu}$ gilt $\E X
=\frac{\nu}{\lambda}$, $\V(X) = \frac{\nu}{\lambda ^ {2}}\, .$
\item
$\Gamma _{\lambda , \nu}$ hat die charakteristische Funktion $\varphi $ mit
\begin{align*}
\ph(u) = {\displaystyle\left( \frac{\lambda }{\lambda -iu}\right)^ {\nu}}.
\end{align*}
\item
Für $\nu _{1,2} >0$ gilt
\begin{align*}
\Gamma _{\lambda , \nu _{1}}\ast \Gamma _{\lambda , \nu _{2}} = \Gamma _{\lambda
, \nu _{1}+\nu_{2}}.
\end{align*}
Insbesondere ist $\Gamma _{\lambda , n }$ die $n$-fache Faltung von $\exp
(\lambda )$.
\item
Die Summe der Quadrate von $n$ unabhängigen jeweils $N(0,1)$-verteilten reellen
Zufallsvariablen ist $\chi^ {2}_{n}$-verteilt.\fishhere
\end{propenum}
\end{prop}

\begin{bem}
\label{bem:7.7}
Sind $X,Y$ zwei unabhängige reelle Zufallsvariablen mit $P_{X}=N(0,1)$
und $P_{Y}=\chi ^ {2}_{n}$, so wird die Verteilung der Zufallsvariablen
$\dfrac{X}{\sqrt{Y/n}}$ als \emph{$t$
-Verteilung} oder \emph{Student-Verteilung} $t_n$ bzw.\ $St_{n}$  mit $n$
Freiheitsgraden bezeichnet $(n\in \N)$.
$t_1$ bzw.\ $St_{1}$ ist eine sogenannte \emph{Cauchy-Verteilung}
--- Anwendung von $\chi^ {2}_{n}$ und $St_{n}$ in der Statistik.\maphere
\end{bem}

Sei $\setdef{T_{n}}{n\in \N}$ eine unabhängige Folge
nichtnegativ-reeller Zufallsvariablen
(d.h.\ $T_{i}$ und~$T_{j}$ haben jeweils dieselbe Verteilung)
mit $P[T_{n}=0] < 1$.

Definiere die Zufallsvariable 
$N_{t}\defl \sup \setdef{n\in \N}{T_{1} +\ldots +T_{n} \leq t}$ für $t\in\R_{+}$
und der Konvention $\sup \emptyset = 0$.

\begin{bsp}
In einem technischen System wird ein Bauelement mit endlicher
zufälliger Lebensdauer bei Ausfall durch ein gleichartiges Element ersetzt. Die
einzelnen Lebensdauern seien Realisierung der Zufallsvariablen $T_{n}$. $N_{t}$ gibt die
Anzahl der Erneuerungen im Zeitintervall $[0,t]$ an.\bsphere
\end{bsp}

\begin{prop}
\label{prop:7.8}
Ist --- mit den obigen Bezeichnungen --- $T_{n}$ $\exp (\lambda
)$-verteilt $(\lambda > 0)$, so ist $N_{t}$ $\pi (\lambda t)$-verteilt, $t\in
\R_{+}$.\fishhere
\end{prop}
\begin{proof}
Setze
$S_k\defl T_1+\ldots+T_k$ mit $k\in\N$, wobei $S_0=0$. Gesucht ist nun
\begin{align*}
P[N_t = k] = P[S_k\le t, S_{k+1}> t] = P[S_k\le t]-P[S_{k+1}\le t],
\end{align*}
denn $[S_{k+1}> t]=[S_{k+1}\le t]^c$ und somit ist
\begin{align*}
[S_k\le t]\cap[S_{k+1}\le t]^c = [S_k\le t]\setminus[S_{k+1}\le t].
\end{align*}
Nun ist $S_k$ $\Gamma_{\lambda,k}$ verteilt und $S_{k+1}$
$\Gamma_{\lambda,k+1}$ verteilt. Für die Dichte erhalten wir somit,
\begin{align*}
P[N_t = k] &= 
\int_0^t \frac{\lambda^k}{(k-1)!}x^{k-1}\e^{-\lambda x}\dx-
\int_0^t \frac{\lambda^{k+1}}{k!}x^{k}\e^{-\lambda x}\dx\\
&= \e^{-\lambda t}\frac{(\lambda t)^k}{k!}
\end{align*}
Der letzte Schritt folgt durch Differenzieren der Integrale nach $t$,
Zusammenfassen und anschließendes Integrieren.\qedhere
\end{proof}
\begin{bem}
\label{bem:7.8}
Die Familie $\{N_{t}\mid t\in \R_{+}\}$ aus Satz
\ref{prop:7.8} ist ein sogenannter \emph{Poisson-Prozess}.\maphere
\end{bem} 

\cleardoublepage
\chapter{Gesetze der großen Zahlen}
\label{chap:8}

\section{Konvergenzbegriffe}

\begin{defn}
\label{defn:8.1}
Seien $X_n$, $X$ reelle Zufallsvariablen (für $n\in\N$) auf einem W-Raum
$(\Omega,\AA,P)$. Die Folge $(X_n)$ heißt gegen $X$
\begin{defnenum}
  \item \emph{vollständig konvergent}, wenn für jedes $\ep>0$ gilt
\begin{align*}
\sum_{n\in\N} P[\abs{X_n-X}] \ge \ep]\to 0.
\end{align*}
Schreibweise $X_n\cto X$.
\item \emph{konvergent $P$-fast sicher} ($P$-f.s., $\fs$), wenn
\begin{align*}
P[X_n\to X] = 1.
\end{align*}
\item \emph{konvergent nach Wahrscheinlichkeit} oder \emph{stochastisch
konvergent}, wenn für jedes $\ep > 0$ gilt
\begin{align*}
P[\abs{X_n-X}\ge \ep]\to 0.
\end{align*}
Schreibweise $X_n\Pto X$,
\item \emph{konvergent im $r$-ten Mittel} ($r>0$), wenn $\E \abs{X_n}^r, \E
\abs{X}^r < \infty$ für $n\in\N$ und
$
\E\abs{X_n-X}^r \to 0.
$

Spezialfall: $r=2$, \ldots konvergenz im quadratischen Mittel.
\item \emph{konvergent nach Verteilung}, wenn für jede beschränkte stetige
Funktion $g: \R\to\R$ gilt
\begin{align*}
\E g(X_n) \to \E g(X),\qquad n\to\infty,
\end{align*}
oder --- hier äquivalent --- wenn für die Verteilungsfunktion $F_n$ und $F$ von
$X_n$ bzw. $X$ gilt
\begin{align*}
F_n(x)\to F(x),\qquad n\to\infty,
\end{align*}
in jedem Stetigkeitspunkt $x$ von $F$. Schreibweise $X_n\Dto X$.\fishhere
\end{defnenum}
\end{defn}

\begin{bem}[Bemerkungen.]
\label{bem:8.1}
\begin{bemenum}
  \item Seien $F_n$ Verteilungsfunktionen für $n\in\N$, dann impliziert $(F_n)$
  konvergiert \textit{nicht}, dass $\lim\limits_{n\to\infty} F_n$ eine
  Verteilungsfunktion ist.
  \item Die Grenzverteilungsfunktion in Definition \ref{defn:8.1} e) ist
  eindeutig bestimmt.
  \item Die Konvergenz in e.) entspricht einer
  schwach*-Konvergenz der $F_n$ im Dualraum von $C_b(\R)$.\maphere
\end{bemenum}
\end{bem}

\begin{prop}
\label{prop:8.1}
Seien $X_n$, $X$ reelle Zufallsvariablen auf einem W-Raum $(\Omega,\AA,P)$.
\begin{propenum}
  \item Falls $X_n\cto X$, so gilt $X_n\to X$ f.s.
  \item Falls $X_n\to X$ f.s., so gilt $X_n\Pto X$.
  \item Falls für $r> 0$, $X_n\to X$ im $r$-ten Mittel, so gilt $X_n\Pto X$.
  \item Falls $X_n\Pto X$, so gilt $X_n\Dto X$.\fishhere
\end{propenum}
\end{prop}
\begin{proof}
\textit{Vorbetrachtung}. Seien $X_n,X$ reelle Zufallsvariablen, so gilt
% $X_n\to X$ f.s.,
%dies ist äquivalent zu
\begin{align*}
X_n\to X\Pfs &\Leftrightarrow P[\lim\limits_{n\to\infty} X_n = X] = 1\\
&\Leftrightarrow
P\left(\bigcap_{0<\ep\in\Q}\bigcup_{n\in\N}\bigcap_{m\ge n} \left[\abs{X_m -
X}<\ep\right]\right) = 1
\end{align*}
Wobei sich die letzte Äquivalenz direkt aus dem $\ep-\delta$-Kriterium für
konvergente Folgen ergibt, wenn man sich für die Wahl von $\ep$ auf
$\Q$ zurückzieht.

Dies ist natürlich äquivalent zu
\begin{align*}
&P\left(\bigcup_{0<\ep\in\Q}\bigcap_{n\in\N}\bigcup_{m\ge n} \left[\abs{X_m -
X}\ge\ep\right]\right) = 0\\
\Leftrightarrow
&\forall \ep > 0 :  P\left(\bigcap_{n\in\N}\underbrace{\bigcup_{m\ge n}
\left[\abs{X_m - X}\ge\ep\right]}_{\defl A_{n}(\ep)}\right) = 0
\end{align*}
Nun ist $A_n\downarrow A$, d.h. wir können dies auch schreiben als
\begin{align*}
&\forall \ep > 0 : 
\lim\limits_{n\to\infty}P\left(A_n(\ep)\right) = 0\\
\Leftrightarrow
&
\forall \ep > 0 : 
\lim\limits_{n\to\infty}P\left(\sup_{m\ge n} \abs{X_m-X}\ge \ep\right) = 0.
\end{align*}
Weiterhin gilt
\begin{align*}
P[\abs{X_n-X}\ge \ep] \le
P\left[\sup \abs{X_n-X}\ge \ep\right] \le
\sum\limits_{m\ge n} P\left[\abs{X_n-X}\ge \ep\right].
\end{align*}
\begin{proofenum}
  \item Falls $X_n\cto X$, dann gilt $\sum_{m\ge n}
  P\left[\abs{X_n-X}\ge \ep\right]\to 0$ und damit auch
\begin{align*}
X_n\to X \fs
\end{align*} 
\item Falls $X_n\to X$ f.s., so gilt $P\left[\sup \abs{X_n-X}\ge \ep\right]\to
0$ und damit auch
\begin{align*}
P[\abs{X_n-X}\ge \ep]\to 0.
\end{align*}
\item Unter Verwendung der Markov-Ungleichung erhalten wir
\begin{align*}
P\left[\abs{X_n-X}\ge \ep\right] \le \frac{1}{\ep^r}\E \abs{X_n-X}^r
\end{align*}
also $X_n\Pto X$, wenn $X_n\to X$ im $r$-ten Mittel.
\item Sei  $g:\R\to\R$ beschränkt und stetig und $X_n\Pto X$. Es existiert also
zu jeder Teilfolge $(X_{n'})$ eine konvergente Teilfolge $(X_{n''})$, die gegen
$X$ konvergiert. Somit gilt
$g(X_{n''})\to g(X)$ f.s.

Mit dem Satz von der dominierten Konvergenz erhalten wir somit
\begin{align*}
\E g(X_{n''})\to \E g(X).
\end{align*}
Nun verwenden wir den Satz aus der Analysis, dass eine Folge genau dann
konvergiert, wenn jede Teilfolge eine konvergente Teilfolge besitzt, und alle
diese Teilfolgen denselben Grenzwert haben und erhalten somit, $\E g(X_n)\to \E
g(X)$, d.h. $X_n\Dto X$.\qedhere
\end{proofenum}
\end{proof}

\section{Schwache und starke Gesetze der großen Zahlen}

\begin{defn}
\label{defn:8.2}
Eine Folge $(X_n)$ von integrierbaren reellen Zufallsvariablen auf einem W-Raum
$(\Omega,\AA,P)$ genügt dem \emph{schwachen} bzw. \emph{starken Gesetz der
großen Zahlen}, wenn
\begin{align*}
\frac{1}{n}\sum\limits_{k=1}^n (X_k-\E X_k) \to 0,\qquad n\to \infty\text{ nach
Wahrscheinlichkeit bzw. } P\text{-f.s.}\fishhere
\end{align*}
\end{defn}

\begin{bem}
\label{bem:8.2}
Genügt eine Folge von Zufallsvariablen dem starken Gesetzt der großen Zahlen,
dann genügt sie auch dem schwachen Gesetz der großen Zahlen. Die Umkehrung gilt
im Allgemeinen nicht.\maphere
\end{bem}

\begin{prop}[Kriterium von Kolmogorov für das starke Gesetz der großen Zahlen]
\label{prop:8.2}
Eine unabhängige Folge $(X_n)$ quadratisch integrierbarer reeller
Zufallsvariablen mit
\begin{align*}
\sum\limits_{n=1}^\infty n^{-2}\V(X_n) < \infty
\end{align*}
genügt dem starken Gesetz der großen Zahlen.\fishhere
\end{prop}

\begin{proof}
Wir wollen den Satz zunächst nur unter der stärkeren Bedingung
\begin{align*}
\sup_n \V(X_n) \defr C < \infty
\end{align*}
(jedoch unter der schwächeren Voraussetzung der paarweisen Unkorreliertheit der
$(X_n)$ anstatt der Unabhängigkeit). Einen vollständigen Beweis werden wir mit
Hilfe der Martingaltheorie geben können.

Wir zeigen nun
\begin{align*}
\forall \ep > 0 : P\left[\abs{\frac{1}{n} \sum\limits_{j=1}^n X_j - \E
X_j} \ge \ep\right] \le \frac{C}{n\ep^2}\tag{1}
\end{align*}
und
\begin{align*}
\frac{1}{n}\sum\limits_{j=1}^n (X_j-\E X_j) \to 0
\end{align*}
im quadratischen Mittel und $\Pfs$. Setze $Y_j \defl X_j-\E X_j$, $Z_n \defl
\frac{1}{n}\sum\limits_{j=1}^n Y_j$. Mit Hilfe der Markov-Ungleichung erhalten
wir
\begin{align*}
\ep^2 P\left[\abs{Z_n} \ge \ep\right] \le \V(Z_n)
= \frac{1}{n^2}\V\left(\sum\limits_{j=1}^n Y_j\right)
\overset{\text{Bienayme}}{=} \frac{1}{n^2}\sum\limits_{j=1}^n\V\left( Y_j\right)
\le \frac{C}{n}.
\end{align*}
Somit folgt (1) und damit
\begin{align*}
\frac{1}{n}\sum\limits_{j=1}^n \left(X_j - \E X_j\right) \to 0,\qquad
\text{im quadratischen Mittel}.
\end{align*}
Es verbleibt die $\Pfs$ konvergenz zu zeigen.
Betrachte dazu
\begin{align*}
\sum\limits_{n=1}^\infty P[\abs{Z_{n^2}}\ge \ep] \le
\frac{C}{\ep^2}\sum\limits_{n=1}^\infty \frac{1}{n^2}.
\end{align*}
also konvergiert $Z_{n^2}$ vollständig und insbesondere auch $\Pfs$ gegen $0$
(mit Satz \ref{prop:8.1}). Sei nun $n\in\N$, setze
\begin{align*}
m(n) \defl \max \setdef{m\in\N}{m^2\le n}
\end{align*}
dann folgt $m(n)^2 \le n < (m(n)+1)^2$. 
\begin{align*}
\abs{Z_n} \le \abs{\frac{1}{n}\sum\limits_{j=1}^{m(n)^2} Y_j} + 
\underbrace{\abs{\frac{1}{n} \sum\limits_{j=m(n)^2+1}^n Y_j}}_{\defr R_n}
\end{align*}
Nun ist $\abs{\frac{1}{n}\sum\limits_{j=1}^{m(n)^2} Y_j} \le
\abs{\frac{1}{m(n)^2}\sum\limits_{j=1}^{m(n)^2} Y_j} \to 0\Pfs$. Es genügt also
zu zeigen, dass $R_n\to 0\Pfs$. Sei also $\ep >0$, dann folgt mit der
Markov-Ungleichung und dem Satz von Bienayme
\begin{align*}
\ep^2P\left[\abs{R_n}\ge \ep\right] &\le \V(R_n)
\le \frac{1}{n^2}\sum\limits_{j=m(n)^2+1}^n \V(Y_j)
\le \frac{C}{n^2} (n-m(n)^2)\\
&\le  \frac{C}{n^2} ((m(n)+1)^2-1-m(n)^2)\\
&=   \frac{C}{n^2} (2m(n)) \le 2C\frac{\sqrt{n}}{n^2}
= 2C n^{-3/2}.
\end{align*}
Da $\sum_{n\ge 0}n^{-3/2}<\infty$, folgt $(R_n)$ konvergiert vollständig gegen
$0$.\qedhere
\end{proof}

\begin{prop}[Kolmogorovsches starkes Gesetz der großen Zahlen]
\label{prop:8.3}
Für eine unabhängige Folge $(X_n)$ identisch verteilter integrierbarer reeller
Zufallsvariablen gilt
\begin{align*}
\frac{1}{n}\sum\limits_{k=1}^n X_k \to \E X_1,\qquad n\to \infty,\;
f.s.\fishhere
\end{align*}
\end{prop}
\begin{proof}
Um die quadratische Integrierbarkeit der Zufallsvariablen zu
gewährleisten, führen wir gestutzte Zufallsvariablen ein durch
\begin{align*}
X_i' \defl X_i\Id_{[\abs{X_i}\le i]}
\end{align*}
Insbesondere folgt aus der Unabhängigkeit der $X_n$ auch die Unabhängigkeit der
$X_n'$. Setze weiterhin
\begin{align*}
&S_n' = \sum\limits_{i=1}^n X_i',\quad
S_n \defl \sum\limits_{i=1}^n X_i,\\
&k_n \defl \lfloor \th^n\rfloor 
\end{align*}
für beliebiges aber fest gewähltes $\th>1$.
\begin{align*}
&\ep^2\sum\limits_{n=1}^\infty P\left[ \frac{1}{k_n}\abs{S_{k_n}'-\E S_{k_n}'}
\right] 
\le \sum\limits_{n=1}^\infty \frac{1}{k_n^2}\V(S_{k_n}')
= \sum\limits_{n=1}^\infty \frac{1}{k_n^2} \sum\limits_{k=1}^{k_n} \V(X_k')\\
&\le \sum\limits_{n=1}^\infty \frac{1}{k_n^2} \sum\limits_{k=1}^{k_n} \E
((X_k')^2)
\le 
\sum\limits_{n=1}^\infty \frac{1}{k_n^2} \sum\limits_{k=1}^{k_n} \E
(X_k^2\Id_{[\abs{X_k}\le k]})\\
&\le 
\sum\limits_{n=1}^\infty \frac{1}{k_n^2} \sum\limits_{k=1}^{k_n} \E
(X_1^2\Id_{[\abs{X_1}\le k_n]}),
\end{align*}
denn da alle Zufallsvariablen dieselbe Verteilung besitzten, ist es
für den Erwartungswert unerheblich, welche Zufallsvariable speziell ausgewählt
wird. Aufgrund der Linearität von $\E$ und dem Satz der monotonen Konvergenz
folgt,
\begin{align*}
\le
\E\left(X_1^2 \sum\limits_{n=1}^\infty \frac{1}{k_n}
\Id_{[\abs{X_1}\le k_n]}\right).
\end{align*}
Setzen wir $n_0 \defl \min\setdef{n\in\N}{\abs{X_1}\le k_n}$, so gilt
\begin{align*}
=
\E\left(X_1^2 \sum\limits_{n=n_0}^\infty \frac{1}{k_n}
\Id_{[\abs{X_1}\le k_n]}\right).
\end{align*}
Außerdem $\abs{X_1}\le k_{n_0} = \lfloor \th^{n_0} \rfloor \le \th^{n_0}$, also
\begin{align*}
\sum\limits_{n=n_0}^\infty \frac{1}{k_n}\Id_{[\abs{X_1}\le k_n]}
\le
\sum\limits_{n=n_0}^\infty \frac{1}{\lfloor\th^n\rfloor}
\le \frac{1}{\th^{n_0}}
\sum\limits_{n=0}^\infty \frac{2}{\th^n}
= \frac{2}{\th^{n_0}}\frac{1}{1-\frac{1}{\th}}
\le \frac{2}{\abs{X_1}}\frac{\th}{\th-1},
\end{align*}
wobei $\frac{1}{\lfloor\th^{n_0}\rfloor} \le \frac{2}{\th^{n_0}}$, denn es gilt
allgemein für positive reelle Zahlen  $\lfloor p\rfloor  > \frac{p}{2}$. Also
erhalten wir für den Erwartungswert,
\begin{align*}
\E\left(X_1^2 \sum\limits_{n=n_0}^\infty \frac{1}{k_n}
\Id_{[\abs{X_1}\le k_n]}\right)
\le \E \left(X_1^2 \frac{2}{\abs{X_1}}\frac{\th}{\th-1}\right)
= \frac{2\th}{\th-1}\E \abs{X_1} < \infty,
\end{align*}
da $\E\abs{X_n}<\infty$ nach Voraussetzung.

Mit Satz \ref{prop:8.1} folgt
\begin{align*}
\frac{1}{k_n}\left(S_{k_n}'-\E S_{k_n}'\right) \to 0,\qquad \Pfs\text{ für }
n\to\infty
\end{align*}
wobei
\begin{align*}
\frac{1}{k_n}\E S_{k_n}' = \frac{1}{k_n}\sum\limits_{k=1}^{k_n} \E (X_1
\Id_{[\abs{X_1}\le k]}).
\end{align*}
Mit dem Satz von der dominierten Konvergenz folgt $\E (X_1
\Id_{[\abs{X_1}\le k]})\to \E X_1$ und damit,
\begin{align*}
\frac{1}{k_n}\sum\limits_{k=1}^{k_n} \E (X_1
\Id_{[\abs{X_1}\le k]})\to \E X
\end{align*}
nach dem Satz von Césaro–Stolz bzw. dem Cauchyschen Grenzwertsatz.

Außerdem konvergiert $\dfrac{S_{k_n}}{k_n}\to\E X_1\Pfs$, denn
\begin{align*}
\sum\limits_{n=1}^\infty P[X_n=X_n'] &= 
\sum\limits_{n=1}^\infty P[\abs{X_n}>n] 
\le
\sum\limits_{n=1}^\infty \int\limits_{n-1}^n P[\abs{X_n}>t]\dt\\
&\overset{\text{dom.konv}}{=}
\int\limits_{0}^\infty P[\abs{X_n}>t]\dt
=
\E X_1.
\end{align*}
Mit Lemma von Borel-Cantelli folgt dass $X_n$ und $X_n'$ für alle bis auf
endlich viele $n$ übereinstimmen, damit
\begin{align*}
\frac{S_{k_n}-S_{k_n}'}{k_n}\to 0,\qquad \Pfs \text{ für }n\to\infty.
\end{align*}
Zusammenfassend erhalten wir also
\begin{align*}
\frac{S_{k_n}}{k_n}
=
\underbrace{\frac{S_{k_n}-S_{k_n}'}{k_n}}_{\to0\fs}
+
\underbrace{\frac{S_{k_n}'-\E S_{k_n}'}{k_n}}_{\to 0\fs}
+
\underbrace{\frac{\E S_{k_n}'}{k_n}}_{\to\E X_1\fs} \to \E X_1\fs
\end{align*}

Sei zunächst $X_i\ge 0$ für alle $i\in\N$. Dann gilt für alle $i$ mit $k_n\le
i\le k_{n+1}$,
\begin{align*}
\frac{k_n}{k_{n+1}}\frac{S_{k_n}}{k_n}
\le \frac{k_n S_i}{k_{n+1}k_n}
\le \frac{S_i}{i}
\le \frac{S_{k_{n+1}}}{i}
\le \frac{S_{k_{n+1}}}{k_{n+1}}\frac{k_{n+1}}{k_n}.
\end{align*}
\textit{per definitionem} ist 
\begin{align*}
\frac{k_n}{k_{n+1}} = \frac{\lfloor \th^n\rfloor}{\lfloor \th^{n+1}\rfloor}
\ge \frac{\th^n-1}{\th^{n+1}} \to \frac{1}{\th} 
\end{align*}
analog erhält man $\frac{k_{n+1}}{k_n}\to \th$. Somit folgt
\begin{align*}
\frac{1}{\th}\E X_1 \le \liminf_n \frac{S_n}{n}
\le \limsup_n \frac{S_n}{n} \le \th \E X_1.
\end{align*}
Da $\th>1$ beliebig und obige Gleichung für alle $\th$ gilt, folgt durch den
Übergang $\th\downarrow 1$,
\begin{align*}
\E X_1 = \lim\limits_{n\to\infty} \frac{S_n}{n},\qquad \Pfs\quad n\to \infty.
\end{align*}
Sei jetzt $X_i$ reellwertig, mit $X_i = X_i^+-X_i^-$, dann ist
\begin{align*}
\frac{1}{n}\sum\limits_{i=1}^n X_i 
= \frac{1}{n}\sum\limits_{i=1}^n X_i^+ -
\frac{1}{n}\sum\limits_{i=1}^n X_i^-
\to \E X_1^+ - \E X_1^- = \E X_1,\quad \Pfs \text{ für }n\to\infty.\qedhere
\end{align*}
\end{proof}


\begin{bem}[Bemerkungen.]
\label{bem:8.3}
\begin{bemenum}
\item In Satz \ref{prop:8.3} darf die Integrierbarkeitsvoraussetzung nicht
weggelassen werden.
\item Die Voraussetzung der Unabhängigkeit in Satz \ref{prop:8.4} kann zur
Voraussetzung der paarweisen Unabhängigkeit abgeschwächt werden (Etemadi).
\item In Satz \ref{prop:8.2} kann bei $\sum n^{-2}\V (X_n)(\log n)^2< \infty$
die Unabhängigkeitsvoraussetzung zur Voraussetzung der paarweisen
Unkorreliertheit (s.u.) abgeschwächt werden (Rademacher, Menchov).\maphere
\end{bemenum}
\end{bem}

\begin{prop}[Satz von Tschebyschev]
\label{prop:8.4}
Eine Folge $(X_n)$ quadratisch integrierbarer paarweise unkorrelierter, d.h. es gilt
\begin{align*}
\forall j\neq k : \E(X_j-\E X_j)(X_k-\E X_k) = 0,
\end{align*}
reeller Zufallsvariablen mit
\begin{align*}
n^{-2}\sum\limits_{k=1}^n \V(X_k)\to 0,\qquad n\to\infty,
\end{align*}
genügt dem schwachen Gesetz der großen Zahlen.\fishhere
\end{prop}

\begin{bem}
\label{bem:8.4}
Für eine unabhängige Folge $(X_n)$ identisch verteilter reeller
Zufallsvariablen mit $P[X_1=1]=p$, $P[X_1=0]=1-p$ (festes $p\in[0,1]$) gilt
\begin{align*}
\frac{1}{n}\sum\limits_{k=1}^n X_k \to p,\qquad n\to\infty \quad \Pfs \text{ und
nach Wahrscheinlichkeit}.
\end{align*}
Borelsches Gesetz der großen Zahlen bzw. Bernoullisches schwaches Gesetzt der
großen Zahlen.\maphere
\end{bem}

Bemerkung \ref{bem:8.4} stellt ein theoretisches Gegenstück zu der
Erfahrungstatsache dar, dass im Allgemeinen die relative Häufigkeit eines
Ereignisses bei großer Zahl der unter gleichen Bedingungen unabhängig 
durchgeführten Zufallsexperimente näherungsweise konstant ist.

Wir betrachten nun zwei Beispiele zu Satz \ref{prop:8.3} und Bemerkung
\ref{bem:8.4}.
\begin{bsp}
Es seien $X_1,X_2,\ldots$ unabhängige identisch verteilte Zufallsvariablen
$Z_n$ mit existierenden (endlichen) $\E X_1\defr a$ und $\V(X_1) \defr
\sigma^2$. Aufgrund der beobachteten Realisierungen $x_1,\ldots,x_n$ von $X_1,\ldots,X_n$
wird $a$ geschätzt durch
\begin{align*}
\overline{x}_n \defl \frac{1}{n}\sum\limits_{i=1}^n x_i
\end{align*}
das sogenannte \emph{empirische Mittel} und $\sigma^2$ durch
\begin{align*}
s_n^2 \defl \frac{1}{n-1}\sum\limits_{i=1}^n (x_i-\overline{x}_n)^2
\end{align*}
die sogenannte \emph{empirische Varianz}.
Für
\begin{align*}
\overline{X}_n \defl
\frac{1}{n}\sum\limits_{i=1}^n X_i,\qquad S_n^2 \defl \frac{1}{n-1}
\sum\limits_{i=1}^n (X_i-\overline{X}_n)^2
\end{align*}
gilt $\E\overline{X}_n=a$, $\E S_n^2 = \sigma^2$ (sogenannte Erwartungstreue der
Schätzung) und ferner für $n\to \infty$,
\begin{align*}
\overline{X}_n\to a\fs,\qquad S_n^2\to \sigma^2 \fs,
\end{align*}
sogenannte Konsistzenz der Schätzfolgen.\bsphere
\begin{proof}
Wir zeigen $S_n^2\to \sigma^2\Pfs$. Setzen wir $\mu=\E X_1$, so gilt
\begin{align*}
S_n^2 &\defl \frac{1}{n-1}\sum\limits_{i=1}^n (X_i-\overline{X}_n)^2
=
\frac{n}{n-1}\left(\frac{1}{n}\sum\limits_{i=1}^n
(X_i-\mu+\mu-\overline{X}_n)^2\right)\\
&=
\underbrace{\frac{n}{n-1}}_{\to
1}\left(\underbrace{\frac{1}{n}\sum\limits_{i=1}^n (X_i-\mu)^2}_{\to
\sigma^2\Pfs} + \underbrace{\frac{2}{n}\sum\limits_{i=1}^n (X_i-\mu)}_{\to 0
\Pfs}\underbrace{(\mu-\overline{X}_n)}_{\to 0\fs} +
\underbrace{\frac{1}{n}\sum\limits_{i=1}^n
(\mu-\overline{X}_n)^2}_{(\mu-\overline{X}_n)^2\to 0\fs} \right),\\
&\to \sigma^2\fs\bsphere
\end{align*}
\end{proof}
\end{bsp}

\begin{bsp}
Für eine Menge von $n$ radioaktiven Atomen mit unabhängigen
$\exp(\lambda)$-verteilten Lebensdauern (mit Verteilungsfunktion $F$) gebe -
bei festem $T>0$ - die Zufallsvariable $Z_n$ die zufällige Anzahl der Atome an,
die von jetzt (Zeitpunkt $0$) an bis zum Zeitpunkt $T$ zerfallen. Die
Wahrscheinlichkeit, dass ein gewisses Atom im Zeitintervall $[0,T]$ zerfällt,
ist $p=F(T) = 1-\e^{-\lambda T}$. Nach Bemerkung \ref{bem:8.4} konvergiert mit
Wahrscheinlichkeit Eins $Z_n/n\to p$ für $n\to\infty$. Wählt man $T$ so, dass
$F(T) = \frac{1}{2}$, d.h. $T=(\ln 2) / \lambda$ ($T$ sog. Median von
$\exp(\lambda)$), dann gilt
\begin{align*}
\fs\quad \frac{Z_n}{n}\to \frac{1}{2},\qquad n\to\infty.
\end{align*}
Dieses $T$ wird als Halbwertszeit des radioaktiven Elements bezeichnet (nach
$T$ Zeiteinheiten ist i.A. bei großem $n$ ungefähr die Hälfte der Atome
zerfallen).\bsphere
\end{bsp}

\begin{bsp}
Aus der Analysis ist der Satz von Weierstraß bekannt, dass jede stetige
Funktion $f:\R\to \R$ auf einem kompakten Intervall gleichmäßiger Limes von
Polynomen ist.

Wir wollen diesen Satz nun unter Verwendung des
Kolmogorovschen starken Gesetz der großen Zahlen beweisen.
Dazu verwenden wir Bernstein-Polynome, um stetige Funktionen über einem
kompakten Intervall zu approximieren.

Sei $f: [0,1]\to\R$ stetig. Die \emph{Bernsteinpolynome $n$-ten Grades} sind
definiert durch
\begin{align*}
f_n(p) = \sum\limits_{k=0}^n
f\left(\frac{k}{n}\right)\binom{n}{k}p^k(1-p)^{n-k},\qquad p\in[0,1].
\end{align*}
Wir versehen also $[0,1]$ mit einem Gitter mit Gitterabstand $\frac{1}{n}$,
werten $f$ an $\frac{k}{n}$ aus und multiplizieren mit speziellen Gewichten.

Seien $U_1,\ldots,U_n$ unabhängige auf $[0,1]$ identisch gleichverteilte
Zufallsvariablen. Wähle $p\in[0,1]$ fest und setze
\begin{align*}
X_i \defl \Id_{[0,p]}(U_i) = \Id_{U_i\in [0,p]}.
\end{align*}
Die $X_i$ sind dann unabhängige und $b(1,p)$-verteilte Zufallsvariablen.
\begin{align*}
&\E\left(f\left(\frac{1}{n}\sum\limits_{i=1}^n \Id_{[0,p]}(U_i)\right)\right)
= 
\E\left(f\left(\underbrace{\frac{1}{n}\sum\limits_{i=1}^n
X_i}_{\in\setd{0,1/n,2/n,\ldots,1}}\right)\right)\\
&= \sum\limits_{k=0}^n f\left(\frac{k}{n}\right)\binom{n}{k}p^k(1-p)^k = f_n(p)
\end{align*}
Sei $\ep > 0$ beliebig. Da $f$ auf dem kompakten Intervall $[0,1]$ gleichmäßig
stetig, existiert ein $\delta>0$, so dass
\begin{align*}
\abs{x-y}<\delta \Rightarrow \abs{f(x)-f(y)}<\ep,\qquad \text{für } x,y\in[0,1]. 
\end{align*}
Somit gilt
\begin{align*}
&\abs{f_n(p)-f(p)} = \abs{\E\left(f\left(\frac{1}{n}\sum\limits_{i=1}^n
\Id_{[0,p]}(U_i)\right)\right)- f(p) } \\ &=
\abs{\E\left(f\left(\frac{1}{n}\sum\limits_{i=1}^n
\Id_{[0,p]}(U_i)\right)- f(p)\right)}
\le
\E\abs{f\left(\frac{1}{n}\sum\limits_{i=1}^n
\Id_{[0,p]}(U_i)\right)- f(p)}
\end{align*}
Falls nun $\abs{\frac{1}{n}\sum\limits_{i=0}^n
\Id_{[0,p]}(U_i)-p}<\delta$, ist der Erwartungswert $<\ep$, andernfalls
schätzen wir grob durch die Dreiecksungleichung ab,
\begin{align*}
&\le \E\abs{\ep+2\norm{f}_\infty\cdot
\Id_{\left[\abs{\frac{1}{n}\sum\limits_{i=0}^n
\Id_{[0,p]}(U_i)-p}\ge\delta\right]}}\\
&= \ep + 2\norm{f}_\infty P\left[\abs{\frac{1}{n}\sum\limits_{i=0}^n
\Id_{[0,p]}(U_i)-p}\ge\delta\right]
\le \ep + 2\norm{f}_\infty \frac{\V\left(\Id_{[0,p]}(U_1) \right)}{n\delta^2}\\
&= \ep  + 2\norm{f}_\infty \frac{p(1-p)}{n\delta^2}
\le \ep  + 2\norm{f}_\infty \frac{1}{n\delta^2}
%P\left[\abs{\frac{1}{n}\sum\limits_{i=0}^n \Id_{[0,p]}(U_i)-p}\ge\delta\right]
\end{align*}
Somit $f_n\unito f$ gleichmäßig auf $[0,1]$.

Bernstein-Polynome haben sehr angenehme Eigenschaften. Ist beispielsweise $f$
monoton, so ist auch $f_n$ monoton. Ist $f$ konvex, so ist auch $f_n$ konvex.
\bsphere
\end{bsp}

\cleardoublepage
\chapter{Zentrale Grenzwertsätze}
\label{chap:9}

\section{Schwache Konvergenz von Wahrscheinlichkeitsmaßen in $\R $}

Im Folgenden sei stets $n\in \N$ und ``$\to$'' bezeichne die Konvergenz für
$n\to\infty$.

\begin{defn}
\label{defn:9.1}
\begin{defnenum}%[label=\alph{*})]
\item Seien $Q_{n}$, $Q$ W-Maße auf der $\sigma$-Algebra
  ${\BB}$ der Borelschen Mengen in $\R$. Die Folge $(Q_{n})$ heißt
  gegen $Q$ \emph{schwach konvergent (weakly convergent)} --- Schreibweise
  $Q_{n} \to Q $ schwach --- , wenn für jede beschränkte stetige Funktion $g:
  \R\to \R$ gilt,
\begin{align*}
\int_{\R} g\dQ_{n} \to \int_{\R}g\dQ, \qquad n\to
\infty.
\end{align*}
\item Seien $X_{n}$, $X$ reelle Zufallsvariablen (nicht notwendig
auf demselben W-Raum definiert). Die Folge $(X_{n})$ heißt gegen $X$ 
\emph{nach Verteilung konvergent (convergent in distribution, convergent in
law)} --- Schreibweise $X_{n}\Dto X\;\, (n\to\infty)$ --- ,
wenn $P_{X_{n}}\to P_{X}$ schwach, d.h. für jede beschränkte stetige Funktion $g:\R\to \R$ gilt
\begin{align*}
\int_{\R} g\dP_{X_{n}}\to \int_{\R} g\dP_{X}, \qquad n\to
\infty
\end{align*}
oder (äquivalent nach dem Transformationssatz für Integrale)
\begin{align*}
\E g (X_{n}) \to \E g (X), \qquad n\to \infty.\fishhere
\end{align*}
\end{defnenum}
\end{defn}

\begin{bem}[Bemerkungen.]
\label{bem:9.1}
\begin{bemenum}%[label=\alph{*})]
\item In Definition \ref{defn:9.1}a ist das Grenz-W-Maß $Q$ eindeutig 
bestimmt.
\item In Definition \ref{defn:9.1}b können $X_{n}$, $X$ durch reelle
Zufallsvariablen $X'_{n}$, $X'$ mit $P_{X'_{n}} = P_{X_{n}}$, $P_{X'} =P_{X}$
ersetzt werden.\maphere
\end{bemenum}
\end{bem}

Bevor wir die Eindeutigkeit des Grenz-W-Maßes beweisen, betrachten wir
zunächst den Poissonschen Grenzwertsatz
\begin{align*}
\forall k\in\N_0 : b(n,p_n;k) \to \pi(\lambda;k),\qquad n\to \infty,
\end{align*}
falls $n p_n \to \lambda$. Dies ist (hier) äquivalent zu
\begin{align*}
\forall x\in\R : F_n(x)\to F(x),
\end{align*}
wobei $F_n$ Verteilungsfunktion einer $b(n,p)$-verteilten und $F$
Verteilungsfunktion einer $\pi(\lambda)$-verteilten Zufallsvariable ist.

Seien $X_n,X$ reelle Zufallsvariablen mit $X_n=\frac{1}{n}\fs$, $X\defl0\fs$, so
gilt $X_n\to X\fs$ für $n\to\infty$.

\begin{figure}[!htpb]
\centering
\begin{pspicture}(-1.5,-1)(4,2.5)
\psaxes[labels=none,ticks=none,linecolor=gdarkgray,tickcolor=gdarkgray]{->}%
 (0,0)(-1.2,-0.5)(3.2,2)[\color{gdarkgray}$t$,-90][\color{gdarkgray},0]

\psline[linecolor=darkblue](-1,0.01)(1,0.01)
\psline[linecolor=purple](-1,0)(0,0)

\psline[linecolor=darkblue](1,1.01)(3,1.01)
\psline[linecolor=purple](0,1)(3,1)

\rput(0.4,1.4){\color{gdarkgray}$F$}

\rput(1.4,1.4){\color{gdarkgray}$F_n$}

\psxTick(1){\color{gdarkgray}\frac{1}{n}}
\psyTick(1){\color{gdarkgray}1}

\end{pspicture} 
\caption{Verteilungsfunktionen von $X$ und $X_n$.}
\end{figure}
Für die Verteilungsfunktionen folgt
\begin{align*}
\forall x\in\R\setminus\setd{0} : F_n(x)\to F(x),
\end{align*}
denn $0$ ist Unstetigkeitsstelle von $F$. Wir sehen an diesem Beispiel, dass
die Konvergenz der Verteilungsfunktion in Unstetigkeitsstellen im Allgemeinen
\textit{nicht} fordern können. Wir können jedoch die Konvergenz nach Verteilung
durch die Konvergenz der $F_n$ in Stetigkeitspunkten charakterisieren.

\begin{defnn}[Klassische Definition der Verteilungskonvergenz]
Seien $X_n,X$ reelle Zufallsvariablen bzw. $Q_n$, $Q$ W-Maße mit
Verteilungsfunktion $F_n$ bzw. $F$. Dann ist
\begin{align*}
X_n\Dto X,
\end{align*}
bzw.
\begin{align*}
Q_n\to Q\text{ schwach},
\end{align*}
wenn
\begin{align*}
\forall \text{Stetigkeitspunkte $x$ von $F$} : F_n(x)\to F(x),\qquad n\to
\infty.\fishhere
\end{align*}
\end{defnn}
Diese Definition ist jedoch nicht so allgemein wie die in \ref{defn:9.1}, da
sie sich nicht auf unendlichdimensionale Räume übertragen lässt.

\begin{proof}[Beweis der Eindeutigkeit
des Grenz-W-Maßes.]
Falls $Q_n\to Q^{*}$ schwach und $Q_n\to Q^{**}$ schwach, so gilt nach Satz
\ref{prop:9.2} (noch zu zeigen),
\begin{align*}
F_n(x) \to F^{*}(x),\qquad F_n(x)\to F^{**}(x)
\end{align*}
für jede Stetigkeitsstelle $x$ von $F^{*}$ bzw. $F^{**}$. Aufgrund der
rechtsseitigen Stetigkeit einer Verteilungsfunktion folgt,
\begin{align*}
\forall x\in \R : F^{*}(x) = F^{**}(x)
\end{align*} 
und damit auch $Q^{*}=Q^{**}$.\qedhere
\end{proof}

\begin{prop}
\label{prop:9.1}
Seien $X_{n}$, $X$ reelle Zufallsvariablen auf $(\Omega, {\AA}, P)$. Dann gilt
\begin{align*}
X_{n} \Pto X \Rightarrow X_{n}\Dto X.\fishhere
\end{align*}
\end{prop}
\begin{proof}
Sei $g: \R\to\R$ stetig und beschränkt. Angenommen $\E (g(X_n)-g(X))$
konvergiert nicht gegen Null, es gibt also eine Teilfolge $(n_k)$, so dass
\begin{align*}
\abs{\E (g(X_{n_k})-g(X))} > \ep,\qquad \forall k\in\N.
\end{align*}
Aber da auch $X_{n_k}\Pto X$, besitzt $X_{n_k}$ eine Teilfolge, die $\Pfs$
konvergiert. Da $g$ stetig, folgt somit auch
\begin{align*}
g(X_{n_{k_l}}) \to  g(X) \Pfs
\end{align*}
Da außerdem $g$ beschränkt, können wir den Satz von der dominierten Konvergenz
anwenden und erhalten somit
\begin{align*}
\E g(X_{n_{k_l}}) \to  \E g(X),
\end{align*}
im Widerspruch zur Annahme, dass $\E g(X_n)$ nicht gegen $\E g(X)$
konvergiert.\qedhere
\end{proof}

\begin{prop}
\label{prop:9.2}
Seien $Q_{n}$, $Q$ W-Maße auf ${\BB}$ bzw.\ reelle Zufallsvariablen $X_{n}$,
$X$ mit Verteilungsfunktion $F_{n}$,~$F$.

$Q_{n} \to Q$ schwach bzw.\ $X_{n} \Dto X$ gilt genau dann, wenn $F_{n}(x) \to
F(x)$ für alle Stetigkeitspunkte $x$ von $F$.\fishhere
\end{prop}
\begin{proof}
``$\Rightarrow$'': Es gelte $X_n\Dto X$. Sei $D\defl\setdef{x\in\R}{F(x-)=F(x)}$
die Menge der Stetigkeitspunkte von $F$. $D$ ist dicht in $\R$, denn $F$ ist
monoton und daher ist $D^c$ höchstens abzählbar. Sei $x\in D$. Für jedes
$p\in \N$ setze
\begin{align*}
&f_p(y) \defl
\begin{cases}
1, & y\le x,\\
p(x-y), & x\le y\le x+\frac{1}{p},\\
0, & x+\frac{1}{p}\le y,
\end{cases}\\
&g_p(y) \defl
\begin{cases}
1, & y\le x -\frac{1}{p},\\
p(x-y), & x-\frac{1}{p}\le y\le x,\\
0, & x\le y.
\end{cases}
\end{align*}
\begin{figure}[!htpb]
\centering
\begin{pspicture}(-1.5,-1)(4,2.5)
\psaxes[labels=none,ticks=none,linecolor=gdarkgray,tickcolor=gdarkgray]{->}%
 (0,0)(-1.2,-0.5)(3.2,2)[\color{gdarkgray}$t$,-90][\color{gdarkgray},0]

\psline[linecolor=darkblue](-1,1.01)(0.5,1.01)(1.5,0.01)(3,0.01)
\psline[linecolor=purple](-1,1)(1.5,1.0)(2.5,0.0)(3,0)
\psline[linecolor=yellow](-1,1)(1.5,1.0)(1.5,0.0)(3,0)


\rput(0.4,1.4){\color{darkblue}$g_p$}

\rput(2.4,1.4){\color{purple}$f_p$}

\rput(1.4,1.4){\color{yellow}$\Id_{(-\infty,x]}$}

\psxTick(0.5){\color{gdarkgray}x-\frac{1}{p}}
\psxTick(1.5){\color{gdarkgray}x}
\psxTick(2.5){\color{gdarkgray}x+\frac{1}{p}}

\end{pspicture} 
\caption{Verteilungsfunktionen von $X$ und $X_n$.}
\end{figure}

$f_p,g_p: \R\to[0,1]$ sind stetige, beschränkte Approximationen von
$\Id_{(-\infty,x]}(y)$, wobei
\begin{align*}
f_p(y) \downarrow \Id_{(-\infty,x]}(y),\quad
g_p(y) \uparrow \Id_{(-\infty,x]}(y),\qquad
p\to\infty.
\end{align*}
Da $X_n$ nach Verteilung konvergiert, gilt
\begin{align*}
&\forall p\in\N : \lim\limits_{n\to\infty} \E f_p(X_n) = \E f_p(X),\\
&\forall p\in\N : \lim\limits_{n\to\infty} \E g_p(X_n) = \E g_p(X).
\end{align*}
Nach Definition der Verteilungsfunktion ist $F_n(x)= \E \Id_{(-\infty,x]}(X)$,
also gilt auch
\begin{align*}
&\E g_p(X_n)\le F_n(x) \le \E f_p(X_n).
\end{align*}
Mit dem Lemma von Fatou folgt außerdem
\begin{align*}
\forall p\in\N : \E g_p(X) &\le \liminf\limits_{n\to\infty} \E g_p(X_n) \le
\liminf\limits_{n\to\infty} F_n(x) \le \limsup\limits_{n\to\infty} F_n(x)\\
&\le \limsup\limits_{n\to\infty} \E f_p(X_n) \le \E f_p(X),\tag{*}
\end{align*}
%wobei wir $f_p(X_n)$ durch $\Id_{(-\infty,x]}(X_n)$ majorisieren können mit $\E
%\Id_{(-\infty,x]}(X_n) \le 1$.
da $f_p$ beschränkt. 
Weiterhin folgt mit dem Satz von der monotonen Konvergenz,
\begin{align*} 
&\lim\limits_{p\to\infty} \E f_p(X) = \E \Id_{(-\infty,x]}(X) = P[X\le x] =
F(x),\\
&\lim\limits_{p\to\infty} \E g_p(X) = \E \Id_{(-\infty,x)}(X) = P[X < x] =
F(x-).
\end{align*}
Mit (*) folgt $F(x-0)\le\liminf\limits_{n\to\infty} F_n(x) \le
\limsup\limits_{n\to\infty} F_n(x) \le  F(x)$. Für $x\in D$ gilt
\begin{align*}
F(x-) = F(x),
\end{align*}
also auch
\begin{align*}
\lim\limits_{n\to\infty} F_n(x) = F(x).
\end{align*}
``$\Leftarrow$'': Diese Richtung ist etwas mühsamer. Hier approximiert man
umgekehrt zur Hinrichtung eine stetige, beschränkte Funktion $g$ durch
Stufenfunktionen. Details findet man in [Jacod J, Protter P. Probability Essentials].\qedhere
\end{proof}

\begin{prop}
\label{prop:9.3}
Seien $X_{n}$, $X$ reelle Zufallsvariablen auf $(\Omega,{\AA},P)$. Ist $X$
P-f.s. konstant, so gilt:
\begin{align*}
X_{n}\Dto X \Leftrightarrow
X_{n}\Pto X.\fishhere
\end{align*}
\end{prop}
\begin{proof}
``$\Leftarrow$'': Klar nach Satz \ref{prop:9.1}.

``$\Rightarrow$'': Sei $X=c\Pfs$ mit Verteilungsfunktion $F$. $c$ ist die
einzige Unstetigkeitsstelle von $F$.


\begin{figure}[!htpb]
\centering
\begin{pspicture}(-1.5,-1)(4,2.5)
\psaxes[labels=none,ticks=none,linecolor=gdarkgray,tickcolor=gdarkgray]{->}%
 (0,0)(-1.2,-0.5)(3.2,2)[\color{gdarkgray}$x$,-90][\color{gdarkgray},0]

\psline[linecolor=darkblue,arrows=*-](-1,0)(1.5,0)
\psline[linecolor=darkblue](1.5,1)(3,1)

\rput(1.6,1.4){\color{darkblue}$F$}

\psxTick(1.5){\color{gdarkgray}c}
\psyTick(1){\color{gdarkgray}1}

\end{pspicture} 
\caption{Verteilungsfunktionen von $X$.}
\end{figure}
Da $X_n\Dto X$ nach Voraussetzung folt mit Satz \ref{prop:9.1},
\begin{align*}
\forall \R \setminus\setd{c} : F_n(x)\to F(x).
\end{align*}
Sei $\ep > 0$ beliebig aber fest, so gilt
\begin{align*}
P[\abs{X_n-X}>\ep] &= 1-P[\abs{X_n-X}\le\ep]\\
P[\abs{X_n-X}\le \ep] &=
P[c-\ep \le X_n \le c+\ep]
\ge
P[c-\ep < X_n \le c+\ep]\\
&= \underbrace{F_n(c+\ep)}_{\to F(c+\ep)}-\underbrace{F_n(c-\ep)}_{\to
F(c-\ep)}\to 1.
\end{align*}
Somit konvergiert $P[c-\ep < X_n \le c+\ep]\to 1$, also $P[\abs{X_n-X}>\ep]\to
0$.\qedhere
\end{proof}

\begin{prop}
\label{prop:9.4}
Seien $X_{n}$, $Y_{n}$, $X$ reelle Zufallsvariablen auf $(\Omega, {\AA}, P)$.
Gilt
\begin{align*}
X_{n}\Dto X,\qquad
|X_{n} -Y_{n}| \Pto 0,
\end{align*}
so folgt
\begin{align*}
Y_{n} \Dto X.\fishhere
\end{align*}
\end{prop}
\begin{proof}
Sei $g: \R\to\R$ stetig und beschränkt. Ohne Einschränkung können wir annehmen,
dass $g$ gleichmäßig stetig ist (siehe Satz \ref{prop:9.9}). Außerdem gilt ohne
Einschränkung
\begin{align*}
\abs{X_n-Y_n}\to 0\Pfs
\end{align*}
ansonsten gehen wir zu einer Teilfolge $(n_k)$ über. Betrachte
\begin{align*}
\E g(Y_n) = \E g(X_n) + \E[g(Y_n)-g(X_n)].
\end{align*}
da $g$ gleichmäßig stetig und $\abs{X_n-Y_n}\to 0 \Pfs$ gilt $g(Y_n)-g(X_n)\to
0\Pfs$. $\abs{g(Y_n)-g(X_n)}\le c$, da $g$ beschränkt, also können wir den
Satz von der dominierten Konvergenz anwenden und erhalten $\E[g(Y_n)-g(X_n)] \to 0$.

Somit gilt
\begin{align*}
\E g(Y_n) = \underbrace{\E g(X_n)}_{\to \E g(X) \Pfs} +
\underbrace{\E[g(Y_n)-g(X_n)]}_{\to 0 \Pfs} \to \E g(X)\Pfs.\qedhere
\end{align*}
\end{proof}

\begin{prop}[Spezialfall des Darstellungssatzes von Skorokhod]
\label{prop:9.5}
Seien $X_{n}$, $X$ reelle Zufallsvariablen mit $X_{n}\Dto X$. Dann existiert
ein W-Raum $(\Omega ^ {\ast}, {\AA}^ {\ast}, P^ {\ast})$ --- wobei man 
$\Omega^{\ast} = [0,1]$, ${\AA}^ {\ast} = [0,1] \cap {\cal B}$, $P^ {\ast} = $
L-B-Maß auf ${\AA}^ {\ast}$ wählen kann --- und reelle Zufallsvariablen 
$X^{\ast}_{n}$, $X^{\ast}$ auf $(\Omega ^ {\ast}, {\AA}^ {\ast}, P^ {\ast})$ derart, dass
\begin{align*}
\forall n\in\N : P_{X_{n}} =P^*_{X^ {\ast}_{n}},\;\, P_{X} =
P^*_{X^ {\ast}},\,\; X^ {\ast}_{n} \to X^ {\ast}\;
P^{\ast}\mbox{-f.s.}\fishhere
\end{align*}
\end{prop}
\begin{proof}
\textit{Beweisidee}. Wähle als Ansatz für $X_n^*$, $X^*$ die verallgemeinerte
Inverse von $F_n$ und $F$,
\begin{align*}
&X_n^*(\omega) \defl \inf\setdef{t\in\R}{F_n(t)\ge \omega},\\
&X^*(\omega) \defl\inf\setdef{t\in\R}{F(t)\ge \omega}.
\end{align*}
Nach Annahme konvergiert $X_n\Dto X$ und daher $F_n(x)\to F(x)$ in den
Stetigkeitspunkten von $F$. Zu zeigen ist nun
\begin{align*}
X_n^*(\omega) \to X^*(\omega)
\end{align*}
in den Stetigkeitspunkten $\omega$ von $X^*$, also $\fu{\lambda}$.

\textit{Formaler Beweis}. Wähle $\Omega^*=(0,1)$, $\AA^*=(0,1)\cap \BB$ und
$P^*=\lambda\big|_{\AA^*}$. Seien $F_n$ und $F$ die Verteilungsfunktionen von
$X_n$ und $X$. Nun setzen wir
\begin{align*}
&X_n^*(\omega) \defl \inf\setdef{t\in\R}{F_n(t)\ge \omega},\\
&X^*(\omega) \defl\inf\setdef{t\in\R}{F(t)\ge \omega},
\end{align*}
d.h. $X_n^*$ und $X^*$ sind die verallgemeinerten Inversen von $F_n$ bzw. $F$.
Somit sind $X_n^*$ und $X^*$ monotone Funktionen und damit insbesondere messbar.
Wähle für die Verteilung von $X^*$,
\begin{align*}
P^*[X^*\le x] \defl F(x) = P[X\le x],
\end{align*}
so gilt $P^*_{X^*} = P_X$, entsprechend für $X_n$ und $X_n^*$. Somit ist
auch klar, dass $X^*$ höchstens abzählbar viele Unsteigkeitsstellen hat.

Es genügt nun zu zeigen $X_n^*\to X^*$ für alle Stetigkeitspunkte $\omega$ von
$X^*$. Sei $\omega$ Stetigkeitspunkt von $X^*$ und $\ep > 0$. Es existieren
dann Stetigkeitspunkte $x_1$ und $x_2$ von $F$ mit
\begin{align*}
x_1< X^*(\omega) < x_2 \text{ und }x_2-x_1< \ep,\tag{*}
\end{align*}
Hierbei gilt auch $F(x_1)<\omega < F(x_2)$, da $\omega$ Stetigkeitspunkt von
$X^*$, sowie $F_n(x_1)\to F(x_1)$ und $F_n(x_2)\to F(x_2)$, da $x_1,x_2$
Stetigkeitspunkte von $F$ und $X_n\Dto X$. Also
\begin{align*}
\exists n_0\in\N \forall n\ge n_0 : F_n(x_1) < \omega < F_n(x_2)
\end{align*}
und damit auch
\begin{align*}
\forall n\ge n_0 : x_1\le X_n^*(\omega) \le x_2\tag{**},
\end{align*}
wobei lediglich benötigt wird, dass $\omega$ Stetigkeitspunkt von
$X^*$, nicht unbedingt von $X_n^*$.

Aus (*) und (**) folgt nun $\abs{X_n^*(\omega)-X^*(\omega)}<\ep$.\qedhere
\end{proof}

\begin{prop}[Satz von der stetigen Abbildung]
\label{prop:9.6}
Seien $X_{n}$, $X$ reelle Zufallsvariablen mit $X_{n}\Dto X$. Sei die Abbildung
\begin{align*}
h: (\R, {\BB}) \to (\R, {\BB})
\end{align*}
$P_{X}$-f.ü.\ stetig. Dann gilt $h(X_{n})\Dto h(X)$.\fishhere
\end{prop}

\begin{proof}
Sei $D\defl\setdef{x\in\R}{h\text{ unstetig in }x}$. Den Beweis $D\in\BB$
überspringen wir. Nach Voraussetzung ist $P_X(D)=0$.

Nach Satz \ref{prop:9.5} existieren ein W-Raum $(\Omega^*,\AA^*,P^*)$ und
Zufallsvariablen $X_n^*$, $X^* : (\Omega^*,\AA^*,P^*)\to (\R,\BB)$ mit
$X_n^*\to X^*\fu{P^*}$ und
\begin{align*}
P^*_{X_n^*} = P_{X_n},\qquad P^*_{X^*} = P_X.
\end{align*}
Da $h$ stetig, gilt
\begin{align*}
h(X_n^*(\omega)) \to h(X^*(\omega))
\end{align*}
für $\omega\in D^c$, aber
\begin{align*}
P^*(\setdef{\omega\in\Omega^*}{X^*(\omega)\in D}) = P^*[X^*\in D] =
P^*_{X^*}(D) = P_X(D) = 0,
\end{align*}
also gilt $h(X_n^*)\to h(X^*)$ $P^*\text{ f.s.}$ Somit folgt nach Satz
\ref{prop:9.1} $h(X_n^*)\Dto h(X^*)$, also
\begin{align*}
P^*_{h(X_n^*)}\to P^*_{h(X^*)}\quad \text{schwach}.
\end{align*}
Nun ist $P^*_{h(X_n^*)} = (P^*_{X_n^*})_h = (P_{X_n})_h = P_{h(X_n)}$, analog
$P^*_{h(X^*)} = P_{h(X^*)}$. Somit
\begin{align*}
P_{h(X_n)}\to P_{h(X)}\quad \text{schwach}.\qedhere
\end{align*}
\end{proof}

Wie wir gesehen haben, vertauscht der Limes Operator der
Verteilungskonvergenz im Allgemeinen nicht mit der Addition oder der
Multiplikation. Für spezielle Zufallsvariablen erhalten wir jedoch
Vertauschbarkeit.

\begin{prop}[Satz von Slutsky]
\label{prop:9.7}
Seien $X_{n}$, $Y_{n}$ und $X$ reelle Zufallsvariablen auf
$(\Omega, \AA, P)$ und $c\in \R$.
\begin{align*}
\begin{rcases}
X_{n}\Dto X,\\
Y_{n} \Pto c,
\end{rcases}
\Rightarrow
\begin{cases}
X_{n} +Y_{n} \Dto X+c,\\
Y_{n}X_{n} \Dto cX.\fishhere
\end{cases}
\end{align*}
\end{prop}
\begin{proof}
Sei $c\in\R$ und $Z_n\defl X_n+(Y_n-c)$, so gilt
\begin{align*}
\abs{X_n-Z_n} = \abs{Y_n-c} = 0 \fs,
\end{align*}
insbesondere  $\abs{X_n-Z_n}\Pto 0$. Somit ist \ref{prop:9.4} anwendbar und
daher,
\begin{align*}
X_n+(Y_n-c) = Z_n\Dto X.
\end{align*}
Nun wenden wir den Satz von der stetigen Abbildung an mit
\begin{align*}
h : \R\to\R,\quad x\mapsto x+c,
\end{align*}
so gilt
\begin{align*}
X_n + Y_n = h(X_n+(Y_n-c)) \Dto h(X) = X+c.\qedhere
\end{align*}
\end{proof}

\begin{prop}
\label{prop:9.8}
Für reellwertige Zufallsvariablen $X_n$, $X$ mit Dichten $f_n$
bzw. $f$ gilt:
\begin{align*}
f_n \rightarrow f \text{ $\lambda$-f.ü.} \Rightarrow X_n \Dto
X.\fishhere
\end{align*}
\end{prop}
\begin{proof}
Sei $g\in C_b(\R)$. Wähle $c\in\R$, so dass $\abs{g}\le c$, so folgt
\begin{align*}
g+c \ge 0.
\end{align*}
Da $\dP_X = f\dlambda$ erhalten wir,
\begin{align*}
\int_\R g\dP_X = - c +\int_\R (c+g) \dP_X =
-c+ \int_\R (c+g) f\dlambda.
\end{align*}
Nun ist $f = \lim\limits_{n\to\infty} f_n = \liminf\limits_{n\to\infty} 
f_n\fu{\lambda}$ nach Vorraussetzung. Da $f,f_n$ Dichten und daher positiv,
können wir das Lemma von Fatou anwenden und erhalten,
\begin{align*}
\int_\R g\dP_X &\le - c  +\liminf\limits_{n\to\infty} \int_\R (c+g) f_n\dlambda
=  \liminf\limits_{n\to\infty} \int_\R g f_n\dlambda\\
&\le \limsup\limits_{n\to\infty} \int_\R g f_n\dlambda
= -\liminf\limits_{n\to\infty} \int_\R - g f_n\dlambda\\
&= c-\liminf\limits_{n\to\infty} \int_\R \underbrace{(c-g)}_{\ge 0}f_n\dlambda
\le c-\int_\R \liminf\limits_{n\to\infty}(c-g)f_n\dlambda\\
&= \int_\R gf \dlambda.
\end{align*}
Es gilt also
\begin{align*}
\int_\R g\dP_X \le \liminf\limits_{n\to\infty} \int_\R g f_n \dlambda
\le \limsup\limits_{n\to\infty} \int_\R g f_n \dlambda \le \int_\R g\dP_X.
\end{align*}
Somit existiert $\lim\limits_{n\to\infty} \int_\R g f_n\dlambda$ und es gilt
\begin{align*}
\lim\limits_{n\to\infty} \int_\R g \dP_{X_n} = \int_\R g\dP_X.\qedhere
\end{align*}
\end{proof}

Um $X_n\Dto X$ nachzuweisen, müssen wir für jedes stetige beschränkte $g$
nachweisen
\begin{align*}
\E g(X_n) \to \E g(X).
\end{align*}
Der nächste Satz zeigt, dass es genügt, sich auf eine kleinere Menge von
Funktionen zurückzuziehen.

\begin{defnn}
Eine Funktion $g:\R\to\R$ heißt \emph{Lipschitz-stetig}, falls eine Konstante
$L\in\R$ existiert, so dass
\begin{align*}
\abs{g(x)-g(y)} \le L\abs{x-y},\qquad \forall x,y\in\R.\fishhere
\end{align*}
\end{defnn} 
Jede lipschitzstetige Funktion ist gleichmäßig stetig und jede gleichmäßig
stetige Funktion ist stetig. Die Umkehrungen gelten im Allgemeinen
\textit{nicht}.

Die gleichmäßig stetigen Funktionen auf $\R$ bilden eine echte Teilmenge der
stetigen Funktionen.

\begin{prop}
\label{prop:9.9}
Für reellwertige Zufallsvariablen $X_n$, $X$ gilt:
\begin{align*}
X_n \Dto X \Leftrightarrow \E f(X_n) \to \E f(X)
\end{align*}
für alle beschränkten \textit{gleichmäßig} stetigen Funktionen
$f:\R\to\R$.\fishhere
\end{prop} 
\begin{proof}
``$\Rightarrow$'': Klar, denn die Aussage gilt nach Voraussetzung für stetige
beschränkte  und somit insbesondere für gleichmäßig stetige
beschränkte Funktionen.

``$\Leftarrow$'': Sei $f\in C_b(\R)$. Zu zeigen ist $\E f(X_n)\to \E f(X)$, da
$f$ lediglich stetig und beschränkt. Setzen wir
\begin{align*}
\alpha \defl \sup_{x\in\R} \abs{f(x)} = \norm{f}_\infty.
\end{align*}
Wir zeigen nun, dass für jedes $i\in\N$ eine Lipschitz-stetige Funktion
$g_i:\R\to\R$ existiert mit
\begin{align*}
-\alpha \le g_i \le g_{i+1}\le \alpha,\qquad \forall x\in \R : g_i(x)\to
f(x),\quad i\to\infty.\tag{*}
\end{align*}
Haben wir die Existenz der $g_i$ gezeigt, so gilt für $i\in\N$
\begin{align*}
\liminf_n \E f(X_n) \ge \liminf\limits_{n\to\infty} \E g_i(X_n)
= \lim\limits_{n\to\infty} \E g_i(X_n) = \E g_i(X),\tag{**}
\end{align*}
denn die $g_i$ sind gleichmäßig stetig und daher konvergiert der Erwartungswert
nach Voraussetzung.

Da die $g_i$ durch $\alpha$ beschränkt sind, ist $g_i(X)+\alpha \ge 0$ und es
gilt
\begin{align*}
\lim\limits_{i\to\infty} \E(g_i(X)+\alpha)
\overset{\text{mon.konv}}{=}
\E (f(X)+\alpha).
\end{align*}
Somit gilt auch $\lim\limits_{i\to\infty} \E(g_i(X)) = \E f(X)$. Zusammen mit
(**) erhalten wir also
\begin{align*}
\liminf\limits_{n\to\infty} \E f(X_n) \ge \E f(X).
\end{align*}
Analog erhalten wir mit $f$ ersetzt durch $-f$,
\begin{align*}
\limsup\limits_{n\to\infty} \E f(X_n) \le \E f(X).
\end{align*}
Insgesamt gilt also $\E f(X_n)\to \E f(X)$ und die Behauptung ist gezeigt.

Wir müssen also noch die Behauptung (*) nachweisen. Es genügt, eine Folge
$(h_k)_{k\in\N}$ Lipschitz-stetiger Funktionen zu finden mit
\begin{align*}
h_k \ge -\alpha\text{ und } \forall x\in\R : \sup_{k\in\N} h_k(x) = f(x). 
\end{align*} 
Denn dann leistet $g_i$ mit $g_i(x)=\max\setd{h_1(x),\ldots,h_i(x)}$ das
Gewünschte, denn das Maximum über endlich viele Lipschitz-stetige Funktionen
ist wieder Lipschitz-stetig.

Wir können ohne Einschränkung davon ausgehen, dass $f\ge 0$, ansonsten ersetzen
wir $f$ durch $\tilde{f}=f+\alpha\ge 0$. Wähle $A\in\BB$ und setze
\begin{align*}
d_A(x) = \inf\setdef{\abs{x-y}}{y\in A}.
\end{align*}
$d_A$ ist der Hausdorffabstand des Punktes $x$ von der Menge $A$. Sei $r \ge
0$ rational, $m\in\N$ und 
\begin{align*}
h_{m,r}(x) = \min\setd{r, (m\cdot d_{\setd{t:f(t)\le r}}(x))},
\end{align*}
so sind die $h_{m,r}$ lipschitz, denn
\begin{align*}
\abs{h_{m,r}(x)-h_{m,r}(y)} \le m \abs{d_{\setd{t : f(t)\le r}}(x)-d_{\setd{t :
f(t)\le r}}(y)} \le m \abs{x-y},
\end{align*}
außerdem sind die $h_{m,r}$ beschränkt, denn $\abs{h_{m,r}}\le r$, und
$\abs{h_{m,r}}(x) = 0$ für $x$ mit $f(x)\le r$. Insbesondere
\begin{align*}
0 \le h_{m,r}(x)\le f(x),\qquad \forall x\in\R.
\end{align*}

\begin{figure}[!htpb]
\centering
\begin{pspicture}(-3.2,-0.7)(3.2,4.7)

 \psaxes[labels=none,ticks=none,linecolor=gdarkgray,tickcolor=gdarkgray]{->}%
 (0,0)(-3,-0.5)(3,4.5)[\color{gdarkgray}$x$,-90][,0]


\psplot[linewidth=1.2pt,%
	     linecolor=darkblue,%
	     algebraic=true]%
	     {-2}{2}{%
x^2
}

\psline[linewidth=1.2pt,linecolor=yellow](-3,0.98)(-2,0.98)(-1,-0.02)(1,-0.02)(2,0.98)(3,0.98)
\psline[linewidth=1.2pt,linecolor=purple](-3,1.02)(-1.5,1.02)(-1,0.02)(1,0.02)(1.5,1.02)(3,1.02)

\rput(2.5,4){\color{darkblue}$f$}
\rput(2.5,1.4){\color{purple}$h_{2,r}$}
\rput(2.5,0.6){\color{yellow}$h_{1,r}$}
\end{pspicture}
\caption{$h_{m,r}(x)$ für $f(x)=x^2$, $r=1$ und $m=1,2$.}
\end{figure}
Wähle $x\in\R$, $\ep > 0$ beliebig aber fest. Wähle außerdem $0\le r\in\Q$ so,
dass
\begin{align*}
f(x)-\ep < r < f(x).
\end{align*}
Es gilt $f(y)> r$ für alle $y$ aus einer hinreichend kleinen Umgebung von
$x$, da $f$ stetig. Somit folgt
\begin{align*}
d_{\setd{t : f(t)\le r}}(x) > 0,
\end{align*}
also ist auch
\begin{align*}
h_{m,r}(x) = r
\begin{cases}
 < f(x),\\
 > f(x)-\ep,
\end{cases}
\end{align*}
für $m$ hinreichend groß.

Die Menge $\setdef{h_{m,r}}{m\in\N,\; 0\le r\in \Q}$ ist abzählbar. Sei
\begin{align*}
\setdef{h_k}{k\in\N}
\end{align*}
eine Abzählung dieser Menge. Nach Konstruktion gilt nun
\begin{align*}
\sup_{k\in\N} h_k(x)
\begin{cases}
\le f(x),\\
\ge f(x)-\ep,
\end{cases}
\end{align*}  
also $\sup\limits_{k\in\N} h_k(x) = f(x)$, da $\ep > 0$ beliebig.\qedhere
\end{proof}

% \begin{figure}[!htpb]
% \centering
% \begin{pspicture}(-1,-0.7)(5,4.7)
% 
%  \psaxes[labels=none,ticks=none,linecolor=gdarkgray,tickcolor=gdarkgray]{->}%
%  (0,0)(-0.5,-0.5)(3.5,4.5)[\color{gdarkgray}$x$,-90][,0]
% 
% 
% \psplot[linewidth=1.2pt,%
% 	     linecolor=darkblue,%
% 	     algebraic=true]%
% 	     {-0.5}{3.5}{%
% 3 + 3*x - 4*x^2 + x^3
% }
% 
% % \psline(-3,3)(5,3)
% % 
% % \psline[linewidth=1.2pt,linecolor=yellow]%
% %  	(-3,0)(0,0)(0.5,1)(1,0)(3,0)(3.5,1)(4,1)
% 
% 
% % \psline[linewidth=1.2pt,linecolor=purple](-3,1.02)(-1.5,1.02)(-1,0.02)(1,0.02)(1.5,1.02)(3,1.02)
% 
% % \rput(2.5,4){\color{darkblue}$f$}
% % \rput(2.5,1.4){\color{purple}$h_{2,r}$}
% % \rput(2.5,0.6){\color{yellow}$h_{1,r}$}
% \end{pspicture}
% \caption{$h_{m,r}(x)$ für $f(x)=x^2$, $r=1$ und $m=1,2$.}
% \end{figure}

\begin{prop}[Satz von L\'{e}vy-Cram\'{e}r; Stetigkeitssatz]
\label{prop:9.10}
Seien $Q_{n}$, $Q$ W-Maße auf ${\BB}$ mit
charakteristischen Funktionen $\ph_{n}$, $\ph :\R\to \C$.
Dann gilt:
\begin{align*}
Q_{n}\to Q \text{ schwach} \Leftrightarrow \forall u\in
\R : \ph_{n} (u) \to \ph (u).\fishhere
\end{align*}
\end{prop}
\begin{proof}
``$\Rightarrow$'': Sei $\ph_n$ die charakteristische Funktion von $Q_n$, also
\begin{align*}
\ph_n(u) = \int_\R \e^{iux}\dQ_n(x)
= \int_\R \cos(ux)\dQ_n(x) + i\int_\R \sin(ux)\dQ_n(x).
\end{align*}
$\sin$ und $\cos$ sind beschränkte Funktionen also,
\begin{align*}
\ph_n(u)
\to &\int_\R \cos(ux)\dQ(x) + i\int_\R \sin(ux)\dQ(x)
= \int_\R \e^{iux}\dQ(x) \\ &= \ph(u).
\end{align*}
``$\Leftarrow$'': Seien $Y_n$, $Y_0$, $X$ unabhängige Zufallsvariablen auf
$(\Omega,\AA,P)$ mit $Y_n\sim Q_n$, $Y_0\sim Q$, $X\sim N(0,1)$. Dann gilt für
alle $\alpha>0$,
\begin{align*}
P_{Y_n+\alpha X} = P_{Y_n}*P_{\alpha X}.
\end{align*}
% Aufgrund der Unabhängigkeit ist die charakteristische Funktion dieser Faltung
% gegeben durch das Produkt,
% \begin{align*}
% \ph_n(u)\e^{-\frac{(\alpha u)^2}{2}}.
% \end{align*}
Nach Übungsaufgabe 40 besitzt $P_{Y_n+\alpha X}$ die Dichte
\begin{align*}
g_{n,\alpha}(x) = \frac{1}{2\pi}\int \e^{-ux}\ph_n(u)\e^{-\frac{(\alpha
u)^2}{2}}\du.
\end{align*}
Der Integrand besitzt eine integrierbare Majorante, denn
\begin{align*}
\abs{\e^{-ux}\ph_n(u)\e^{-\frac{(\alpha
u)^2}{2}}}
\le
\e^{-ux}\e^{-\frac{(\alpha u)^2}{2}},
\end{align*}
also können wir den Satz von der dominierten Konvergenz anwenden und erhalten,
\begin{align*}
\forall x\in\R : 
g_{n,\alpha}(x) \to \frac{1}{2\pi}\int_\R \e^{-ux}\ph(u)\e^{-\frac{(\alpha
x)^2}{2}}\du.
\end{align*}
Unter Verwendung des Satzes \ref{prop:9.8} erhalten wir
\begin{align*}
Y_n+\alpha X \Dto Y_0 + \alpha X.\tag{*}
\end{align*}
Zum Nachweis von $Y_n\Dto Y_0$ genügt es nach Satz \ref{prop:9.9} zu
zeigen, dass für beliebige beschränkte gleichmäßig stetige Funktionen gilt
$f:\R\to\R$
\begin{align*}
\int f(Y_n)\dP \to \int f(Y_0)\dP.
\end{align*}
Seien also $f$ wie vorausgesetzt, $\ep > 0$ und $\delta > 0$, so dass
\begin{align*}
\abs{f(y+x)-f(y)} < \frac{\ep}{6},\qquad \forall x,y\in\R\text{ mit } \abs{x}<
\delta.
\end{align*}
Wähle außerdem $\alpha > 0$ mit
\begin{align*}
P[\abs{\alpha X} \ge \delta] \le \frac{\ep}{12\norm{f}_\infty}.
\end{align*}
Nach (*) existiert ein $n_0$ so, dass
\begin{align*}
\forall n\ge n_0 : \abs{\int_\Omega f(Y_n+\alpha X) - f(Y_0 + \alpha X)\dP} < 
\frac{\ep}{3}.
\end{align*}
Dann gilt für $n\ge n_0$,
\begin{align*}
\abs{\int_\Omega f(Y_n)-f(Y_0)\dP} &\le \underbrace{\int_\Omega
\abs{f(Y_n)-f(Y_n+\alpha X)}\dP}_{(1)} \\
&+ \underbrace{\abs{\int_\Omega f(Y_n+\alpha X)- f(Y_0+\alpha
X)\dP}}_{(2)} \\ &+ \underbrace{\int_\Omega \abs{f(Y_0+\alpha
X)-f(Y_0)}\dP}_{(3)}.
\end{align*}
(1): Wir nutzen die gleichmäßige Stetigkeit von $f$ aus,
\begin{align*}
\int_\Omega \abs{f(Y_n)-f(Y_n+\alpha
X)}\dP
&= \int_{\abs{\alpha X}\ge \delta} \abs{f(Y_n)-f(Y_n+\alpha
X)}\dP\\
&+\int_{\abs{\alpha X}< \delta} \abs{f(Y_n)-f(Y_n+\alpha
X)}\dP\\
&< 2\norm{f}_\infty P\left[\abs{\alpha X} \ge \delta\right] +
\frac{\ep}{6}P\left[ \abs{\alpha X} < \delta
\right]\\
&\le
2\norm{f}_\infty \frac{\ep}{12\norm{f}_\infty} + \frac{\ep}{6}\cdot1
=
\frac{\ep}{3}.
\end{align*}
(2) $< \frac{\ep}{3}$ für $n\ge n_0$.\\
(3) $< \frac{\ep}{6}P[\abs{\alpha X}< \delta] + 2\norm{f}_\infty
P[\abs{\alpha X}\ge \delta] \le \frac{\ep}{3}$.\\
Somit $\E f(Y_n)\to \E f(Y_0)$, d.h. $Y_n\Dto  Y_0$.\qedhere
\end{proof}

\begin{bem}
\label{bem:9.2}
Die obigen Definitionen und Sätze lassen sich auf W-Maße auf ${\BB}_{k}$
bzw.\ $k$-dim.\ Zufallsvektoren übertragen.\maphere
\end{bem}

\section{Zentrale Grenzwertsätze}

Ausssagen über die Approximation von Verteilungen, insbesondere der
Verteilungen von Summen von Zufallsvariablen, durch die Normalverteilung im
Sinne der schwachen Konvergenz werden als zentrale Grenzwertsätze bezeichnet.

\begin{prop}[Zentraler Grenzwertsatz von Lindeberg-L\'{e}vy im
  1-dim.\ Fall]
\label{prop:9.11}
Sei $(X_{n})_{n\in \N}$ eine unabhängige Folge identisch verteilter quadratisch
integrierbarer reeller Zufallsvariablen mit $\E X_{1} \defr a$, $\V(X_{1})\defr\sigma
^ {2}$ mit $\sigma >0$. Dann
\begin{align*}
\frac{1}{\sqrt{n} \sigma } \sum\limits^{n}_{k=1} (X_{k}-a)\quad
\overset{\DD}{\longrightarrow}\quad N(0,1)\mbox{-verteilte reelle
Zufallsvariable.}\fishhere
\end{align*}
\end{prop}

\begin{proof}
Ohne Einschränkung können wir $a=0$ und $\sigma=1$ annehmen (andernfalls
ersetzen wir $X_k$ durch $\frac{X_k-a}{\sigma}$). $X_1$ habe die
charakteristische Funktion $\ph$, dann besitzt
\begin{align*}
\sum\limits_{k=1}^n X_k
\end{align*}
nach Satz \ref{prop:6.4} die charakteristische Funktion $\ph^n$. Somit hat
\begin{align*}
\frac{1}{\sqrt{n}}\sum\limits_{k=1}^n X_k
\end{align*}
die charakteristische Funktion $\ph^n(\frac{\cdot}{\sqrt{n}})$.
Nach Satz \ref{prop:9.10} genügt es zu zeigen,
\begin{align*}
\forall x\in\R : \ph^n\left(\frac{u}{\sqrt{n}}\right)\to \e^{-\frac{u^2}{2}}.
\end{align*}
$\e^{-\frac{u^2}{2}}$ ist nach Bemerkung \ref{bem:6.4} die charakterisitsche
Funktion einer $N(0,1)$-Verteilung.

Nach Satz \ref{prop:6.6} besitzt $\ph$ wegen $\E X_1^2 < \infty$ eine stetige
2. Ableitung und daher nach dem Satz von Taylor eine Darstellung
\begin{align*}
\ph(u) &= \ph(0) + u\ph'(0) + \frac{u^2}{2}\ph''(0) + \rho(u)\\ 
&= 1 + iu\E X_1 - \frac{u^2}{2}\E X_1^2 + \rho(u),
\end{align*}
mit einem Restterm $\rho(u)$ der Ordnung $o(u^2)$, d.h.
\begin{align*}
\lim\limits_{u\to 0}\frac{\rho(u)}{u^2} = 0.
\end{align*}
Nach Voraussetzung ist $\E X_1 = a = 0$ und $\E X_1(X_1-1)
= \E X_1^2 = \V X_1 = 1$ und daher
\begin{align*}
\ph(u) =  1 - \frac{u^2}{2} + \rho(u).
\end{align*}
Für $u\in\R$ beliebig aber fest gilt
\begin{align*}
\ph^2\left(\frac{u}{\sqrt{n}}\right) =
\left(1 - \frac{u^2}{2n} + \rho\left(\frac{u}{\sqrt{n}}\right)\right)^n
=  \left(1 - \frac{\frac{u^2}{2} +
n\rho\left(\frac{u}{\sqrt{n}}\right)}{n}\right)^n
\end{align*} 
Nach Voraussetzung $n\rho\left(\frac{u}{\sqrt{n}}\right)\to 0$ für
$n\to\infty$, also
\begin{align*}
\left(1 - \frac{\frac{u^2}{2} +
n\rho\left(\frac{u}{\sqrt{n}}\right)}{n}\right)^n \to
\e^{-\frac{u^2}{2}}.\qedhere
\end{align*}
\end{proof}

Als unmittelbares Korollar erhalten wir.

\addtocounter{cor}{1}

\begin{cor}[Zentraler Grenzwertsatz von de Moivre und Laplace]
\label{cor:9.2}
Für eine unabhängige Folge $(X_{n})_{n\in \N}$ identisch verteilter
reeller Zufallsvariablen auf $(\Omega, \AA,P)$ mit
\begin{align*}
P[X_{1} =1]=p,\quad P[X_{1}=0] = 1-p \defr q,\quad (0<p<1)
\end{align*}
gilt für $n\to\infty$
\begin{align*}
\forall \alpha < \beta \in \R :
P\left[\alpha\mathop{<}\limits_{(=)} \frac{\sum^ {n}_{k=1}
X_{k}-np}{\sqrt{npq}}\mathop{<}\limits_{(=)} \beta \right] \to
\frac{1}{\sqrt{2\pi}}\int^{\beta}_{\alpha }\e^ {-t^ {2}/2}\dt.\fishhere
\end{align*}
\end{cor}

\begin{bsp}
Volksabstimmung zu Vorschlägen $A$ und $B$. Eine resolute Minderheit
von $3.000$ Personen stimmt für $A$. Weitere $1.000.000$ Personen stimmen
zufällig ab.

\textit{Wie groß ist die Wahrscheinlichkeit, dass $A$ angenommen wird?}
Bezeichne die ``gleichgültigen'' Wähler mit den Nummern
$k=1,2,3,\ldots,1.000.000$,
\begin{align*}
X_k = \begin{cases}
1, & \text{Wähler wählt }A,\\
0, & \text{Wähler wählt }B.
\end{cases}
\end{align*}
Somit sind $X_1,X_2,\ldots$ unabhängig und $b(1,1/2)$-verteilt. Seien
$n=1.000.000$, $r=3.000$, so wird der Vorschlag $A$ angenommen, wenn
\begin{align*}
\sum_{k=1}^n X_k + r > n - \sum\limits_{k=1}^n X_k,
\end{align*}
d.h. genau dann, wenn
\begin{align*}
\sum_{k=1}^n X_k > \frac{n-r}{2} = 498.500.
\end{align*}
Die Wahrscheinlichkeit dafür ist gegeben durch,
\begin{align*}
P[\sum_{k=1}^n X_k > 498.500]
= P\left[\frac{\sum_{k=1}^n X_k - \frac{n}{2}}{\sqrt{n\frac{1}{4}}} >
\frac{498.500 - 500.000}{500} = -3\right].
\end{align*}
Korollar \ref{cor:9.2} besagt, dass $\dfrac{\sum_{k=1}^n X_k -
\frac{n}{2}}{\sqrt{n\frac{1}{4}}}$ annähernd $N(0,1)$-verteilt ist, d.h.
\begin{align*}
P\left[\frac{\sum_{k=1}^n X_k - \frac{n}{2}}{\sqrt{n\frac{1}{4}}} >
-3\right] &= 1- P\left[\frac{\sum_{k=1}^n X_k -
\frac{n}{2}}{\sqrt{n\frac{1}{4}}} \le -3\right]\\ &
 \approx 1-\Phi(-3) = 0.9986.
\end{align*}
Obwohl lediglich 3.000 Personen sicher für $A$ stimmen, wird der Vorschlag
mit einer Wahrscheinlichkeit von $99.86\%$ angenommen.

Dies ist auch der Grund dafür, dass Vorschläge, die den Großteil der
Abstimmenden nicht interessieren, bereits von einer kleinen Gruppe von
Entschlossenen durchgesetzt werden können.\bsphere
\end{bsp}

\noindent
\textit{Normal- oder Poisson-Approximation?} 

Seien $X_n$ unabhängige $b(1,p)$-verteilte Zufallsvariablen. Nach dem
Grenzwertsatz von de Moivre-Laplace gilt
\begin{align*}
\underbrace{\sum_{i=1}^n\frac{X_i-p}{\sqrt{np(1-p)}}}_{\Dto N(0,1)-\text{vert.
ZV}} \approx N(0,1)\text{-verteilt für ``große'' }n.\tag{1}
\end{align*}
Andererseits gilt nach dem Grenzwertsatz von Poisson
\begin{align*}
\underbrace{\sum_{j=1}^n X_i}_{\Dto \pi(\lambda)} \approx \pi(np_n),\;
\text{falls $n$ ``groß'' und $p_n$ ``klein''},\tag{2}
\end{align*} 
und $np_n\to \lambda\in (0,\infty)$ (siehe Übungsaufgabe 45).

Im Fall (1) sind die Summanden dem Betrag nach klein für großes $n$, während
im Fall (2) nur die Wahrscheinlichkeit, $p_n=P[X_n=1]$, klein ist,
dass die Summanden \textit{nicht} klein sind.

Als \textit{Faustregel} gilt
\begin{defnpropenum}
  \item Normal-Approximation ist ``gut'', falls $np(1-p)\ge 9$.
  \item Poisson-Approximation ist ``gut'', falls $n\ge 50$ und $p\le 0.05$.
\end{defnpropenum}
Im Allgemeinen können Poisson- und Normal-Approximation (für kleine $n$) nicht
gleichzeitig angewendet werden.


\begin{bem}[Bemerkungen.]
\label{bem:9.3}
\begin{bemenum}
\item Seien $X_n$ unabhängige, identisch verteilte reelle Zufallsvariablen mit
endlichen Varianzen und $\mu=\E X_1$, so gilt nach dem starken Gesetz der großen
Zahlen,
\begin{align*}
\frac{1}{n}S_n \defl \frac{1}{n}\sum_{i=1}^n X_i\to \mu,\qquad
\Pfs\text{ und }L^2. 
\end{align*}
Sei $\ep > 0$. Wie groß ist nun $n_0$ zu wählen, so dass
\begin{align*}
\abs{\frac{1}{n}S_n - \mu} < \ep,\qquad n\ge n_0?
\end{align*}
Mit Standardmethoden der Analysis lässt sich die Fragestellung umformulieren zu
\begin{align*}
\exists \alpha > 0 \exists c\neq 0 :
\lim\limits_{n\to\infty} n^\alpha \abs{\frac{S_n}{n}-\mu} = c,\qquad
\Pfs?
\end{align*}
Ein solches $\alpha$ existiert nicht, da nach dem zentralgen Grenzwertsatz
\begin{align*}
n^{\frac{1}{2}} \left(
\frac{S_n}{n}-\mu
\right)
\Dto
N(0,\V(X_1))\text{-verteilte Zufallsvariable}.
\end{align*}
Für $\alpha<\frac{1}{2}$ liegt nach dem Satz von Slutsky Verteilungs-Konvergenz
gegen Null vor, also auch nach Wahrscheinlichkeit.
\item In der Praxis ist die Verteilungsfunktion $F$ der Zufallsvariablen
$X$ meist unbekannt. Seien dazu $X_1,\ldots,X_n$ unabhängige
Zufallsvariablen mit Verteilungsfunktion $F$. Schätze $F_n$ durch
\begin{align*}
\hat{F}_n(x) \defl \hat{F}_n(x,\omega) \defl
\frac{1}{n}\sum_{i=1}^n \Id_{[X_i(\omega)\le x]},\qquad x\in\R,
\end{align*}
wobei $\hat{F}_n(x)$ die relative Anzahl derjenigen $X_1,\ldots,X_n$
bezeichnet mit $X_i\le x$. $\hat{F}_n$ heißt \emph{empirische
Verteilungsfunktion} zu $X_1,\ldots,X_n$. Nach dem starken Gesetz der großen
Zahlen von Kolmogorov gilt $\Pfs$
\begin{align*}
\lim\limits_{n\to\infty}
\hat{F}_n(x)
&=
\lim\limits_{n\to\infty}
\frac{1}{n}\sum_{i=1}^n \underbrace{\Id_{[X_i\le x]}}_{\defl Y_i(x)}
= \E Y_1(x)
= \E \Id_{[X_1\le x]} = P[X_1\le x] \\ &= F(x),
\end{align*}
also $\hat{F}_n(x)\to F(x)\Pfs$ und $L^2$. Eine Verschärfung dieser Aussage
liefert der
\begin{propn}[Satz von Glivenko-Cantelli]
Seien $X_n$, $X$ unabhängige und identisch verteilte Zufallsvariablen. Die
empirische Verteilungsfunktion $\hat{F}_n$ konvergiert P-f.s. gleichmäßig
gegen $F$, d.h.
\begin{align*}
\forall \ep > 0\exists n_0\in\N \forall x\in \R,n\ge n_0:
\sup_{x\in\R}\abs{\hat{F}_n(x)-F(x)} < \ep\fs.\fishhere
\end{align*}
\end{propn}
Dieser Satz wird in der mathematischen Statistik bewiesen und heißt auch
Hauptsatz der Statistik.

Nach dem zentralen Grenzwertsatz gilt
\begin{align*}
\sqrt{n}\left(\hat{F}_n(x)-F(x) \right) &= \sqrt{n}
\left(\frac{1}{n}\sum_{i=1}^n \Id_{[X_i\le x]} - \E \Id_{[X_1\le x]} \right)\\
&= 
\frac{\sum_{i=1}^n\left( \Id_{[X_i\le x]} - \E \Id_{[X_1\le
x]}\right)}{\sqrt{n}} \Dto N(0,\sigma^2(x))
\end{align*}
wobei $\sigma^2(x) = \V(\Id_{[X_1\le x]}) = F(x)(1-F(x))$ und $\Id_{[X_1\le
x]}$ eine $b(1,F(x))$ verteilte Zufallsvariable.
Somit ist $\hat{F}_n(x)-F(x)$ approximativ
$N\left(0,\frac{F(x)(1-F(x)}{n}\right)$-verteilt.

Da $\sigma(x)\le1/4$  für alle $x\in\R$, gilt für ein vorgegebenes $\ep>0$
und eine $N(0,1)$-verteilte Zufallsvariable $Z$,
\begin{align*}
P[|\hat{F}_n(x)-F(x)| \le \ep]
& \approx P\left[|Z| \le \epsilon\sqrt{\frac{n}{F(x)(1-F(x))}}\right]\\
& \ge P\left[|Z| \le 2 \sqrt{n} \ep \right] \\
&= 2 \Phi(2\ep\sqrt{n}) -1
\end{align*}
\begin{bspn}[Zahlenbeispiel]
$\ep=0.1$, $n=100$, dann ist $P[|\hat{F}_n(x)-F(x)| \le
\ep] \gtrapprox 0,955 $ für jedes $x$ und jede Verteilfungsfunktion $F$.\bsphere
\end{bspn}
\item
\begin{propn}[Satz von Berry-Esseen]
Seien $X_n$ unabhängige identisch verteilte Zufallsvariablen mit $\E X_1 =
\mu$, $\V(X_1) = \sigma^2$ und $\E \abs{X_1}^3 < \infty$. Dann gilt
\begin{align*}
\sup_{x\in\R}\abs{P\left[\sqrt{n}\frac{\sum_{i=1}^n (X_i-\mu)}{n\sigma} \le x
\right] - \Phi(x)} \le c \frac{\E \abs{X_1}^3}{\sigma^3\sqrt{n}},
\end{align*}
mit $c < 0.7975$, wobei
\begin{align*}
P\left[\sqrt{n}\frac{\sum_{i=1}^n (X_i-\mu)}{n\sigma} \le x \right] \to
 \Phi(x) = \int_{-\infty}^x \frac{1}{2\pi} \e^{-\frac{t^2}{2}}\dt. 
\end{align*}
nach dem zentralen Grenzwertsatz.\fishhere\maphere
\end{propn}
\end{bemenum}
\end{bem}

% \begin{lemn}[Hilfsformel]
% \begin{align*}
% \forall {k\in \N}\forall {x\in \R} :  | \e^ {ix}-1-\frac{ix}{1!} -
% \ldots - \frac{(ix)^ {k-1}}{(k-1)!}| \leq \frac{|x|^ {k}}{k!}.\fishhere
% \end{align*}
% \end{lemn}

\begin{prop}[Zentraler Grenzwertsatz von Lindeberg]
\label{prop:9.12}
Die Folge $(X_{n})_{n\in \N}$ quadratisch integrierbarer reeller
Zufallsvariablen mit $\E X^{2}_{1}>0$, $\E X_{n}=0$, sei
unabhängig und erfülle --- mit $s^{2}_{n}\defl\sum^{n}_{i=1}\E X^{2}_{i}$, $s_n =
\sqrt{s^2_n}$ --- die \emph{klassische Lindeberg-Bedingung}
\begin{align*}
\forall {\varepsilon > 0} :
\frac{1}{s^ {2}_{n}}\sum^{n}_{i=1} \E(X^ {2}_{i}\Id_{ [|X_{i}| > \ep s_n]})\to
0.
\tag{LB}
\end{align*}
Dann
\begin{align*}
\frac{1}{s_{n}}\sum\limits^ {n}_{i=1}X_{i}\quad
\overset{\DD}{\longrightarrow}\quad N(0,1)\mbox{-verteilte
Zufallsvariable.}\fishhere
\end{align*}
\end{prop}

Die Lindeberg-Bedingung stellt sicher, dass der Einfluss aller Zufallsvariablen
ungefähr ``gleich groß ist''. Salop kann man sagen, dass ein Zentraler
Grenzwertsatz immer existiert, wenn man eine Größe betrachtet, die aus sehr
vielen aber kleinen und nahezu unabhängigen Einflüssen besteht. In diesem Fall
kann man stets vermuten, dass die Summe der Einflüsse normalverteilt ist.

Als Beispiel sei der Kurs einer Aktien genannt, die sich im Streubesitzt
befindet. Durch unabhängige Kauf- und Verkaufsaktionen haben die Aktienbesitzer
nur einen geringen Einfluss auf den Kurs, in ihrer Gesamtheit führt dies aber
zu normalverteilten Tages-Renditen. Befindet sich die Aktie dagegen im Besitz
einiger weniger Großaktionäre, sind die Aktienkurse  nicht mehr
($\log$)\-normalverteilt.

\begin{bem}[Bemerkungen.]
\label{bem:9.3}
\begin{bemenum}%[label=\alph{*})]
\item In Satz \ref{prop:9.12} dient die Folge $(s_n)$ zur Normierung. Die
  Lindeberg-Bedingung (LB) schränkt den Einfluss der einzelnen Zufallsvariablen ein.
\item In Satz \ref{prop:9.12} --- mit am Erwartungswert zentrierten Zufallsvariablen ---
  impliziert die klassische Lindeberg-Bedingung (LB) die \emph{klassische
  Feller-Bedingung}
\begin{align*}
\max\limits_{i=1,\ldots,n} \left(\frac{1}{s^{2}_{n}}\E X^ {2}_{i}\right)
\to 0
\end{align*}
und wird ihrerseits durch die \emph{klassische Ljapunov-Bedingung}
\begin{align*}
\exists \delta > 0 : \frac{1}{s^ {2+\delta }_{n}}
\sum\limits^ {n}_{i=1} \E|X_{i}| ^ {2+\delta }\to 0
\end{align*}
impliziert. Bei nicht am Erwartungswert zentrierten Zufallsvariablen ist jeweils $X_{i}$
durch $X_{i}-\E X_{i}$, auch in der Definition von $s_{n}$, zu ersetzen.
\item
Satz \ref{prop:9.12} $\Rightarrow $ Satz \ref{prop:9.11}.\maphere
\end{bemenum}
\end{bem}

Satz \ref{prop:9.12} folgt aus

\begin{prop}[Zentraler Grenzwertsatz von Lindeberg für Dreiecksschemata von
  ZVn]
\label{prop:9.13}
Für jedes $n\in\mathbb{N}$ seien
$X_{n,1},\ldots,X_{n,m_n}$ unabhängige quadratisch integrierbare reelle
Zufallsvariablen mit $m_n\to \infty$ für $n\to\infty$. Ferner seien
\begin{align*}
\E X_{n,i}=0,\quad \sum_{i=1}^{m_n} \E X_{n,i}^2=1
\end{align*}
und die Lindeberg-Bedingung
\begin{align*}
\forall \ep > 0 : \sum^{m_n}_{i=1}
\E\left(X^{2}_{n,i}\Id_{ [|X_{i}| > \ep]}\right) \to 0.
\end{align*}
erfüllt.  Dann
\begin{align*}
\sum\limits^{m_n}_{i=1}X_{n,i}\Dto
N(0,1)\mbox{-verteilte Zufallsvariable.}\fishhere
\end{align*}
\end{prop}

Bevor wir den Satz beweisen, betrachten wir folgendes Beispiel.

\begin{bsp}
Seien $X_n$ unabhängige, identisch verteilte, quadratisch integrierbare reelle
Zufallsvariablen mit
\begin{align*}
\E X_1 = 0,\quad \E X_1^2 = 1,\quad S_n \defl \sum_{i=1}^n X_i.
\end{align*}
Wir zeigen
\begin{align*}
n^{-\frac{3}{2}} \sum_{k=1}^n S_k \to N(0,\frac{1}{3})\text{-verteilte
Zufallsvariable}.
\end{align*}
Der zentrale Grenzwertsatz von Lindeberg \ref{prop:9.12} lässt sich so nicht
anwenden, da beispielsweise die Unabhängikeit in Bezug auf die $S_k$ verletzt
ist. Es gilt jedoch,
\begin{align*}
\sqrt{3}n^{-\frac{3}{2}} \sum_{k=1}^n S_k
= \sqrt{3}n^{-\frac{3}{2}} \sum_{k=1}^n \sum_{j=1}^k X_j
= \sqrt{3}n^{-\frac{3}{2}} \sum_{j=1}^n \underbrace{\sum_{k=j}^n
X_j}_{(n-j+1)X_j}.
\end{align*}
Die Vertauschbarkeit der Summen macht man sich schnell an der Skizze klar.

\begin{figure}[!htpb]
\centering
\begin{pspicture}(-0.7,-0.7)(3.2,3.2)
\psaxes[labels=none,ticks=none,linecolor=gdarkgray,tickcolor=gdarkgray]{->}%
 (0,0)(-0.5,-0.5)(3,3)[\color{gdarkgray}$k$,-90][\color{gdarkgray}$j$,0]

\psdots[linecolor=darkblue](0.5,0.5)(1,0.5)(1.5,0.5)(2,0.5)(2.5,0.5)
\psdots[linecolor=darkblue](1,1)(1.5,1)(2,1)(2.5,1)
\psdots[linecolor=darkblue](1.5,1.5)(2,1.5)(2.5,1.5)
\psdots[linecolor=darkblue](2,2)(2.5,2)
\psdots[linecolor=darkblue](2.5,2.5)

\psxTick(2.5){\color{gdarkgray}n}
\psyTick(2.5){\color{gdarkgray}n}

\end{pspicture}
\caption{Zur Vertauschbarkeit der Summen.}
\end{figure}

Setzen wir nun $X_{n,j} = \sqrt{3}n^{-\frac{3}{2}}(n-j+1)X_j$, so sind die
Voraussetzung von Satz \ref{prop:9.13} erfüllt, denn
\begin{align*}
\E X_{n,j} &= 0,\\
\sum_{j=1}^{m_n}\E X_{n,j}^2
&=\sum_{j=1}^{n} \E\left(\sqrt{3}n^{-\frac{3}{2}} (n-j+1) X_j\right)^2
=3 \frac{1}{n}\sum_{j=1}^n\left(1-\frac{j-1}{n}\right)^2\\
&\to 3 \int_0^1(1-t)^2\dt = 1.
\end{align*}
Außerdem ist (LB) erfüllt, da für $\ep > 0$,
\begin{align*}
&3\sum_{j=1}^n \E n^{-3}(n-j+1)^2 X_j^2\Id_{[\sqrt{3}n^{-\frac{3}{2}}
(n-j+1)X_j>\ep]}\\
&\quad\le \frac{3}{n}\sum\limits_{j=1}^n \E
\underbrace{X_1^2\Id_{[\sqrt{3}n^{-\frac{1}{2}}\abs{X_1}>\ep]}}_{\to 0\text{ in
}\Omega} \to 0,
\end{align*}
denn $X_1^2\Id_{[\sqrt{3}n^{-\frac{1}{2}}\abs{X_1}>\ep]} \le X_1^2$ und $\E
X_1^2 < \infty$, also können wir den Satz von der dominierten Konvergenz
anwenden. Somit konvergieren die Erwartungswerte gegen Null und nach dem Satz
von Stolz-Cesàro konvergiert daher auch das arithmetische Mittel gegen Null.

Somit gilt
\begin{align*}
\sqrt{3}n^{-\frac{3}{2}} \sum_{k=1}^n S_k = \sum_{j=1}^n X_{n,j} \Dto
N(0,1)\bsphere
\end{align*}
\end{bsp}

Zur Beweisvorbereitung benötigen wir noch einige Ergebnisse.

\begin{lemn}
Seien $z_i,\eta_i \in\C$ mit $\abs{z_i},\abs{\eta_i} < 1$, so gilt,
\begin{align*}
\abs{\prod_{i=1}^n z_i - \prod_{i=1}^n \eta _i} \le \sum_{i=1}^n
\abs{z_i-\eta_i}.\fishhere
\end{align*}
\end{lemn}
\begin{proof}
Der Beweis erfolgt durch Induktion. Der Induktionsanfang mit $n=1$ ist klar.
``$n=2$'':
\begin{align*}
&z_1z_2 - \eta_1\eta_2 = z_1(z_2-\eta_2)+\eta_2(z_1-\eta_1),\\
\Rightarrow &\abs{z_1z_2 - \eta_1\eta_2} \le \abs{z_2+\eta_2} +
\abs{z_1+\eta_1}.
\end{align*}
Der Induktionsschritt erfolgt analog.\qedhere
\end{proof}

Aus der Analysis kennen wir die

\begin{propn}[Taylor-Formel mit Integralrestglied]
Sei $I\subset\R$ ein Intervall und $f: I\to \C$ mindestens $(n+1)$-mal stetig
differenzierbar, so gilt für $a\in I$
\begin{align*}
f(x) = \sum_{k=0}^n \frac{f^{(k)}(0)}{k!}(x-a)^k + \frac{1}{n!}\int_a^x (x-t)^n
f^{(n+1)}(t)\dt.\fishhere
\end{align*}
\end{propn}

Als unmittelbare Konsequenz erhalten wir
\begin{lemn}[Hilfsformel 1]
$\abs{\e^{ix}-(1+ix-\frac{x^2}{2})} \le \min\setd{\abs{x^2},\abs{x^3}}$.\fishhere
\end{lemn}
\begin{proof}
Nach dem Taylor-Formel gilt für $a=0$ und $m\in\N$,
\begin{align*}
\e^{ix} = \sum_{k=0}^m \frac{1}{k!}(ix)^k + \frac{1}{m!}\int_0^x i^m
(x-s)^m\e^{is}\ds.
\end{align*}
\begin{align*}
&m=1: \abs{\e^{ix}-(1+ix)} = \abs{\int_0^x i\e^{is}(x-s)\ds} \le
\frac{\abs{x}^2}{2},\\ &m=2: \abs{\e^{ix}-(1+ix-\frac{x^2}{2})} \le \frac{\abs{x}^3}{6} \le \abs{x}^3.
\end{align*}
und folglich
\begin{align*}
\abs{\e^{ix}-(1+ix-\frac{x^2}{2})}
\le \abs{\e^{ix}-(1+ix)}+\frac{\abs{x}^2}{2} \le \abs{x}^2.
\end{align*}
Zusammenfassend also
\begin{align*}
\abs{\e^{ix}-(1+ix-\frac{x^2}{2})} \le \min\setd{\abs{x^2},\abs{x^3}}.\qedhere
\end{align*}
\end{proof}
\begin{lemn}[Hilfsformel 2]
Für $x\ge 0$ gilt $\abs{\e^{-x}-(1-x)} \le \frac{1}{2}x^2$.\fishhere
\end{lemn}
\begin{proof}
$\abs{\e^{-x}-(1-x)} \le \int_0^x \abs{\e^{-s}(x-s)}\ds \le \int_0^x \abs{x-s}\ds
= \frac{1}{2}x^2$.\qedhere
\end{proof}

\begin{proof}[Beweis von Satz \ref{prop:9.13}]
Wir zeigen zunächst, dass das Maximum der Varianzen von
$X_{n,1},\ldots,X_{n,n_m}$ gegen Null konvergiert. Sei also $\ep > 0$
\begin{align*}
\max_{i\in\setd{1,\ldots,m_n}} \V(X_{n,i})
&=
\max_{i\in\setd{1,\ldots,m_n}} 
\int_\Omega X_{n,i}^2 \Id_{[\abs{X_{n,i}}\le \ep]} + 
X_{n,i}^2\Id_{[\abs{X_{n,i}}> \ep]}\dP\\
&\le \ep^2 + \underbrace{\int_\Omega
\sum_{i=1}^{m_n}X_{n,i}^2\Id_{[\abs{X_{n,i}}> \ep]}\dP}_{\to 0\text{ nach Vor.
}}.
\end{align*}
Somit konvergiert das Maximum gegen Null, da $\ep$ beliebig.

Seien nun $S_n = \sum_{i=1}^n X_i$ und $C_{n,i}\defl\V X_{n,i}$, so gilt nach
Annahme $\sum_{i=1}^nC_{n,i}=~1$. Nach Satz \ref{prop:6.7} gilt für $t\in\R$,
\begin{align*}
\abs{\ph_{S_n}(t)-\e^{-\frac{t^2}{2}}} = 
\abs{\prod_{i=1}^n \ph_{X_{n,i}}(t)-\prod_{i=1}^{m_n} \e^{-C_{n,i}\frac{t^2}{2}}}
\le \sum_{i=1}^n \abs{\ph_{X_{n,i}}(t)-\e^{-C_{n,i}\frac{t^2}2}},\tag{*}
\end{align*}
nach dem obigen Lemma. Anwendung von Hilfsformel 1 und $\E X_{n,i}=0$ ergibt,
\begin{align*}
\abs{\ph_{n,i}(t)-\left(1-c_{n,i}\frac{t^2}{2}\right)}
&= \abs{\int_\Omega
\e^{itX_{n,i}}-\left(1-itX_{n,i}-X_{n,i}^2\frac{t^2}{2}\right)\dP}\\ 
&\overset{\text{HF1}}{\le} \int_\Omega \min\setd{\abs{tX_{n,i}}^2,
\abs{tX_{n,i}}^3}\dP\\ &\le \underbrace{\max\setd{1,\abs{t}^3}}_{\alpha}\int_\Omega X_{n,i}^2
\min\setd{1,\abs{X_{n,i}}}\dP,
\end{align*}
wobei
\begin{align*}
\int X_{n,i}^2 \min\setd{1,\abs{X_{n,i}}}\dP &\le
\int_{[\abs{X_{n,i}}>\ep]} X_{n,i}^2 \dP
+ \int_{[\abs{X_{n,i}}\le \ep]} X_{n,i}^2\underbrace{\abs{X_{n,i}}}_{\le
\ep}\dP\\
&\le\E\left(X_{n,i}^2\Id_{[\abs{X_{n,i}}>\ep]}\right)
+ \ep C_{n,i}. 
\end{align*}
Durch Summation erhalten wir,
\begin{align*}
\sum_{i=1}^{m_n} \abs{\ph_{n,i}(t)-\left(1-C_{n,i}\frac{t^2}{2}\right)}
\le \alpha
\left(\underbrace{\sum_{i=1}^{m_n}\E\left(X_{n,i}^2\Id_{[X_{n,i}>\ep]}\right)}_{\to
0,\quad n\to\infty} + \ep\underbrace{\sum_{i=1}^{m_n}C_{n,i}}_{=1}\right)\tag{1}
\end{align*}
Also konvergiert der gesamte Ausdruck gegen Null für $n\to\infty$, da $\ep > 0$
beliebig.

Da $C_{n,i}$ Varianz, ist $C_{n,i}\ge0$. Wir können somit
Hilfsformel 2 anwenden und erhalten,
\begin{align*}
\sum_{i=1}^{m_n}\abs{\e^{-C_{n,i}\frac{t^2}{2}}-\left(1-C_{n,i}\frac{t^2}{2}\right)}
\overset{\text{HF2}}{\le} \frac{t^2}{4}\sum_{i=1}^{m_n} C_{n,i}^2
\le \frac{t^2}{4} \underbrace{\max\limits_{1\le i\le m_n}
C_{n,i}}_{\to
0,\quad n\to\infty}
\underbrace{\sum_{j=1}^{m_n} C_{n,i}}_{=1}.\tag{2}
\end{align*}

Wenden wir die Dreieckungsungleichugn sowie (1) und (2) auf (*) an, so folgt
\begin{align*}
\forall t\in \R : \ph_{S_n}(t) \to \e^{-\frac{t^2}{2}}.\qedhere
\end{align*}
\end{proof}

\subsection{Multivariate zentrale Grenzwertsätze}

Ist $X$ ein $d$-dimensionaler integrierbarer Zufallsvektor, d.h. $\E\norm{X} <
\infty$, so heißt
\begin{align*}
\E X= (\E X_1,\ldots,\E X_d)^\top
\end{align*}
\emph{Erwartungsvektor} von $X$.

Ist $X$ ein $d$-dimensionaler quadratisch integrierbarer Zufallsvektor,
d.h. $\E\norm{X}^2 < \infty$, so heißt
\begin{align*}
\Cov(X)\defl (\Cov(X_i,X_j))_{i,j\in\setd{1,\ldots,d}}
\end{align*}
\emph{Kovarianzmatrix} von $X$, wobei die einzelnen Einträge die
\emph{Kovarianzen}
\begin{align*}
\Cov(X_i,X_j)\defl\E(X_i-\E X_i)(X_j-\E X_j)
\end{align*}
der reellwertigen Zufallsvariablen $X_i$ und $X_j$ darstellen.

Auf der Hauptdiagonalen der Kovarianzmatrix stehen die Varianzen der $X_i$.
Insbesondere ist im eindimensionalen Fall gerade $\V X = \Cov(X)$. Die
Nebendiagonalelemente können als Maß für die stochastische Abhängigkeit der
$X_i,X_j$ interpretiert werden. Sind alle Komponenten unabhängig, so ist die
Kovarianzmatrix eine Diagonalmatrix.

\begin{defn}
\label{defn:9.2}
Ein $d$-dimensionaler Zufallsvektor
$X=(X_1,\ldots,X_d)^\top$ heißt \emph{multivariat normalverteilt} (oder auch
$d$-dimensional normalverteilt), falls für jedes $u\in\R^d$ die
Zufallsvariable $\lin{u,X} = u^tX=\sum_{i=1}^d u_iX_i$
eindimensional normalverteilt ist, wobei eine 1-dimensionale Normalverteilung
mit Varianz $0$ als eine Dirac-Verteilung $\delta_a$ im Punkt $a$
interpretiert wird.\fishhere
\end{defn}

\begin{defn}
\label{defn:9.2}
Seien $X_n$, $X$ $d$-dimensionale Zufallsvektoren so konvergiert $X_n\to X$
\emph{nach Verteilung} für $n\to\infty$, wenn
\begin{align*}
\forall f\in C_b(\R^n) : \E f(X_n)\to \E f(X).
\end{align*}
Schreibe $X_n\Dto X$.\fishhere
\end{defn}

\begin{prop}[Satz (Cram\'{e}r-Wold-Device)]
\label{prop:9.14}
Für $d$-dimensionale Zufallsvektoren $X_n$ und $X$ gilt $X_n \Dto X$ genau
dann, wenn $\lin{u,X_n} \Dto \lin{u,X}$
für alle $u\in\R^d$.\fishhere
\end{prop}
\begin{proof}
Der Beweis wird in den Übungen behandelt.\qedhere
\end{proof}

\begin{prop}[Multivariater zentraler Grenzwertsatz]
\label{prop:9.15}
Seien $X_{n,1},\ldots,X_{n,m_n}$ unabhängige quadratisch
integrierbare $d$-dimensionale Zufallsvariablen mit $m_n\to \infty$ für
$n\to\infty$. Des Weiteren gelte
\begin{defnpropenum}
\item $\forall n\in\N \forall i\in\setd{1,\ldots,m_n} :
\E X_{n,i}=0$,
\item $\forall n\in\N  : \sum_{i=1}^{m_n} \Cov(X_{n,i}) =
C \in \R^{d\times d}$,
\item  $\forall \ep > 0 :
\sum^{m_n}_{i=1} \E\left(\norm{X}^{2}_{n,i}\Id_{ [\norm{X_{i}} >
\ep]} \right) \to 0.$
\end{defnpropenum}
Dann
\begin{align*}
\sum\limits^{m_n}_{i=1}X_{n,i}\Dto
N(0,C)\text{-verteilten Zufallsvektor}.\fishhere
\end{align*}
\end{prop}
\begin{proof}
$S_n\defl\sum_{i=1}^{m_n} X_{n,i}$. Wegen Satz \ref{prop:9.14} genügt es zu zeigen,
\begin{align*}
\forall u\in\R^d : \lin{u,S_n}\Dto N(0,\lin{u,Cu}).
\end{align*}
\textit{1. Fall} $\lin{u,Cu} = 0$. Dann ist
\begin{align*}
\V(\lin{u,S_n}) = 0,
\end{align*}
also $\lin{u,S_n}=0\Pfs$\\
\textit{2. Fall} $\lin{u,Cu} > 0$. Wende Satz \ref{prop:9.13} auf
\begin{align*}
\tilde{X}_{n,i} = \frac{1}{\sqrt{\lin{u,Cu}}}\lin{u,X_{n,i}}
\end{align*}
an, so erhalten wir für den Erwartungswert
\begin{align*}
\E \tilde{X}_{n,i} = \frac{1}{\sqrt{\lin{u,Cu}}}\lin{u,\E X_{n,i}} = 0,
\end{align*}  
also
\begin{align*}
\sum_{i=1}^{m_n} \V(\tilde{X}_{n,i}) =
\frac{1}{\lin{u,Cu}}\underbrace{\sum_{i=1}^{m_n}\V(\lin{u,X_{n,i}})}_{\lin{u,Cu}}
= 1.
\end{align*}
Zum Nachweis der Lindebergbedingung betrachte
\begin{align*}
\sum_{i=1}^{m_n} \E \left( \tilde{X}_{n,i}^2\Id_{[\abs{\tilde{X}_{n,i}}>\ep]}
\right) &=
\frac{1}{\lin{u,Cu}}
\sum_{i=1}^{m_n} \E
\underbrace{\lin{u,X_{n,i}}^2}_{\le\norm{u}^2\norm{X_{n,i}}^2}
\Id_{\left[\frac{1}{\sqrt{\lin{u,Cu}}}\abs{\lin{u,X_{n,i}}}
>\ep \right]}\\
&\le
\frac{1}{\lin{u,Cu}}
\sum_{i=1}^{m_n} \E\left(\norm{u}^2\norm{X_{n,i}}^2\right)
\Id_{\left[\frac{\norm{u}}{\sqrt{\lin{u,Cu}}}\norm{X_{n,i}}
>\ep \right]}\\
&\le
\frac{\norm{u}^2}{\lin{u,Cu}}\sum_{i=1}^{m_n}\E \norm{X_{n,i}}^2
\Id_{[\norm{X_{n,i}} > \frac{\sqrt{\lin{u,Cu}}}{\norm{u}}\ep]}
\to 0,
\end{align*}
nach Voraussetzung für $n\to\infty$. Damit sind alle Voraussetzungen von Satz
\ref{prop:9.13} erfüllt.\qedhere
\end{proof}
\begin{cor}
\label{cor:9.3}
Ist $(X_n)_{n\in\mathbb{N}}$ eine unabhängige Folge
  identisch verteilter quadratisch integrierbarer Zufallsvektoren, so gilt
\begin{align*}
\frac{1}{\sqrt{n}} \sum\limits^{m_n}_{i=1}(X_{i}-\E X_i)
\Dto N(0,\Cov(X_1))\text{-verteilte ZV}.\fishhere
\end{align*}
\end{cor}

\cleardoublepage
\chapter{Bedingte Erwartungen}

Bedingte Erwartungen stellen den mathematischen Rahmen zur Untersuchung der
Fragestellung, welchen Mittelwert eine Zufallsvariable $Y$ annimmt unter der 
Voraussetzung, dass eine andere Zufallsvariable $X$ den Wert $x$
annimmt,
\begin{align*}
\E(Y\mid X=c).
\end{align*}
Als Beispiel sei die Suche des mittleren Körpergewichts einer gegebenen
Bevölkerungsschicht, unter Vorraussetzung einer gewissen Körpergröße, genannt.

Erinnern wir uns zurück an die Definition des Erwartungswerts, so können wir
diesen als Verallgemeinerung des Begriffes der Wahrscheinlichkeit für das
Eintreten eines Ereignisses interpretieren.
Allgemeiner gilt
\begin{align*}
P(A\mid X=x)=\E(\Id_A\mid X=x).
\end{align*}

Als Beispiel sei das Ereignis $A$, die Straße ist glatt, unter der
Voraussetzung, dass die Lufttemperatur $X$ den Wert $x$ annimmt.

Unsere bisherige Definition von $P(A\mid X=x)$ schließt den Fall $P[X=x]=0$
nicht mit ein. Wir werden dieses Manko durch die Definition einer allgemeinen
``bedingten Erwartung'' beheben.
\clearpage

\section{Grundlagen}

Wie immer sei $(\Omega,\AA,P)$ ein W-Raum.

Erinnern wir uns zunächst an die Definition der Messbarkeit einer
Abbildung. $f$ ist $\CC$-$\BB$-messbar $\Leftrightarrow$ $f^{-1}(\BB)
\subseteq \CC$.

Eine reellwertige Funktion ist eine Zufallsvariable, falls sie $\AA-\BB$
messbar ist. Bedingte Erwartungen sind spezielle Zufallsvariablen, bei denen
die Messbarkeitsforderung noch verschärft wird.

Zum Beweis des folgenden Satzes benötigen wir den Satz von Radon-Nikodym (siehe
Anhang), der ein Verhältnis zwischen zwei Maßen $\mu$ und $\nu$ herstellt.

\begin{prop}
\label{prop:10.1}
Sei $X: (\Omega, {\AA},P)\to (\RA, \overline{\BB})$ integrierbare
Zufallsvariable und ${\CC} \subset {\AA}$ $\sigma $-Algebra. Dann existiert
eine Zufallsvariable
\begin{align*}
Z : (\Omega, {\AA}, P) \to (\R,{\BB})
\end{align*}
mit folgenden Eigenschaften:
\begin{defnpropenum}
\item\label{prop:10.1:1}
$Z$ ist integrierbar und ${\CC}$-${\BB}$-messbar,
\item\label{prop:10.1:2}
$\forall C\in {\CC} : \int_{C}X\dP = \int_{C}Z\dP$.\fishhere
\end{defnpropenum}
\end{prop}

\begin{proof}
Ohne Einschränkung sei $X\ge 0$, ansonsten zerlegen wir $X$ in $X_+$ und $X_-$.
Sei $\ph: \CC \to \R$ definiert durch
\begin{align*}
\ph(C) \defl \int_C X\dP,\qquad C\in \CC,
\end{align*}
so ist $\ph$ aufgrund der Integrierbarkeit von $X$ wohldefiniert und ein Maß.
Außerdem ist $\ph$ ein $P\big|_{\CC}$-stetiges endliches Maß auf $\CC$.

Somit sind alle Voraussetzungen des Satzes von Radon-Nikodyn erfüllt und es
folgt die Existenz einer bis auf die Äquivalenz ``$=\fu{P\big|_\CC}$''
eindeutige, $\CC-\BB$-messbare Funktion
\begin{align*}
Z: \Omega\to \R_+.
\end{align*}
Weiterhin ist $Z$ $P\big|_{\CC}$-integrierbar, wobei
\begin{align*}
\forall C\in\CC : \ph(C) = \int_C Z \dP\big|_{\CC}
=  \int_C Z \dP.\qedhere
\end{align*}
$Z$ ist auch $\AA-\BB$-messbar und bezüglich $\PP$ integrierbar.\qedhere
\end{proof}

Die Aussage des Satzes mag zunächst unscheinbar sein. Es ist jedoch zu
beachten, dass die Zufallsvariable $Z$ tatsächlich so konstruiert werden kann,
dass sie $\CC$-$\BB$-messbar ist. Da man $\CC$ als echte Teilmenge von $\AA$
wählen kann, ist dies überhaupt nicht offensichtlich

$Z$ ist eindeutig bis auf die Äquivalenz ``$\fu{P\big|_{\CC}}$'

\begin{defn}
\label{defn:10.1}
Sei $X : (\Omega, {\AA},P)\to (\R, \overline{\BB})$
integrierbare Zufallsvariable und ${\CC}\subset {\AA}$
$\sigma$-Algebra. Die Äquivalenzklasse (im oberen
Sinne) der Zufallsvariablen $Z$: $(\Omega , {\AA}, P) \to (\R , {\BB})$ mit
\ref{prop:10.1:1} und \ref{prop:10.1:2} --- oder auch ein Repräsentant dieser
Äquivalenzklasse --- heißt \emph{bedingte Erwartung von $X$ bei gegebenem
  ${\CC}$}\index{Bedingte!Erwartung}. Man bezeichnet $Z$ mit $\E(X\mid {\CC})$.

Häufig wird ein Repräsentant dieser Äquivalenzklasse als eine Version von
$\E(X\mid {\CC})$ bezeichnet.\fishhere
\end{defn}

Die bedingte Erwartung $\E(X\mid {\CC})$ stellt im Allgemeinen also
\textit{keine} reelle Zahl sondern eine Zufallsvariable dar! $\E(X\mid {\CC})$
kann man als eine ``Vergröberung'' von $X$ betrachten, da $\CC$ gröber als
$\AA$ und $Z$ daher weniger Möglichkeiten der Variation als $X$ besitzt.
(Siehe Abbildung \ref{abb:10.1})

\begin{bemn}[Bemerkungen.]
\begin{bemenum}
\item Zur Eindeutigkeit von Satz \ref{prop:10.1} betrachte zwei Zufallsvariablen
$Z_1$ und $Z_2$, so gilt
\begin{align*}
\forall C\in\CC : \int_C (Z_1-Z_2)\dP = 0 \Rightarrow Z_1=Z_2 \fu{P\big|_{\CC}}
\end{align*}
\item $\E(X\mid\CC) = \frac{\dph}{\dP\big|_{\CC}}$, wobei $\ph(C) = \int_C
X\dP$ für $C\in\CC$.\maphere
\end{bemenum}
\end{bemn}

\begin{figure}[!htpb]
\centering
\begin{pspicture}(-0.7,-1)(4.7,4.7)

 \psaxes[labels=none,ticks=none,linecolor=gdarkgray,tickcolor=gdarkgray]{->}%
 (0,0)(-0.5,-0.5)(4.5,4.5)[\color{gdarkgray}$\omega$,-90][,0]

\psline[linewidth=1.2pt,linecolor=darkblue](0,1)(1,1)
\psline[linewidth=1.2pt,linecolor=darkblue](1,2)(2,2)
\psline[linewidth=1.2pt,linecolor=darkblue](2,3)(3,3)
\psline[linewidth=1.2pt,linecolor=darkblue](3,4)(4,4)
\psline[linewidth=1.2pt,linecolor=purple](0,1.5)(2,1.5)
\psline[linewidth=1.2pt,linecolor=purple](2,3.5)(4,3.5)

\rput(2.7,4){\color{darkblue}$X$}
\rput(3.7,3.2){\color{purple}$Z$}

\psxTick(1){\color{gdarkgray}\frac{1}{4}}
\psxTick(2){\color{gdarkgray}\frac{1}{2}}
\psxTick(3){\color{gdarkgray}\frac{3}{4}}
\psxTick(4){\color{gdarkgray}1}

\psyTick(1){\color{gdarkgray}1}
\psyTick(2){\color{gdarkgray}2}
\psyTick(3){\color{gdarkgray}3}
\psyTick(4){\color{gdarkgray}4}

\end{pspicture}
\caption{$\Omega=[0,1]$, %
$\AA=\sigma\left(\setd{[0,\frac{1}{4}),[\frac{1}{4},\frac{1}{2}),[\frac{1}{2},\frac{3}{4}),[\frac{3}{4},1]}\right)$, %
$\CC=\sigma\left(\setd{[0,\frac{1}{2}),[\frac{1}{2},1]}\right)\subsetneq\AA$.}
\label{abb:10.1}
\end{figure}


\begin{bsp}
\label{bsp:10.1}
\begin{bspenum}
\item Sei ${\CC} = {\AA}$, so gilt $\E(X\mid {\CC}) = X $ f.s.
\item Sei $\CC =  \setd{\emptyset, \Omega}$ so gilt $\E(X\mid {\CC}) = \E
X$, denn
\begin{align*}
\underbrace{\int_\emptyset X\dP}_{=0} = \underbrace{\int_\varnothing \E
X\dP}_{=0},\quad
\int_\Omega X\dP = \int_\Omega \E X \dP.
\end{align*}
\item Sei ${\CC} = \setd{\emptyset, B, B^ {c},\Omega}$ für festes $B$
mit $0<P(B)<1$. So gilt
\begin{align*}
(\E(X\mid {\CC}))(\omega)=
\begin{cases}
\frac{1}{P(B)} \int_{B} X\dP \defr \E(X\mid B),
  & \omega \in B\\
\frac{1}{P(B^ {c})} \int_{B^{c}}X\dP, & \omega \in B^ {c}.
\end{cases}
\end{align*}
$\E(X\mid B)$ heißt \emph{bedingter Erwartungswert von $X$ unter der
Hypothese $B$}.
\begin{proof}
Scharfes Hinsehen liefert, dass die rechte Seite nach obiger Definition
$\CC-\BB$-messbar ist. Weiterhin gilt
\begin{align*}
\int_\varnothing ``\text{rechte Seite}"\dP &= 0 = \int_\varnothing X\dP,\\
\int_B ``\text{rechte Seite}" \dP& = \frac{\int_B 1\dP}{P(B)}\int_B X\dP =
\int_B X\dP,\\
\int_{B^c} ``\text{rechte Seite}" \dP& = \frac{\int_{B^c}
1\dP}{P(B^c)}\int_{B^c} X\dP = \int_{B^c} X\dP,\\
\int_\Omega ``\text{rechte Seite}" \dP
&= \int_{B} ``\text{rechte Seite}" \dP
+ \int_{B^c} ``\text{rechte Seite}" \dP\\
&= \int_{B} X \dP
+ \int_{B^c} X \dP
= \int_\Omega X\dP.\qedhere\bsphere
\end{align*}
\end{proof}
\end{bspenum}
\end{bsp}

\begin{prop}
\label{prop:10.2}
Seien $X$, $X_{i}$ integrierbar, ${\CC} \subset
{\AA}$ $\sigma$-Algebra und $c,\alpha _{1,2} \in \R$.
\begin{propenum}
\item\label{prop:10.2:1} $\forall {C\in {\CC}} : \int_{C}  \E(X\mid {\CC})\dP =
\int_{C} X\dP$.
\item\label{prop:10.2:2}
$X=c$ P-f.s. $\Rightarrow \E(X\mid {\CC})=c$ f.s.
\item\label{prop:10.2:3}
$X\geq 0$ P-f.s. $\Rightarrow \E(X\mid {\CC}) \geq 0$ f.s.
\item $\E(\alpha _{1} X_{1} +\alpha _{2} X_{2} \mid {\CC}) = \alpha _{1}
  \E(X_{1} \mid {\CC})+\alpha _{2}\E (X_{2} \mid {\CC})$ f.s.
\item\label{prop:10.2:4}
$X_{1} \leq X_{2} $ P-f.s. $\Rightarrow \E(X_{1} \mid {\CC})\leq \E(X_{2}
\mid {\CC})$ f.s.
\item\label{prop:10.2:5}
$X$ ${\CC}$-${\BB}$-messbar $\Rightarrow X=\E(X\mid {\CC})$ f.s.
\item\label{prop:10.2:6}
$X$ integrierbar, $Y$ ${\CC}$-${\BB}$-messbar, $XY$ integrierbar
\begin{align*}
\Rightarrow \E(XY\mid {\CC})= Y\E (X\mid {\CC}) \fs
\end{align*}
\item[g')]\label{prop:10.2:6a}
$X,X'$ integrierbar, $X\E(X'\mid {\CC})$ integrierbar
\begin{align*}
\Rightarrow \E(X\E(X'\mid {\CC})\mid {\CC})=\E( X\mid {\CC})\E(X'\mid
{\CC})\fs
\end{align*}
\item[h)]\label{prop:10.2:7} $\sigma$-Algebra ${\CC}_{1,2}$ mit ${\CC}_{1}
\subset {\CC}_{2} \subset {\AA}$, $X$ integrierbar
\begin{align*}
&\E(\E(X\mid {\CC}_{1})\mid {\CC}_{2})=\E(X\mid{\CC}_{1})\fs\\
&\E(\E(X\mid {\CC}_{2})\mid {\CC}_{1})=\E(X\mid{\CC}_{1})\fs
\end{align*}
Hier bedeutet f.s., Rest$\,_{{\CC}_{2}}$ P-f.s. bzw.
Rest$\,_{{\CC}_{1}}$ P-f.s.\fishhere
\end{propenum}
\end{prop}
\begin{proof}
\begin{proofenum}
\item Folgt sofort aus der Definition der bedingten Erwartung.
\item Klar, denn Glättung einer Konstanten ergibt die Konstante selbst.
\item Für eine $\CC-\BB$-messbare Funktion $Z$ mit $\forall C\in\CC : \int_C
Z\dP \ge 0$ folgt $Z\ge 0$ $P\big|_{\CC}$-$\fs$
\item Folgt direkt aus der Linearität des Intergrals.
\item Folgt direkt aus der Monotonie des Integrals, denn für $\CC-\BB$-messbare
Zufallsvariablen $Z_{1,2}$ mit
\begin{align*}
\forall C\in\CC : \int_C Z_1\dP \le \int_C Z_2\dP
\end{align*}
gilt auch $Z_1\le Z_2$ $P\big|_{\CC}$-$\fs$
\item Klar.
\item Es folgt sofort, dass $XY$ und $Y\E(X\mid \CC)$ $\CC-\BB$-messbar. Es
verbleibt zu zeigen, dass
\begin{align*}
\forall C\in \CC : \int_C Y \E (X\mid \CC)\dP = \int_C XY\dP.
\end{align*}
Ohne Einschränkung ist $X\ge 0$, ansonsten gehen wir zu $X_+$ und $X_-$ über.
Sei $C\in\CC$ so gilt
\begin{align*}
\int_C Y \E (X\mid \CC)\dP
=
\int_C Y \E (X\mid \CC)\dP\big|_{\CC},
\end{align*}
wobei $\E(X\mid \CC)=\frac{\dph}{\dP\big|_{\CC}}$ mit $\ph(C) = \int_C X\dP$ für
$C\in\CC$, so dass nach dem Zusatz zum Satz von Radon-Nikodym,
\begin{align*}
\int_C Y \E (X\mid \CC)\dP\big|_{\CC} = 
\int_C Y \dph = \int_C Y \dph^*, 
\end{align*}
mit $\ph^*(A) = \int_A X\dP$ für $A\in\AA$ (also $\ph=\ph^*\big|_{\CC}$). Somit
ist $\ph^*$ ein endliches Maß auf $\AA$. Erneute Anwendung des Zusatzes zum
Satz von Radon-Nikodym ergibt,
\begin{align*}
\int_C Y \dph^* = \int_C Y \frac{\dph^*}{\dP}\dP.
\end{align*}
Nun ist $\frac{\dph^*}{\dP} = X\Pfs$ und daher
\begin{align*}
\int_C Y \frac{\dph^*}{\dP}\dP=\int_C Y X\dP.
\end{align*}
\item[g')] Folgt sofort aus g), wenn wir $Y=\E(X\mid\CC)$ setzen.
\item[h)]
\begin{enumerate}[leftmargin=16pt]
\item[($\alpha$)] 
Aus g) folgt direkt, dass
\begin{align*}
\E(\E(X\mid\CC_1)\mid \CC_2) = \E(X\mid\CC_1)\underbrace{\E(1\mid\CC_2)}_{1},
\end{align*}
da $\E(X\mid\CC_1)$ $\CC_1-\BB$-messbar.
\item[($\beta$)]
Die zweite Gleichung ist plausibel, da die Vergröberung von $X$ über $\CC_2$
zur noch kleineren $\sigma$-Algebra $\CC_1$ dasselbe liefert, wie die unmittelbare
Vergröberung über $\CC_1$.

Seien also $Z\defl\E(X\mid\CC_1)$ und $Y\defl\E(X\mid\CC_2)$. Es ist zu zeigen, dass
$Z=\E(Y\mid\CC_1)$. Zunächst ist nach Definition $Z$ auch
$\CC_1-\BB$-messbar. Weiterhin sei $C\in\CC_1\subseteq \CC_2$, so gilt per
definitionem,
\begin{align*}
&\int_C Z \dP = \int_C X\dP,\qquad \int_C Y \dP = \int_ C X\dP,\\
\Rightarrow & \int_C Y \dP = \int_C Z\dP.\qedhere  
\end{align*}
\end{enumerate}
\end{proofenum}
\end{proof}

\begin{defn}
\label{defn:10.2}
Sei ${\CC}\subset {\AA}$ $\sigma$-Algebra und $A\in {\AA}$.
\begin{align*}
P(A\mid {\CC})\defl \E(\Id _{A}\mid {\CC})
\end{align*}
heißt \emph{bedingte Wahrscheinlichkeit von $A$ bei gegebenem
${\CC}$}\index{Bedingte!Wahrscheinlichkeit}.\fishhere
\end{defn}

Setzen wir $\CC=\setd{\varnothing,\Omega}$, so ist $\E(\Id_A\mid\CC) = \E
\Id_A = P(A)$ (vgl. \textsc{Bsp} \ref{bsp:10.1}). Nach Definition
\ref{defn:10.2} gilt dann
\begin{align*}
\E(\Id_A\mid\CC) = P(A\mid\CC).
\end{align*}

\begin{bem}[Bemerkung zu Definition \ref{defn:10.2}.]
\label{bem:10.1}
\begin{align*}
\forall{C\in {\CC}} : \int_{C} P(A\mid {\CC})\dP =P(A\cap C).\maphere
\end{align*}
\end{bem}
\begin{proof}
Sei $C\in\CC$, so gilt
\begin{align*}
\int_{C} P(A\mid {\CC})\dP = \int_{C} \E(\Id_A\mid \CC)\dP
= \int_{C} \Id_A\dP=\int_{\Omega} \Id_A\Id_C\dP = P(A\cap C).\qedhere
\end{align*}
\end{proof}

\begin{bsp}
\label{bsp:10.2}
Sei ${\CC} = \setd{\emptyset,B,B^ {c},\Omega}$ mit $0< P(B) < 1$. Sei
$A\in\AA$, so gilt
\begin{align*}
(P(A\mid {\CC}))(\omega )=
\begin{cases}
\frac{P(A\cap B)}{P(B)}\defr P(A\mid B), & \omega \in B\\
\frac{P(A\cap B^{c})}{P(B^ {c})}\defr P(A\mid B^c), &
\omega \in B^ {c},
\end{cases}
\end{align*}
denn
\begin{align*}
(P(A\mid {\CC}))(\omega) &\overset{\text{vgl. \textsc{Bsp} \ref{bsp:10.1}}}{=}
\E(\Id_A\mid\CC)(\omega) \\
&=
\begin{cases}
\frac{1}{P(B)}\int_B \Id_A \dP = \frac{P(A\cap B)}{P(B)} = P(A\mid B), &
\omega\in B,\\
\frac{1}{P(B^c)}\int_{B^c} \Id_A \dP = \frac{P(A\cap B^c)}{P(B^c)} = P(A\mid
B^c), &\omega\notin B.\bsphere
\end{cases} 
\end{align*}
\end{bsp}

\begin{defn}
\label{defn:10.3}
\begin{defnenum}
\item
Seien $X:(\Omega, {\AA}, P) \to (\RA,\overline{\BB})$, $Y:~(\Omega, {\AA},  P)
\to (\Omega', {\AA}')$ Zufallsvariablen und $X$ integrierbar.
\begin{align*}
\E(X\mid Y)\defl\E(X\mid {Y^ {-1} ({\AA}')})
\end{align*}
heißt \emph{bedingte Erwartung von $X$ bei
gegebenem $Y$}\index{Bedingte!Erwartung}.
$Y^{-1}(\AA)$ bezeichnet hier die kleinste $\sigma$-Algebra in $\Omega$, bzgl.\
der $Y$ messbar ist.
\item
Seien $X : (\Omega, {\AA}, P) \to (\RA,\overline{\BB})$,  
$Y_{i} : (\Omega, {\AA}, P)\to (\Omega'_{i}, {\AA}'_{i})$ für $i\in\II$
Zufallsvariablen und $X$ integrierbar. 

$\CC \subset {\AA}$ sei die kleinste $\sigma$-Algebra in $\Omega $, bzgl.\ der
alle $Y_{i}$ messbar sind, d.h.
${\CC}= {\FF} \left(\bigcup_{i\in\II} Y^ {-1}_{i}  ({\AA}_{i})\right)$. So heißt
\begin{align*}
\E(X\mid (Y_{i})_{i\in\II}) \defl \E(X\mid {\CC})
\end{align*}
\emph{bedingte Erwartung von $X$ bei
gegebenen~$Y_{i}$}, $i\in\II$\index{Bedingte!Erwartung}.
\item
Sei $A\in {\AA}$ und $Y: (\Omega, {\AA}, P) \to (\Omega' ,
{\AA}')$ Zufallsvariable.
\begin{align*}
P(A\mid Y) \defl \E(\Id_{A}\mid Y)
\end{align*}
heißt \emph{bedingte Wahrscheinlichkeit von $A$ bei
gegebenem $Y$}\index{Bedingte!Wahrscheinlichkeit}.\fishhere
\end{defnenum}
\end{defn}

\begin{bem}[Bemerkungen.]
\label{bem:10.2}
Sei $X : (\Omega, {\AA},P)\to (\R, \overline{\BB})$ eine integrierbare
Zufallsvariable.
\begin{bemenum}
\item Sei ${\CC}$ $\sigma$-Algebra in ${\AA}$. Dann gilt

$(X^ {-1}(\overline{\BB}), {\CC})$ unabhängig $\Rightarrow \E(X \mid {\CC}) =
\E X$ f.s.
\item
Sei $Y: (\Omega, {\AA}, P) \to (\Omega', {\AA}')$ eine Zufallsvariable. Dann
gilt

$(X,Y)$ unabhängig $ \Rightarrow \E(X\mid Y) =\E X $ f.s.\maphere
\end{bemenum}
\end{bem}
\begin{proof}
\begin{bemenum}
\item Zu zeigen ist, dass
\begin{align*}
\forall C\in\CC : \int_C \E X \dP = \int_C X \dP. 
\end{align*}
Schreiben wir
\begin{align*}
\int_C \E X \dP = (\E X)\int_\Omega \Id_C \dP = (\E X)(P(C)),
\end{align*}
sowie
\begin{align*}
\int_C X\dP = \int_\Omega \Id_C X\dP = \E(\Id_C X) = \E(\Id_C)\E X,
\end{align*}
mit Hilfe der Unabhängigkeit, so ist die Gleichheit offensichtlich.
\item Folgt direkt mit a).\qedhere
\end{bemenum}
\end{proof}

\begin{prop}
\label{prop:10.3}
Seien $X: (\Omega, {\AA}, P)\to (\RA,\overline{\BB})$, $Y: (\Omega, {\AA}, P)
\to (\Omega ', {\AA}')$ Zufallsvariablen. Dann existiert eine Abbildung
\begin{align*}
g: (\Omega ', {\AA}') \to (\R, {\BB}),
\end{align*}
mit $\E(X\mid Y) = g \circ Y$. $g$ ist die sog.\ \emph{Faktorisierung der
bedingten Erwartung}\index{Faktorisierung}.
$g$ ist eindeutig bis auf die Äquivalenz ``$=P_{Y}$-f.ü.\ ''.\maphere
\end{prop}
\begin{proof}
Sei $Z\defl \E(X\mid Y)$, dann
\begin{align*}
Z: (\Omega,\FF(Y)) \to (\R,\BB).
\end{align*}
\begin{proofenum}
\item \textit{Existenz}
\textit{1. Schritt}. Rückführung auf Indikatorfunktionen. Sei $Z=\Id_A$ für
ein $A\in\FF(Y)=Y^{-1}(\AA')$, also existiert ein $A'\in\AA'$, so dass
$Y^{-1}(A') = A$.
\begin{align*}
\Id_{A'}(Y(\omega)) = 
\begin{cases}
1, & Y(\omega)\in A',\text{ d.h. }\omega\in A,\\
0, & \text{sonst} 
\end{cases}
= \Id_A(\omega).
\end{align*}
Also ist $Z=g\circ Y$ mit $g=\Id_{A'}$.

\textit{2. Schritt}. $Z$ sei nichtnegativ und einfach. Dann folgt die Existenz
von $g$ mit $Z=g\circ Y$ direkt aus dem 1. Schritt.

\textit{3. Schritt}. $Z$ sei nichtnegativ und $\CC-\BB$-messbar. Dann existiert
einer Folge von einfachen Zufallsvariablen $Z_n$ mit $Z_n\uparrow Z$. Nach dem
2. Schritt existieren $g_n: (\Omega',\AA')\to(\R,\BB')$ mit $Z_n = g_n\circ Y$.
Setzen wir
\begin{align*}
g^* \defl \sup_{n\in\N} g_n,
\end{align*}
so ist $g^*$ messbar und setzten wir weiterhin
\begin{align*}
g = g^*\Id_{[g^*< \infty]},
\end{align*}
so ist $g$ reellwertig. Da $Z= \sup_{n\in\N} Z_n$, gilt auch $\Pfs$
\begin{align*}
Z = g\circ Y.
\end{align*}
\textit{4. Schritt}. Sei $Z$ messbar, so besitzt $Z$
eine Darstellung $Z=Z_+-Z_-$, mit $Z_+,Z_-$ positiv und messbar. Nach dem 3.
Schritt existieren Abbildungen $g_+,g_-$ mit $Z_+ = g_+\circ Y$ und $Z_- =
g_-\circ Y$. $g\defl g_+-g_-$ ist die gesuchte Abbildung.
\item\textit{Eindeutigkeit}.
Es gelte $\E(X\mid Y) = g_1\circ Y \fs = g_2\circ Y \fs$ und somit
\begin{align*}
&\forall {C\in\FF(Y)} : \int_C g_1\circ Y \dP = \int_C
X \dP = \int_C g_2\circ Y \dP.\\
\Rightarrow &\forall {C'\in\AA'} : \int_{Y^{-1}(C')} g_1\circ Y \dP =
\int_{Y^{-1}(C')} g_2\circ Y \dP.
\end{align*}
Nach dem Transformationssatz gilt somit
\begin{align*}
\forall {C'\in\AA'} : \int_{C'} g_1 \dP_Y =
\int_{C'} g_2 \dP_Y,
\end{align*}
d.h. gerade
\begin{align*}
\forall {C'\in\AA'} : \int_{C'} (g_1-g_2) \dP_Y = 0
\end{align*}
und daher $g_1=g_2 \fu{P_{Y}}$\qedhere
\end{proofenum}
\end{proof}

\begin{defn}
\label{defn:10.4}
Sei $X: (\Omega, {\AA}, P)\to (\RA,\overline{\BB})$  
bzw. $A\in {\AA}$ eine integrierbare Zufallsvariable und $Y : (\Omega,
{\AA},P)\to (\Omega ', {\AA}')$ Zufallsvariable. Sei $g$ bzw.\ $g_{A}$ eine ---
bis auf Äquivalenz ``$=P_{Y}$ - f.ü.'' eindeutig bestimmte --- Faktorisierung
von $\E(X|Y)$ bzw.\ von $P(A\mid Y)$.
\begin{align*}
\E(X\mid Y=y)\defl g(y)
\end{align*}
heißt \emph{bedingte Erwartung von $X$ unter der Hypothese
$Y=y$}.\index{Bedingte!Erwartung}
\begin{align*}
P(A\mid Y=y) \defl g_{A}(y)
\end{align*}
heißt \emph{bedingte Wahrscheinlichkeit von $A$ unter der
Hypoththese.~$Y=y$}\index{Bedingte!Wahrscheinlichkeit}
\begin{align*}
&\E(X\mid Y=\cdot)=g,\\
&P(A\mid Y=\cdot )=g_{A}.\fishhere
\end{align*}
\end{defn}

\begin{prop}
\label{prop:10.4}
Sei $X: (\Omega, {\AA}, P)\to (\overline{\R},
\overline{\BB})$ bzw. $A\in {\AA}$ integrierbare Zufallsvariable und
$Y : (\Omega, {\AA},P)\to (\Omega ', {\AA }')$ Zufallsvariable. Dann gelten
\begin{propenum}
\item
$\forall {A' \in {\AA}'} : \int_{A'} \E(X\mid Y=y)
\dP_{Y}(y) =\int_{Y^ {-1}(A')} X\dP$,

insbesondere $\int_{\Omega '} \E( X \mid Y=y)\dP_{Y}(y) = \E X$.
\item
$\forall A'\in {\AA}' : \int_{A'} P(A\mid Y=y) \dP_{Y}(y)= P(Y^ {-1}
(A')\cap A)$.

insbesondere $\int_{\Omega '} P(A\mid Y=y)\dP_{Y}(y) = P(A)$.\fishhere
\end{propenum} 
\end{prop}

In vielen Anwendungen ist es tatsächlicher einfacher, die bedingte Erwartung
$\int_{\Omega '} \E( X \mid Y=y)\dP_{Y}(y)$ anstelle des Erwartungswerts
$\E Y$ zu berechnen.

\begin{proof}
\begin{proofenum}
\item Sei $g$ eine Faktorisierung von $\E(X\mid Y)$ (deren Existenz sichert
Satz \ref{prop:10.3}) und $A'\in\AA'$. So gilt
\begin{align*} 
\int_{Y^{-1}(A')} X \dP &= \int_{Y^{-1}(A')} \E(X\mid Y)\dP
 \overset{\text{Satz \ref{prop:10.3}}}{=} \int_{Y^{-1}(A')} g\circ Y\dP
 \\&= \int_{A'} g(y)\dP_Y(y)\overset{\text{Def \ref{defn:10.4}}}=
 \int_{A'} \E(X\mid Y=y) \dP_Y(y).
\end{align*}
\item Folgt direkt aus dem Vorangegangenen mit $X=\Id_A$.\qedhere
\end{proofenum}
\end{proof}

\begin{bsp}
Seien $X$ bzw. $A$ sowie $Y$ wie zuvor. Sei $y \in \Omega '$ wobei
$\setd{y} \in {\AA}'$ und $P[Y=y]=P_{Y}(\setd{y})>0$.
\begin{bspenum}
\item
$\underbrace{\E(X\mid Y=y)}_{\text{s.\ Definition \ref{defn:10.4}}}
=
\underbrace{ \E(X\mid [Y=y])}_{\text{s.\ \textsc{Bsp} \ref{bsp:10.1}}}$.
\begin{proof}
Sei also $y\in\Omega'$ fest mit $\setd{y}\in\AA'$ und $P_Y(\setd{y})>0$. Nach
Satz \ref{prop:10.4} ist $\E(X\mid Y=y)$ eindeutig, denn für zwei
Faktorisierungen $g_1$ und $g_2$ von $\E(X\mid Y)$ stimmen $g_1$ und $g_2$ bis
auf eine Menge vom $P_Y$-Maß Null überein, aber $P_Y(\setd{y})>0$, also
$g_1(y)=g_2(y)$.

Sei also $g$ eine Faktorisierung von $\E(X\mid Y)$, so ist $\E(X\mid Y)=g(y)$
unabhängig von der Wahl von $g$. Nach Satz \ref{prop:10.4}  gilt
\begin{align*}
&\int_{\setd{y}} g(y')\dP_Y(y') = \int_{[Y=y]} X\dP,\\
\Rightarrow &
g(y) = \frac{1}{P[Y=y]} \int_{[Y=y]} X\dP.\qedhere
\end{align*}
\end{proof}
\item $\underbrace{P(A\mid Y=y)}_{\text{s.\ Definition \ref{defn:10.4}}}
=
\underbrace{P(A\mid [Y=y])}_{\text{s.\ \textsc{Bsp} \ref{bsp:10.2}}}$.\bsphere
\end{bspenum}
\end{bsp}

\begin{prop}
\label{prop:10.5}
Sei $X: (\Omega, {\AA}, P) \to (\RA, \overline{\BB})$ integrierbare
Zufallsvariable und $Y$: $(\Omega, {\AA})\to (\Omega ', {\AA}')$
Zufallsvariable.
\begin{propenum}
\item
$X=c $f.s. $\Rightarrow \E(X\mid Y=\cdot ) = c \fu{P_{Y}}$.
\item
$X\geq 0 $ f.s. $\Rightarrow \E(X\mid Y= \cdot) \geq 0 \fu{P_{Y}}$.
\item
$\E(\alpha X_{1} +\beta X_{2} \mid Y=\cdot ) = \alpha \E(X_{1}\mid Y=\cdot)
+ \beta \E(X_{2} \mid Y=\cdot ) \fu{P_{Y}}$.
\item $X_{1} \leq X_{2}$ f.s. $\Rightarrow \E(X_{1}\mid Y=\cdot )\leq
  \E(X_{2} \mid Y=\cdot ) \fu{P_{Y}}$.\fishhere
\end{propenum}
\end{prop}
\begin{proof}
Diese Eigenschaften ergeben sich sofort aus Satz \ref{prop:10.2} und
Satz \ref{prop:10.3}.\qedhere
\end{proof}


\begin{prop}
\label{prop:10.6}
Seien ${\CC}\subset {\AA}$ eine Sub-$\sigma$-Algebra und $X$, $X_n$
integrierbare Zufallsvariablen. Dann gilt:
\begin{propenum}
\item Ist $0 \le X_n \uparrow X \fs$, so folgt
\begin{align*}
\E(X_n\mid {\CC}) \to \E(X\mid {\CC}) \fs
\end{align*}
(Satz von der monotonen Konvergenz für bedingte Erwartungen).
\item Ist $ X_n \to X\fs$, $|X_n| \le Y\fs$  und
$Y$ eine integrierbare Zufallsvariable, so folgt
\begin{align*}
\E(X_n\mid {\CC}) \to \E(X\mid {\CC}) \fs
\end{align*}
(Satz von der dominierten Konvergenz für bedingte Erwartungen).\fishhere
\end{propenum}
\end{prop}
\begin{proof}
\begin{proofenum}
\item Sei $0\le X_n \uparrow X\fs$, so folgt mit dem klassischen Satz von der
monotonen Konvergenz, dass
\begin{align*}
\E \abs{X_n-X} = \E X_n - \E X \to 0.
\end{align*}
Außerdem folgt aus der Dreiecksungleichung oder aus Satz \ref{prop:10.7},
\begin{align*}
\E(\E(\abs{X_n-X}\mid \CC))
\ge \E(\abs{\E((X_n-X)\mid \CC)})
=
\E(\abs{\E(X_n\mid \CC)-\E(X\mid \CC)}),
\end{align*}
wobei die linke Seite gegen Null konvergiert, also
$\E(X_n\mid\CC)\overset{L^1}{\to}\E(X\mid\CC)$. Nun existiert eine Teilfolge,
so dass $\E(X_{n_k}\mid \CC)\to \E(X\mid \CC)\fs$ und da $X_{n_k}\uparrow X\fs$
ist auch die Konvergenz von $(\E(X_{n_k}\mid\CC))$ monoton und da
$(\E(X_n\mid\CC))$ monoton folgt
\begin{align*}
\E(X_n\mid \CC)\uparrow \E(X\mid \CC).
\end{align*}
\item Sei $X_n\to X\fs$ Setzen wir $Z_n\defl\sup\limits_{k\ge n} \abs{X_k-X}$, so
ist klar, dass $Z_n\downarrow 0\Pfs$ und es gilt
\begin{align*}
\abs{\E(X_n\mid\CC)-\E(X\mid\CC)} \le 
\E(\abs{X_n-X}\mid\CC) \le \E(Z_n\mid\CC),
\end{align*}
aufgrund der Monotonie der bedingten Erwartung. Es genügt nun zu zeigen, dass
$\E(Z_n\mid\CC)\downarrow 0\Pfs$

Da $Z_n \downarrow 0\Pfs$, konvergiert auch $\E(Z_n\mid\CC)$ $P$-f.s. punktweise
gegen eine nichtnegative Zufallsvariable. Sei also $U\defl \lim\limits_{n\to\infty}
\E(Z_n\mid\CC)$. Nun gilt
\begin{align*}
0 \le Z_n \le \sup_{n\in\N} \abs{X_n} + \abs{X} \le 2 Y.
\end{align*}
Wir können schreiben
\begin{align*}
\E(U) = \E(\E(U\mid\CC)) \le \E(\E(Z_n\mid\CC)) = \E(Z_n)\to 0,
\end{align*}
nach dem klassischen Satz von der dominierten Konvergenz, denn die $Z_n$
konvergieren f.s. und werden durch eine integrierbare Zufallsvariable
majorisiert. Somit ist $U=0\fs$\qedhere
\end{proofenum}
\end{proof}

\begin{prop}[Jensen'sche Ungleichung]
\label{prop:10.7}
Sei ${\CC} \subset {\AA}$ eine Sub-$\sigma$-Algebra, $I \subset \R$ ein
Intervall, $f: I \to \R$ eine konvexe Funktion und $X:\Omega\to I$ eine
integrierbare Zufallsvariable. Dann ist $\E(X\mid {\CC}) \in I$ fast sicher.
Ist $f(X)$ integrierbar, so gilt
\begin{align*}
f(\E(X\mid {\CC})) \le \E(f(X)\mid {\CC}) \fs\fishhere
\end{align*}
\end{prop}
\begin{proof}
Falls $a \le X$ bzw. $X\le b$, so ist $a\le \E(X\mid\CC)$ bzw. $\E(X\mid\CC)
\le b$ und daher $\E(X\mid\CC)\in I$.

Eine konvexe Funktion $f:I\to\R$ besitzt eine Darstellung als
\begin{align*}
f(x) = \sup_{v\in V} v(x),\qquad V\defl \setdef{v: I\to\R}{v(t)=a+bt\le f(t),\quad
t\in I}.
\end{align*}
Somit gilt aufgrund der Linearität
\begin{align*}
f(\E(X\mid\CC)) = \sup\limits_{v\in V}v(\E(X\mid\CC))
= \sup\limits_{v\in V}(\E(v(X)\mid\CC)).
\end{align*}
Nun ist $\sup\limits_{v\in\V} v(X)\le f(X)$ und somit,
\begin{align*}
\sup\limits_{v\in V}(\E(v(X)\mid\CC)) \le \E(f(X)\mid\CC).\qedhere
\end{align*}
%TODO: Bild konvexe Funktion, Geraden an zwei Stützpunkte, sodass geraden f
% nicht schneiden (höchstens berühren)
\end{proof}


\cleardoublepage
\chapter{Martingale}

Ziel dieses Kapitels ist
es, Kriterien für ein starkes Gesetz der großen Zahlen auch für abhängige
Zufallsvariablen zu finden. Dazu untersuchen wir zunächst spezielle Folgen von
Zufallsvariablen, die Martingale, für die sehr angenehme
Konvergenzsätze existieren, auch wenn die Zufallsvariablen der Folge abhängig
sind.

Martingale spielen eine große Rolle in der Spieltheorie, in der
Finanzmathematik und in der stochastischen Analysis.

Für alles Weitere sei $(\Omega,\AA,P)$ ein W-Raum, $n\in\N$ und sofern nicht
anders angegeben, mit ``$\to$'' die Konvergenz für $n\to\infty$
bezeichnet.

\begin{defn}
\label{defn:11.1}
Eine Folge $(X_{n})$ integrierbarer Zufallsvariablen $X_{n}: (\Omega ,
{\AA},P)\to (\RA, \overline{\BB})$ heißt bei gegebener monoton
wachsender Folge $({\AA}_{n})$ von $\sigma $-Algebren ${\AA}_{n}\subset {\AA}$
mit ${\AA}_{n}$-$\overline{\BB}$-Messbarkeit von $X_{n}$
\begin{propenum}
\item
ein \emph{Martingal}\index{Martingal} bzgl.\ $({\AA}_{n})$, wenn
\begin{align*}
\E(X_{n+1}\mid {\AA}_{n}) 
= X_{n}\fs,\qquad n\ge 1,
\end{align*}
\item
ein \emph{Submartingal}\index{Submartingal} bzgl. $({\AA}_{n})$, wenn
\begin{align*}
\E(X_{n+1}\mid {\AA}_{n})\geq X_{n} \fs,\qquad n\ge 1,
\end{align*}
\item
ein \emph{Supermartingal}\index{Supermatringal}  bzgl. $({\AA}_{n})$, wenn
$(-X_{n})$ ein Submartingal bzgl. $({\AA}_{n})$ ist.~\fishhere
\end{propenum}
\end{defn}

Eine Folge von Zufallsvariablen ist nicht per se ein Martingal sondern
immer nur in Bezug auf eine Folge von $\sigma$-Algebren. Ein wichtiger
Spezialfall ist ${\AA}_{n}= {\FF}(X_{1},\ldots ,X_{n})$ und sofern nicht anders
angegeben, gehen wir immer von dieser Wahl von $\AA_n$ aus.

\begin{bem}
\label{bem:11.1}
Ein Martingal $(X_{n})$ bezüglich $({\AA}_{n})$ ist stets auch
ein Martingal bezüglich $({\FF} (X_{1},$ $\ldots, X_{n}))$. Denn nach
Voraussetzung ist jedes $X_n$ $\AA_n$-$\overline{\BB}$-messbar, so dass
$X_n^{-1}(\overline{\BB}) = \FF(X_n) \subset \AA_n$. Aus der Monotonie von
$\AA_n$ folgt dass $\FF(X_1,\ldots,X_n)\subset\AA_n$ und somit
\begin{align*}
\E(X_{n+1}\mid X_1,\ldots,X_n) &= 
\E(\underbrace{\E(X_{n+1}\mid \AA_n}_{\ge X_n})\mid \FF(X_1,\ldots,X_n))\\
&\ge \E(X_n\mid \FF(X_1,\ldots,X_n)) = X_n. 
\end{align*}
Entsprechend für Sub-, Supermartingal.\maphere
\end{bem}

\begin{prop}
\label{prop:11.1}
Sei $(V_{n})$ eine Folge von Zufallsvariablen $V_{n} : (\Omega, {\AA},P)\to
(\R, {\BB})$.  Die Partialsummenfolge $S_n = \sum^{n}_{j=1}
V_{j}\right$ ist genau dann ein Martingal bzw. Submartingal bzgl. 
$(\FF(V_{1},V_{1}+V_{2}, \ldots , V_{1}+ \ldots +V_{n})) =
(\FF(V_{1},\ldots,V_{n}))$, wenn für jedes $n$ gilt
\begin{propenum}
\item $V_{n}$ ist integrierbar, und
\item $\E(V_{n+1} \mid 
V_{1} , \ldots , V_{n})= 0$ bzw. $\geq 0\fs$\fishhere 
\end{propenum}
\end{prop}
\begin{proof}
Die Integrierbarkeit sichert die Existenz der bedingten Erwartungen. Weiterhin
gilt für jedes $n\ge 0$
\begin{align*}
\E\left(\sum_{j=1}^{n+1} V_j\mid \FF_n \right)
= 
\E\biggl(V_{n+1}\mid \FF_n \biggr) + \sum_{j=1}^{n} V_j,
\end{align*}
nach Satz \ref{prop:10.2}, denn $V_j$ ist $\FF_n$-messbar für $j\le n$.  Somit
ist ist die Partialsummenfolge genau dann ein Martingal, wenn
$\E\biggl(V_{n+1}\mid \FF_n \biggr) = 0$, bzw. ein Submartingal, wenn $\ge
0$.\qedhere
\end{proof}

\begin{defn}
\label{defn:11.2}
Ein Spiel mit zufälligen Gewinnständen $X_{1},X_{2},\ldots $ nach dem 1., 2.,
\ldots\ Schritt heißt \emph{fair}\index{faires Spiel}, wenn $\E X_{1} =0 $ und
$(X_{n})$ ein Martingal ist, d.h. für jedes $n$ gilt  $\E X_n = 0$ und
\begin{align*}
\E(X_{n+1}\mid X_{1} = x_{1}, \ldots , X_{n} = x_{n})=x_n \text{ für }
P_{(X_{1}, \ldots , X_{n})}\text{-f.a. } (x_{1},\ldots ,x_{n}).\fishhere
\end{align*}
\end{defn}

\begin{prop}
\label{prop:11.2}
Seien die $V_{n} : (\Omega , {\AA}, P) \to (\R, {\BB})$
quadratisch integrierbare Zufallsvariablen und
die Partialsummenfolge $S_n = \sum_{j=1}^n V_j$ ein Martingal.
Dann sind die $V_n$ paarweise unkorreliert, d.h. 
\begin{align*}
\E(V_i\, V_j) = 0,\qquad i\neq j.\fishhere
\end{align*}
\end{prop}
\begin{proof}
Da $V_i$ und $V_j$ quadratisch integrierbar sind, folgt mit
der Cauchy-Schwartz-Ungleichung, dass
\begin{align*}
(\E(V_jV_i))^2 \le (\E V_j^2)(\E V_i^2) < \infty.
\end{align*}
Also ist $V_jV_i$ integrierbar. Falls $i< j$, so ist $V_i$ $\FF_j$-messbar und
wir erhalten
\begin{align*}
\E(V_jV_i) &= \E(\E(V_jV_i\mid V_i)) \overset{\ref{prop:10.2}}{=}
\E(V_i\,\E(V_j\mid V_i)) \\ &
= \E(\underbrace{\E(V_j\mid
V_1,\ldots,V_{j-1})}_{=0\fs\text{ nach }\ref{defn:11.1}}\mid V_i).\qedhere
\end{align*}
\end{proof}

\begin{bsp}
Ein Beispiel für
ein Martingal ist die Partialsummenfolge $(\sum^ {n}_{i=1} V_{j})_{n}$ zu einer unabhängigen Folge $(V_{n})$ von
integrierbaren reellen Zufallsvariablen mit Erwartungswerten $0$.\bsphere
\begin{proof}

Nach Satz \ref{prop:11.1} genügt es zu zeigen, dass 
\begin{align*}
\E(V_{n+1}\mid V_1,\ldots,V_n) = 0 \fs
\end{align*}
da $(\FF(V_{n+1}), \FF(V_1,\ldots,V_n))$ ein unabhängiges Paar von
$\sigma$-Algebren ist, gilt nach Bemerkung \ref{prop:10.2},
\begin{align*}
\E(V_{n+1}\mid V_1,\ldots,V_n) = \E V_{n+1} = 0.\qedhere
\end{align*}
\end{proof}
\end{bsp}

\begin{prop}[(Sub-/Super-)Martingalkonvergenztheorem von Doob]
\label{prop:11.3}
Sei $(X_{n})$ ein Sub-, Super- oder Martingal mit $\limsup\limits_{n\to\infty}
\E|X_{n}| < \infty$.
Dann existiert eine integrierbare reelle Zufallsvariable $X$, so dass $X_{n}\to
X\Pfs$\fishhere
\end{prop}

Zur Beweisvorbereitung benötigen wir noch
Definition \ref{defn:11.3} und \ref{defn:11.4}, sowie Satz \ref{prop:11.4} und
\ref{prop:11.5}.

\begin{defn}
\label{defn:11.3}
Sei $({\AA }_{n})$ eine monoton wachsende Folge von $\sigma$-Algebren
${\AA}_{n}\subset {\AA}$. Eine Zufallsvariable
\begin{align*}
T : (\Omega, {\AA },P)\to (\overline{\N}, {\PP}(\overline{\N})),\qquad\qquad
\overline{\N}\defl \N\cup \setd{\infty}
\end{align*}
heißt \emph{Stoppzeit}\index{Stoppzeit} bzgl.\ $({\AA}_{n})$, wenn
\begin{align*}
\forall {k\in \N} : [T=k]\in {\AA}_{k}.
\end{align*}
Hierbei heißt $T$ \emph{ Stoppzeit im engeren Sinne}, falls $P[T<\infty] =1$
(``Kein Vorgriff auf die Zukunft'').

Wichtiger Spezialfall: ${\AA}_{n} = {\FF} (X_{1},\ldots ,X_{n})$ mit
Zufallsvariablen $X_{n}$.\fishhere
\end{defn}

Man kann $(X_{n})$ als Folge der Gewinnstände in einem Spiel interpretieren.
Ein Spieler ohne prophetische Gaben bricht das Spiel im zufälligen Zeitpunkt
$T$ aufgrund des bisherigen Spielverlaufs, d.h. aufgrund der Informationen, die
bis zu diesem Zeitpunkt zur Verfügung stehen, ab.

\begin{bsp}
$T(\omega ) = \inf\setdef{n\in \N}{X_{n}(\omega ) \in B}$,
$\omega \in \Omega $ --- festes messbares $B$.

Zum Beispiel habe sich ein Aktienhändler einen festen Minimalwert vorgegegeben,
bei dessen Unterschreitung er seine Aktien verkaufen will. $B$ stellt dann das
Ereignis dar, dass eine Aktie den Minimalwert unterschreitet, und $X_n$ den
Aktienkurs seiner Aktien. Seine Bedingung lautet damit ``Verkaufe die Aktien für
das kleinste n, so dass $X_n\in B$''.\bsphere
\end{bsp}

\begin{defn}
\label{defn:11.4}
Sei $(X_{n})$ eine Folge von Zufallsvariablen $X_{n} : (\Omega, {\AA},P)\to
(\RA, \overline{\BB})$ und $(T_{n})$ eine Folge von Stoppzeiten bzgl.\
$(X_{n})$ [d.h.\ bzgl.\ $({\FF}(X_{1},\ldots ,X_{n}))$] mit $T_{1} \leq T_{2} \leq
\ldots < \infty$.

So wird eine neue Folge $(X_{T_{n}})_{n}$ von Zufallsvariablen definiert durch
\begin{align*}
(X_{T_{n}})(\omega ) \defl X_{T_{n}(\omega)} (\omega),\quad \omega \in
\Omega.
\end{align*}
Der Übergang von $(X_{n})$ zu $(X_{T_{n}})$ heißt \emph{optional sampling}
[frei gewählte Stichprobenbildung].\fishhere
\end{defn}

Man kann optional sampling z.B. als ein Testen des Spielverlaufs zu den
Zeitpunkten $T_{n}(\omega )$ interpretieren.

Anschaulich greift man aus einer vorgegebenen Menge von Zufallsvariablen eine
zufällige Teilmenge heraus. Dabei ist es durchaus möglich, dass eine
Zufallsvariable mehrfach auftritt.

\textit{Vorsicht}: Die $(X_{T_n})$ stellen keine Teilfolge von $(X_n)$ dar!

\begin{prop}[Optional Sampling Theorem]
\label{prop:11.4}
Sei $(X_{n})$ ein Submartingal, $M\in \N$ fest und $(T_{n})$
eine Folge von Stoppzeiten bzgl.\ $(X_{n})$ mit $T_{1}\leq T_{2} \leq \ldots
\leq M$.

Die durch optional sampling erhaltene Folge $(X_{T_{n}})_{n\in \N}$
ist ebenfalls ein Submartingal. --- Entsprechend für Martingal statt
Submartingal.\fishhere
\end{prop}

Die Martingaleigenschaft ist invariant unter ``Stoppen''. Interpretieren wir
$(X_n)$ z.B. als Folge von Gewinnständen in einem Spiel, so besagt der Satz,
dass sich die Fairness eines Spielverlaufs nicht ändert.

\begin{proof}
Wir führen den Beweis für Submartingale. Martingale werden analog behandelt.

Für alles weitere sei $n\in\N$ fest und $C\in\FF(X_{T_1},\ldots,X_{T_n})$.
Setzen wir $T = T_{n+1}$ und $S = T_n$, so ist $S\le T \le M$ und wir haben zu
zeigen, dass
\begin{align*}
\int_C X_T \dP \ge \int_C X_S \dP. 
\end{align*}
Da $C=\sum_{j=1}^M (C\cap[S=j])$, genügt es obige Ungleichung auf $D_j \defl C
\cap [S=j]$ zu zeigen. Es sei noch bemerkt, dass $D_j\in
\AA_j = \FF(X_1,\ldots,X_j)$.

Wir zeigen nun per Induktion, dass für jedes $m \ge j$
\begin{align*}
\int_{D_j\cap [T\ge m]} X_T \dP \ge
\int_{D_j\cap [T\ge m]} X_{m} \dP.  
\end{align*}
Somit folgt da $j\le T$ auf $D_j$,
\begin{align*}
\int_{D_j} X_T \dP = \int_{D_j\cap [T\ge j]} X_T\dP
\ge \int_{D_j} X_j\dP = \int_{D_j} X_S\dP.
\end{align*}

Für den Induktionsanfang bemerken wir, dass $[T\ge M] = [T=M]$, also folgt die
Behauptung unmittelbar. Für $m\ge j$ ist 
$D_j\cap [T\ge m+1] = D_j\cap [T\le m]^c\in \AA_{m}$, also folgt mit der
Submartingaleigenschaft von $X$
\begin{align*}
\int_{D_j\cap [T\ge m+1]} X_{m+1} \dP
&\ge
\int_{D_j\cap [T\ge m+1]} X_{m} \dP.
\end{align*}
Zusammen mit der Induktionsvoraussetzung gilt also
\begin{align*}
\int_{D_j\cap [T\ge m]} X_T \dP
&=
\int_{D_j\cap [T\ge m+1]} X_T \dP
+
\int_{D_j\cap [T= m]} X_T \dP\\
&\ge 
\int_{D_j\cap [T\ge m+1]} X_{m+1} \dP
+
\int_{D_j\cap [T= m]} X_T \dP\\
&\ge
\int_{D_j\cap [T\ge m]} X_{m} \dP.
\end{align*}
Damit ist die Induktion komplett und es gilt $\E(X_{T_{n+1}}\mid
X_{T_1},\ldots,X_{T_n}) \ge X_{T_n}$, was zu zeigen war.\qedhere


% 
% 
% 
% Wir beweisen den Satz für Submartingale. Für Martingale erfolgt er analog.
% 
% Sei $n\in\N$ beliebig aber fest und $C\in\FF(X_{T_1},\ldots,X_{T_n})$. Zu
% zeigen ist
% \begin{align*}
% \int_C X_{T_{n+1}}\dP \ge \int_C X_{T_n}\dP.\tag{*} 
% \end{align*}
% Sei $D_j=C\cap [T_n=j]$ für $j\in\setd{1,\ldots,M}$. Wegen $C=\sum_{j=1}^M D_j$
% genügt es (*) auf $D_j$ nachzuweisen. Sei nun $j\in\setd{1,\ldots,M}$ fest, so
%  gilt $D_j\in \FF(X_1,\ldots,X_j)$, denn 
% $C=[(X_{T_1},\ldots,X_{T_n})\cap B]$ für ein $B\in\BB$ und damit
% \begin{align*}
% D_j &= [(X_{T_1},\ldots,X_{T_n})\cap B,\; T_n = j] \\ &=
% \sum_{\atop{j_1,\ldots,j_{n-1}=1}{j_1\le j_2\le \ldots \le j_{n-1}\le j}}
% [X_{j_1},\ldots,X_{j_n}\in B,T_1=j_1,\ldots,T_{n-1}=j_{n-1},T_n=j]\\
% &\in \FF(X_1,\ldots,X_j). 
% \end{align*}
% Das Integral über $D_j$ ist also definiert und es gilt
% \begin{align*}
% \int_{D_j} X_{T_{n+1}}\dP
% = \sum_{k=1}^{M} \int_{D_j\cap [T_{n+1}=k]} X_{T_{n+1}}\dP
% = \sum_{k=j}^{M} \int_{D_j\cap [T_{n+1}=k]} X_{k}\dP.
% \end{align*}
% Wir behandeln lediglich $j=M-2$, die übrigen Fälle zeigt man analog,
% \begin{align*}
% \int_{D_{M-2}} X_{T_{n+1}}\dP &= \sum_{k=M-2}^{M} \int_{D_{M-2}\cap [T_{n+1}=k]}
% X_{k}\dP.
% \end{align*}
% Wir schätzen nun jeden der drei Summanden ab.
% 
% ``$k=M$'':
% Da $D_j\cap [T_{n+1}=M] =  D_j\cap[T_{n+1}\le M-1]^c\in
% \FF(X_1,\ldots,X_{M-1})$, gilt aufgrund der Submartingaleigeschaft
% \begin{align*}
% \int_{D_{M-2}\cap [T_{n+1}=M]}
% X_{M}\dP
% \ge  
% \int_{D_{M-2}\cap [T_{n+1}= M]} X_{M-1}\dP.
% \end{align*}
% ``$k=M-1$'': Zusammen mit dem eben gezeigten gilt
% \begin{align*}
% \sum_{k=M-1}^{M} \int_{D_{M-2}\cap [T_{n+1}=k]}
% X_{k}\dP
% &\ge \int_{D_{M-2}\cap [T_{n+1}=M-1]}
% X_{M-1}\dP
% +
% \int_{D_{M-2}\cap [T_{n+1}= M]} X_{M-1}\dP\\
% & =
% \int_{D_{M-2}\cap [T_{n+1}\ge M-1]}
% X_{M-1}\dP.
% \end{align*}
% Da $D_{M-2}\cap [T_{n+1}\ge M-1]= D_{M-2}\cap [T_{n+1}\le M-2]^c \in
% \FF(X_1,\ldots,X_{M-2})$ gilt
% \begin{align*}
% \int_{D_{M-2}\cap [T_{n+1}\ge M-1]}
% X_{M-1}\dP \ge
% \int_{D_{M-2}\cap [T_{n+1}\ge M-1]}
% X_{M-2}\dP.
% \end{align*}
% ``$k=M-2$": Folglich ist mit dem eben gezeigten,
% \begin{align*}
% \int_{D_j\cap [T_{n+1}=M-2]} X_{T_{n+1}} &\ge
% \int_{D_{M-2}\cap [T_{n+1}= M-2]}
% X_{M-2}\dP
% +
% \int_{D_{M-2}\cap [T_{n+1}\le M-2]^c}
% X_{M-2}\dP\\
% &=
% \int_{D_{M-2}\cap [T_{n+1}\ge M-2]}
% X_{M-2}\dP.
% \end{align*}
% Da $D_{M-2}\subset [T_{n}\ge M-2] \subset [T_{n+1}\ge M-2]$ gilt
% \begin{align*}
% \int_{D_j} X_{T_{n+1}} &\ge
% \int_{D_{M-2}}
% X_{M-2}\dP
% =
% \int_{D_{M-2}}
% X_{T_n}\dP,
% \end{align*}
% da auf $D_{M-2}$, $T_n=M-2$.\qedhere
\end{proof}


\begin{prop}[Upcrossing inequality von Doob]
\label{prop:11.5}
Sei $(X_{1},\ldots, X_{n})$ ein bei $n\ge 1$
abbrechendes Submartingal und $a < b$ reelle Zahlen. Die
Zufallsvariable $U [a,b]$ gebe die Anzahl der aufsteigenden Überquerungen des
Intervalls $[a,b]$ durch $X_{1},\ldots ,X_{n}$ an (d.h. die Anzahl der
Übergänge der abbrechenden Folge von einem Wert $\leq a $ zu einem Wert $\geq
b$). Dann gilt
\begin{align*}
(b-a)\E U [a,b] \leq \E(X_{n}-a)^{+} - \E(X_{1} -a)^{+}.\fishhere
\end{align*}
\end{prop}

\begin{figure}[!htpb]
\centering
\begin{pspicture}(-0.7,-1)(11,5.3)

 \psaxes[labels=none,ticks=none,linecolor=gdarkgray,tickcolor=gdarkgray]{->}%
 (0,0)(-0.5,-0.5)(10.4,4.5)%
 [\color{gdarkgray}$n$,-90][\color{gdarkgray}$X_n(\omega)$,0]

\psline[linestyle=dashed](0,1)(10,1)
\psline[linestyle=dashed](0,4)(10,4)
\psline[linestyle=dotted,linecolor=purple](2,0.8)(4,4.9)
\psline[linestyle=dotted,linecolor=purple](8,0.3)(10,5)

\psxTick(1){\color{gdarkgray}1}
\psxTick(2){\color{gdarkgray}2}
\psxTick(3){\color{gdarkgray}3}
\psxTick(4){\color{gdarkgray}4}
\psxTick(5){\color{gdarkgray}5}
\psxTick(6){\color{gdarkgray}6}
\psxTick(7){\color{gdarkgray}7}
\psxTick(8){\color{gdarkgray}8}
\psxTick(9){\color{gdarkgray}9}
\psxTick(10){\color{gdarkgray}10}

\psyTick(1){\color{gdarkgray}a}
\psyTick(4){\color{gdarkgray}b}

\psdots[linecolor=darkblue](1,1.2)(2,0.8)(3,3.8)(4,4.9)(5,3.2)(6,0.9)(7,1.2)(8,0.3)(9,3)(10,5)

\end{pspicture}
\caption{Zur Upcrossing Inequality. $X_n(\omega)$ mit Überquerungen,
$U_{10}[a,b]=2$.}
\label{abb:10.1}
\end{figure}

\begin{proof}
Da $(X_1,\ldots,X_n)$ ein Submartingal ist, ist auch $(X_1-a,\ldots,X_n-a)$
ein Submartingal und ebenso $((X_1-a)^+,\ldots,(X_n-a)^+)$, aufgrund der
Monotonie der bedingten Erwartung. Somit gibt $U[a,b]$ die Anzahl der
aufsteigenden Überschreitungen des Intervalls $[0,b-a]$ durch
$((X_1-a)^+,\ldots,(X_n-a)^+)$ an. Deshalb können wir ohne Einschränkung $a=0$
und $X_i \ge 0$ annehmen.

Zu zeigen ist nun $b\,\E U[0,b] \le \E X_n - \E X_1$. Wir definieren uns
Zufallsvariablen $T_1,\ldots,T_{n+1}$ durch folgende Vorschrift:
\begin{align*}
&T_1(\omega) &&= 1,\\
&T_{2j}(\omega) &&= 
\begin{cases}
\min\setdef{i}{T_{2j-1}\le i\le n,\,X_i(\omega)=0},\quad\; &
\text{falls existent},\\
n, & \text{sonst},
\end{cases}\\
&T_{2j+1}(\omega) &&=
\begin{cases}
\min\setdef{i}{T_{2j}(\omega)\le i \le n,\,X_i(\omega)\ge b},
& \text{falls existent},\\
n, & \text{sonst},
\end{cases}\\
&T_{n+1}(\omega) &&= n.
\end{align*}
Die geraden Zeiten $T_{2j}$ geben die Zeitpunkte der
Unterschreitungen von $[0,b-a]$ an, und die ungeraden Zeiten $T_{2j+1}$ geben
die Zeitpunkte der Überschreitungen an. So definiert sind $T_1,\ldots,T_{n+1}$
sind Stoppzeiten mit $T_1\le \ldots\le T_{n+1}=n$. 

Nach Satz \ref{prop:11.4} überträgt sich die Submartingaleigenschaft von $X$ auf
$(X_{T_1},\ldots,X_{T_n})$ und ein Teleskopsummenargument liefert
\begin{align*}
X_n - X_1 = \sum_{k=1}^n (X_{T_{k+1}}-X_{T_k})
= \underbrace{\sum\underbrace{\left(X_{T_{2j+1}} -
X_{T_{2j}}\right)}_{\ge b,\; \text{da Aufsteigung}}}_{\ge bU[0,b]} + 
\sum\underbrace{\left(X_{T_{2j}} -
X_{T_{2j-1}}\right)}_{\E \ldots \ge 0, \;\text{da Submartingal}}
\end{align*}
Also $\E X_n - \E X_1 \ge b\E U[0,b]$.\qedhere
\end{proof}

Nun können wir einen Beweis für Satz \ref{prop:11.3} geben.

\begin{proof}[Beweis von \ref{prop:11.3}.]
Es gilt $\liminf_n X_n \le \limsup_n X_n$ und beide Zufallsvariablen sind
messbar. Angenommen  $[\liminf_n X_n \neq \limsup_n X_n]$ habe positives Maß, so
gilt aufgrund der Stetigkeit von $P$, dass für ein $\ep > 0$
\begin{align*}
P[\limsup_n X_n - \liminf_n X_n > \ep] > 0.
\end{align*}
Also existieren reelle Zahlen $a < b$, so dass auch
\begin{align*}
P[\liminf_n X_n < a < b < \limsup_n X_n] > 0.\tag{\ensuremath{\star}}
\end{align*} 

Gebe nun $U_n[a,b]$ die Anzahl der aufsteigenden Überquerungen des Intervalls
$[a,b]$ durch $X_1,\ldots,X_n$ an. Nach Voraussetzung ist $\sup_{n\ge
1}\E \abs{X_n} \le M < \infty$, also gilt nach Doobs Upcrossing Inequality
\ref{prop:11.5}
\begin{align*}
(b-a)\E U_n[a,b] \le \E (X_n-a)^+  \le M + \abs{a}
<\infty,\qquad n\ge 1.
\end{align*} 
Nach ($\star$) überqueren die $X_n$ auf einer Menge mit positivem Maß  das
Intervall $[a,b]$ jedoch unendlich oft, d.h. $\E U_n[a,b] \uparrow \infty$ im
Widerspruch zur Upcrossing Inequality. Somit ist die Annahme falsch und es gilt
\begin{align*}
P[\liminf_n X_n = \limsup_n X_n] = 1.
\end{align*}
Setzen wir also $X = \lim\limits_{n\to\infty} X_n$ wann immer der Limes erklärt
ist, so gilt $X_n\to X$ \fs.~\qedhere
% 
% 
% \textit{1. Schritt}. Wir zeigen,
% \begin{align*}
% \forall -\infty < a < b < \infty : P
% \underbrace{\left[\liminf_n X_n < a < b < \limsup_n X_n\right]}_{\defr A(a,b)} =
% 0
% \end{align*}
% Wähle also $a<b$ beliebig aber fest. $U_n[a,b]$ gebe die Anzahl der
% aufsteigenden Überquerungen des Intervalls $[a,b]$ durch $X_1,\ldots,X_n$ an.
% Für jedes $n\in\N$ gilt
% \begin{align*}
% (b-a)\E U_n[a,b] &= (b-a)\int_{A(a,b)} U_n[a,b]\dP
% \le \E (X_n-a)^+ \dP\\ 
% &\overset{\text{Satz }\ref{prop:11.5}}{\le} \E\abs{X_n} + a
% \le \const  < \infty. 
% \end{align*}
% Angenommen $P(A(a,b)) > 0$. Dann gilt aber $U_n[a,b] \uparrow \infty$ für
% $n\to\infty$ auf $A(a,b)$. Nach dem Satz von der monotonen Konvergenz gilt
% \begin{align*}
% \int_{A(a,b)} U_n[a,b]\dP  \to \infty \cdot \underbrace{P(A(a,b))}_{>0} =
% \infty.
% \end{align*}
% Im Widerspruch zu $\int_{A(a,b)} U_n[a,b]\dP \le \const < \infty$.
% 
% \textit{2. Schritt}. Sei $A\defl[\liminf_n X_n < \limsup_n X_n] =
% \bigcup_{a,b\in\Q} A(a,b)$, so ist $A$ messbar.
% \begin{align*}
% P(A) \le \sum_{a,b\in\Q} \underbrace{P(A(a,b))}_{=0} = 0.
% \end{align*}
% 
% Also existiert eine erweitert reellwertige Zufallsvariable $X^*$ mit $X_n\to
% X^*\fs$ Nun gilt
% \begin{align*}
% \E \abs{X^*} \le \liminf_n \E \abs{X_n} < \infty
% \end{align*}
% nach Voraussetzung. Also ist $X^*$ integrierbar und daher $X^*\in\R\fs$ Setzen
% wir $X=X^*\Id_{[X^*\in\R]}$, so $X_n\to X\fs$ und $X$ ist reellwertige
% Zufallsvariable.
% \qedhere
\end{proof}

\begin{cor}
\label{cor:11.1}
Ist $(U_{n})$ eine Folge integrierbarer nichtnegativ-reeller Zufallsvariablen
auf $(\Omega, {\AA},P)$ und $({\AA}_{n})$ eine monoton wachsende Folge von
Sub-$\sigma $-Algebren von ${\AA}$ mit ${\AA}_{n}$-${\BB}_{+}$-Messbarkeit
von $U_{n}$ und gilt weiterhin
\begin{align*}
\E(U_{n+1}\mid {\AA}_{n})\leq (1+\alpha _{n}) U_{n} +\beta _{n}, 
\end{align*}
wobei $\alpha _{n}, \;\, \beta _{n} \in \R_{+}$ mit $\sum \alpha_{n} < \infty$,
$\sum \beta _{n}< \infty$, dann konvergiert $(U_{n})$ f.s. --- $(U_{n})$ ist
``fast ein nichtnegatives Supermartingal''.\\ Auch $(\E U_{n})$
konvergiert.\fishhere
\end{cor}

\begin{propn}[Zusatz zum Martingalkonvergenztheorem]
Ist das Martingal $(X_n)$ bezüglich der Folge $(\AA_n)$ gleichgradig
integrierbar, d.h.
\begin{align*}
\sup\limits_{n\ge 1} \E(\abs{X_n}\cdot \Id_{[\abs{X_n}\ge c]}) \to 0,\qquad
c\to \infty,
\end{align*}
so gilt zusätzlich
\begin{align*}
X_n\overset{L^1}{\longrightarrow} X,\qquad X_n = \E(X\mid \AA_n).\fishhere
\end{align*}
\end{propn}
\begin{proof}
Sei $\ep > 0$ fest. Dann existiert ein $c > 0$, so dass
\begin{align*}
\sup_{n\ge 1} \E(\abs{X_n}\cdot \Id_{[\abs{X_n} > c]}) < \ep,
\end{align*}
da $(X_n)$ gleichgradig integrierbar. Nun gilt
\begin{align*}
\E\abs{X_n} = \underbrace{\E(\abs{X_n}\mid\Id_{[\abs{X_n}>c]})}_{<\ep}
+\underbrace{\E(\abs{X_n}\mid\Id_{[\abs{X_n}\le c]})}_{\le c}
< c + \ep.
\end{align*}
Also ist $(X_n)$ $L^1$-beschränkt. Mit Satz \ref{prop:11.3} folgt somit
$\lim\limits_{n\to\infty} X_n \defr X$ existiert f.s. und $X\in L^1$.

Wir zeigen nun, dass $X_n\to X$ in $L^1$, d.h. $\E\abs{X_n-X}\to 0$. Setze
dazu
\begin{align*}
f_c(x) \defl  
\begin{cases}
c, & x > c,\\
x, & -c\le x \le c,\\
-c, & x < -c,
\end{cases}
\end{align*}
so ist $f$ lipschitz stetig und wegen der gleichgradigen Integrierbarkeit der
$X_n$ existiert ein $c>0$, so dass
\begin{align*}
&\E\abs{f_c(X_n)-X_n} < \frac{\ep}{3},\qquad n\ge 1\tag{1}\\
&\E\abs{f_c(X)-X} < \frac{\ep}{3},\tag{2}
\end{align*}
Da $\lim\limits_{n\to\infty} X_n = X\fs$ folgt
\begin{align*}
f_c(X_n)\to f_c(X)\fs
\end{align*}
und $\abs{f_c(X_n)} \le c$. Mit dem Satz von der dominierten Konvergenz folgt
\begin{align*}
\E\abs{f_c(X_n)-f_c(X)}\to 0\fs
\end{align*}
Zusammenfassend ergibt sich,
\begin{align*}
\E\abs{X_n-X} \le \E\abs{X_n-f_c(X_n)} + \E\abs{f_c(X_n)-f_c(X)}
+ \E\abs{f_c(X)-X} < \ep
\end{align*}
also $X_n\to X$ in $L^1$.

Es verbleibt zu zeigen, dass $X_n = \E(X\mid\AA_n)\fs$.
Sei $C\in\AA_m$ und $n\ge m$. Da $(X_n)$ ein Martingal ist, gilt $\E(X_n
\Id_{C}) = \E(X_m \Id_C)$. Betrachte
\begin{align*}
\abs{\E(X_n\Id_C) - \E(X\Id_C)}
\le
\E(\abs{X_n-X}\Id_C)
\le
\E(\abs{X_n-X})\to 0,
\end{align*}
so folgt $\E (X_m \Id_C) = \E (X\Id_C)$.\qedhere
\end{proof}

\begin{figure}[!htpb]
\centering
\begin{pspicture}(-0.7,-1)(2.5,3)

 \psaxes[labels=none,ticks=none,linecolor=gdarkgray,tickcolor=gdarkgray]{->}%
 (0,0)(-0.5,-0.5)(2.2,2.8)[\color{gdarkgray}$\omega$,-90][,0]

\psline[linecolor=purple,linestyle=dotted](0.5,0)(0.5,1.8)
\psline[linecolor=purple,linestyle=dotted](1.2,0)(1.2,1.8)

\psplot[linewidth=1.2pt,%
	     linecolor=darkblue,%
	     algebraic=true]%
	     {0}{1.7}{%
2/(1+(2*x-1.7)^2)*cos((2*x-1.7)^2)+0.6 %
}

\psline(0,1.8)(1.7,1.8)

\psyTick(1.8){\color{gdarkgray}c}

\end{pspicture}
\caption{Zur gleichgradigen Integrierbarkeit.}
\label{abb:10.1}
\end{figure}

\begin{prop}
\label{prop:11.6}
Sei $(V_{n})$ eine Folge von quadratisch integrierbaren reellen Zufallsvariablen
mit $\sum_{n=1}^\infty \V(V_{n})< \infty$. Dann ist
\begin{align*}
\sum_{n=1}^\infty (V_{n}-\E(V_{n} \mid V_{1},\ldots ,V_{n-1})) \quad \text{f.s.\ 
konvergent.}
\end{align*}
Sind die $V_n$ zusätzlich unabhängig, dann ist $\sum_{n=1}^\infty
(V_{n}-\E V_{n})$ f.s.\ konvergent.\fishhere
\end{prop}
\begin{proof}
Setze $W_n = V_n - \E(V_n\mid V_1,\ldots,V_{n-1})$,
so ist $W_n$ integrierbar und $\FF_n = \FF(V_1,\ldots,V_n)$-messbar. Wir haben
zu zeigen, dass $S_n = \sum_{i=1}^n W_i$ f.s. konvergiert.
Zunächst ist
\begin{align*}
&\E(W_{n+1}\mid \FF_n)  = 
\E(V_{n+1}\mid \FF_n)  -
\E(\E(V_{n+1}\mid \FF_n)\mid \FF_n)
= 0,  
\end{align*}
also ist $S_n$ ein Martingal. Weiterhin gilt $\E W_n^2 \le 4 \E
V_n^2<\infty$, also sind die $W_n$ nach Satz \ref{prop:11.2} paarweise
unkorreliert. Mit dem Satz von Bienaymé
\ref{prop:5.5} folgt unter Verwendung von $\E S_n = 0$, dass
\begin{align*}
\E S_n^2 = \V S_n = \sum_{i=1}^n \V W_i \le
\sum_{i\ge 1} \V V_i < \infty.
\end{align*}
Daher ist $\E\abs{S_n}$ beschränkt, und mit dem Martingalkonvergensatz
\ref{prop:11.3} folgt nun die Behauptung, dass $S_n$ $\fs$ konvergiert.

Sind die $V_n$ außerdem unabhängig, so können wir Bemerkung \ref{bem:10.2}
anwenden und erhalten $\E(V_n\mid V_1,\ldots,V_{n-1}) = \E V_n \fs$\qedhere
\end{proof}

\begin{prop}
\label{prop:11.7}
 Sei $(V_{n})$ eine Folge von quadratisch integrierbaren reellen
 Zufallsvariablen mit $\sum_{n=1}^\infty n^ {-2}\, \V(V_{n})< \infty$.  Dann
 gilt
 \begin{align*}
\frac{1}{n} \sum\limits^ {n}_{j=1} (V_{j}-\E(V_{j}\mid
V_{1},\ldots , V_{j-1}))\to 0 \quad \fs 
 \end{align*}
Falls zusätzlich $(V_{n})$ unabhängig, dann $\frac{1}{n} \sum^ {n}_{j=1}
(V_{j}-\E V_{j})\to 0 \fs$\\ (Kriterium von Kolmogorov zum starken Gesetz
der großen Zahlen)\fishhere
\end{prop}

\begin{proof}
Wenden wir Satz \ref{prop:11.6} auf $n^{-1}\, V_n$ an, erhalten wir
\begin{align*}
\sum_{n=1}^\infty \frac{V_n - \E(V_n\mid V_1,\ldots,V_{n-1})}{n}\text{
konvergiert f.s.}.
\end{align*}
Daraus folgt mit dem Lemma von Kronecker (s.u.),
\begin{align*}
\frac{1}{n}\sum_{i=1}^n \left(V_i - \E(V_i\mid V_1,\ldots,V_{i-1}\right) \to 0
\fs.
\end{align*}
Falls die $V_n$ unabhängig sind, können wir Bemerkung \ref{prop:10.2}
anwenden und erhalten so den Zusatz.\qedhere
\end{proof}


\begin{lem}[Lemma von Kronecker]
Sei $(c_{n})$ eine Folge reeller Zahlen.
\begin{align*}
\sum_{n=1}^\infty \frac{c_{n}}{n}\mbox{ konvergiert }\Rightarrow \frac{1}{n}
\sum\limits^ {n}_{j=1} c_{j}\to 0.\fishhere
\end{align*}
\end{lem}
\begin{proof}
Sei $s_n = \sum_{j=1}^n j^{-1} c_j$, so ist $(s_n)$ nach Voraussetzung
konvergent, sei $s$ der Grenzwert. Offensichtlich ist
$s_n-s_{n-1} = n^{-1}c_n$ mit $s_0=0$ und daher
\begin{align*}
\sum_{j=1}^n c_j = \sum_{j=1}^n j(s_j-s_{j-1}) = 
n s_n - \sum_{j=1}^n \underbrace{(j-(j-1))}_{1}s_{j-1}.
\end{align*}
Somit gilt
\begin{align*}
\frac{1}{n}\sum_{j=1}^n c_j = s_n - \underbrace{\frac{1}{n}\sum_{j=1}^n
s_{j-1}}_{\to s} \to 0.\qedhere
\end{align*}
\end{proof}

\begin{bem}
\label{bem:11.2}
Aus Satz \ref{prop:11.6} bzw. \ref{prop:11.7} ergibt sich unmittelbar für eine
Folge $(V_{n})$ quadratisch integrierbarer reeller Zufallsvariablen eine hinreichende
Bedingung für die $\fs$-Konvergenz der Reihe $\sum_{j=1}^n V_{n}$ bzw.\ für
$\frac{1}{n} \sum^{n}_{j=1} V_{j}\to 0 \fs$\fishhere
\end{bem}  

\begin{prop}[Kriterium von Kolmogorov für das starke Gesetz der großen
  Zahlen]
\label{prop:11.8}
Eine unabhängige Folge $(X_{n})$ quadratisch integrierbarer reeller
Zufallsvariablen mit
\begin{align*}
\sum\limits^ {\infty}_{n=1} n^ {-2} V(X_{n})< \infty
\end{align*}
genügt dem starken Gesetz der großen Zahlen.\fishhere
\end{prop}

\begin{prop}[Kolmogorovsches starkes Gesetz der großen Zahlen]
\label{prop:11.9}
Für eine unabhängige Folge $(X_{n})_{n\in\N}$ identisch verteilter
integrierbarer reeller Zufallsvariablen gilt
\begin{align*}
\frac{1}{n} \sum\limits^ {n}_{k=1}X_{k} \to \E X_{1} \fs\fishhere
\end{align*}
\end{prop}

Der Satz wurde bereits im Kapitel \ref{chap:8} beweisen, wir wollen nun unter
Verwendung der Martingaltheorie einen eleganteren Beweis geben.

\begin{lemn}[Stutzungslemma]
Sei $(X_n)$ eine Folge unabhängiger, identisch verteilter Zufallsvariablen mit
$\E \abs{X_1}< \infty$. Setze $Y_n \defl X_n\Id_{[\abs{X_1}\le n]}$. Dann gilt
\begin{propenum}
\item $\E Y_n\to \E X_1$,
\item $P[X_n = Y_n\text{ für fast alle }n]=1$,
\item $(Y_n)$ erfüllt die Kolmogorov-Bedingung,
\begin{align*}
\sum_{n=1}^\infty n^{-2} \V(Y_n) < \infty.\fishhere
\end{align*}
\end{propenum}
\end{lemn}
\begin{proof}
\begin{proofenum}
\item Für jedes $n\ge 1$ besitzen $X_n$ und $X_1$ dieselbe Verteilung, folglich
ist $\E Y_n = \E (X_1\Id_{[\abs{X_1}\le n]})$. Weiterhin ist
$\abs{X_1\Id_{[\abs{X_1}\le n]}}\le \abs{X_1}$ und $X_1$ ist integrierbar. Mit
dem Satz von der dominierten Konvergenz folgt somit
\begin{align*}
\E Y_n \to \E X_1\fs
\end{align*}
\item Wir stellen zunächst fest, dass
\begin{align*}
[X_n = Y_n\text{ für fast alle
}n]^c &= [X_n \neq Y_n\text{ für unendlich viele
}n]^c\\ &= \limsup [X_n\neq Y_n].
\end{align*}
Ferner gilt
\begin{align*}
\sum_{n=1}^\infty P[X_n\neq Y_n] &= \sum_{n=1}^\infty P[\abs{X_1} >n] \le
\int_0^\infty P[\abs{X_1}>t]\dt\\
&\overset{\text{Lemma }\ref{lem:4.1}}= \E\abs{X_1} < \infty.
\end{align*}
Mit dem 1. Lemma von Borel und Cantelli folgt somit
\begin{align*}
P(\limsup [X_n\neq Y_n]) = 0.
\end{align*}
\item Es gilt $\V(Y_n) = \E Y_n^2 + (\E Y_n)^2$, wobei $\E Y_n$ konvergiert und
folglich beschränkt ist. Somit ist
\begin{align*}
\sum_{n\ge 1} \frac{\V(Y_n)}{n^2}
=
\sum_{n\ge 1} \frac{\E Y_n^2 - (\E Y_n)^2}{n^2}
\le
\sum_{n\ge 1} \frac{\E Y_n^2}{n^2} + M \sum_{n\ge 1}\frac{1}{n^2},   
\end{align*}
und nur der erste Term bedarf weiterer Untersuchung.
Für jedes $n\in\N$ gilt nach Partialbruchzerlegung,
\begin{align*}
\frac{1}{n^2} \le \frac{2}{n(n+1)}
\le
2\left(\frac{1}{n} - \frac{1}{n+1}\right).
\end{align*}
Damit erhalten wir folgende Abschätzung für den Reihenrest
\begin{align*}
\sum_{n\ge k}\frac{1}{n^2} \le \frac{2}{k},\tag{*}
\end{align*}
und somit
\begin{align*}
\sum_{n=1}^\infty \frac{\abs{X_n}^2\Id_{[\abs{X_1}\le n]}}{n^2}
\le
\sum_{n\ge 1\lor\abs{X_1}} \frac{\abs{X_1}^2}{n^2}
\le
\frac{2\abs{X_1}^2}{1\lor\abs{X_1}}
\le 2 \abs{X_1}.
\end{align*}
Also ist $\sum_{n\ge 1} \E Y_n^2/n^2 \le 2 \E \abs{X_1}$.\qedhere
\end{proofenum}
\end{proof}

\begin{proof}[Beweis von Satz \ref{prop:11.9}.]
Wir betrachten die gestutzten Zufallsvariablen
\begin{align*}
Y_n \defl X_n \Id_{[\abs{X_1}\le n]},\qquad S_n \defl \sum_{i=1}^n X_i.
\end{align*}
Nach Betrachtung b) im obigen Lemma genügt es zu zeigen,
\begin{align*}
\frac{1}{n}\sum_{i=1}^n Y_i \to \E X_1\fs
\end{align*}
Dazu betrachten wir die Zerlegung 
\begin{align*}
\frac{1}{n}\sum_{i=1}^n Y_i = 
\frac{1}{n}\sum_{i=1}^n \E Y_i
+ \frac{1}{n}\sum_{i=1}^n (Y_i-\E Y_i).
\end{align*}
Nach dem Grenzwertsatz von Caesaro gilt
\begin{align*}
\frac{1}{n}\sum_{i=1}^n \E Y_i \to \E X_1\fs
\end{align*}
und nach Satz \ref{prop:11.8} gilt außerdem
\begin{align*}
\frac{1}{n}\sum_{i=1}^n (Y_i-\E Y_i) \to 0\fs\qedhere
\end{align*}
\end{proof}

\begin{cor}
Sei $(X_n)$ eine Folge von unabhängigen identisch verteilten
reell-erweiterten Zufallsvariablen mit $\E X_1^- < \infty$ und $\E X_1^+ =
\infty$, d.h. $\E X_1 = \infty$. Dann gilt
\begin{align*}
\frac{1}{n}\sum_{n=1}^\infty X_i \to \infty\fs\fishhere
\end{align*}
\end{cor}
\begin{proof}
Mit Satz \ref{prop:11.9} gilt
\begin{align*}
\frac{1}{n}\sum_{i=1}^\infty X_i^- \to \E X_1^- \fs
\end{align*}
Deshalb können wir ohne Einschränkung davon ausgehen, dass $X_i \ge 0$. Sei
$k\in\N$ beliebig aber fest und setze
\begin{align*}
Z_n^{(k)} \defl X_n \Id_{[X_n \le k]}.
\end{align*}
So folgt $Z_n^{(k)} \le k$ und
\begin{align*}
\frac{1}{n}\sum_{i=1}^n X_i \ge
\frac{1}{n}\sum_{i=1}^n Z_i^{(k)}
\to \E Z_1^{(k)} \fs\tag{**}
\end{align*}
Aber $Z_1^{(k)}\uparrow X_1$ und mit dem Satz von der monotonen Konvergenz folgt
\begin{align*}
\E Z_1^{(k)}\uparrow \E X_1.
\end{align*}
Mit (**) folgt jedoch, dass
\begin{align*}
\frac{1}{n}\sum_{i=1}^n X_i \to \infty \fs\qedhere
\end{align*}
\end{proof}
% 
% Ein direktes Analogon zu Satz \ref{prop:9.13} für Dreiecksschemata von
% Zufallsvariablen ist.
% 
% \begin{prop}
% Das Schema reeller Zufallsvariablen $X_{n,}$ mit $j=1,\ldots.,j_n$ und das
% Schema der $\sigma$-Algebren $\AA_{nj}$ in $\Omega$ mit $\FF(X_{nj}\subset
% \AA_{nj} \subset \AA_{n,j+1}\subset \AA$ sollen die folgenden Bedingungen
% erfüllen
% \begin{propenum}
% \item $X_{n,j}$ ist quadratisch integrierbar, $\E_{j-1} X_{n-j}\defl
% \E(X_{nj}\mid\AA_{n,j-1}) = 0$,\\
% die $X_{n,j}$ bilden ein sogenanntes Martingaldifferenzschema.
% \item $\sum_{j=1}^{j_n} \E_{j-1}X_{n,j}^2 \overset{P}{\to} 1$.
% \item Für jedes $\ep > 0$ sei
% \begin{align*}
% \sum_{j=1}^{j_n} \E_{j-1}\left[X_{n,j}^2\Id_{[X_{n,j}>\ep]}\overset{P}{\to} 0
% \right]
% \end{align*}
% eine sogenannte bedingte Lindeberg-Bedingung.\fishhere
% \end{propenum}
% \end{prop}

% \cleardoublepage
\appendix

\chapter{Begriffe und Sätze der Maß- und Integrationstheorie}

Sei $(\Omega, {\mathcal A}, \mu)$ ein Maßraum.

\begin{propn}[Satz von der monotonen Konvergenz (B.\ Levi)]
Für erweitert reellwertige
messbare Funktionen $f_{n}$ mit $f_{n} \geq0$ und $f_{n}
\uparrow f$ existiert $\lim\limits_{n} \int f_{n}\dmu$ und es gilt
\begin{align*}
\lim\limits_{n} \int f_{n}\dmu = \int \lim\limits_{n}
f_{n}\dmu.\fishhere
\end{align*}
\end{propn}

\begin{propn}[Lemma von Fatou]
Für jede Folge $(f_n)$ von erweitert reellwertigen
messbaren Funktionen mit $f_n \ge 0$ $\mu$-f.ü. gilt
\begin{align*}
\int \liminf f_n \dmu \le \liminf \int f_n \dmu.\fishhere 
\end{align*}
\end{propn}

\begin{propn}[Satz von der dominierten Konvergenz (Lebesgue)]
Für erweitert-reellwertige messbare Funktionen $f_n$, $f$ und
$g$ mit $f_n \to f$ $\mufu$, $|f_n| \le g$ $\mufu$ für
alle $n$ und $\int g \dmu < \infty$ existiert $\lim_{n\to\infty} \int f_n\dmu$
und es gilt
\begin{align*}
\lim_{n\to\infty} \int f_n \dmu = \int f\dmu.\fishhere
\end{align*}
\end{propn}

\begin{defnn}
\begin{defnenum}
\item
Das Maß $\mu$ heißt \emph{$\sigma$-endlich}, wenn es ein
Folge von Mengen $A_n\in\AA$ gibt mit $A_n \uparrow
\Omega$ und $\mu(A_n) < \infty$.
\item Ein Maß $\nu$ auf ${\mathcal A}$ heißt \emph{$\mu$-stetig}, falls
\begin{align*}
\forall A \in \AA :  \mu(A)=0 \Rightarrow \nu(A)=0.\fishhere
\end{align*} 
\end{defnenum}
\end{defnn}

\begin{propn}[Satz von Radon-Nikodym]
Seien $(\Omega,\AA)$ Messraum, $\mu$ und $\nu$ $\sigma$-endliche Maße auf
$\AA$ und $\nu$ sei $\mu$-stetig.  Dann existiert eine Funktion
$f:(\Omega,\AA) \to (\R_+, \BB_+)$ mit
\begin{align*}
\forall A \in \AA : \nu(A)=\int_A f\dmu
\end{align*}
$f$ ist eindeutig bis auf Äquivalenz ``=$\mu$-f.ü.''. \\ $f$ ist die
sog.\ \emph{Radon-Nikodym-Ableitung} von $\nu$ nach $\mu$ und wird häufig
kurz durch $\frac{\dnu}{\dmu}$ angegeben.\fishhere
\end{propn}
\begin{proof}
Zum Beweis siehe Elstrodt.\qedhere
\end{proof}

\begin{bsp}
Sei $Q$ ein W-Maß auf $\BB$, $\lambda$ das LB-Maß auf $\BB$ und $\Q$ sei
$\lambda$-stetig. Dann ist die Radon-Nikodyn-Ableitung
\begin{align*}
\frac{\dQ}{\dlambda}
\end{align*}
die gewöhnliche Dichte von $Q$.\bsphere
\end{bsp}

\begin{propn}[Zusatz]
Seien $\mu,\nu,\ph$ $\sigma$-endliche Maße, $\ph$ sei $\nu$-stetig und $\nu$
sei $\mu$-stetig. Dann gilt
\begin{align*}
\frac{\dph}{\dmu} = \frac{\dph}{\dnu}\cdot \frac{\dnu}{\dmu}.\fishhere
\end{align*}
\end{propn}

%\end{appendix}

% 
% \addcontentsline{toc}{section}{Literatur}
%  
\begin{thebibliography}{99}

\bibitem{bauerII} Bauer, H. Wahrscheinlichkeitstheorie; de Gruyter, Berlin
  (1990), 4.\ Aufl.
\bibitem{bauerI} Bauer, H. Maß- und Integrationstheorie; de Gruyter, Berlin
  (1998), 2.~Aufl.
\bibitem{billingsley} Billingsley, P. Probability and Measure; Wiley, New York
  (1995), 3rd ed.
\bibitem{capinski-kopp} Capi\'{n}ski, M., Kopp, E. Measure Integral and
  Probability; Springer (2004), 2nd ed.
\bibitem{dudley} Dudley, R.M. Real Analysis and Probability; Cambridge
  University Press (2002), 2nd ed.
\bibitem{elstrodt} Elstrodt, J. Maß- und Integrationstheorie; Springer (2009),
  6.~Aufl.
\bibitem{feller} Feller, W. An Introduction to Probability and Its
  Applications I,II; Wiley, New York (1970/71), 3rd ed./2nd ed.
\bibitem{fisz} Fisz, M. Wahrscheinlichkeitsrechnung und mathematische
  Statistik; VEB Deutscher Verlag der Wissenschaften, Berlin (1970),
  5.\ Aufl.
\bibitem{gaenssler-stute} Gänssler, P., Stute, W. Wahrscheinlichkeitstheorie;
  Springer, Berlin (1977).
\bibitem{henze} Henze, N. Stochastik für Einsteiger: Eine Einführung in die
  faszinierende Welt des Zufalls; Vieweg+Teubner (2008), 7.~Aufl.
\bibitem{hesse} Hesse, C. Wahrscheinlichkeitstheorie --- Eine Einführung mit
  Beispielen und Anwendungen. Vieweg+Teubner (2009), 2.~Aufl.
\bibitem{hinderer} Hinderer, K. Grundbegriffe der Wahrscheinlichkeitstheorie;
  Springer, Berlin (1985), 3.~korr.\ Nachdruck.\\[-8mm]
\bibitem{jacod-protter} Jacod, J., Protter, P. Probability
  Essentials. Springer, Berlin (2004), 2nd ed.
\bibitem{kallenberg} Kallenberg, O.  Foundations of Modern
  Probability. Springer, New York (2001), 2nd ed.
\bibitem{klenke} Klenke, A. Wahrscheinlichkeitstheorie; Springer, Heidelberg
  (2006)
\bibitem{krengel} Krengel, U. Einführung in die Wahrscheinlichkeitstheorie und
  Statistik; Vieweg, Braunschweig (2003), 7.\ Aufl.
\bibitem{loeve} Loève, M. Probability Theory I,II; Springer, Berlin (1977/78),
  4th ed.
\bibitem{meintrup-schaeffler} Meintrup D., Schäffler S. Stochastik --- Theorie
  und Anwendungen; Springer (2004).
\bibitem{williams-1} Williams, D. Probability with Martingales; Cambridge
  University Press (1991).
\bibitem{wengenroth} Wengenroth, J. Wahrscheinlichkeitstheorie. De Gruyter
  (2008).
\bibitem{williams} Williams, D. Weighing the Odds --- A Course in Probability
  and Statistics; Cambridge University Press (2001).

\end{thebibliography}

%\printindex

\end{document}