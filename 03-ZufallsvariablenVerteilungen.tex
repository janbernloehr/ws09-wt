\chapter{Zufallsvariablen und Verteilungen}

\section{Meßbare Abbildungen und Zufallsvariablen}

Wir betrachten folgende Zufallsexperimente:
\begin{defnenum}
  \item Zufällige Anzahl der Erfolge bei $n$ Bernoulli-Versuchen.
  \item Zufällige Anzahl der an einem Schalter in einem bestimmten
  Zeitintervall $[0,T]$ eintreffenden Kunden.
  \item  Zufällige Wartezeit (von $0$ an gemessen) bis zum Eintreffen des
  nächsten Kunden.
  \item Zufälliger Abstand eines aus dem Intervall $[-1/2,1/2]$
  herausgegriffenen Punktes von $0$.
  \item Zufällige Lage eines Partikels im $\R^3$ zu einem bestimmten Zeitpunkt.
\end{defnenum}

Die zufälligen Größen können durch sogenannte ``Zufallsvariablen''
modelliert werden.

\begin{bsp}
\label{bsp:3.1}
Zu 4.): Betrachte den W-Raum $(\Omega,\AA,P)$ mit $\Omega = [-1/2,1/2]$,
$\AA=\BB\cap\Omega$ und $P=\lambda\big|_{\AA}$ (Restriktion
des LB-Maßes auf Mengen aus $\AA$). Ein Punkt $\omega\in\Omega$ hat Abstand
$Y(\omega)\defl\abs{\omega}$ von Null. $Y$ nennen wir Zufallsvariable.

Die Wahrscheinlichkeit, dass $Y\le x$ mit $x\in[0,1/2]$ fest, ist gegeben durch
\begin{align*}
P[Y\le x] \defl P\left(\setdef{\omega\in \Omega}{Y(\omega)\le x} \right)
= [-x,x]\in \AA.
\end{align*}
Allgemeiner ist die Wahrscheinlichkeit, dass $Y\in B$ für $B\in\BB$, gegeben
durch
\begin{align*}
P[Y\in B] \defl P\left(\setdef{\omega\in\Omega}{Y(\omega)\in B}\right).
\end{align*}
Damit $P[Y\in B]$ überhaupt definiert ist, muss $[Y\in B]$ eine messbare Menge
sein, d.h. im Definitionsbereich von $P$ liegen. Im Folgenden
erarbeiten wir die notwendigen Vorraussetzungen, dass dem so ist.\bsphere
\end{bsp}

\begin{defn}
\label{defn:3.1}
Seien $\Omega$, $\Omega '$ zwei nichtleere Mengen, $A' \subset
\Omega '$ und $X: \Omega \to \Omega '$ eine Abbildung. Die Menge
\begin{align*}
\setdef{\omega \in \Omega}{X(\omega ) \in A'} \defr X^{-1} (A')\defr [X \in
A']
\end{align*}
(in $\Omega $) heißt das \emph{Urbild} von $A'$ bezüglich der Abbildung $X$; 
die somit
definierte Abbildung $X^{-1}:{\cal P} (\Omega ') \to {\cal P} (\Omega )$ heißt
\emph{Urbildfunktion} (zu unterscheiden von einer inversen Funktion!).\fishhere
\end{defn}

In vielen Fällen untersuchen wir nicht nur einzelne Mengen sondern ganze
Mengensysteme und verwenden daher folgende Bezeichnung als Abkürzung.

\begin{bemn}[Bezeichnung.]
Sei $X: \Omega \to \Omega '$ und ${\cal C}'$ Mengensystem in
$\Omega '$. So ist
\begin{align*}
X^{-1}({\cal C}') \defl \setdef{ X^{-1} (A ')}{ A' \in {\cal C}'},
\end{align*}
wobei $X^{-1}({\CC}')$ ist ein Mengensystem in $\Omega$ ist.\maphere 
\end{bemn}

Um zu klären, unter welchen Voraussetzungen $[X\in B]$ messbar ist, wenn $B$
messbar ist, müssen wir untersuchen, wie sich die Urbildfunktion $X^{-1}$ auf
$\sigma$-Algebren im Bildraum auswirkt.

\begin{prop}
\label{prop:3.1}
Sei $X:\Omega \to \Omega' $.
\begin{propenum}
\item\label{prop:3.1:1}
$X^{-1}$ und Mengenoperationen $\cup$, $\sum$, $\cap$, $^{c}$, $\backslash$
sind vertauschbar.\\
z.B. ist $X^{-1}\left(\bigcup\limits_{\alpha \in I} A'_{\alpha}\right) =
\bigcup\limits_{\alpha \in I} X^{-1}(A'_{\alpha})\qquad$ für $A'_{\alpha}\subset
\Omega ', \alpha \in $ Indexbereich $I$.
\item\label{prop:3.1:2}
$X^{-1}(\emptyset) = \emptyset; \quad X^{-1}(\Omega ') = \Omega$.\\ 
$A' \subset B'\subset \Omega ' \Longrightarrow X^{-1} (A') \subset X^{-1}(B')$.
\item\label{prop:3.1:3}
Ist $\AA'$ $\sigma$-Algebra in $\Omega '$, so ist $X^{-1}(\AA') $
$\sigma$-Algebra in $\Omega$.
\item\label{prop:3.1:4}
${\CC}'\subset {\cal P}(\Omega') \Longrightarrow X^{-1} ({\cal F}_{\Omega' }
({\CC}')) = {\cal F}_{\Omega}(X^{-1}({\CC}'))$.\fishhere
\end{propenum}
\end{prop}
\begin{proof}
\ref{prop:3.1:1}-\ref{prop:3.1:4}: Der Beweis sei als Übungsaufgabe
überlassen.\qedhere
\end{proof}

Sei $X:(\Omega,\AA)\to (\Omega',\AA')$, so ist $X^{-1}(\AA')$ stets eine
$\sigma$-Algebra. Damit die Urbilder messbarer Mengen auch tatsächlich messbar
sind, müssen wir also fordern, dass $X^{-1}(\AA')\subseteq \AA$.

\begin{defn}
\label{3.2}
Seien $(\Omega ,\AA)$, $(\Omega ', \AA')$ Messräume.
Die Abbildung $X:\Omega \to \Omega '$ heißt $\AA$-${\cal A'}$-messbar
[kurz: \emph{messbar}; measurable], wenn gilt:
\begin{align*}
\forall {A' \in \AA'} : X^{-1}(A') \in \AA,
\quad \text{d.h. } X^{-1}(\AA') \subseteq \AA,
\end{align*}
d.h. Urbilder von messbaren Mengen in $\Omega '$ sind messbare Mengen in
$\Omega $.\\
 In diesem Falle verwenden wir die Schreibweise $X$: $(\Omega, \AA) \to (\Omega
 ', \AA')$.\fishhere
\end{defn}

Wie wir noch sehen werden, ist die Messbarkeit einer Abbildung eine
\textit{wesentlich} schwächere Voraussetzung als beispielsweise Stetigkeit oder
gar Differenzierbarkeit. Es erfordert sogar einiges an Aufwand, eine nicht
messbare Abbildung zu konstruieren.

\begin{bem}
\label{bem:3.1}
In Satz \ref{prop:3.1} \ref{prop:3.1:3} ist $X^{-1}(\AA')$ die kleinste der
$\sigma$-Algebren $\AA$ in $\Omega$ mit $\AA$-$\AA'$-Messbarkeit
von $X$.\maphere
\end{bem}

Wir fassen nun alle nötigen Vorraussetzungen in der folgenden Definition
zusammen.

\begin{defn}
\label{defn:3.3}
Sei $(\Omega, \AA, P)$ ein W-Raum, $(\Omega ', {\cal
A}')$ ein Messraum. Die Abbildung
\begin{align*}
X:(\Omega , \AA) \to (\Omega ', \AA')
\end{align*}
heißt \emph{Zufallsvariable $(ZV)$} auf $(\Omega, \AA,P)$
[mit Zustandsraum $\Omega '$] (ausführlich: $(\Omega ', \AA')$-Zufallsvariable auf
$(\Omega, \AA, P)$; random variable).

\begin{defnenum}
  \item $X: (\Omega,\AA) \to (\R, {\cal B})$ heißt \emph{reelle Zufallsvariable} 
(oft auch nur ZV).
\item $X: (\Omega, \AA) \to (\overline{\R}, \overline{\cal B})$ heißt 
\emph{ erweitert-reelle Zufallsvariable}.
\item $X: (\Omega, \AA) \to (\R^{n}, {\cal B}_{n})$ heißt 
\emph{$n$-dimensionaler Zufallsvektor}.
\item $X(\omega)$ für ein $\omega \in \Omega $ heißt eine \emph{Realisierung}
der Zufallsvariable $X$.   
\end{defnenum}
Bezeichnung. $\overline{\BB}\defl \setdef{B, B \cup \{+ \infty \}, B \cup \{
-\infty \},  B\cup \{- \infty , + \infty \}}{B \in {\BB}}$.\fishhere
\end{defn}

Zufallsvariablen ermöglichen es uns für ein Experiment lediglich einmal einen
W-Raum $(\Omega,\AA,P)$ zu modellieren und dann für jeden Aspekt, der uns
interessiert, eine Zufallsvariable zu konstruieren. $X(\omega)$ kann man als
Messung interpretieren, $X(\omega)$ ist der von $\omega$ abhängige Messwert.

Außerdem lassen sich Zufallsvariablen verknüpfen, wir können sie addieren,
multiplizieren, hintereinanderausführen, \ldots

Messbarkeit für alle Elemente der Bild-$\sigma$-Algebra nachzuweisen, kann sich
als äußert delikat herausstellen. Der folgende Satz besagt jedoch, dass
es genügt sich auf ein Erzeugersystem der Bild-$\sigma$-Algebra
zurückzuziehen.

\begin{prop}
\label{prop:3.2}
Seien $(\Omega, \AA)$, $(\Omega ', {\cal A '})$ Messräume, $ X:
\Omega \to \Omega '$ und ${\EE}'$ Erzeugersystem  von $\AA'$.
Dann ist $X$ genau dann messbar, wenn
\begin{align*}
\forall M\in {\EE}' : X^{-1}(M)\in \AA,\qquad
\text{d.h. } X^{-1}({\EE}') \subseteq \AA.\fishhere
\end{align*}
\end{prop}
\begin{proof}
$\Rightarrow$: Gilt \textit{per definitionem}.\\
$\Leftarrow$: Sei $X^{-1}(\EE')\subseteq \AA $, zu zeigen ist nun, dass
$X^{-1}(\AA')\subseteq \AA$. Es gilt
\begin{align*}
X^{-1}(\AA') = X^{-1}(\FF_{\Omega'}(\EE')) = \FF_\Omega(X^{-1}(\EE')) \subseteq
\AA,
\end{align*}
denn $X^{-1}(\EE')\subseteq \AA$ und $\FF_\Omega(X^{-1}(\EE'))$ ist die
kleinste $\sigma$-Algebra, die $X^{-1}(\EE')$ enthält und $\AA$ ist
$\sigma$-Algebra.\qedhere
\end{proof}
 
Ein einfaches aber äußerst nützliches Korollar ergibt sich, wenn wir uns auf
reellwertige messbare Abbildungen beschränken.
\begin{prop}[Korollar]
\label{prop:3.3}
Sei $(\Omega, \AA)$ ein Messraum und $X: \Omega \to \overline{\R}$
Abbildung. Dann sind folgende Aussagen äquivalent:
\begin{equivenum}
  \item $X$ $\AA$-$\overline{\BB}$-messbar.
  \item $\forall \alpha \in \R : [X \leq \alpha] \in \AA$.
  \item $\forall \alpha \in \R : [X < \alpha] \in \AA$.
  \item $\forall \alpha \in \R : [X \ge \alpha] \in \AA$.\fishhere
\end{equivenum}
\end{prop}
\begin{proof}
Wird geführt mit Satz \ref{prop:3.2} und der Tatsache, dass
$\setdef{(-\infty,\alpha]}{\alpha\in\R}$ ein Erzeugersystem von
$\overline{\BB}$ ist.\qedhere
\end{proof}

Insbesondere sind somit die Mengen $[X\le \alpha]$, $[X<\alpha]$, \ldots\ für
jede reelle Zufallsvariable $X$ messbar.

\begin{cor}
\label{cor:3.1}
Für zwei Abbildungen $X,Y : (\Omega, \AA)\to
(\overline{\R}, \overline{\BB})$ gilt
\begin{align*}
[X< Y], \quad 
[X\leq Y], \quad 
[X = Y], \quad 
[X \neq Y] \in \AA,
\end{align*}
wobei $[X<Y]\defl \setdef{\omega \in \Omega}{X(\omega) < Y(\omega )}$.\fishhere
\end{cor}
\begin{proof}
Nach dem Prinzip von Archimedes gilt
$[X<Y]=\bigcup_{\alpha\in\R} [X<\alpha < Y]$. Außerdem ist
\begin{align*}
\bigcup_{\alpha\in\R} [X<\alpha < Y] = \bigcup_{\alpha\in\Q} [X<\alpha <
Y] = \bigcup_{\alpha\in\Q} [X<\alpha] \cap [Y\le \alpha]^c.
\end{align*}
Für jedes $\alpha\in\Q$ ist $[X<\alpha] \cap [Y\le \alpha]^c$ messbar und die abzählbare
Vereinigung messbarer Mengen ist messbar.

Die übrigen Fälle folgen sofort, denn
\begin{align*}
&[X\le Y] = [X>Y]^c \in \AA,\\
&[X=Y] = [X\le Y]\cap [X\ge Y]\in \AA\\
&[X\neq Y] = [X=Y]^c \in \AA.\qedhere
\end{align*}
\end{proof}

\begin{prop}
\label{prop:3.4}
Sei $\varnothing \neq A \subset \R^{m}$ und $A \cap {\BB}_{m} \defl
\setdef{A \cap B}{B\in {\BB}_{m}}$. Jede stetige Abbildung $g: A \to
\R^{n}$ ist $A \cap {\BB}_{m}$-${\BB}_{n}$-messbar.\fishhere
\end{prop}
\begin{proof}
$\OO_n$, das System der offenen Mengen auf $\R^n$, ist ein
Erzeugersystem von $\BB_n$. Nach \ref{prop:3.2} genügt es  zu
zeigen, dass
\begin{align*}
g^{-1}(\OO_n) \subseteq A\cap\BB_m.
\end{align*} 
$g$ ist stetig, daher sind die Urbilder offener Mengen offene Mengen
$\cap\; A$.\qedhere
\end{proof}

Für Abbildungen $\R^m\opento\R^n$ ist somit Messbarkeit eine \textit{wesentlich}
schwächere Voraussetzung als Stetigkeit.
\begin{bsp}
Die \emph{Dirichletfunktion} $f: \R\to\R$ gegeben durch,
\begin{align*}
f(x) = 
\begin{cases}
1, & x\in\Q,\\
0, & x\in\R\setminus\Q,
\end{cases}
\end{align*}
ist messbar aber sicher \textit{nicht} stetig.\bsphere
\end{bsp}

\begin{prop}
\label{prop:3.5}
Seien $X: (\Omega_{1}, \AA_{1}) \to (\Omega _{2}, \AA_{2})$,
$Y:(\Omega _{2}, \AA_{2}) \to (\Omega _{3}, \AA_{3})$.
Die zusammengesetzte Abbildung $Y \circ X: \Omega _{1} \to \Omega _{3}$ ist dann
$\AA_{1}$-$\AA_{3}$-messbar.\fishhere
\end{prop}
\begin{proof}
Offensichtlich gilt $(Y\circ X)^{-1}(\AA_3) = X^{-1}\circ Y^{-1}(\AA_3) 
\subseteq \AA_1$.\qedhere
\end{proof}

\begin{cor}
\label{cor:3.2}
Für $X: (\Omega , \AA)\to (\R^{m}, {\BB}_{m})$ und
eine stetige Abbildung $g: \R^{m} \to \R^{n}$ ist $g \circ X: \Omega \to \R^{n}$
 $\AA$-${\BB}_{n}$-messbar.\fishhere
\end{cor}

Insbesondere ist die Komposition von Zufallsvariablen bzw. von
Zufallsvariablen mit stetigen Funktionen wieder eine Zufallsvariable.  Für eine
reelle Zufallsvariable $X$ sind somit auch $\abs{X}$, $X^2$, $\sqrt{\abs{X}}$,
\ldots\ Zufallsvariablen.

\begin{prop}
\label{prop:3.6}
Sei $(\Omega , \AA)$ Messraum.
\begin{propenum}
\item
  Seien $X_{n} :(\Omega , \AA) \to (\R , {\BB})$ für $n = 1, \ldots, m$. Dann
  ist
\begin{align*}
 Y: \Omega \to \R^{m},\; \omega\mapsto Y(\omega)\defl (X_{1} (\omega )),
  \ldots , X_{m}(\omega )),\qquad \omega\in\Omega,
\end{align*}
$\AA$-${\cal B}_{m}$-messbar. Für $g: (\R^{m},
{\BB}_{m}) \to (\R, {\BB})$ ist
\begin{align*}
g \circ Y: \Omega \to \R
\end{align*}
$\AA$-${\BB}$-messbar.
\item
Seien $X_{1,2}: (\Omega, \AA) \to (\R, {\BB})$.
Dann sind auch die Abbildungen
\begin{defnpropenum}
  \item $\alpha X_{1} + \beta X_{2} \quad (\alpha , \beta \in \R)'$,
  \item $X_{1}X_{2}$,
  \item $\dfrac{X_{1}}{X_{2}}$ (falls existent)   
\end{defnpropenum}
$\AA$-$\BB$-messbar.\fishhere
\end{propenum}
\end{prop}
\begin{proof}
\begin{proofenum}
\item Ein Erzeugersystem $\CC$ von $\BB_m$ ist gegeben durch
\begin{align*}
(-\infty,\alpha_1]\times \ldots \times (-\infty,\alpha_m],
\end{align*}
wobei $(\alpha_1,\ldots,\alpha_m)\in\R^m$.
\begin{figure}[!htbp]
\centering
\begin{pspicture}(0,-2.13)(6,2.15)
\psframe[linestyle=none,fillcolor=glightgray,fillstyle=solid](3.1,1.13)(0,-2.13)

\psline{->}(2.12,-2.13)(2.1,1.93)
\psline{->}(0.0,0.51)(4.04,0.53)

\psline[linecolor=darkblue](0,1.13)(3.1,1.13)(3.1,-2.13)

\rput(2.57,1.955){\color{gdarkgray}$\R^2$}

\rput(4.1,1.3){\color{gdarkgray}$(\alpha_1,\alpha_2)$}

\rput(1.88,1.3){\color{gdarkgray}$\alpha_2$}

\rput(3.36,0.32){\color{gdarkgray}$\alpha_1$}
\end{pspicture}
\caption{Erzeuger der $\BB_n$.}
\end{figure}
Es genügt zu zeigen, dass $Y^{-1}(\CC)\subseteq \AA$.
Nun gilt
\begin{align*}
&Y^{-1}\left((-\infty,\alpha_1]\times \ldots \times (-\infty,\alpha_m]
\right)\\
&\qquad = \setdef{\omega\in \Omega}{x_1(\omega) \le \alpha_1, \ldots,
X_m(\omega) \le \alpha_m} \\
&\qquad = [X_1\le \alpha_1]\cap \ldots \cap [X_m\le
\alpha_m]\in \AA.\qedhere
\end{align*}
\item Die Messbarkeit folgt mit Korollar \ref{cor:3.2} und der Stetigkeit von
Summen- und Produktabbildung $(x,y)\mapsto x+y$, $(x,y)\mapsto x\cdot
y$.\qedhere
\end{proofenum}
\end{proof}

Eine Abbildung $X:\Omega\to \R^n$ ist also genau dann messbar, wenn jede
Komponente $X_n:\Omega\to\R$ messbar ist. Insbesondere sind Produkte und
Summen von Zufallsvariablen stets messbar.

\begin{prop}
\label{prop:3.7}
Sei  $(\Omega, \AA)$ Messraum, $X_{n} : (\Omega, \AA) \to
(\overline{\R}, \overline{\BB})$ für $n= 1,2, \ldots$.
Dann sind
\begin{defnpropenum}
  \item $\inf\limits_{n} X_{n}$,
  \item $\sup\limits_{n} X_{n}$,
  \item $\limsup\limits_{n} X_{n}$,
  \item $\liminf \limits_{n} X_{n}$,
  \item $\lim\limits_{n} X_{n}$ (falls existent)
\end{defnpropenum}
$\AA$-$\overline{\BB}$-messbar.\fishhere
\end{prop}
\begin{proof}
\begin{proofenumarabicbr}
  \item $\forall\alpha\in\R : [\inf X_n < \alpha] = \bigcup_{n\in\N} [X_n <
  \alpha] \in \AA$, d.h. $\inf X_n$ ist $\AA$-$\overline{\BB}$-messbar.
  \item $\sup=-\inf(-X_n)$ ist $\AA$-$\overline{\BB}$-messbar.
  \item $\limsup X_n = \inf_n \sup_{k\ge n} X_k$ 
 ist $\AA$-$\overline{\BB}$-messbar.
 \item $\liminf X_n = \sup_n \inf_{k\ge n} X_k$ ist
 $\AA$-$\overline{\BB}$-messbar.
 \item $\lim X_n = \limsup X_n$, falls $\lim_n X_n$ existiert.\qedhere
\end{proofenumarabicbr}
\end{proof}

Da Messbarkeit eine so schwache Voraussetzung ist, ist sie auch ein sehr
stabiler Begriff, da sie auch unter Grenzwertbildung und Komposition erhalten
bleibt. Probleme treten erst bei überabzählbar vielen Operationen auf.

\clearpage
\section{Bildmaße und Verteilungen}

In diesem Abschnitt werden wir eine Beziehung zwischen
Zufallsvariablen und Maßen herstellen. Jeder Zufallsvariablen
\begin{align*}
X:(\Omega,\AA,P)\to (\Omega',\AA')
\end{align*}
lässt sich eindeutig ein Maß
\begin{align*}
P_X : \AA' \to \R,\quad A'\mapsto P_X(A) \defl P(X^{-1}(A)) 
\end{align*}
das sogenannte \emph{Bildmaß} zuordnen. Die Verteilungsfunktion des
Bildmaßes
\begin{align*}
F(t) = P_X((-\infty,t])
\end{align*}
nennen wir auch \emph{Verteilungsfunktion von $X$}. $F$ lässt sich oft leichter
handhaben als $X$, und aus den Eigenschaften von $F$ lassen sich Rückschlüsse
auf $X$ machen.

Analog lässt sich zu jedem Maß
\begin{align*}
\mu:\AA'\to\R
\end{align*}
über $\Omega'$ eine Zufallsvariable
\begin{align*}
Y: (\Omega,\AA,Q)\to (\Omega',\AA')
\end{align*}
finden, so dass $P_Y=\mu$. 

\clearpage

Zunächst ist natürlich zu klären, ob das Bildmaß überhaupt ein Maß ist.

\begin{prop}
\label{prop:3.8}
Seien $(\Omega, \AA)$ und $ (\Omega ', \AA')$ Messräume und
 $X:(\Omega, \AA) \to (\Omega ', \AA')$ eine Abbildung. Sei $\mu$ ein
Maß auf $\AA$. Durch
\begin{align*}
\mu_{X}(A') & \defl  \mu(X^{-1}(A')) \\
              & =   \mu \left(\setdef{\omega \in \Omega}{X(\omega ) \in
              A'}\right) \defr \mu [ X \in A']; \quad A' \in \AA'
\end{align*}
wird ein Maß (das sogenannte \emph{Bildmaß}) $\mu _{X}$ auf $\AA'$
definiert.\\
Ist $\mu $ ein W-Maß auf $\AA$, dann ist $\mu _{X}$ ebenfalls ein W-Maß auf
$\AA'$.\fishhere
\end{prop}
\begin{proof}
\begin{proofenumroman}
  \item $\mu_X \ge 0$ ist klar.
  \item $\mu_X(\varnothing) = \mu(X^{-1}(\varnothing)) = \mu(\varnothing) = 0$.
  \item Seien $A_1',A_2',\ldots\in\AA'$ paarweise disjunkt, so gilt
\begin{align*}
\mu_X\left(\sum\limits_{i\ge 1} A_i'\right) &= 
\mu\left(X^{-1}\left(\sum\limits_{i\ge 1} A_i'\right)\right)
= \mu\left(\sum\limits_{i\ge 1}\left(X^{-1}(A_i')\right)\right)\\
&= \sum\limits_{i\ge 1}\mu\left(X^{-1}(A_i')\right).
\end{align*}
\item Sei $\mu$ ein Wahrscheinlichkeitsmaß so ist
\begin{align*}
\mu_X(\Omega') = \mu(X^{-1}(\Omega')) = \mu(\Omega) = 1.\qedhere
\end{align*}
\end{proofenumroman}
\end{proof}

Es genügt also, dass $X$ messbar ist, damit $P_X$ tatsächlich ein Maß ist.
Insbesondere ist für jede Zufallsvariable $P_X$ ein Maß.

\begin{defn}
\label{defn:3.4}
Sei $X$ eine $(\Omega ', \AA')$-Zufallsvariable auf dem W-Raum
$(\Omega, \AA, P)$.
Das W-Maß $P_{X}$ im Bild-W-Raum $(\Omega ', \AA', P_{X})$ heißt
\emph{Verteilung} der Zufallsvariable $X$.\\
Sprechweise:
\begin{itemize}[label=-]
  \item Die Zufallsvariable $X$ liegt in $A'\in \AA'$.
  \item $X$ nimmt Werte in $A' \in \AA'$ an mit Wahrscheinlichkeit $P_{X}(A') =
  P[X \in A']$.
  \item Wenn $P[X \in A'] = 1$, sagt man $X$ liegt $P$-fast sicher
  ($P$-f.s.) in $A'$.\fishhere
\end{itemize}
\end{defn}

Betrachten wir die Verteilung $P_X$ einer reellen Zufallsvariablen $X$, so
besitzt diese eine Verteilungsfunktion
\begin{align*}
F(t) = P_X((-\infty,t]) = P(X^{-1}(-\infty,t]).
\end{align*}
Da es sich bei der Verteilungsfunktion um eine reelle Funktion
handelt, kann man oft viel leichter mit ihr Rechnen, als mit der
Zufallsvariablen selbst.

Die Eigenschaften der Verteilungsfunktion charakterisieren die Zufallsvariable,
man klassifiziert Zufallsvariablen daher nach Verteilung (binomial-, poisson-~,
exponentialverteilt, \ldots).

\begin{defnn}
Besitzen zwei Zufallsvariaben dieselbe Verteilung (also dieselbe
Verteilungsfunktion) heißen sie \emph{gleichverteilt}.\fishhere
\end{defnn}

\begin{bsp}
Wir betrachten $n$ Bernoulli-Versuche mit jeweiliger Erfolgswahrscheinlichkeit
$p$. $\Omega$ besteht aus der Menge der Elementarereignisse $\omega$ $(=$
$n$-Tupel aus Nullen und Einsen$)$, $\AA=\PP(\Omega)$.

Das Wahrscheinlichkeitsmaß $P$ auf $\AA$ ist gegeben durch
\begin{align*}
P(\setd{\omega}) = p^k(1-p)^{n-k},
\end{align*}
falls $\omega$ aus $k$ Einsen und $n-k$ Nullen besteht.

Wir definieren uns eine Zufallsvariable $X$ durch:
\begin{align*}
X: \Omega \to \R,\quad X(\omega) = \text{Anzahl der Einsen in $\omega$}.
\end{align*}
$X$ ist $\AA-\BB$-messbar, denn $X^{-1}(\BB)\subseteq \PP(\Omega) = \AA$.

$X : (\Omega,\AA,P)\to(\R,\BB)$ gibt also die zufällige Anzahl der Erfolge in
$n$ Bernoulli-Versuchen an.

Das Bildmaß zu $X$ ist gegeben durch
\begin{align*}
P_X(\setd{k}) = P[X=k] = \binom{n}{k}p^k(1-p)^{n-k} \defr b(n,p;k).
\end{align*}
Um das Bildmaß auf allgemeine Mengen in $\BB$ fortzusetzen, setzen wir zu
$B\in\BB$
\begin{align*}
P_X(B) = \sum\limits_{k\in\N_0\cap B} b(n,p;k).
\end{align*}
$P_X$ ist die sogenannte \emph{Binominalverteilung} $b(n,p)$. $X$ ist
$b(n,p)$-verteilt.\bsphere
\end{bsp}

\begin{bem}
\label{bem:3.2}
Seien $(\Omega, \AA, P)$ ein W-Raum und
$(\Omega ', \AA')$, $(\Omega '', \AA'')$  Messräume.
Seien
$X: (\Omega , \AA) \to (\Omega ', \AA')$ und $Y: (\Omega ', \AA')\to
(\Omega '', \AA'')$, dann gilt
\begin{align*}
P_{Y \circ X} = (P_{X})_{Y}.\maphere
\end{align*}
\end{bem}
\begin{proof}
$Y\circ X$ ist wieder messbar. Für $A''\in\AA''$ gilt somit,
\begin{align*}
P_{Y\circ X}(A'') &= P((Y\circ X)^{-1}(A'')) =
P(X^{-1}(Y^{-1}(A''))) \\ &= P_X(Y^{-1}(A'') = (P_X)_Y(A'').\qedhere
\end{align*}
\end{proof}

\subsection{Produktmessräume}

Betrachten wir die Vektorräume $\R^n$ und $\R^m$, so können wir diese durch das
karthesische Produkt verknüpfen zu $V=\R^n\times \R^m$. $V$ ist dann wieder
ein Vektorraum und jedes Element $v\in V$ lässt sich darstellen als
$v=(x,y)$, wobei $x\in\R^n$ und $y\in\R^m$.

Eine solche Verknüpfung lässt sich auch für Messräume definieren.

\begin{defn}
\label{defn:3.5}
Seien $(\Omega _{i}, \AA_{i})$ Messräume für $i= 1, \ldots ,n$.
Die \emph{Produkt-$\sigma$-Algebra}
$\bigotimes^{n}_{i=1} \AA_{i} $ wird definiert als die von dem
Mengensystem
\begin{align*}
\setdef{
\prod\limits^{n}_{i=1}A_{i}
}{
A_{i} \in \AA_{i},\quad  i= 1, \ldots ,n}
\end{align*}
erzeugte $\sigma$-Algebra.\\
$\bigotimes^{n}_{i=1}(\Omega _{i}, \AA_{i}) \defl
\left(\prod\limits^{n}_{i=1} \Omega _{i},\,
\bigotimes^{n}_{i=1} \AA_{i}\right)$
heißt \emph{Produkt-Messraum}.\fishhere
\end{defn}

Man bildet also das karthesische Produkt der $\Omega_i$ und die
Produkt-$\sigma$-Algebra der $\AA_i$ und erhält so wieder einen Messraum
$\bigotimes_{i=1}^n (\Omega_i,\AA_i)$.

\begin{bsp}
$(\R^n\times\R^m$, ${\BB}_{n} \bigotimes {\BB}_{m}) =
(\R^{n+m},{\BB}_{n+m})$.\bsphere
\end{bsp}

\begin{bem}
\label{bem:3.3}
Sei $X_{i}: (\Omega , \AA) \to (\Omega _{i},
\AA_{i})$ messbar für $i=1, \ldots , n$. Die Abbildung
\begin{align*}
X: \Omega \to \prod^{n}_{i=1} \Omega _{i}
\end{align*}
mit
\begin{align*}
X(\omega )\defl (X_{1} (\omega ), \ldots , X_{n}(\omega )), \qquad \omega \in
\Omega
\end{align*}
ist dann $\AA$-$\bigotimes^{n}_{i=1}\AA_{i}$-messbar.\maphere
\end{bem}
\begin{proof}
Betrachte ein Erzeugendensystem von $\bigotimes_{i=1}^n A_i$, z.B. die Quader
\begin{align*}
\CC = \setdef{\prod\limits_{i=1}^n A_i}{A_i\in\AA_i}.
\end{align*}
Der Beweis wird dann wie im eindimensionalen Fall geführt.\qedhere
\end{proof}

Insbesondere ist eine Abbildung
\begin{align*}
X: \Omega\to\R^n,\qquad \omega\mapsto (x_1,\ldots,x_n)
\end{align*}
genau dann messbar, wenn es jede Komponete $X_i:\Omega\to \R$ ist.

\begin{bem}
\label{bem:3.4}
Seien $(\Omega _{i}, \AA_{i})$ Messräume für $i=1,\ldots ,n$. Ein W-Maß auf
$\bigotimes^{n}_{i=1} \AA_{i}$ ist eindeutig festgelegt durch seine Werte
auf dem Mengensystem
\begin{align*}
\setdef{\prod\limits^{n}_{i=1} A_{i}}{A_{i}
\in \AA_{i},\;i=1, \ldots ,n}.\maphere
\end{align*}
\end{bem}
\begin{proof}
Fortsetzungs- und Eindeutigkeitssatz \ref{prop:1.4} mit der $\sigma$-Algebra
gegeben durch die endliche Summe von Quadern der Form
\begin{align*}
\prod\limits_{i=1}^n A_i,\qquad A_i\in\AA_i.\qedhere
\end{align*}
\end{proof}

\begin{defn}
\label{defn:3.6}
Seien $X_{i}: (\Omega,\AA,P)\to(\Omega _{i}, \AA_{i})$ Zufallsvariablen für
$i=1, \ldots ,n$ und $X\defl (X_{1},
\ldots, X_{n})$ Zufallsvariable auf $(\Omega, \AA, P)$.

\begin{defnenum}
  \item Die Verteilung $P_{X}$ der Zufallsvariablen $X$  - erklärt durch
\begin{align*}
P_{X} (A) \defl P\left[(X_{1}, \ldots , X_{n}) \in A \right]
\end{align*}
für $A \in \bigotimes^{n}_{i=1} \AA_{i}$ oder auch nur für
$A= \prod\limits^{n}_{i=1} A_{i} \; (A_{i} \in \AA_i, \; i=
1, \ldots ,n)$ - heißt die \emph{gemeinsame Verteilung} der Zufallsvariablen $X_{i}$.
\item
Die Verteilungen ${P_{X_{i}}}$ - erklärt durch
\begin{align*}
P_{X_{i}}(A_{i}) &\defl P[X_{i} \in A_{i}] \\ &= P[X \in \Omega_{1}
\times \ldots \times \Omega _{i-1} \times A_{i} \times \Omega _{i+1} \times
\ldots \times \Omega _{n}]
\end{align*}
für $A_{i} \in \AA_{i}$ - heißen \emph{Randverteilungen} von
$P_{X} \quad (i=1,\ldots , n)$.\fishhere  
\end{defnenum}
\end{defn}

\begin{bemn}[Wichtiger Spezialfall.]
$(\Omega _{i}, \AA_{i}) =
(\R, {\BB})$ für $i=1, \ldots, n$. Dann sind die $X_i$ reelle
Zufallsvariablen und $X$ ein Zufallsvektor.\maphere
\end{bemn}

\begin{bem}[Bemerkungen.]
\label{bem:3.5}
Seien die Bezeichnungen wie in Definition \ref{defn:3.6}.
\begin{bemenum}
\item
Die Verteilung $P_{X}$ ist ohne Zusatzvoraussetzung durch ihre Randverteilungen
\textit{nicht} eindeutig festgelegt.
\item
Die Projektionsabbildung
\begin{align*}
\pi _{i} : \prod^{n}_{k=1} \Omega _{k} \to \Omega_{i},\qquad (\omega _{1},
\ldots, \omega _{n}) \to \omega _{i}
\end{align*}
ist messbar
und somit gilt
\begin{align*}
P_{X_{i}} = (P_{X})_{\pi _{i}},\qquad i=1, \ldots , n.\maphere
\end{align*}
\end{bemenum}
\end{bem}

Aufgrund der Messbarkeit der Projektionen sind die Randverteilungen
einer gemeinsamen Verteilung eindeutig bestimmt. Die Umkehrung, dass die
Randverteilungen auch die gemeinsame Verteilung eindeutig bestimmen ist im
Allgemeinen falsch. Wir werden uns in Kapitel \ref{chap:5} ausführlicher damit
beschäftigen.

\begin{bem}[Bemerkungen.]
\label{bem:3.6}
\begin{bemenum}
\item
Sei $Q$ ein W-Maß auf $(\R^n, {\BB}_{n})$.
Dann existieren reelle Zufallsvariablen $X_{1}, \ldots , X_{n}$
auf einem geeigneten W-Raum $(\Omega ,\AA,P)$ so,
dass ihre gemeinsame Verteilung mit $Q$ übereinstimmt: \\
Auf dem W-Raum $(\Omega, \AA,P) \defl (\R^{n}, {\BB}_{n}, Q)$, wird
$X_{i} : \Omega \to \R$ definiert als die Projektionsabbildung
\begin{align*}
(x_{1}, \ldots, x_{n}) \to x_{i} \quad (i=1, \ldots,n);
\end{align*}
hierbei ist also $X\defl (X_{1}, \ldots ,X_{n})$ die auf
$\R^n$ definierte identische Abbildung.
\item
Wir können die Aussage aus a) unmittelbar auf einen beliebigen Produkt-Meßraum
$\left(\prod\limits^{n}_{i=1}\Omega _{i}, \;
\bigotimes^{n}_{i=1} \AA_{i}\right)$
statt $(\R^{n}, {\BB}_{n})$ verallgemeinern.
\item Sonderfall zu b): Ist $(\Omega, \AA,Q)$ ein W-Raum, $X:\Omega \to
\Omega$ die identische Abbildung, so gilt $P_{X} =Q$. Jedes W-Maß lässt sich somit
als eine Verteilung auffassen (und - definitionsgemäß - umgekehrt).\maphere
\end{bemenum}
\end{bem}

Dieser Zusammenhang zwischen W-Maß und Verteilung hat eine große Bedeutung, wie
wir mit der Einführung des Maßintegrals im folgenden Kapitel sehen werden.
